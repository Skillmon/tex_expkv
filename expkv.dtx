% \iffalse meta-comment
%
% File: expkv.dtx Copyright (C) 2020 Jonathan P. Spratte
%
% This work  may be  distributed and/or  modified under  the conditions  of the
% LaTeX Project Public License (LPPL),  either version 1.3c  of this license or
% (at your option) any later version.  The latest version of this license is in
% the file:
%
%   http://www.latex-project.org/lppl.txt
%
% ------------------------------------------------------------------------------
%
%<*driver>^^A>>=
\def\nameofplainTeX{plain}
\ifx\fmtname\nameofplainTeX\else
  \expandafter\begingroup
\fi
\input l3docstrip.tex
\askforoverwritefalse
\preamble

--------------------------------------------------------------
expkv -- an expandable key=val implementation
E-mail: jspratte@yahoo.de
Released under the LaTeX Project Public License v1.3c or later
See http://www.latex-project.org/lppl.txt
--------------------------------------------------------------

Copyright (C) 2020 Jonathan P. Spratte

This  work may be  distributed and/or  modified under  the conditions  of the
LaTeX Project Public License (LPPL),  either version 1.3c  of this license or
(at your option) any later version.  The latest version of this license is in
the file:

  http://www.latex-project.org/lppl.txt

This work is "maintained" (as per LPPL maintenance status) by
  Jonathan P. Spratte.

This work consists of the file  expkv.dtx
and the derived files           expkv.pdf
                                expkv.sty
                                expkv.tex

\endpreamble
% stop docstrip adding \endinput
\postamble
\endpostamble
\generate{\file{expkv.sty}{\from{expkv.dtx}{pkg}}}
\generate{\file{expkv.tex}{\from{expkv.dtx}{tex}}}
\ifx\fmtname\nameofplainTeX
  \expandafter\endbatchfile
\else
  \expandafter\endgroup
\fi
%
\ProvidesFile{expkv.dtx}
  [\csname expkv@date\endcsname\ an expandable key=val implementation]
\PassOptionsToPackage{full}{textcomp}
\documentclass{l3doc}
\RequirePackage[oldstylenums]{kpfonts}
\RequirePackage{listings}
\RequirePackage{booktabs}
\RequirePackage{siunitx}
\RequirePackage{xcolor}
\RequirePackage{caption}
\RequirePackage{microtype}
\lstset
  {
    ,flexiblecolumns=false
    ,basewidth=.52em
    ,gobble=2
    ,basicstyle=\itshape
    ,morekeywords={\ekvdef,\ekvdefNoVal,\ekvset}
    ,morecomment=[l]\%
    ,commentstyle=\color[gray]{0.4}
    ^^A,literate=*{<key>}{\key}{4}{<set>}{\set}{4}
  }
\RequirePackage{randtext}
\IfFileExists{expkv.tex}{% \iffalse meta-comment
%
% File: expkv.dtx Copyright (C) 2020 Jonathan P. Spratte
%
% This work  may be  distributed and/or  modified under  the conditions  of the
% LaTeX Project Public License (LPPL),  either version 1.3c  of this license or
% (at your option) any later version.  The latest version of this license is in
% the file:
%
%   http://www.latex-project.org/lppl.txt
%
% ------------------------------------------------------------------------------
%
%<*driver>^^A>>=
\def\nameofplainTeX{plain}
\ifx\fmtname\nameofplainTeX\else
  \expandafter\begingroup
\fi
\input l3docstrip.tex
\askforoverwritefalse
\preamble

--------------------------------------------------------------
expkv -- an expandable key=val implementation
E-mail: jspratte@yahoo.de
Released under the LaTeX Project Public License v1.3c or later
See http://www.latex-project.org/lppl.txt
--------------------------------------------------------------

Copyright (C) 2020 Jonathan P. Spratte

This  work may be  distributed and/or  modified under  the conditions  of the
LaTeX Project Public License (LPPL),  either version 1.3c  of this license or
(at your option) any later version.  The latest version of this license is in
the file:

  http://www.latex-project.org/lppl.txt

This work is "maintained" (as per LPPL maintenance status) by
  Jonathan P. Spratte.

This work consists of the file  expkv.dtx
and the derived files           expkv.pdf
                                expkv.sty
                                expkv.tex

\endpreamble
% stop docstrip adding \endinput
\postamble
\endpostamble
\generate{\file{expkv.sty}{\from{expkv.dtx}{pkg}}}
\generate{\file{expkv.tex}{\from{expkv.dtx}{tex}}}
\ifx\fmtname\nameofplainTeX
  \expandafter\endbatchfile
\else
  \expandafter\endgroup
\fi
%
\IfFileExists{expkv.tex}{% \iffalse meta-comment
%
% File: expkv.dtx Copyright (C) 2020 Jonathan P. Spratte
%
% This work  may be  distributed and/or  modified under  the conditions  of the
% LaTeX Project Public License (LPPL),  either version 1.3c  of this license or
% (at your option) any later version.  The latest version of this license is in
% the file:
%
%   http://www.latex-project.org/lppl.txt
%
% ------------------------------------------------------------------------------
%
%<*driver>^^A>>=
\def\nameofplainTeX{plain}
\ifx\fmtname\nameofplainTeX\else
  \expandafter\begingroup
\fi
\input l3docstrip.tex
\askforoverwritefalse
\preamble

--------------------------------------------------------------
expkv -- an expandable key=val implementation
E-mail: jspratte@yahoo.de
Released under the LaTeX Project Public License v1.3c or later
See http://www.latex-project.org/lppl.txt
--------------------------------------------------------------

Copyright (C) 2020 Jonathan P. Spratte

This  work may be  distributed and/or  modified under  the conditions  of the
LaTeX Project Public License (LPPL),  either version 1.3c  of this license or
(at your option) any later version.  The latest version of this license is in
the file:

  http://www.latex-project.org/lppl.txt

This work is "maintained" (as per LPPL maintenance status) by
  Jonathan P. Spratte.

This work consists of the file  expkv.dtx
and the derived files           expkv.pdf
                                expkv.sty
                                expkv.tex

\endpreamble
% stop docstrip adding \endinput
\postamble
\endpostamble
\generate{\file{expkv.sty}{\from{expkv.dtx}{pkg}}}
\generate{\file{expkv.tex}{\from{expkv.dtx}{tex}}}
\ifx\fmtname\nameofplainTeX
  \expandafter\endbatchfile
\else
  \expandafter\endgroup
\fi
%
\IfFileExists{expkv.tex}{% \iffalse meta-comment
%
% File: expkv.dtx Copyright (C) 2020 Jonathan P. Spratte
%
% This work  may be  distributed and/or  modified under  the conditions  of the
% LaTeX Project Public License (LPPL),  either version 1.3c  of this license or
% (at your option) any later version.  The latest version of this license is in
% the file:
%
%   http://www.latex-project.org/lppl.txt
%
% ------------------------------------------------------------------------------
%
%<*driver>^^A>>=
\def\nameofplainTeX{plain}
\ifx\fmtname\nameofplainTeX\else
  \expandafter\begingroup
\fi
\input l3docstrip.tex
\askforoverwritefalse
\preamble

--------------------------------------------------------------
expkv -- an expandable key=val implementation
E-mail: jspratte@yahoo.de
Released under the LaTeX Project Public License v1.3c or later
See http://www.latex-project.org/lppl.txt
--------------------------------------------------------------

Copyright (C) 2020 Jonathan P. Spratte

This  work may be  distributed and/or  modified under  the conditions  of the
LaTeX Project Public License (LPPL),  either version 1.3c  of this license or
(at your option) any later version.  The latest version of this license is in
the file:

  http://www.latex-project.org/lppl.txt

This work is "maintained" (as per LPPL maintenance status) by
  Jonathan P. Spratte.

This work consists of the file  expkv.dtx
and the derived files           expkv.pdf
                                expkv.sty
                                expkv.tex

\endpreamble
% stop docstrip adding \endinput
\postamble
\endpostamble
\generate{\file{expkv.sty}{\from{expkv.dtx}{pkg}}}
\generate{\file{expkv.tex}{\from{expkv.dtx}{tex}}}
\ifx\fmtname\nameofplainTeX
  \expandafter\endbatchfile
\else
  \expandafter\endgroup
\fi
%
\IfFileExists{expkv.tex}{\input{expkv.tex}}{}
\ProvidesFile{expkv.dtx}
  [\csname ekvDate\endcsname\ an expandable key=val implementation]
\PassOptionsToPackage{full}{textcomp}
\documentclass{l3doc}
\RequirePackage[oldstylenums,nott]{kpfonts}
\input{glyphtounicode}
\pdfgentounicode=1
\RequirePackage{xfp} ^^A required for an example
\RequirePackage{listings}
\RequirePackage{booktabs}
\RequirePackage{array}
\RequirePackage{collcell}
\RequirePackage{siunitx}
\RequirePackage{xcolor}
\RequirePackage{caption}
\RequirePackage{microtype}
\RequirePackage{accsupp}
\lstset
  {
    ,flexiblecolumns=false
    ,basewidth=.53em
    ,gobble=2
    ,basicstyle=\fontfamily{jkp}\itshape
    ,morekeywords=^^A
      {^^A
        \ekvdef,\ekvdefNoVal,\ekvlet,\ekvletNoVal,\ekvletkv,\ekvletkvNoVal,^^A
        \ekvset,\ekvparse,\ekvVersion,\ekvDate,\ekvifdefined,^^A
        \ekvifdefinedNoVal,\ekvbreak,\ekvbreakPreSneak,\ekvbreakPostSneak,^^A
        \ekvsneak,\ekvsneakPre,\ekvchangeset
      }
    ,morecomment=[l]\%
    ,commentstyle=\color[gray]{0.4}
    ,literate={\{}{{\CodeSymbol\{}}{1}
              {\}}{{\CodeSymbol\}}}{1}
    ^^A,literate=*{<key>}{\key}{4}{<set>}{\set}{4}
  }
\newcommand*\CodeSymbol[1]{\textbf{#1}}
\RequirePackage{randtext}
\let\metaORIG\meta
\protected\def\meta #1{\texttt{\metaORIG{#1}}}
\renewcommand*\thefootnote{\fnsymbol{footnote}}
\definecolor{expkvred}{HTML}{9F393D}
\colorlet{expkvgrey}{black!75}
\makeatletter
\newcommand*\expkv
  {^^A
    \texorpdfstring
      {^^A
        \mbox
          {^^A
            \BeginAccSupp{ActualText=expkv}^^A
            \href{https://github.com/Skillmon/tex_expkv}
              {^^A
                \rmfamily
                \bfseries
                {\color{expkvgrey}e\kern-.05em x\kern-.05em}^^A
                \lower.493ex
                  \hbox{{\color{expkvgrey}P}\kern-.1em{\color{expkvred}k}}^^A
                \kern-.18em{\color{expkvred}v}^^A
              }^^A
            \EndAccSupp{}^^A
          }^^A
      }
      {expkv}^^A
  }
\newcommand*\expkvd
  {^^A
    \texorpdfstring
      {^^A
        \mbox
          {^^A
            \BeginAccSupp{ActualText=expkv-def}^^A
            \rmfamily
            \bfseries
            {\color{expkvgrey}e\kern-.05em x\kern-.05em}^^A
            \lower.493ex
              \hbox{{\color{expkvgrey}P}\kern-.1em{\color{expkvred}k}}^^A
            \kern-.18em{\color{expkvred}v}^^A
            {\color{expkvgrey}^^A
              \kern.05em\rule[-.1ex]{.08em}{1.2ex}\kern.05em\textsc{def}^^A
            }^^A
            \EndAccSupp{}^^A
          }^^A
      }
      {expkv-def}^^A
  }
\newcommand\kv{\meta{key}=\meta{value}}
\newcommand\key{\meta{key}}
\newcommand\val{\meta{value}}
\newcommand\set{\meta{set}}
\hypersetup{linkcolor=red!80!black,urlcolor=purple!80!black}
\DoNotIndex{\def,\edef,\,,\=,\begingroup,\catcode,\chardef,\csname,\endcsname}
\DoNotIndex{\endgroup,\endinput,\errmessage,\expandafter,\input,\let,\long}
\DoNotIndex{\noexpand,\protected,\ProvidesFile,\ProvidesPackage,\relax,\space}
\DoNotIndex{\@,\unexpanded,\string,\expanded}
\DoNotIndex{\ifcsname}
\DoNotIndex{\ifx}
\DoNotIndex{\else}
\DoNotIndex{\fi}
\@gobble\fi ^^A ignoring \ifx and \ifcsname, but only one \fi
\@ifdefinable\gobbledocstriptag{\def\gobbledocstriptag#1>{}}
\newcommand*\pmso[1] ^^A poor man's strike out
  {^^A
    \leavevmode
    \begingroup
      \sbox0{#1}^^A
      \rlap{\vrule height .6ex depth -.5ex width \wd0\relax}^^A
      \usebox0\relax
    \endgroup
  }
\makeatother
\begin{document}
  \title
    {^^A
      \texorpdfstring
        {^^A
          \huge\expkv
          \\[\medskipamount]
          \Large an expandable \kv\ implementation^^A
        }{expkv - an expandable <key>=<value> implementation}^^A
    }
  \date{\ekvDate\space v\ekvVersion}
  \author{Jonathan P. Spratte\thanks{\protect\randomize{jspratte@yahoo.de}}}
  \DocInput{expkv.dtx}
\end{document}
%</driver>^^A=<<
% \fi
%
% \maketitle
% \renewcommand*\thefootnote{\arabic{footnote}}
%
% \begin{abstract}
% \noindent\parfillskip=0pt
% \expkv\ provides a small interface for \kv\ parsing. The parsing macro is
% fully expandable, the \meta{code} of your keys might be not. \expkv\ is pretty
% fast, but not the fastest available \kv\ solution (\pkg{keyval} for instance
% is thrice as fast, but not expandable and it might strip braces it shouldn't
% have stripped).
% \end{abstract}
%
% \tableofcontents
%
% \begin{documentation}^^A>>=
%
% \section{Documentation}
%
% \expkv\ provides an expandable \kv\ parser. The \kv\ pairs should be given as
% a comma separated list and the separator between a \key\ and the associated
% \val\ should be an equal sign. Both, the commas and the equal signs, might be
% of category 12 (other) or 13 (active). To support this is necessary as for
% example \pkg{babel} turns characters active for some languages, for instance
% the equal sign is turned active for Turkish.
%
% \expkv\ is usable as generic code or as a \LaTeX\ package. To use it, just
% use one of:
% \begin{lstlisting}
% \usepackage{expkv} % LaTeX
% \input expkv       % plainTeX
% \end{lstlisting}
% The \LaTeX\ package doesn't do more than \file{expkv.tex}, except calling
% |\ProvidesPackage| and setting things up such that \file{expkv.tex} will use
% |\ProvidesFile|.
%
% \subsection{Setting up Keys}\label{sec:define}
% \expkv\ provides a rather simple approach to setting up keys, similar to
% \pkg{keyval}. However there is an auxiliary package named
% \href{https://github.com/Skillmon/tex_expkv-def}{\expkvd} (the files use
% \file{expkv-def} instead because \string| isn't allowed in package names by
% CTAN) which provides a more sophisticated interface, similar to well
% established packages like \pkg{pgfkeys} or \pkg{l3keys}.
%
% Keys in \expkv\ (as in almost all other \kv\ implementations) belong to a
% \emph{set} such that different sets can contain keys of the same name. Unlike
% many other implementations \expkv\ doesn't provide means to set a default
% value, instead we have keys that take values and keys that don't (the latter
% are called |NoVal| keys by \expkv), but both can have the same name (on the
% user level).
%
% The following macros are available to define new keys. Those macros containing
% ``|def|'' in their name can be prefixed by anything allowed to prefix |\def|,
% prefixes allowed for |\let| can prefix those with ``|let|'' in their name,
% accordingly. Neither \set\ nor \key\ are allowed to be empty for new keys and
% must not contain a |\par| or tokens that expand to it -- they must be legal
% inside of |\csname ...\endcsname|.
%
% \begin{function}{\ekvdef}
%   \begin{syntax}
%     \cs{ekvdef}\marg{set}\marg{key}\marg{code}
%   \end{syntax}
%   Defines a \key\ taking a value in a \set\ to expand to \meta{code}. In
%   \meta{code} you can use |#1| to refer to the given value.
% \end{function}
%
% \begin{function}{\ekvdefNoVal}
%   \begin{syntax}
%     \cs{ekvdefNoVal}\marg{set}\marg{key}\marg{code}
%   \end{syntax}
%   Defines a no value taking \key\ in a \set\ to expand to \meta{code}.
% \end{function}
%
% \begin{function}{\ekvlet}
%   \begin{syntax}
%     \cs{ekvlet}\marg{set}\marg{key}\meta{cs}
%   \end{syntax}
%   Let the value taking \key\ in \set\ to \meta{cs}, there are no checks on
%   \meta{cs} enforced.
% \end{function}
%
% \begin{function}{\ekvletNoVal}
%   \begin{syntax}
%     \cs{ekvletNoVal}\marg{set}\marg{key}\meta{cs}
%   \end{syntax}
%   Let the no value taking \key\ in \set\ to \meta{cs}, it is not checked
%   whether \meta{cs} exists or that it takes no parameter.
% \end{function}
%
% \begin{function}{\ekvletkv}
%   \begin{syntax}
%     \cs{ekvletkv}\marg{set}\marg{key}\marg{set2}\marg{key2}
%   \end{syntax}
%   Let the \key\ in \set\ to \meta{key2} in \meta{set2}, it is not checked
%   whether that second key exists.
% \end{function}
%
% \begin{function}{\ekvletkvNoVal}
%   \begin{syntax}
%     \cs{ekvletkvNoVal}\marg{set}\marg{key}\marg{set2}\marg{key2}
%   \end{syntax}
%   Let the \key\ in \set\ to \meta{key2} in \meta{set2}, it is not checked
%   whether that second key exists.
% \end{function}
%
% \subsection{Parsing Keys}
%
% \begin{function}{\ekvset}
%   \begin{syntax}
%     \cs{ekvset}\marg{set}\{\kv,\ldots\}
%   \end{syntax}
%   Splits \kv\ pairs on commas. From both \key\ and \val\ up to one space is
%   stripped from both ends, if then only a braced group remains the braces are
%   stripped as well. So |\ekvset{foo}{bar=baz}| and
%   |\ekvset{foo}{ {bar}= {baz} }| will both do
%   \texttt{\cs[no-index]{\meta{foobarcode}}\{baz\}}, so you can hide commas,
%   equal signs and spaces at the ends of either \key\ or \val\ by putting
%   braces around them. If you omit the equal sign the code of the key created
%   with the |NoVal| variants described in \autoref{sec:define} will be
%   executed. If \kv\ contains more than a single unhidden equal sign, it will
%   be split at the first one and the others are considered part of the value.
%   |\ekvset| should be nestable.
% \end{function}
%
% \begin{function}{\ekvparse}
%   \begin{syntax}
%     \cs{ekvparse}\meta{cs1}\meta{cs2}\{\kv,\ldots\}
%   \end{syntax}
%   This macro parses the \kv\ pairs and provides those list elements which are
%   only keys as the argument to \meta{cs1}, and those which are a \kv\ pair to
%   \meta{cs2} as two arguments. It is fully expandable as well and returns the
%   parsed list in |\unexpanded|, which has no effect outside of an |\expanded|
%   or |\edef| context\footnotemark.
%   If you need control over the necessary steps of expansion you
%   can use |\expanded| around it.
%
%   |\ekvbreak|, |\ekvsneak|, and |\ekvchangeset| and their relatives don't work
%   in |\ekvparse|.  It is analogue to \pkg{expl3}'s |\keyval_parse:NNn|, but
%   not with the same parsing rules -- |\keyval_parse:NNn| throws an error on
%   multiple equal signs per \kv\ pair and on empty \key\ names in a \kv\ pair,
%   both of which |\ekvparse| doesn't deal with.
%
%   As a small example:
% \begin{lstlisting}
% \ekvparse\handlekey\handlekeyval{foo = bar, key, baz={zzz}}
% \end{lstlisting}
%   would expand to
% \begin{lstlisting}
% \handlekeyval{foo}{bar}\handlekey{key}\handlekeyval{baz}{zzz}
% \end{lstlisting}
%   and afterwards |\handlekey| and |\handlekeyval| would have to further handle
%   the \key. There are no macros like these two contained in \expkv, you have
%   to set them up yourself if you want to use |\ekvparse| (of course the names
%   might differ). If you need the results of |\ekvparse| as the argument for
%   another macro, you should use |\expanded| as only then the input stream will
%   contain the output above:
% \begin{lstlisting}
% \expandafter\handle\expanded{\ekvparse\k\kv{foo = bar, key, baz={zzz}}}
% \end{lstlisting}
%   would expand to
% \begin{lstlisting}
% \handle\kv{foo}{bar}\k{key}\kv{baz}{zzz}
% \end{lstlisting}
% \end{function}
% \footnotetext 
%   {^^A
%     This is a change in behaviour, previously (v0.3 and before) \cs{ekvparse}
%     would expand in exactly two steps. This isn't always necessary, but makes
%     the parsing considerably slower. If this is necessary for your application
%     you can put an \cs[no-index]{expanded} around it and will still be faster
%     since you need only a single \cs[no-index]{expandafter} this way.^^A
%   }.
%
% \subsection{Miscellaneous}
%
% \subsubsection{Other Macros}
%
% \expkv\ provides some other macros which might be of interest.
%
% \begin{function}{\ekvVersion,\ekvDate}
%   These two macros store the version and date of the package.
% \end{function}
% \begin{function}{\ekvifdefined,\ekvifdefinedNoVal}
%   \begin{syntax}
%     \cs{ekvifdefined}\marg{set}\marg{key}\marg{true}\marg{false}
%     \cs{ekvifdefinedNoVal}\marg{set}\marg{key}\marg{true}\marg{false}
%   \end{syntax}
%   These two macros test whether there is a \key\ in \set. It is false if
%   either a hash table entry doesn't exist for that key or its meaning is
%   |\relax|.
% \end{function}
%
% \begin{function}{\ekvbreak,\ekvbreakPreSneak,\ekvbreakPostSneak}
%   \begin{syntax}
%     \cs{ekvbreak}\marg{after}
%   \end{syntax}
%   Gobbles the remainder of the current |\ekvset| macro and its argument list
%   and reinserts \meta{after}. So this can be used to break out of |\ekvset|.
%   The first variant will also gobble anything that has been sneaked out using
%   |\ekvsneak| or |\ekvsneakPre|, while |\ekvbreakPreSneak| will put
%   \meta{after} before anything that has been smuggled and |\ekvbreakPostSneak|
%   will put \meta{after} after the stuff that has been sneaked out.
% \end{function}
%
% \begin{function}{\ekvsneak,\ekvsneakPre}
%   \begin{syntax}
%     \cs{ekvsneak}\marg{after}
%   \end{syntax}
%   Puts \meta{after} after the effects of |\ekvset|. The first variant will put
%   \meta{after} after any other tokens which might have been sneaked before,
%   while |\ekvsneakPre| will put \meta{after} before other smuggled stuff.
%   This reads and reinserts the remainder of the current |\ekvset| macro and
%   its argument list to do its job. A small usage example is shown in
%   \autoref{sec:sneakex}.
% \end{function}
%
% \begin{function}{\ekvchangeset}
%   \begin{syntax}
%     \cs{ekvchangeset}\marg{new-set}
%   \end{syntax}
%   Replaces the current set with \meta{new-set}, so for the rest of the current
%   |\ekvset| call, that call behaves as if it was called with
%   \texttt{\cs[no-index]{ekvset}\marg{new-set}}. Just like |\ekvsneak| this
%   reads and reinserts the remainder of the current |\ekvset| macro to do its
%   job. It is comparable to using \texttt{\meta{key}/.cd} in \pkg{pgfkeys}.
% \end{function}
%
% \bigskip
%
% \begin{function}{\ekv@name,\ekv@name@set,\ekv@name@key}
%   \begin{syntax}
%     \cs{ekv@name}\marg{set}\marg{key}\\
%     \cs{ekv@name@set}\marg{set}\\
%     \cs{ekv@name@key}\marg{key}
%   \end{syntax}
%   The names of the macros that correspond to a key in a set are build with
%   these macros. The default definition of |\ekv@name@set| is
%   ``\texttt{\csname ekv@name@set\endcsname{\set}}'' and the default of
%   |\ekv@name@key| is ``\texttt{\csname ekv@name@key\endcsname{\key}}''.
%   The complete name is build using |\ekv@name| which is equivalent to
%   \texttt{\cs[no-index]{ekv@name@set}\marg{set}\cs{ekv@name@key}\marg{key}}.
%   For |NoVal| keys an additional |N| gets appended irrespective of these
%   macros' definition, so their name is
%   \cs[no-index]{\csname ekv@name\endcsname{\set}{\key}N}. You might redefine
%   |\ekv@name@set| and |\ekv@name@key| locally but \emph{don't redefine}
%   |\ekv@name|!
% \end{function}
%
% \subsubsection{Bugs}
%
% Just like \pkg{keyval}, \expkv\ is bug free. But if you find \pmso{bugs}hidden
% features\footnote{Thanks, David!} you can tell me about them either via mail
% (see the first page) or directly on GitHub if you have an account there:
% \url{https://github.com/Skillmon/tex_expkv}
%
% \subsubsection{Comparisons}
%
% Comparisons of speed are done with a very simple test key and the help of the
% \pkg{l3benchmark} package. The key and its usage should be equivalent to
% \begin{lstlisting}
% \protected\ekvdef{test}{height}{\def\myheight{#1}}
% \ekvset{test}{ height = 6 }
% \end{lstlisting}
% and only the usage of the key, not its definition, is benchmarked. For the
% impatient, the essence of these comparisons regarding speed and buggy
% behaviour is contained in \autoref{tab:comp}.
%
% As far as I know \expkv\ is the only fully expandable \kv\ parser. I tried to
% compare \expkv\ to every \kv\ package listed on
% \href{https://ctan.org/topic/keyval}{CTAN}, however, one might notice that
% some of those are missing from this list. That's because I didn't get the
% others to work due to bugs, or because they just provide wrappers around other
% packages in this list.
%
% In this subsubsection is no benchmark of |\ekvparse| and |\keyval_parse:NNn|
% contained, as most other packages don't provide equivalent features to my
% knowledge. |\ekvparse| is slightly faster than |\ekvset|, but keep in mind
% that it does less. The same is true for |\keyval_parse:NNn| compared to
% |\keys_set:nn| of \pkg{expl3} (where the difference is much bigger).
%
% \paragraph{\pkg{keyval}} is about two times faster and has a comparable
% feature set just a slightly different way how it handles keys without values.
% That might be considered a drawback, as it limits the versatility, but also as
% an advantage, as it might reduce doubled code. Keep in mind that as soon as
% someone loads \pkg{xkeyval} the performance of \pkg{keyval} gets replaced by
% \pkg{xkeyval}'s.
%
% Also \pkg{keyval} has a bug, which unfortunately can't really be resolved
% without breaking backwards compatibility for \emph{many} documents, namely it
% strips braces from the argument before stripping spaces if the argument isn't
% surrounded by spaces, also it might strip more than one set of braces. Hence
% all of the following are equivalent in their outcome, though the last two
% lines should result in something different than the first two:
% \begin{lstlisting}[belowskip=0pt]
% \setkeys{foo}{bar=baz}
% \setkeys{foo}{bar= {baz}}
% \setkeys{foo}{bar={ baz}}
% \setkeys{foo}{bar={{baz}}}
% \end{lstlisting}
%
% \paragraph{\pkg{xkeyval}} is roughly fourteen times slower, but it provides
% more functionality, e.g., it has choice keys, boolean keys, and so on. It
% contains the same bug as \pkg{keyval} as it has to be compatible with it by
% design (it replaces \pkg{keyval}'s frontend), but also adds even more cases in
% which braces are stripped that shouldn't be stripped, worsening the situation.
%
% \paragraph{\pkg{ltxkeys}} is over 300 times slower -- which is funny, because
% it aims to be ``[\ldots] faster than these earlier packages [referring to
% \pkg{keyval} and \pkg{xkeyval}].'' Since it aims to have a bigger feature set
% than \pkg{xkeyval}, it most definitely also has a bigger feature set than
% \expkv. Also, it can't parse |\long| input, so as soon as your values contain
% a |\par|, it'll throw errors. Furthermore, \pkg{ltxkeys} doesn't strip outer
% braces at all by design, which, imho, is a weird design choice. In addition
% \pkg{ltxkeys} loads \pkg{catoptions} which is known to introduce bugs (e.g.,
% see \url{https://tex.stackexchange.com/questions/461783}).
%
% \paragraph{\pkg{l3keys}} is almost five times slower, but has an, imho, great
% interface to define keys. It strips \emph{all} outer spaces, even if somehow
% multiple spaces ended up on either end. It offers more features, but is pretty
% much bound to \pkg{expl3} code. Whether that's a drawback is up to you.
%
% \paragraph{\pkg{pgfkeys}} is a bit more than two times slower for one key,
% but has an \emph{enormous} feature set. However, since adding additional keys
% doesn't add as much needed time for \pkg{pgfkeys} compared to \expkv, it gets
% faster than \expkv\ at around eight \kv\ pairs.
% It has the same or a very similar bug \pkg{keyval} has. The brace bug (and
% also the category fragility) can be fixed by \pkg{pgfkeyx}, but this package
% was last updated in 2012 and it slows down |\pgfkeys| by factor~$8$. Also I
% don't know whether this might introduce new bugs.
%
% \paragraph{\pkg{kvsetkeys} with \pkg{kvdefinekeys}} is about three times
% slower, but it works even if commas and equals have category codes different
% from 12 (just as some other packages in this list). Else the features of the
% keys are equal to those of \pkg{keyval}, the parser has more features, though.
%
% \paragraph{\pkg{options}} is a bit slower for only a single value, but gets a
% tad faster than \expkv\ at around 10 \kv\ pairs. It has a much bigger feature
% set. Unfortunately it also suffers from the premature unbracing bug
% \pkg{keyval} has.
%
% \paragraph{\pkg{simplekv}} is hard to compare because I don't speak French (so
% I don't understand the documentation) and from what I can see, there is no
% direct way to define the equivalent test key. Nevertheless, I tested the
% closest possible equivalent of my test key while siding for \pkg{simplekv}'s
% design not forcing something into it it doesn't seem to be designed for. It is
% more than five times slower and has hard to predict behaviour regarding brace
% and space stripping, similar to \pkg{keyval}. The tested definition was:
% \begin{lstlisting}[belowskip=0pt]
% \usepackage{simplekv}
% \setKVdefault[simplekv]{height={ abc}}                     % key setup
% \setKV[simplekv]{ height = 6 }                             % benchmarked
% \end{lstlisting}
%
% \paragraph{\pkg{yax}} is over eighteen times slower. It has a pretty strange
% syntax, imho, and again a direct equivalent is hard to define. It has the
% premature unbracing bug, too. Also somehow loading \pkg{yax} broke
% \pkg{options} for me. The tested definition was:
% \begin{lstlisting}[belowskip=0pt]
% \usepackage{yax}
% \defactiveparameter yax {\storevalue\myheight yax:height } % key setup
% \setparameterlist{yax}{ height = 6 }                       % benchmarked
% \end{lstlisting}
%
% \begin{table}
%   \def\fnsym{\textcolor{red!80!black}{*}}%
%   \sisetup{round-precision=1, round-mode=places}%
%   \begingroup
%   \centering
%   \newcommand*\yes{\textcolor{red!80!black}  {yes}}^^A
%   \newcommand*\no {\textcolor{green!80!black}{no}}^^A
%   \caption[Comparison of \kv\ packages]
%     {^^A
%       Comparison of \kv\ packages. The packages are ordered from
%       fastest to slowest for one \kv\ pair. Benchmarking was done using
%       \pkg{l3benchmark} and the scripts in the \file{Benchmarks} folder
%       of the \href{https://github.com/Skillmon/tex_expkv}{git repository}.
%       The columns $p_i$ are the polynomial coefficients of a linear fit to the
%       run-time, $p_0$ can be interpreted as the overhead for initialisation
%       and $p_1$ the cost per key. The $T_0$ column is the actual mean ops
%       needed for an empty list argument, as the linear fit doesn't match that
%       point well in general. The column ``BB'' lists whether the
%       parsing is affected by some sort of brace bug, ``CF'' stands for
%       category code fragile and lists whether the parsing breaks with active
%       commas or equal signs.^^A
%       \label{tab:comp}^^A
%     }
%   \begin{tabular}
%       {>{\collectcell\pkg}l<{\endcollectcell}*3{S[table-format=4.1]}ccc}
%     \toprule
%     \rmfamily Package & {$p_1$} & {$p_0$} & {$T_0$}& BB & CF & Date \\
%     \midrule
%     keyval    &   13.539 &    1.844 &    6.969 & \yes & \yes & 2014-10-28 \\
%     \expkv    &   26.411 &    2.391 &   12.610 & \no  & \no  & 2020-02-22 \\
%     options   &   25.532 &   10.518 &   21.393 & \yes & \yes & 2015-03-01 \\
%     pgfkeys   &   26.444 &   40.939 &   55.790 & \yes & \yes & 2020-01-08 \\
%     kvsetkeys & {\fnsym} & {\fnsym} &   41.620 & \no  & \no  & 2019-12-15 \\
%     l3keys    &  114.456 &   35.357 &   52.667 & \no  & \no  & 2020-02-14 \\
%     simplekv  &  160.523 &   10.809 &    8.698 & \yes & \yes & 2017-08-08 \\
%     xkeyval   &  260.872 &  180.435 &  164.167 & \yes & \yes & 2014-12-03 \\
%     yax       &  507.643 &   62.863 &  123.533 & \yes & \yes & 2010-01-22 \\
%     ltxkeys   & 3932.759 & 4737.818 & 5883.000 & \no  & \no  & 2012-11-17 \\
%     \bottomrule
%   \end{tabular}
%   \par
%   \endgroup
%   \medskip
%   \fnsym For \pkg{kvsetkeys} the linear model used for the other
%   packages is a poor fit, \pkg{kvsetkeys} seems to have approximately
%   quadratic run-time, the coefficients of the second degree polynomial fit are
%   $p_2=\num{9.911}$, $p_1=\num{40.523}$, and $p_0=\num{61.500}$. Of course the
%   other packages might not really have linear run-time, but at least from 1~to
%   20~keys the fits don't seem too bad (the maximum ratio $p_2/p_1$
%   for the other packages is \num{9.697e-3}). If one extrapolates the fits
%   for 100 \kv\ pairs one finds that most of them match pretty well, the
%   exception being \pkg{ltxkeys}, which behaves quadratic as well with
%   $p_2=\num{31.685}$, $p_1=\num{3267.368}$, and $p_0=\num{7177.582}$.
% \end{table}
%
% \subsection{Examples}
%
% \subsubsection{Standard Use-Case}
%
% Say we have a macro for which we want to create a \kv\ interface. The macro
% has a parameter, which is stored in the dimension |\ourdim| having a default
% value from its initialization. Now we want to be able to change that dimension
% with the |width| key to some specified value. For that we'd do
% \begin{lstlisting}
% \newdimen\ourdim
% \ourdim=150pt
% \protected\ekvdef{our}{width}{\ourdim=#1\relax}
% \end{lstlisting}
% as you can see, we use the set |our| here. We want the key to behave different
% if no value is specified. In that case the key should not use its initial
% value, but be smart and determine the available space from
% \cs[no-index]{hsize}, so we also define
% \begin{lstlisting}
% \protected\ekvdefNoVal{our}{width}{\ourdim=.9\hsize}
% \end{lstlisting}
% Now we set up our macro to use this \kv\ interface
% \begin{lstlisting}
% \protected\def\ourmacro#1{\begingroup\ekvset{our}{#1}\the\ourdim\endgroup}
% \end{lstlisting}
% Finally we can use our macro like in the following\\
% \begin{minipage}{6cm}
% \begin{lstlisting}
% \ourmacro{}\par
% \ourmacro{width}\par
% \ourmacro{width=5pt}\par
% \end{lstlisting}
% \end{minipage}^^A
% \begin{minipage}{\dimexpr\linewidth-6cm\relax}
%   \newdimen\ourdim\ourdim=150pt
%   \protected\ekvdef{our}{width}{\ourdim=#1\relax}^^A
%   \protected\ekvdefNoVal{our}{width}{\ourdim=.9\hsize}^^A
%   \protected\def\ourmacro#1^^A
%     {\begingroup\ekvset{our}{#1}\the\ourdim\endgroup}^^A
%   \ourmacro{}\par
%   \ourmacro{width}\par
%   \ourmacro{width=5pt}\par
% \end{minipage}
%
% \paragraph{The same key using \protect\expkvd}
% Using \expkvd\ we can set up the equivalent key using a \kv\ interface, after
% the following we could use |\ourmacro| in the same way as above. \expkvd\ will
% allocate and initialise |\ourdim| and define the |width| key |\protected| for
% us, so the result will be exactly the same -- with the exception that the
% default will use |\ourdim=.9\hsize\relax| instead.
% \begin{lstlisting}
% \input expkv-def                 % or \usepackage{expkv-def}
% \ekvdefinekeys{our}
%   {
%     dimen    width = \ourdim,
%     qdefault width = .9\hsize,
%     initial  width = 150pt
%   }
% \end{lstlisting}
%
% \subsubsection{An Expandable \kv\ Macro Using \cs[no-index]{ekvsneak}}
% \label{sec:sneakex}
%
% Let's set up an expandable macro, that uses a \kv\ interface. The problems
% we'll face for this are:
% \begin{enumerate}
%   \item ignoring duplicate keys
%   \item default values for keys which weren't used
%   \item providing the values as the correct argument to a macro (ordered)
% \end{enumerate}
%
% First we need to decide which \kv\ parsing macro we want to do this with,
% |\ekvset| or |\ekvparse|. For this example we also want to show the
% usage of |\ekvsneak|, hence we'll choose |\ekvset|. And we'll have to use
% |\ekvset| such that it builds a parsable list for our macro internals. To
% gain back control after |\ekvset| is done we have to put an internal of our
% macro at the start of that list, so we use an internal key that uses
% |\ekvsneakPre| after any user input.
%
% To ignore duplicates will be easy if the value of the key used last will be
% put first in the list, so the following will use |\ekvsneakPre| for the
% user-level keys. If we wanted some key for which the first usage should be the
% the binding one we would use |\ekvsneak| instead for that key.
%
% Providing default values can be done in different ways, we'll use a simple
% approach in which we'll just put the outcome of our keys if they were used
% with default values before the parsing list terminator.
%
% Ordering the keys can be done simply by searching for a specific token for
% each argument which acts like a flag, so our sneaked out values will include
% specific tokens acting as markers.
%
% Now that we have answers for our technical problems, we have to decide what
% our example macro should do. How about we define a macro that calculates the
% sine of a number and rounds that to a specified precision? As a small extra
% this macro should understand input in radian and degree and the used
% trigonometric function should be selectable as well. For the hard part of this
% task (expandably evaluating trigonometric functions) we'll use the \pkg{xfp}
% package.
%
% First we set up our keys according to our earlier considerations and set up
% the user facing macro |\sine|. The end marker of the parsing list will be a
% |\sine@stop| token, which we don't need to define and we put our defaults
% right before it.
%
% \begin{lstlisting}
% \RequirePackage{xfp}
% \makeatletter
% \ekvdef{expex}{f}{\ekvsneakPre{\f{#1}}}
% \ekvdef{expex}{round}{\ekvsneakPre{\rnd{#1}}}
% \ekvdefNoVal{expex}{degree}{\ekvsneakPre{\deg{d}}}
% \ekvdefNoVal{expex}{radian}{\ekvsneakPre{\deg{}}}
% \ekvdefNoVal{expex}{internal}{\ekvsneakPre{\sine@rnd}}
% \newcommand*\sine[2]
%   {\ekvset{expex}{#1,internal}\rnd{3}\deg{d}\f{sin}\sine@stop{#2}}
% \end{lstlisting}
% For the sake of simplicity we defined the macro |\sine| with two mandatory
% arguments, the first being the \kv\ list, the second the argument to the
% trigonometric function. We could've used \pkg{xparse}'s facilities here to
% define an expandable macro which takes an optional argument instead.
%
% Now we need to define some internal macros to extract the value of each key's
% last usage (remember that this will be the group after the first special
% flag-token). For that we use one delimited macro per key.
% \begin{lstlisting}
% \def\sine@rnd#1\rnd#2#3\sine@stop{\sine@deg#1#3\sine@stop{#2}}
% \def\sine@deg#1\deg#2#3\sine@stop{\sine@f#1#3\sine@stop{#2}}
% \def\sine@f#1\f#2#3\sine@stop{\sine@final{#2}}
% \end{lstlisting}
% After the macros |\sine@rnd|, |\sine@deg|, and |\sine@f| the macro
% |\sine@final| will see
% \texttt
%   {^^A
%     \cs[no-index]{sine@final}^^A
%     \marg{f}\marg{degree/radian}\marg{round}\marg{num}^^A
%   }.
% Now |\sine@final| has to expandably deal with those arguments such that the
% |\fpeval| macro of \pkg{xfp} gets the correct input. Luckily this is pretty
% straight forward in this example. In |\fpeval| the trigonometric functions
% have names such as |sin| or |cos| and the degree taking variants |sind| or
% |cosd|. And since the |degree| key puts a |d| in |#2| and the |radian| key
% leaves |#2| empty all we have to do to get the correct function name is stick
% the two together.
% \begin{lstlisting}
% \newcommand*\sine@final[4]{\fpeval{round(#1#2(#4),#3)}}
% \makeatother
% \end{lstlisting}
% Let's test our macro:\\
% \begin{minipage}{.7\linewidth}
%   \begin{lstlisting}[gobble=4]
%   \sine{}{60}\par
%   \sine{round=10}{60}\par
%   \sine{f=cos,radian}{pi}\par
%   \edef\myval{\sine{f=tan}{1}}\texttt{\meaning\myval}
%   \end{lstlisting}
% \end{minipage}\hfill
% \begin{minipage}{.2\linewidth}
%   \makeatletter
%   \ekvdef{expex}{f}{\ekvsneakPre{\f{#1}}}^^A
%   \ekvdef{expex}{round}{\ekvsneakPre{\rnd{#1}}}^^A
%   \ekvdefNoVal{expex}{degree}{\ekvsneakPre{\deg{d}}}^^A
%   \ekvdefNoVal{expex}{radian}{\ekvsneakPre{\deg{}}}^^A
%   \ekvdefNoVal{expex}{internal}{\ekvsneakPre{\sine@rnd}}^^A
%   \newcommand*\sine[2]
%     {\ekvset{expex}{#1,internal}\rnd{3}\deg{d}\f{sin}\sine@stop{#2}}^^A
%   \def\sine@rnd#1\rnd#2#3\sine@stop{\sine@deg#1#3\sine@stop{#2}}^^A
%   \def\sine@deg#1\deg#2#3\sine@stop{\sine@f#1#3\sine@stop{#2}}^^A
%   \def\sine@f#1\f#2#3\sine@stop{\sine@final{#2}}^^A
%   \newcommand*\sine@final[4]{\fpeval{round(#1#2(#4),#3)}}^^A
%   \makeatother
%   \sine{}{60}\par
%   \sine{round=10}{60}\par
%   \sine{f=cos,radian}{pi}\par
%   \edef\myval{\sine{f=tan}{1}}\texttt{\meaning\myval}
% \end{minipage}
%
%
% \subsection{Error Messages}
%
% \expkv\ should only send messages in case of errors, there are no warnings and
% no info messages. In this subsection those errors are listed.
%
% \subsubsection{Load Time}
%
% \file{expkv.tex} checks whether \eTeX\ is available. If it isn't, an error
% will be thrown using |\errmessage|:
% \begin{lstlisting}
% ! expkv Error: e-TeX required.
% \end{lstlisting}
%
% \subsubsection{Defining Keys}
%
% If you get any error from \expkv\ while you're trying to define a key, the
% definition will be aborted and gobbled.
%
% If you try to define a key with an empty set name you'll get:
% \begin{lstlisting}
% ! expkv Error: empty set name not allowed.
% \end{lstlisting}
%
% Similarly, if you try to define a key with an empty key name:
% \begin{lstlisting}
% ! expkv Error: empty key name not allowed.
% \end{lstlisting}
%
% Both of these messages are done in a way that doesn't throw additional errors
% due to |\global|, |\long|, etc., not being used correctly if you prefixed one
% of the defining macros.
%
% \subsubsection{Using Keys}
%
% This subsubsection contains the errors thrown during |\ekvset|. The errors are
% thrown in an expandable manner by providing an undefined macro. In the
% following messages \lstinline|<key>| gets replaced with the problematic key's
% name, and \lstinline|<set>| with the corresponding set. If any errors during
% \kv\ handling are encountered, the entry in the comma separated list will be
% omitted after the error is thrown and the next \kv\ pair will be parsed.
%
% If you're using an undefined key you'll get:
% \begin{lstlisting}
% ! Undefined control sequence.
% <argument> \! expkv Error:
%                            unknown key (`<key>', set `<set>').
% \end{lstlisting}
%
% If you're using a key for which only a normal version and no |NoVal| version
% is defined, but don't provide a value, you'll get:
% \begin{lstlisting}
% ! Undefined control sequence.
% <argument> \! expkv Error:
%                            value required (`<key>', set `<set>').
% \end{lstlisting}
%
% If you're using a key for which only a |NoVal| version and no normal version
% is defined, but provide a value, you'll get:
% \begin{lstlisting}
% ! Undefined control sequence.
% <argument> \! expkv Error:
%                            value forbidden (`<key>', set `<set>').
% \end{lstlisting}
%
% If you're using a set for which you never executed one of the defining macros
% from \autoref{sec:define} you'll get a low level \TeX\ error, as that isn't
% actively tested by the parser (and hence will lead to undefined behaviour and
% not be gracefully ignored). The error will look like
% \begin{lstlisting}
% ! Missing \endcsname inserted.
% <to be read again>
%                    \! expkv Error: Set `<set>' undefined.
% \end{lstlisting}
%
% \subsection{License}
%
% Copyright \textcopyright\ 2020\unless\ifnum\year=2020--\the\year\fi\
% Jonathan P. Spratte
%
% \medskip\noindent
% This work may be distributed and/or modified under the conditions of the
% \LaTeX\ Project Public License (LPPL), either version 1.3c of this license or
% (at your option) any later version. The latest version of this license is in
% the file:
%
% \url{http://www.latex-project.org/lppl.txt}
%
% \noindent
% This work is ``maintained'' (as per LPPL maintenance status) by
%
% \mbox{Jonathan P. Spratte}.
%
% \end{documentation}^^A=<<
%
% \begin{implementation}^^A>>=
%
% \clearpage
%
% \section{Implementation}^^A>>=
%^^A the LaTeX package >>=
% \subsection{The \LaTeX\ Package}
% First we set up the \LaTeX\ package. That one doesn't really do much except
% |\input|ting the generic code and identifying itself as a package.
% \gobbledocstriptag
%<*pkg>
%    \begin{macrocode}
\def\ekv@tmp
  {%
    \ProvidesFile{expkv.tex}%
      [\ekvDate\space v\ekvVersion\space an expandable key=val implementation]%
  }
\input{expkv.tex}
\ProvidesPackage{expkv}%
  [\ekvDate\space v\ekvVersion\space an expandable key=val implementation]
%    \end{macrocode}
% \gobbledocstriptag
%</pkg>
%^^A=<<
%^^A main file >>=
% \subsection{The Generic Code}
% The rest of this implementation will be the generic code.
% \gobbledocstriptag
%<*tex>
%
% Check whether \eTeX\ is available -- \expkv\ requires \eTeX.
%    \begin{macrocode}
\begingroup\expandafter\expandafter\expandafter\endgroup
\expandafter\ifx\csname numexpr\endcsname\relax
  \errmessage{expkv requires e-TeX}
  \expandafter\endinput
\fi
%    \end{macrocode}
%
% We make sure that it's only input once:
%    \begin{macrocode}
\expandafter\ifx\csname ekvVersion\endcsname\relax
\else
  \expandafter\endinput
\fi
%    \end{macrocode}
% \begin{macro}{\ekvVersion,\ekvDate}
% We're on our first input, so lets store the version and date in a macro.
%    \begin{macrocode}
\def\ekvVersion{0.5a}
\def\ekvDate{2020-02-27}
%    \end{macrocode}
% \end{macro}
%
% If the \LaTeX\ format is loaded we want to be a good file and report back who
% we are, for this the package will have defined |\ekv@tmp| to use
% |\ProvidesFile|, else this will expand to a |\relax| and do no harm.
%    \begin{macrocode}
\csname ekv@tmp\endcsname
%    \end{macrocode}
%
% Store the category code of |@| to later be able to reset it and change it to
% 11 for now.
%    \begin{macrocode}
\expandafter\chardef\csname ekv@tmp\endcsname=\catcode`\@
\catcode`\@=11
%    \end{macrocode}
% |\ekv@tmp| might later be reused to gobble any prefixes which might be
% provided to |\ekvdef| and similar in case the names are invalid, we just
% temporarily use it here as means to store the current category code of |@| to
% restore it at the end of the file, we never care for the actual definition of
% it.
%
% \begin{macro}[internal]
%   {
%     \@gobble,\@firstofone,\@firstoftwo,\@secondoftwo,
%     \ekv@gobbleto@stop,\ekv@fi@secondoftwo,\ekv@gobble@mark
%   }
% Since branching tests are often more versatile than |\if...\else...\fi|
% constructs, we define helpers that are branching pretty fast. Also here are
% some other utility functions that just grab some tokens. The ones that are
% also contained in \LaTeX\ don't use the |ekv| prefix.
%    \begin{macrocode}
\long\def\@gobble#1{}
\long\def\@firstofone#1{#1}
\long\def\@firstoftwo#1#2{#1}
\long\def\@secondoftwo#1#2{#2}
\long\def\ekv@fi@secondoftwo\fi\@firstoftwo#1#2{\fi#2}
\long\def\ekv@gobbleto@stop#1\ekv@stop{}
\def\ekv@gobble@mark\ekv@mark{}
%    \end{macrocode}
% \end{macro}
% As you can see |\ekv@gobbleto@stop| uses a special marker |\ekv@stop|. The
% package will use three such markers, the one you've seen already, |\ekv@mark|
% and |\ekv@nil|. Contrarily to how for instance \pkg{expl3} does things, we
% don't define them, as we don't need them to have an actual meaning. This has
% the advantage that if they somehow get expanded -- which should never happen
% if things work out -- they'll throw an error directly.
%
% \begin{macro}[internal]
%   {
%     \ekv@ifempty,\ekv@ifempty@,\ekv@ifempty@true,\ekv@ifempty@false,
%     \ekv@ifempty@true@F
%   }
% We can test for a lot of things building on an if-empty test, so lets define a
% really fast one. Since some tests might have reversed logic (true if something
% is not empty) we also set up macros for the reversed branches.
%    \begin{macrocode}
\long\def\ekv@ifempty#1%
  {%
    \ekv@ifempty@\ekv@ifempty@A#1\ekv@ifempty@B\ekv@ifempty@true
      \ekv@ifempty@A\ekv@ifempty@B\@secondoftwo
  }
\long\def\ekv@ifempty@#1\ekv@ifempty@A\ekv@ifempty@B{}
\long\def\ekv@ifempty@true\ekv@ifempty@A\ekv@ifempty@B\@secondoftwo#1#2{#1}
\long\def\ekv@ifempty@false\ekv@ifempty@A\ekv@ifempty@B\@firstoftwo#1#2{#2}
\long\def\ekv@ifempty@true@F\ekv@ifempty@A\ekv@ifempty@B\@firstofone#1{}
%    \end{macrocode}
% \end{macro}
%
% \begin{macro}[internal]{\ekv@ifblank@}
% The obvious test that can be based on an if-empty is if-blank, meaning a test
% checking whether the argument is empty or consists only of spaces. Our version
% here will be tweaked a bit, as we want to check this, but with one leading
% |\ekv@mark| token that is to be ignored.
%    \begin{macrocode}
\long\def\ekv@ifblank@\ekv@mark#1{\ekv@ifempty@\ekv@ifempty@A}
%    \end{macrocode}
% \end{macro}
%
% \begin{macro}[internal]{\ekv@ifdefined}
% We'll need to check whether something is defined quite frequently, so why not
% define a macro that does this. The following test is expandable, slower than
% the typical expandable test for undefined control sequences, but faster for
% defined ones. Since we want to be as fast as possible for correct input, this
% is to be preferred.
%    \begin{macrocode}
\def\ekv@ifdefined#1%
  {%
    \expandafter
    \ifx\csname\ifcsname #1\endcsname #1\else relax\fi\endcsname\relax
      \ekv@fi@secondoftwo
    \fi
    \@firstoftwo
  }
%    \end{macrocode}
% \end{macro}
  % \begin{macro}[internal]
%   {
%     \ekv@ifdefined@pair,\ekv@ifdefined@pair@,\ekv@ifdefined@key,
%     \ekv@ifdefined@key@
%   }
% Since we can save some time if we only have to create the control sequence
% once when we know beforehand how we want to use it, we build some other macros
% for those cases (which we'll have quite often, once per key usage).
%    \begin{macrocode}
\def\ekv@ifdefined@pair#1#2%
  {%
    \expandafter\ekv@ifdefined@pair@
      \csname
        \ifcsname #1{#2}\endcsname
          #1{#2}%
        \else
          relax%
        \fi
      \endcsname
  }
\def\ekv@ifdefined@pair@#1%
  {%
    \ifx#1\relax
      \ekv@fi@secondoftwo
    \fi
    \@firstoftwo
    {\ekv@set@pair@#1\ekv@mark}%
  }
\def\ekv@ifdefined@key#1#2%
  {%
    \expandafter\ekv@ifdefined@key@
      \csname
        \ifcsname #1{#2}N\endcsname
          #1{#2}N%
        \else
          relax%
        \fi
      \endcsname
  }
\def\ekv@ifdefined@key@#1%
  {%
    \ifx#1\relax
      \ekv@fi@secondoftwo
    \fi
    \@firstoftwo#1%
  }
%    \end{macrocode}
% \end{macro}
%
% \begin{macro}{\ekv@name,\ekv@name@set,\ekv@name@key}
% The keys will all follow the same naming scheme, so we define it here.
%    \begin{macrocode}
\def\ekv@name#1#2{\ekv@name@set{#1}\ekv@name@key{#2}}
\def\ekv@name@set#1{ekv#1(}
\def\ekv@name@key#1{#1)}
%    \end{macrocode}
% \end{macro}
% \begin{macro}[internal]{\ekv@undefined@set}
% We can misuse the macro name we use to expandably store the set-name in a
% single token -- since this increases performance drastically, especially for
% long set-names -- to throw a more meaningful error message in case a set isn't
% defined. The name of |\ekv@undefined@set| is a little bit misleading, as it is
% called in either case inside of |\csname|, but the result will be a control
% sequence with meaning |\relax| if the set is undefined, hence will break the
% |\csname| building the key-macro which will throw the error message.
%    \begin{macrocode}
\def\ekv@undefined@set#1{! expkv Error: Set `#1' undefined.}
%    \end{macrocode}
% \end{macro}
% \begin{macro}[internal]{\ekv@checkvalid}
% We place some restrictions on the allowed names, though, namely sets and
% keys are not allowed to be empty -- blanks are fine (meaning \mbox{set-
% or} key-names consisting of spaces).
%    \begin{macrocode}
\protected\def\ekv@checkvalid#1#2%
  {%
    \ekv@ifempty{#1}%
      {%
        \def\ekv@tmp{}%
        \errmessage{expkv Error: empty set name not allowed}%
      }%
      {%
        \ekv@ifempty{#2}%
          {%
            \def\ekv@tmp{}%
            \errmessage{expkv Error: empty key name not allowed}%
          }%
          \@secondoftwo
      }%
    \@gobble
  }
%    \end{macrocode}
% \end{macro}
% \begin{macro}{\ekvifdefined,\ekvifdefinedNoVal}
% And provide user-level macros to test whether a key is defined.
%    \begin{macrocode}
\def\ekvifdefined#1#2{\ekv@ifdefined{\ekv@name{#1}{#2}}}
\def\ekvifdefinedNoVal#1#2{\ekv@ifdefined{\ekv@name{#1}{#2}N}}
%    \end{macrocode}
% \end{macro}
%
% \begin{macro}
%   {\ekvdef,\ekvdefNoVal,\ekvlet,\ekvletNoVal,\ekvletkv,\ekvletkvNoVal}
% Set up the key defining macros |\ekvdef| etc.
%    \begin{macrocode}
\protected\long\def\ekvdef#1#2#3%
  {%
    \ekv@checkvalid{#1}{#2}%
      {%
        \expandafter\def\csname\ekv@name{#1}{#2}\endcsname##1{#3}%
        \ekv@defset{#1}%
      }%
  }
\protected\long\def\ekvdefNoVal#1#2#3%
  {%
    \ekv@checkvalid{#1}{#2}%
      {%
        \expandafter\def\csname\ekv@name{#1}{#2}N\endcsname{#3}%
        \ekv@defset{#1}%
      }%
  }
\protected\def\ekvlet#1#2#3%
  {%
    \ekv@checkvalid{#1}{#2}%
      {%
        \expandafter\let\csname\ekv@name{#1}{#2}\endcsname#3%
        \ekv@defset{#1}%
      }%
  }
\protected\def\ekvletNoVal#1#2#3%
  {%
    \ekv@checkvalid{#1}{#2}%
      {%
        \expandafter\let\csname\ekv@name{#1}{#2}N\endcsname#3%
        \ekv@defset{#1}%
      }%
  }
\protected\def\ekvletkv#1#2#3#4%
  {%
    \ekv@checkvalid{#1}{#2}%
      {%
        \expandafter\let\csname\ekv@name{#1}{#2}\expandafter\endcsname
        \csname\ekv@name{#3}{#4}\endcsname
        \ekv@defset{#1}%
      }%
  }
\protected\def\ekvletkvNoVal#1#2#3#4%
  {%
    \ekv@checkvalid{#1}{#2}%
      {%
        \expandafter\let\csname\ekv@name{#1}{#2}N\expandafter\endcsname
        \csname\ekv@name{#3}{#4}N\endcsname
        \ekv@defset{#1}%
      }%
  }
%    \end{macrocode}
% \end{macro}
% \begin{macro}[internal]{\ekv@defset}
%   In order to enhance the speed the set name given to |\ekvset| will be turned
%   into a control sequence pretty early, so we have to define that control
%   sequence.
%    \begin{macrocode}
\protected\def\ekv@defset#1%
  {%
    \expandafter\edef\csname\ekv@undefined@set{#1}\endcsname##1%
      {\ekv@name@set{#1}\ekv@name@key{##1}}%
  }
%    \end{macrocode}
% \end{macro}
%
% \begin{macro}{\ekvset}
% Set up |\ekvset|, which should not be affected by active commas and equal
% signs. The equal signs are a bit harder to cope with and we'll do that later,
% but replacing the active commas with commas of category other can be done
% beforehand. That's why we define |\ekvset| here with a temporary meaning just
% to set up the things with two different category codes. |#1| will be a
% \texttt{,\textsubscript{13}} and |#2| will be a \texttt{=\textsubscript{13}}.
%    \begin{macrocode}
\def\ekvset#1#2{%
\endgroup
\long\def\ekvset##1##2%
  {%
    \expandafter\ekv@set\csname\ekv@undefined@set{##1}\endcsname
      \ekv@mark##2#1\ekv@stop#1{}%
  }
%    \end{macrocode}
% \end{macro}
% \begin{macro}[internal]{\ekv@set}
% |\ekv@set| will split the \kv\ list at active commas. Then it has to check
% whether there were unprotected other commas and resplit there.
%    \begin{macrocode}
\long\def\ekv@set##1##2#1%
  {%
%    \end{macrocode}
% Test whether we're at the end, if so invoke |\ekv@endset|,
%    \begin{macrocode}
    \ekv@gobbleto@markstop##2\ekv@endset\ekv@mark\ekv@stop
%    \end{macrocode}
% else go on with other commas,
%    \begin{macrocode}
    \ekv@set@other##1##2,\ekv@stop,%
%    \end{macrocode}
% and get the next active comma delimited \kv\ pair.
%    \begin{macrocode}
    \ekv@set##1\ekv@mark
  }
%    \end{macrocode}
% \end{macro}
% \begin{macro}[internal]{\ekv@endset}
% |\ekv@endset| is a hungry little macro. It will eat everything that remains
% of |\ekv@set| and unbrace the sneaked stuff.
%    \begin{macrocode}
\long\def\ekv@endset
    \ekv@mark\ekv@stop\ekv@set@other##1,\ekv@stop,\ekv@set##2\ekv@mark
    ##3%
  {##3}
%    \end{macrocode}
% \end{macro}
% \begin{macro}[internal]{\ekv@set@other}
% The macro |\ekv@set@other| is guaranteed to get only single \kv\ pairs.
% First we test whether we're done, if not split at equal signs. It is faster to
% first split at category 12 equal signs and only after that on actives.
% If there is no equal sign, we need to test whether we got a blank argument and
% if not this is a |NoVal| key.
%    \begin{macrocode}
\long\def\ekv@set@other##1##2,%
  {%
    \ekv@gobbleto@markstop##2\ekv@endset@other\ekv@mark\ekv@stop
    \ekv@ifhas@eq@other##2=\ekv@ifempty@B\ekv@ifempty@false
      \ekv@ifempty@A\ekv@ifempty@B\@firstoftwo
      {\ekv@set@eq@other##1##2\ekv@stop}%
      {%
        \ekv@ifhas@eq@active##2#2\ekv@ifempty@B\ekv@ifempty@false
          \ekv@ifempty@A\ekv@ifempty@B\@firstoftwo
          {\ekv@set@eq@active##1##2\ekv@stop}%
          {%
            \ekv@ifblank@##2\ekv@nil\ekv@ifempty@B\ekv@ifempty@true@F
              \ekv@ifempty@A\ekv@ifempty@B\@firstofone
              {\ekv@strip{##2}\ekv@set@key##1}%
          }%
      }%
    \ekv@set@other##1\ekv@mark%
  }
%    \end{macrocode}
% \end{macro}
% \begin{macro}[internal]{\ekv@set@eq@other}
% |\ekv@set@eq@other| might not be the correct break point, there might be an
% active equal sign in the currently parsed key-name. If so, we have to resplit.
% If the split is correct strip the key-name of outer spaces and braces and feed
% it to |\ekv@set@pair|.
%    \begin{macrocode}
\long\def\ekv@set@eq@other##1##2=%
  {%
    \ekv@ifhas@eq@active##2#2\ekv@ifempty@B\ekv@ifempty@false
      \ekv@ifempty@A\ekv@ifempty@B\@firstoftwo
      {\ekv@set@eq@active##1##2=}%
      {\ekv@strip{##2}\ekv@set@pair##1}%
  }
%    \end{macrocode}
% \end{macro}
% \begin{macro}[internal]{\ekv@set@eq@active}
% We need to handle the active equal signs.
%    \begin{macrocode}
\long\def\ekv@set@eq@active##1##2#2%
  {%
    \ekv@strip{##2}\ekv@set@pair##1%
  }
%    \end{macrocode}
% \end{macro}
% \begin{macro}[internal]
%   {\ekv@ifhas@eq@other,\ekv@ifhas@eq@active,\ekv@endset@other}
% And we have to set up the testing macros for our equal signs and
% |\ekv@endset@other|.
%    \begin{macrocode}
\long\def\ekv@ifhas@eq@other\ekv@mark##1={\ekv@ifempty@\ekv@ifempty@A}
\long\def\ekv@ifhas@eq@active\ekv@mark##1#2{\ekv@ifempty@\ekv@ifempty@A}
\long\def\ekv@endset@other
    \ekv@mark\ekv@stop
    \ekv@ifhas@eq@other##1=\ekv@ifempty@B\ekv@ifempty@false
    \ekv@ifempty@A\ekv@ifempty@B\@firstoftwo
    ##2%
    \ekv@set@other##3\ekv@mark
  {}
%    \end{macrocode}
% \end{macro}
% \begin{macro}{\ekvbreak,\ekvbreakPreSneak,\ekvbreakPostSneak}
% Provide macros that can completely stop the parsing of |\ekvset|, who knows
% what it'll be useful for.
%    \begin{macrocode}
\long\def\ekvbreak##1##2\ekv@stop#1##3{##1}
\long\def\ekvbreakPreSneak ##1##2\ekv@stop#1##3{##1##3}
\long\def\ekvbreakPostSneak##1##2\ekv@stop#1##3{##3##1}
%    \end{macrocode}
% \end{macro}
% \begin{macro}{\ekvsneak,\ekvsneakPre}
% One last thing we want to do for |\ekvset| is to provide macros that just
% smuggle stuff after |\ekvset|'s effects.
%    \begin{macrocode}
\long\def\ekvsneak##1##2\ekv@stop#1##3%
  {%
    ##2\ekv@stop#1{##3##1}%
  }
\long\def\ekvsneakPre##1##2\ekv@stop#1##3%
  {%
    ##2\ekv@stop#1{##1##3}%
  }
%    \end{macrocode}
% \end{macro}
%
% \begin{macro}{\ekvparse}
% Additionally to the |\ekvset| macro we also want to provide an |\ekvparse|
% macro, that has the same scope as |\keyval_parse:NNn| from \pkg{expl3}.
% This is pretty analogue to the |\ekvset| implementation, we just put an
% |\unexpanded| here and there instead of other macros to stop the |\expanded|
% on our output.
%    \begin{macrocode}
\long\def\ekvparse##1##2##3%
  {%
    \ekv@parse##1##2\ekv@mark##3#1\ekv@stop#1%
  }
%    \end{macrocode}
% \end{macro}
% \begin{macro}[internal]{\ekv@parse}
%    \begin{macrocode}
\long\def\ekv@parse##1##2##3#1%
  {%
    \ekv@gobbleto@markstop##3\ekv@endparse\ekv@mark\ekv@stop
    \ekv@parse@other##1##2##3,\ekv@stop,%
    \ekv@parse##1##2\ekv@mark
  }
%    \end{macrocode}
% \end{macro}
% \begin{macro}[internal]{\ekv@endparse}
%    \begin{macrocode}
\long\def\ekv@endparse
    \ekv@mark\ekv@stop\ekv@parse@other##1,\ekv@stop,\ekv@parse##2\ekv@mark
  {}
%    \end{macrocode}
% \end{macro}
% \begin{macro}[internal]{\ekv@parse@other}
%    \begin{macrocode}
\long\def\ekv@parse@other##1##2##3,%
  {%
    \ekv@gobbleto@markstop##3\ekv@endparse@other\ekv@mark\ekv@stop
    \ekv@ifhas@eq@other##3=\ekv@ifempty@B\ekv@ifempty@false
      \ekv@ifempty@A\ekv@ifempty@B\@firstoftwo
      {\ekv@parse@eq@other##3\ekv@stop##2}%
      {%
        \ekv@ifhas@eq@active##3#2\ekv@ifempty@B\ekv@ifempty@false
          \ekv@ifempty@A\ekv@ifempty@B\@firstoftwo
          {\ekv@parse@eq@active##3\ekv@stop##2}%
          {%
            \ekv@ifblank@##3\ekv@nil\ekv@ifempty@B\ekv@ifempty@true@F
              \ekv@ifempty@A\ekv@ifempty@B\@firstofone
              {\ekv@strip{##3}\ekv@parse@key##1}%
          }%
      }%
    \ekv@parse@other##1##2\ekv@mark
  }
%    \end{macrocode}
% \end{macro}
% \begin{macro}[internal]{\ekv@parse@eq@other}
%    \begin{macrocode}
\long\def\ekv@parse@eq@other##1=%
  {%
    \ekv@ifhas@eq@active##1#2\ekv@ifempty@B\ekv@ifempty@false
      \ekv@ifempty@A\ekv@ifempty@B\@firstoftwo
      {\ekv@parse@eq@active##1=}%
      {\ekv@strip{##1}\ekv@parse@pair\ekv@mark}%
  }
%    \end{macrocode}
% \end{macro}
% \begin{macro}[internal]{\ekv@parse@eq@active}
%    \begin{macrocode}
\long\def\ekv@parse@eq@active##1#2%
  {%
    \ekv@strip{##1}\ekv@parse@pair\ekv@mark
  }
%    \end{macrocode}
% \end{macro}
% \begin{macro}[internal]{\ekv@endparse@other}
%    \begin{macrocode}
\long\def\ekv@endparse@other
    \ekv@mark\ekv@stop
    \ekv@ifhas@eq@other##1=\ekv@ifempty@B\ekv@ifempty@false
    \ekv@ifempty@A\ekv@ifempty@B\@firstoftwo
    ##2%
    \ekv@parse@other##3\ekv@mark
  {}
%    \end{macrocode}
% \end{macro}
% \begin{macro}[internal]{\ekv@parse@pair,\ekv@parse@pair@}
%    \begin{macrocode}
\long\def\ekv@parse@pair##1##2\ekv@stop
  {%
    \ekv@strip{##2}\ekv@parse@pair@{##1}%
  }
\long\def\ekv@parse@pair@##1##2##3%
  {%
    \unexpanded{##3{##2}{##1}}%
  }
%    \end{macrocode}
% \end{macro}
% \begin{macro}[internal]{\ekv@parse@key}
%    \begin{macrocode}
\long\def\ekv@parse@key##1##2%
  {%
    \unexpanded{##2{##1}}%
  }
%    \end{macrocode}
% \end{macro}
%
% Finally really setting things up with |\ekvset|'s temporary meaning:
%    \begin{macrocode}
}
\begingroup
\catcode`\,=13
\catcode`\==13
\ekvset,=
%    \end{macrocode}
%
% \begin{macro}{\ekvchangeset}
% Provide a macro that is able to switch out the current \set\ in |\ekvset|.
% This operation is slow (by comparison, it should be slightly faster than
% |\ekvsneak|), but allows for something similar to \pkg{pgfkeys}'s
% \texttt{\meta{key}/.cd} mechanism. However this operation is more expensive
% than |/.cd| as we can't just redefine some token to reflect this, but have to
% switch out the set expandably, so this works similar to the |\ekvsneak| macros
% reading and reinserting the remainder of the \kv\ list.
%    \begin{macrocode}
\def\ekvchangeset#1%
  {%
    \expandafter\ekv@changeset\csname\ekv@undefined@set{#1}\endcsname\ekv@mark
  }
%    \end{macrocode}
% \end{macro}
% \begin{macro}[internal]{\ekv@changeset}
% This macro does the real change-out of |\ekvchangeset|. We introduced an
% |\ekv@mark| to not accidentally remove some braces which we have to remove
% again.
%    \begin{macrocode}
\long\def\ekv@changeset#1#2\ekv@set@other#3#4\ekv@set#5%
  {%
    \ekv@gobble@mark#2\ekv@set@other#1#4\ekv@set#1%
  }
%    \end{macrocode}
% \end{macro}
%
%
% \begin{macro}[internal]{\ekv@gobbleto@markstop}
% The |\ekv@gobbleto@markstop| can be used to test for |\ekv@stop| similar to
% our if-empty test, but instead of using tokens which are used nowhere else
% (|\ekv@ifempty@A| and |\ekv@ifempty@B|) it uses |\ekv@mark| and |\ekv@stop|.
%    \begin{macrocode}
\long\def\ekv@gobbleto@markstop#1\ekv@mark\ekv@stop{}
%    \end{macrocode}
% \end{macro}
%
% \begin{macro}[internal]{\ekv@set@pair,\ekv@set@pair@}
% |\ekv@set@pair| gets invoked with the space and brace stripped key-name as its
% first argument, the set-macro as the second argument, and following that is
% the key-value right delimited by an |\ekv@stop|.
%    \begin{macrocode}
\long\def\ekv@set@pair#1#2%
  {%
    \ekv@ifdefined@pair#2{#1}%
      {%
%    \end{macrocode}
% This branch will be executed if the key is not defined as an argument grabbing
% one. If so test whether there is a |NoVal| key of the same name or whether the
% key is unknown. Throw a meaningful error message and gobble the value.
%    \begin{macrocode}
        \ekv@ifdefined{#2{#1}N}%
          \ekv@err@noarg
          \ekv@err@unknown
          #2{#1}%
        \ekv@gobbleto@stop
      }%
  }
%    \end{macrocode}
% |\ekv@ifdefined@pair| will call |\ekv@set@pair@| if the key is correctly
% defined. This will then grab the value, strip outer spaces and braces from it
% and feed it to the key-macro. Afterwards |\ekv@set@other| will take control
% again.
%    \begin{macrocode}
\long\def\ekv@set@pair@#1#2\ekv@stop
  {%
    \ekv@strip{#2}#1%
  }
%    \end{macrocode}
% \end{macro}
%
% \begin{macro}[internal]{\ekv@set@key}
% Analogous to |\ekv@set@pair|, |\ekv@set@key| lets |\ekv@ifdefined@key| test
% whether a |NoVal| key is defined, else it'll throw a meaningful error message.
% Since we don't have to grab any value |\ekv@ifdefined@key| will invoke the
% key-macro and we're done here, |\ekv@set@other| will take over again.
%    \begin{macrocode}
\long\def\ekv@set@key#1#2%
  {%
    \ekv@ifdefined@key#2{#1}%
      {%
        \ekv@ifdefined{#2{#1}}%
          \ekv@err@reqval
          \ekv@err@unknown
          #2{#1}%
      }%
  }
%    \end{macrocode}
% \end{macro}
%
% \begin{macro}[internal]{\ekv@err,\ekv@err@}
% Since |\ekvset| is fully expandable as long as the code of the keys is (which
% is unlikely) we want to somehow throw expandable errors, in our case via
% undefined control sequences.
%    \begin{macrocode}
\begingroup
\edef\ekv@err
  {%
    \endgroup
    \unexpanded{\long\def\ekv@err}##1%
      {%
        \unexpanded{\expandafter\ekv@err@\@firstofone}%
        {\expandafter\noexpand\csname ! expkv Error:\endcsname ##1.}%
        \unexpanded{\ekv@stop}%
      }%
  }
\ekv@err
\def\ekv@err@{\expandafter\ekv@gobbleto@stop}
%    \end{macrocode}
% \end{macro}
% \begin{macro}[internal]
%   {
%     \ekv@err@common,\ekv@err@common@,
%     \ekv@err@unknown,\ekv@err@noarg,\ekv@err@reqval
%   }
% Now we can use |\ekv@err| to set up some error messages so that we can later
% use those instead of the full strings.
%    \begin{macrocode}
\long\def\ekv@err@common #1#2{\expandafter\ekv@err@common@\string#2{#1}}
\long\def\ekv@err@common@#1`#2' #3.#4#5{\ekv@err{#4 (`#5', set `#2')}}
\long\def\ekv@err@unknown#1#2{\ekv@err@common{unknown key}#1{#2}}
\long\def\ekv@err@noarg  #1#2{\ekv@err@common{value forbidden}#1{#2}}
\long\def\ekv@err@reqval #1#2{\ekv@err@common{value required}#1{#2}}
%    \end{macrocode}
% \end{macro}
%
% \begin{macro}[internal]{\ekv@strip,\ekv@strip@a,\ekv@strip@b,\ekv@strip@c}
% Finally we borrow some ideas of \pkg{expl3}'s \pkg{l3tl} to strip spaces
% from keys and values. This |\ekv@strip| also strips one level of outer braces
% \emph{after} stripping spaces, so an input of | {abc} | becomes |abc| after
% stripping. It should be used with |#1| prefixed by |\ekv@mark|. Also this
% implementation at most strips \emph{one} space from both sides.
%    \begin{macrocode}
\def\ekv@strip#1%
  {%
    \long\def\ekv@strip##1%
      {%
        \ekv@strip@a
          ##1%
          \ekv@nil
          \ekv@mark#1%
          #1\ekv@nil{}%
        \ekv@stop
      }%
    \long\def\ekv@strip@a##1\ekv@mark#1##2\ekv@nil##3%
      {%
        \ekv@strip@b##3##1##2\ekv@nil
      }%
    \long\def\ekv@strip@b##1#1\ekv@nil
      {%
        \ekv@strip@c##1\ekv@nil
      }%
    \long\def\ekv@strip@c\ekv@mark##1\ekv@nil##2\ekv@stop##3%
      {%
        ##3{##1}%
      }%
  }
\ekv@strip{ }
%    \end{macrocode}
% \end{macro}
%
% Now everything that's left is to reset the category code of |@|.
%    \begin{macrocode}
\catcode`\@=\ekv@tmp
%    \end{macrocode}
%
% \gobbledocstriptag
%</tex>
%^^A=<<
%
%^^A=<<
%
% \end{implementation}^^A=<<
%
% \clearpage
% \PrintIndex
%
\endinput
%
^^A vim: ft=tex fdm=marker fmr=>>=,=<<
}{}
\ProvidesFile{expkv.dtx}
  [\csname ekvDate\endcsname\ an expandable key=val implementation]
\PassOptionsToPackage{full}{textcomp}
\documentclass{l3doc}
\RequirePackage[oldstylenums,nott]{kpfonts}
\input{glyphtounicode}
\pdfgentounicode=1
\RequirePackage{xfp} ^^A required for an example
\RequirePackage{listings}
\RequirePackage{booktabs}
\RequirePackage{array}
\RequirePackage{collcell}
\RequirePackage{siunitx}
\RequirePackage{xcolor}
\RequirePackage{caption}
\RequirePackage{microtype}
\RequirePackage{accsupp}
\lstset
  {
    ,flexiblecolumns=false
    ,basewidth=.53em
    ,gobble=2
    ,basicstyle=\fontfamily{jkp}\itshape
    ,morekeywords=^^A
      {^^A
        \ekvdef,\ekvdefNoVal,\ekvlet,\ekvletNoVal,\ekvletkv,\ekvletkvNoVal,^^A
        \ekvset,\ekvparse,\ekvVersion,\ekvDate,\ekvifdefined,^^A
        \ekvifdefinedNoVal,\ekvbreak,\ekvbreakPreSneak,\ekvbreakPostSneak,^^A
        \ekvsneak,\ekvsneakPre,\ekvchangeset
      }
    ,morecomment=[l]\%
    ,commentstyle=\color[gray]{0.4}
    ,literate={\{}{{\CodeSymbol\{}}{1}
              {\}}{{\CodeSymbol\}}}{1}
    ^^A,literate=*{<key>}{\key}{4}{<set>}{\set}{4}
  }
\newcommand*\CodeSymbol[1]{\textbf{#1}}
\RequirePackage{randtext}
\let\metaORIG\meta
\protected\def\meta #1{\texttt{\metaORIG{#1}}}
\renewcommand*\thefootnote{\fnsymbol{footnote}}
\definecolor{expkvred}{HTML}{9F393D}
\colorlet{expkvgrey}{black!75}
\makeatletter
\newcommand*\expkv
  {^^A
    \texorpdfstring
      {^^A
        \mbox
          {^^A
            \BeginAccSupp{ActualText=expkv}^^A
            \href{https://github.com/Skillmon/tex_expkv}
              {^^A
                \rmfamily
                \bfseries
                {\color{expkvgrey}e\kern-.05em x\kern-.05em}^^A
                \lower.493ex
                  \hbox{{\color{expkvgrey}P}\kern-.1em{\color{expkvred}k}}^^A
                \kern-.18em{\color{expkvred}v}^^A
              }^^A
            \EndAccSupp{}^^A
          }^^A
      }
      {expkv}^^A
  }
\newcommand*\expkvd
  {^^A
    \texorpdfstring
      {^^A
        \mbox
          {^^A
            \BeginAccSupp{ActualText=expkv-def}^^A
            \rmfamily
            \bfseries
            {\color{expkvgrey}e\kern-.05em x\kern-.05em}^^A
            \lower.493ex
              \hbox{{\color{expkvgrey}P}\kern-.1em{\color{expkvred}k}}^^A
            \kern-.18em{\color{expkvred}v}^^A
            {\color{expkvgrey}^^A
              \kern.05em\rule[-.1ex]{.08em}{1.2ex}\kern.05em\textsc{def}^^A
            }^^A
            \EndAccSupp{}^^A
          }^^A
      }
      {expkv-def}^^A
  }
\newcommand\kv{\meta{key}=\meta{value}}
\newcommand\key{\meta{key}}
\newcommand\val{\meta{value}}
\newcommand\set{\meta{set}}
\hypersetup{linkcolor=red!80!black,urlcolor=purple!80!black}
\DoNotIndex{\def,\edef,\,,\=,\begingroup,\catcode,\chardef,\csname,\endcsname}
\DoNotIndex{\endgroup,\endinput,\errmessage,\expandafter,\input,\let,\long}
\DoNotIndex{\noexpand,\protected,\ProvidesFile,\ProvidesPackage,\relax,\space}
\DoNotIndex{\@,\unexpanded,\string,\expanded}
\DoNotIndex{\ifcsname}
\DoNotIndex{\ifx}
\DoNotIndex{\else}
\DoNotIndex{\fi}
\@gobble\fi ^^A ignoring \ifx and \ifcsname, but only one \fi
\@ifdefinable\gobbledocstriptag{\def\gobbledocstriptag#1>{}}
\newcommand*\pmso[1] ^^A poor man's strike out
  {^^A
    \leavevmode
    \begingroup
      \sbox0{#1}^^A
      \rlap{\vrule height .6ex depth -.5ex width \wd0\relax}^^A
      \usebox0\relax
    \endgroup
  }
\makeatother
\begin{document}
  \title
    {^^A
      \texorpdfstring
        {^^A
          \huge\expkv
          \\[\medskipamount]
          \Large an expandable \kv\ implementation^^A
        }{expkv - an expandable <key>=<value> implementation}^^A
    }
  \date{\ekvDate\space v\ekvVersion}
  \author{Jonathan P. Spratte\thanks{\protect\randomize{jspratte@yahoo.de}}}
  \DocInput{expkv.dtx}
\end{document}
%</driver>^^A=<<
% \fi
%
% \maketitle
% \renewcommand*\thefootnote{\arabic{footnote}}
%
% \begin{abstract}
% \noindent\parfillskip=0pt
% \expkv\ provides a small interface for \kv\ parsing. The parsing macro is
% fully expandable, the \meta{code} of your keys might be not. \expkv\ is pretty
% fast, but not the fastest available \kv\ solution (\pkg{keyval} for instance
% is thrice as fast, but not expandable and it might strip braces it shouldn't
% have stripped).
% \end{abstract}
%
% \tableofcontents
%
% \begin{documentation}^^A>>=
%
% \section{Documentation}
%
% \expkv\ provides an expandable \kv\ parser. The \kv\ pairs should be given as
% a comma separated list and the separator between a \key\ and the associated
% \val\ should be an equal sign. Both, the commas and the equal signs, might be
% of category 12 (other) or 13 (active). To support this is necessary as for
% example \pkg{babel} turns characters active for some languages, for instance
% the equal sign is turned active for Turkish.
%
% \expkv\ is usable as generic code or as a \LaTeX\ package. To use it, just
% use one of:
% \begin{lstlisting}
% \usepackage{expkv} % LaTeX
% \input expkv       % plainTeX
% \end{lstlisting}
% The \LaTeX\ package doesn't do more than \file{expkv.tex}, except calling
% |\ProvidesPackage| and setting things up such that \file{expkv.tex} will use
% |\ProvidesFile|.
%
% \subsection{Setting up Keys}\label{sec:define}
% \expkv\ provides a rather simple approach to setting up keys, similar to
% \pkg{keyval}. However there is an auxiliary package named
% \href{https://github.com/Skillmon/tex_expkv-def}{\expkvd} (the files use
% \file{expkv-def} instead because \string| isn't allowed in package names by
% CTAN) which provides a more sophisticated interface, similar to well
% established packages like \pkg{pgfkeys} or \pkg{l3keys}.
%
% Keys in \expkv\ (as in almost all other \kv\ implementations) belong to a
% \emph{set} such that different sets can contain keys of the same name. Unlike
% many other implementations \expkv\ doesn't provide means to set a default
% value, instead we have keys that take values and keys that don't (the latter
% are called |NoVal| keys by \expkv), but both can have the same name (on the
% user level).
%
% The following macros are available to define new keys. Those macros containing
% ``|def|'' in their name can be prefixed by anything allowed to prefix |\def|,
% prefixes allowed for |\let| can prefix those with ``|let|'' in their name,
% accordingly. Neither \set\ nor \key\ are allowed to be empty for new keys and
% must not contain a |\par| or tokens that expand to it -- they must be legal
% inside of |\csname ...\endcsname|.
%
% \begin{function}{\ekvdef}
%   \begin{syntax}
%     \cs{ekvdef}\marg{set}\marg{key}\marg{code}
%   \end{syntax}
%   Defines a \key\ taking a value in a \set\ to expand to \meta{code}. In
%   \meta{code} you can use |#1| to refer to the given value.
% \end{function}
%
% \begin{function}{\ekvdefNoVal}
%   \begin{syntax}
%     \cs{ekvdefNoVal}\marg{set}\marg{key}\marg{code}
%   \end{syntax}
%   Defines a no value taking \key\ in a \set\ to expand to \meta{code}.
% \end{function}
%
% \begin{function}{\ekvlet}
%   \begin{syntax}
%     \cs{ekvlet}\marg{set}\marg{key}\meta{cs}
%   \end{syntax}
%   Let the value taking \key\ in \set\ to \meta{cs}, there are no checks on
%   \meta{cs} enforced.
% \end{function}
%
% \begin{function}{\ekvletNoVal}
%   \begin{syntax}
%     \cs{ekvletNoVal}\marg{set}\marg{key}\meta{cs}
%   \end{syntax}
%   Let the no value taking \key\ in \set\ to \meta{cs}, it is not checked
%   whether \meta{cs} exists or that it takes no parameter.
% \end{function}
%
% \begin{function}{\ekvletkv}
%   \begin{syntax}
%     \cs{ekvletkv}\marg{set}\marg{key}\marg{set2}\marg{key2}
%   \end{syntax}
%   Let the \key\ in \set\ to \meta{key2} in \meta{set2}, it is not checked
%   whether that second key exists.
% \end{function}
%
% \begin{function}{\ekvletkvNoVal}
%   \begin{syntax}
%     \cs{ekvletkvNoVal}\marg{set}\marg{key}\marg{set2}\marg{key2}
%   \end{syntax}
%   Let the \key\ in \set\ to \meta{key2} in \meta{set2}, it is not checked
%   whether that second key exists.
% \end{function}
%
% \subsection{Parsing Keys}
%
% \begin{function}{\ekvset}
%   \begin{syntax}
%     \cs{ekvset}\marg{set}\{\kv,\ldots\}
%   \end{syntax}
%   Splits \kv\ pairs on commas. From both \key\ and \val\ up to one space is
%   stripped from both ends, if then only a braced group remains the braces are
%   stripped as well. So |\ekvset{foo}{bar=baz}| and
%   |\ekvset{foo}{ {bar}= {baz} }| will both do
%   \texttt{\cs[no-index]{\meta{foobarcode}}\{baz\}}, so you can hide commas,
%   equal signs and spaces at the ends of either \key\ or \val\ by putting
%   braces around them. If you omit the equal sign the code of the key created
%   with the |NoVal| variants described in \autoref{sec:define} will be
%   executed. If \kv\ contains more than a single unhidden equal sign, it will
%   be split at the first one and the others are considered part of the value.
%   |\ekvset| should be nestable.
% \end{function}
%
% \begin{function}{\ekvparse}
%   \begin{syntax}
%     \cs{ekvparse}\meta{cs1}\meta{cs2}\{\kv,\ldots\}
%   \end{syntax}
%   This macro parses the \kv\ pairs and provides those list elements which are
%   only keys as the argument to \meta{cs1}, and those which are a \kv\ pair to
%   \meta{cs2} as two arguments. It is fully expandable as well and returns the
%   parsed list in |\unexpanded|, which has no effect outside of an |\expanded|
%   or |\edef| context\footnotemark.
%   If you need control over the necessary steps of expansion you
%   can use |\expanded| around it.
%
%   |\ekvbreak|, |\ekvsneak|, and |\ekvchangeset| and their relatives don't work
%   in |\ekvparse|.  It is analogue to \pkg{expl3}'s |\keyval_parse:NNn|, but
%   not with the same parsing rules -- |\keyval_parse:NNn| throws an error on
%   multiple equal signs per \kv\ pair and on empty \key\ names in a \kv\ pair,
%   both of which |\ekvparse| doesn't deal with.
%
%   As a small example:
% \begin{lstlisting}
% \ekvparse\handlekey\handlekeyval{foo = bar, key, baz={zzz}}
% \end{lstlisting}
%   would expand to
% \begin{lstlisting}
% \handlekeyval{foo}{bar}\handlekey{key}\handlekeyval{baz}{zzz}
% \end{lstlisting}
%   and afterwards |\handlekey| and |\handlekeyval| would have to further handle
%   the \key. There are no macros like these two contained in \expkv, you have
%   to set them up yourself if you want to use |\ekvparse| (of course the names
%   might differ). If you need the results of |\ekvparse| as the argument for
%   another macro, you should use |\expanded| as only then the input stream will
%   contain the output above:
% \begin{lstlisting}
% \expandafter\handle\expanded{\ekvparse\k\kv{foo = bar, key, baz={zzz}}}
% \end{lstlisting}
%   would expand to
% \begin{lstlisting}
% \handle\kv{foo}{bar}\k{key}\kv{baz}{zzz}
% \end{lstlisting}
% \end{function}
% \footnotetext 
%   {^^A
%     This is a change in behaviour, previously (v0.3 and before) \cs{ekvparse}
%     would expand in exactly two steps. This isn't always necessary, but makes
%     the parsing considerably slower. If this is necessary for your application
%     you can put an \cs[no-index]{expanded} around it and will still be faster
%     since you need only a single \cs[no-index]{expandafter} this way.^^A
%   }.
%
% \subsection{Miscellaneous}
%
% \subsubsection{Other Macros}
%
% \expkv\ provides some other macros which might be of interest.
%
% \begin{function}{\ekvVersion,\ekvDate}
%   These two macros store the version and date of the package.
% \end{function}
% \begin{function}{\ekvifdefined,\ekvifdefinedNoVal}
%   \begin{syntax}
%     \cs{ekvifdefined}\marg{set}\marg{key}\marg{true}\marg{false}
%     \cs{ekvifdefinedNoVal}\marg{set}\marg{key}\marg{true}\marg{false}
%   \end{syntax}
%   These two macros test whether there is a \key\ in \set. It is false if
%   either a hash table entry doesn't exist for that key or its meaning is
%   |\relax|.
% \end{function}
%
% \begin{function}{\ekvbreak,\ekvbreakPreSneak,\ekvbreakPostSneak}
%   \begin{syntax}
%     \cs{ekvbreak}\marg{after}
%   \end{syntax}
%   Gobbles the remainder of the current |\ekvset| macro and its argument list
%   and reinserts \meta{after}. So this can be used to break out of |\ekvset|.
%   The first variant will also gobble anything that has been sneaked out using
%   |\ekvsneak| or |\ekvsneakPre|, while |\ekvbreakPreSneak| will put
%   \meta{after} before anything that has been smuggled and |\ekvbreakPostSneak|
%   will put \meta{after} after the stuff that has been sneaked out.
% \end{function}
%
% \begin{function}{\ekvsneak,\ekvsneakPre}
%   \begin{syntax}
%     \cs{ekvsneak}\marg{after}
%   \end{syntax}
%   Puts \meta{after} after the effects of |\ekvset|. The first variant will put
%   \meta{after} after any other tokens which might have been sneaked before,
%   while |\ekvsneakPre| will put \meta{after} before other smuggled stuff.
%   This reads and reinserts the remainder of the current |\ekvset| macro and
%   its argument list to do its job. A small usage example is shown in
%   \autoref{sec:sneakex}.
% \end{function}
%
% \begin{function}{\ekvchangeset}
%   \begin{syntax}
%     \cs{ekvchangeset}\marg{new-set}
%   \end{syntax}
%   Replaces the current set with \meta{new-set}, so for the rest of the current
%   |\ekvset| call, that call behaves as if it was called with
%   \texttt{\cs[no-index]{ekvset}\marg{new-set}}. Just like |\ekvsneak| this
%   reads and reinserts the remainder of the current |\ekvset| macro to do its
%   job. It is comparable to using \texttt{\meta{key}/.cd} in \pkg{pgfkeys}.
% \end{function}
%
% \bigskip
%
% \begin{function}{\ekv@name,\ekv@name@set,\ekv@name@key}
%   \begin{syntax}
%     \cs{ekv@name}\marg{set}\marg{key}\\
%     \cs{ekv@name@set}\marg{set}\\
%     \cs{ekv@name@key}\marg{key}
%   \end{syntax}
%   The names of the macros that correspond to a key in a set are build with
%   these macros. The default definition of |\ekv@name@set| is
%   ``\texttt{\csname ekv@name@set\endcsname{\set}}'' and the default of
%   |\ekv@name@key| is ``\texttt{\csname ekv@name@key\endcsname{\key}}''.
%   The complete name is build using |\ekv@name| which is equivalent to
%   \texttt{\cs[no-index]{ekv@name@set}\marg{set}\cs{ekv@name@key}\marg{key}}.
%   For |NoVal| keys an additional |N| gets appended irrespective of these
%   macros' definition, so their name is
%   \cs[no-index]{\csname ekv@name\endcsname{\set}{\key}N}. You might redefine
%   |\ekv@name@set| and |\ekv@name@key| locally but \emph{don't redefine}
%   |\ekv@name|!
% \end{function}
%
% \subsubsection{Bugs}
%
% Just like \pkg{keyval}, \expkv\ is bug free. But if you find \pmso{bugs}hidden
% features\footnote{Thanks, David!} you can tell me about them either via mail
% (see the first page) or directly on GitHub if you have an account there:
% \url{https://github.com/Skillmon/tex_expkv}
%
% \subsubsection{Comparisons}
%
% Comparisons of speed are done with a very simple test key and the help of the
% \pkg{l3benchmark} package. The key and its usage should be equivalent to
% \begin{lstlisting}
% \protected\ekvdef{test}{height}{\def\myheight{#1}}
% \ekvset{test}{ height = 6 }
% \end{lstlisting}
% and only the usage of the key, not its definition, is benchmarked. For the
% impatient, the essence of these comparisons regarding speed and buggy
% behaviour is contained in \autoref{tab:comp}.
%
% As far as I know \expkv\ is the only fully expandable \kv\ parser. I tried to
% compare \expkv\ to every \kv\ package listed on
% \href{https://ctan.org/topic/keyval}{CTAN}, however, one might notice that
% some of those are missing from this list. That's because I didn't get the
% others to work due to bugs, or because they just provide wrappers around other
% packages in this list.
%
% In this subsubsection is no benchmark of |\ekvparse| and |\keyval_parse:NNn|
% contained, as most other packages don't provide equivalent features to my
% knowledge. |\ekvparse| is slightly faster than |\ekvset|, but keep in mind
% that it does less. The same is true for |\keyval_parse:NNn| compared to
% |\keys_set:nn| of \pkg{expl3} (where the difference is much bigger).
%
% \paragraph{\pkg{keyval}} is about two times faster and has a comparable
% feature set just a slightly different way how it handles keys without values.
% That might be considered a drawback, as it limits the versatility, but also as
% an advantage, as it might reduce doubled code. Keep in mind that as soon as
% someone loads \pkg{xkeyval} the performance of \pkg{keyval} gets replaced by
% \pkg{xkeyval}'s.
%
% Also \pkg{keyval} has a bug, which unfortunately can't really be resolved
% without breaking backwards compatibility for \emph{many} documents, namely it
% strips braces from the argument before stripping spaces if the argument isn't
% surrounded by spaces, also it might strip more than one set of braces. Hence
% all of the following are equivalent in their outcome, though the last two
% lines should result in something different than the first two:
% \begin{lstlisting}[belowskip=0pt]
% \setkeys{foo}{bar=baz}
% \setkeys{foo}{bar= {baz}}
% \setkeys{foo}{bar={ baz}}
% \setkeys{foo}{bar={{baz}}}
% \end{lstlisting}
%
% \paragraph{\pkg{xkeyval}} is roughly fourteen times slower, but it provides
% more functionality, e.g., it has choice keys, boolean keys, and so on. It
% contains the same bug as \pkg{keyval} as it has to be compatible with it by
% design (it replaces \pkg{keyval}'s frontend), but also adds even more cases in
% which braces are stripped that shouldn't be stripped, worsening the situation.
%
% \paragraph{\pkg{ltxkeys}} is over 300 times slower -- which is funny, because
% it aims to be ``[\ldots] faster than these earlier packages [referring to
% \pkg{keyval} and \pkg{xkeyval}].'' Since it aims to have a bigger feature set
% than \pkg{xkeyval}, it most definitely also has a bigger feature set than
% \expkv. Also, it can't parse |\long| input, so as soon as your values contain
% a |\par|, it'll throw errors. Furthermore, \pkg{ltxkeys} doesn't strip outer
% braces at all by design, which, imho, is a weird design choice. In addition
% \pkg{ltxkeys} loads \pkg{catoptions} which is known to introduce bugs (e.g.,
% see \url{https://tex.stackexchange.com/questions/461783}).
%
% \paragraph{\pkg{l3keys}} is almost five times slower, but has an, imho, great
% interface to define keys. It strips \emph{all} outer spaces, even if somehow
% multiple spaces ended up on either end. It offers more features, but is pretty
% much bound to \pkg{expl3} code. Whether that's a drawback is up to you.
%
% \paragraph{\pkg{pgfkeys}} is a bit more than two times slower for one key,
% but has an \emph{enormous} feature set. However, since adding additional keys
% doesn't add as much needed time for \pkg{pgfkeys} compared to \expkv, it gets
% faster than \expkv\ at around eight \kv\ pairs.
% It has the same or a very similar bug \pkg{keyval} has. The brace bug (and
% also the category fragility) can be fixed by \pkg{pgfkeyx}, but this package
% was last updated in 2012 and it slows down |\pgfkeys| by factor~$8$. Also I
% don't know whether this might introduce new bugs.
%
% \paragraph{\pkg{kvsetkeys} with \pkg{kvdefinekeys}} is about three times
% slower, but it works even if commas and equals have category codes different
% from 12 (just as some other packages in this list). Else the features of the
% keys are equal to those of \pkg{keyval}, the parser has more features, though.
%
% \paragraph{\pkg{options}} is a bit slower for only a single value, but gets a
% tad faster than \expkv\ at around 10 \kv\ pairs. It has a much bigger feature
% set. Unfortunately it also suffers from the premature unbracing bug
% \pkg{keyval} has.
%
% \paragraph{\pkg{simplekv}} is hard to compare because I don't speak French (so
% I don't understand the documentation) and from what I can see, there is no
% direct way to define the equivalent test key. Nevertheless, I tested the
% closest possible equivalent of my test key while siding for \pkg{simplekv}'s
% design not forcing something into it it doesn't seem to be designed for. It is
% more than five times slower and has hard to predict behaviour regarding brace
% and space stripping, similar to \pkg{keyval}. The tested definition was:
% \begin{lstlisting}[belowskip=0pt]
% \usepackage{simplekv}
% \setKVdefault[simplekv]{height={ abc}}                     % key setup
% \setKV[simplekv]{ height = 6 }                             % benchmarked
% \end{lstlisting}
%
% \paragraph{\pkg{yax}} is over eighteen times slower. It has a pretty strange
% syntax, imho, and again a direct equivalent is hard to define. It has the
% premature unbracing bug, too. Also somehow loading \pkg{yax} broke
% \pkg{options} for me. The tested definition was:
% \begin{lstlisting}[belowskip=0pt]
% \usepackage{yax}
% \defactiveparameter yax {\storevalue\myheight yax:height } % key setup
% \setparameterlist{yax}{ height = 6 }                       % benchmarked
% \end{lstlisting}
%
% \begin{table}
%   \def\fnsym{\textcolor{red!80!black}{*}}%
%   \sisetup{round-precision=1, round-mode=places}%
%   \begingroup
%   \centering
%   \newcommand*\yes{\textcolor{red!80!black}  {yes}}^^A
%   \newcommand*\no {\textcolor{green!80!black}{no}}^^A
%   \caption[Comparison of \kv\ packages]
%     {^^A
%       Comparison of \kv\ packages. The packages are ordered from
%       fastest to slowest for one \kv\ pair. Benchmarking was done using
%       \pkg{l3benchmark} and the scripts in the \file{Benchmarks} folder
%       of the \href{https://github.com/Skillmon/tex_expkv}{git repository}.
%       The columns $p_i$ are the polynomial coefficients of a linear fit to the
%       run-time, $p_0$ can be interpreted as the overhead for initialisation
%       and $p_1$ the cost per key. The $T_0$ column is the actual mean ops
%       needed for an empty list argument, as the linear fit doesn't match that
%       point well in general. The column ``BB'' lists whether the
%       parsing is affected by some sort of brace bug, ``CF'' stands for
%       category code fragile and lists whether the parsing breaks with active
%       commas or equal signs.^^A
%       \label{tab:comp}^^A
%     }
%   \begin{tabular}
%       {>{\collectcell\pkg}l<{\endcollectcell}*3{S[table-format=4.1]}ccc}
%     \toprule
%     \rmfamily Package & {$p_1$} & {$p_0$} & {$T_0$}& BB & CF & Date \\
%     \midrule
%     keyval    &   13.539 &    1.844 &    6.969 & \yes & \yes & 2014-10-28 \\
%     \expkv    &   26.411 &    2.391 &   12.610 & \no  & \no  & 2020-02-22 \\
%     options   &   25.532 &   10.518 &   21.393 & \yes & \yes & 2015-03-01 \\
%     pgfkeys   &   26.444 &   40.939 &   55.790 & \yes & \yes & 2020-01-08 \\
%     kvsetkeys & {\fnsym} & {\fnsym} &   41.620 & \no  & \no  & 2019-12-15 \\
%     l3keys    &  114.456 &   35.357 &   52.667 & \no  & \no  & 2020-02-14 \\
%     simplekv  &  160.523 &   10.809 &    8.698 & \yes & \yes & 2017-08-08 \\
%     xkeyval   &  260.872 &  180.435 &  164.167 & \yes & \yes & 2014-12-03 \\
%     yax       &  507.643 &   62.863 &  123.533 & \yes & \yes & 2010-01-22 \\
%     ltxkeys   & 3932.759 & 4737.818 & 5883.000 & \no  & \no  & 2012-11-17 \\
%     \bottomrule
%   \end{tabular}
%   \par
%   \endgroup
%   \medskip
%   \fnsym For \pkg{kvsetkeys} the linear model used for the other
%   packages is a poor fit, \pkg{kvsetkeys} seems to have approximately
%   quadratic run-time, the coefficients of the second degree polynomial fit are
%   $p_2=\num{9.911}$, $p_1=\num{40.523}$, and $p_0=\num{61.500}$. Of course the
%   other packages might not really have linear run-time, but at least from 1~to
%   20~keys the fits don't seem too bad (the maximum ratio $p_2/p_1$
%   for the other packages is \num{9.697e-3}). If one extrapolates the fits
%   for 100 \kv\ pairs one finds that most of them match pretty well, the
%   exception being \pkg{ltxkeys}, which behaves quadratic as well with
%   $p_2=\num{31.685}$, $p_1=\num{3267.368}$, and $p_0=\num{7177.582}$.
% \end{table}
%
% \subsection{Examples}
%
% \subsubsection{Standard Use-Case}
%
% Say we have a macro for which we want to create a \kv\ interface. The macro
% has a parameter, which is stored in the dimension |\ourdim| having a default
% value from its initialization. Now we want to be able to change that dimension
% with the |width| key to some specified value. For that we'd do
% \begin{lstlisting}
% \newdimen\ourdim
% \ourdim=150pt
% \protected\ekvdef{our}{width}{\ourdim=#1\relax}
% \end{lstlisting}
% as you can see, we use the set |our| here. We want the key to behave different
% if no value is specified. In that case the key should not use its initial
% value, but be smart and determine the available space from
% \cs[no-index]{hsize}, so we also define
% \begin{lstlisting}
% \protected\ekvdefNoVal{our}{width}{\ourdim=.9\hsize}
% \end{lstlisting}
% Now we set up our macro to use this \kv\ interface
% \begin{lstlisting}
% \protected\def\ourmacro#1{\begingroup\ekvset{our}{#1}\the\ourdim\endgroup}
% \end{lstlisting}
% Finally we can use our macro like in the following\\
% \begin{minipage}{6cm}
% \begin{lstlisting}
% \ourmacro{}\par
% \ourmacro{width}\par
% \ourmacro{width=5pt}\par
% \end{lstlisting}
% \end{minipage}^^A
% \begin{minipage}{\dimexpr\linewidth-6cm\relax}
%   \newdimen\ourdim\ourdim=150pt
%   \protected\ekvdef{our}{width}{\ourdim=#1\relax}^^A
%   \protected\ekvdefNoVal{our}{width}{\ourdim=.9\hsize}^^A
%   \protected\def\ourmacro#1^^A
%     {\begingroup\ekvset{our}{#1}\the\ourdim\endgroup}^^A
%   \ourmacro{}\par
%   \ourmacro{width}\par
%   \ourmacro{width=5pt}\par
% \end{minipage}
%
% \paragraph{The same key using \protect\expkvd}
% Using \expkvd\ we can set up the equivalent key using a \kv\ interface, after
% the following we could use |\ourmacro| in the same way as above. \expkvd\ will
% allocate and initialise |\ourdim| and define the |width| key |\protected| for
% us, so the result will be exactly the same -- with the exception that the
% default will use |\ourdim=.9\hsize\relax| instead.
% \begin{lstlisting}
% \input expkv-def                 % or \usepackage{expkv-def}
% \ekvdefinekeys{our}
%   {
%     dimen    width = \ourdim,
%     qdefault width = .9\hsize,
%     initial  width = 150pt
%   }
% \end{lstlisting}
%
% \subsubsection{An Expandable \kv\ Macro Using \cs[no-index]{ekvsneak}}
% \label{sec:sneakex}
%
% Let's set up an expandable macro, that uses a \kv\ interface. The problems
% we'll face for this are:
% \begin{enumerate}
%   \item ignoring duplicate keys
%   \item default values for keys which weren't used
%   \item providing the values as the correct argument to a macro (ordered)
% \end{enumerate}
%
% First we need to decide which \kv\ parsing macro we want to do this with,
% |\ekvset| or |\ekvparse|. For this example we also want to show the
% usage of |\ekvsneak|, hence we'll choose |\ekvset|. And we'll have to use
% |\ekvset| such that it builds a parsable list for our macro internals. To
% gain back control after |\ekvset| is done we have to put an internal of our
% macro at the start of that list, so we use an internal key that uses
% |\ekvsneakPre| after any user input.
%
% To ignore duplicates will be easy if the value of the key used last will be
% put first in the list, so the following will use |\ekvsneakPre| for the
% user-level keys. If we wanted some key for which the first usage should be the
% the binding one we would use |\ekvsneak| instead for that key.
%
% Providing default values can be done in different ways, we'll use a simple
% approach in which we'll just put the outcome of our keys if they were used
% with default values before the parsing list terminator.
%
% Ordering the keys can be done simply by searching for a specific token for
% each argument which acts like a flag, so our sneaked out values will include
% specific tokens acting as markers.
%
% Now that we have answers for our technical problems, we have to decide what
% our example macro should do. How about we define a macro that calculates the
% sine of a number and rounds that to a specified precision? As a small extra
% this macro should understand input in radian and degree and the used
% trigonometric function should be selectable as well. For the hard part of this
% task (expandably evaluating trigonometric functions) we'll use the \pkg{xfp}
% package.
%
% First we set up our keys according to our earlier considerations and set up
% the user facing macro |\sine|. The end marker of the parsing list will be a
% |\sine@stop| token, which we don't need to define and we put our defaults
% right before it.
%
% \begin{lstlisting}
% \RequirePackage{xfp}
% \makeatletter
% \ekvdef{expex}{f}{\ekvsneakPre{\f{#1}}}
% \ekvdef{expex}{round}{\ekvsneakPre{\rnd{#1}}}
% \ekvdefNoVal{expex}{degree}{\ekvsneakPre{\deg{d}}}
% \ekvdefNoVal{expex}{radian}{\ekvsneakPre{\deg{}}}
% \ekvdefNoVal{expex}{internal}{\ekvsneakPre{\sine@rnd}}
% \newcommand*\sine[2]
%   {\ekvset{expex}{#1,internal}\rnd{3}\deg{d}\f{sin}\sine@stop{#2}}
% \end{lstlisting}
% For the sake of simplicity we defined the macro |\sine| with two mandatory
% arguments, the first being the \kv\ list, the second the argument to the
% trigonometric function. We could've used \pkg{xparse}'s facilities here to
% define an expandable macro which takes an optional argument instead.
%
% Now we need to define some internal macros to extract the value of each key's
% last usage (remember that this will be the group after the first special
% flag-token). For that we use one delimited macro per key.
% \begin{lstlisting}
% \def\sine@rnd#1\rnd#2#3\sine@stop{\sine@deg#1#3\sine@stop{#2}}
% \def\sine@deg#1\deg#2#3\sine@stop{\sine@f#1#3\sine@stop{#2}}
% \def\sine@f#1\f#2#3\sine@stop{\sine@final{#2}}
% \end{lstlisting}
% After the macros |\sine@rnd|, |\sine@deg|, and |\sine@f| the macro
% |\sine@final| will see
% \texttt
%   {^^A
%     \cs[no-index]{sine@final}^^A
%     \marg{f}\marg{degree/radian}\marg{round}\marg{num}^^A
%   }.
% Now |\sine@final| has to expandably deal with those arguments such that the
% |\fpeval| macro of \pkg{xfp} gets the correct input. Luckily this is pretty
% straight forward in this example. In |\fpeval| the trigonometric functions
% have names such as |sin| or |cos| and the degree taking variants |sind| or
% |cosd|. And since the |degree| key puts a |d| in |#2| and the |radian| key
% leaves |#2| empty all we have to do to get the correct function name is stick
% the two together.
% \begin{lstlisting}
% \newcommand*\sine@final[4]{\fpeval{round(#1#2(#4),#3)}}
% \makeatother
% \end{lstlisting}
% Let's test our macro:\\
% \begin{minipage}{.7\linewidth}
%   \begin{lstlisting}[gobble=4]
%   \sine{}{60}\par
%   \sine{round=10}{60}\par
%   \sine{f=cos,radian}{pi}\par
%   \edef\myval{\sine{f=tan}{1}}\texttt{\meaning\myval}
%   \end{lstlisting}
% \end{minipage}\hfill
% \begin{minipage}{.2\linewidth}
%   \makeatletter
%   \ekvdef{expex}{f}{\ekvsneakPre{\f{#1}}}^^A
%   \ekvdef{expex}{round}{\ekvsneakPre{\rnd{#1}}}^^A
%   \ekvdefNoVal{expex}{degree}{\ekvsneakPre{\deg{d}}}^^A
%   \ekvdefNoVal{expex}{radian}{\ekvsneakPre{\deg{}}}^^A
%   \ekvdefNoVal{expex}{internal}{\ekvsneakPre{\sine@rnd}}^^A
%   \newcommand*\sine[2]
%     {\ekvset{expex}{#1,internal}\rnd{3}\deg{d}\f{sin}\sine@stop{#2}}^^A
%   \def\sine@rnd#1\rnd#2#3\sine@stop{\sine@deg#1#3\sine@stop{#2}}^^A
%   \def\sine@deg#1\deg#2#3\sine@stop{\sine@f#1#3\sine@stop{#2}}^^A
%   \def\sine@f#1\f#2#3\sine@stop{\sine@final{#2}}^^A
%   \newcommand*\sine@final[4]{\fpeval{round(#1#2(#4),#3)}}^^A
%   \makeatother
%   \sine{}{60}\par
%   \sine{round=10}{60}\par
%   \sine{f=cos,radian}{pi}\par
%   \edef\myval{\sine{f=tan}{1}}\texttt{\meaning\myval}
% \end{minipage}
%
%
% \subsection{Error Messages}
%
% \expkv\ should only send messages in case of errors, there are no warnings and
% no info messages. In this subsection those errors are listed.
%
% \subsubsection{Load Time}
%
% \file{expkv.tex} checks whether \eTeX\ is available. If it isn't, an error
% will be thrown using |\errmessage|:
% \begin{lstlisting}
% ! expkv Error: e-TeX required.
% \end{lstlisting}
%
% \subsubsection{Defining Keys}
%
% If you get any error from \expkv\ while you're trying to define a key, the
% definition will be aborted and gobbled.
%
% If you try to define a key with an empty set name you'll get:
% \begin{lstlisting}
% ! expkv Error: empty set name not allowed.
% \end{lstlisting}
%
% Similarly, if you try to define a key with an empty key name:
% \begin{lstlisting}
% ! expkv Error: empty key name not allowed.
% \end{lstlisting}
%
% Both of these messages are done in a way that doesn't throw additional errors
% due to |\global|, |\long|, etc., not being used correctly if you prefixed one
% of the defining macros.
%
% \subsubsection{Using Keys}
%
% This subsubsection contains the errors thrown during |\ekvset|. The errors are
% thrown in an expandable manner by providing an undefined macro. In the
% following messages \lstinline|<key>| gets replaced with the problematic key's
% name, and \lstinline|<set>| with the corresponding set. If any errors during
% \kv\ handling are encountered, the entry in the comma separated list will be
% omitted after the error is thrown and the next \kv\ pair will be parsed.
%
% If you're using an undefined key you'll get:
% \begin{lstlisting}
% ! Undefined control sequence.
% <argument> \! expkv Error:
%                            unknown key (`<key>', set `<set>').
% \end{lstlisting}
%
% If you're using a key for which only a normal version and no |NoVal| version
% is defined, but don't provide a value, you'll get:
% \begin{lstlisting}
% ! Undefined control sequence.
% <argument> \! expkv Error:
%                            value required (`<key>', set `<set>').
% \end{lstlisting}
%
% If you're using a key for which only a |NoVal| version and no normal version
% is defined, but provide a value, you'll get:
% \begin{lstlisting}
% ! Undefined control sequence.
% <argument> \! expkv Error:
%                            value forbidden (`<key>', set `<set>').
% \end{lstlisting}
%
% If you're using a set for which you never executed one of the defining macros
% from \autoref{sec:define} you'll get a low level \TeX\ error, as that isn't
% actively tested by the parser (and hence will lead to undefined behaviour and
% not be gracefully ignored). The error will look like
% \begin{lstlisting}
% ! Missing \endcsname inserted.
% <to be read again>
%                    \! expkv Error: Set `<set>' undefined.
% \end{lstlisting}
%
% \subsection{License}
%
% Copyright \textcopyright\ 2020\unless\ifnum\year=2020--\the\year\fi\
% Jonathan P. Spratte
%
% \medskip\noindent
% This work may be distributed and/or modified under the conditions of the
% \LaTeX\ Project Public License (LPPL), either version 1.3c of this license or
% (at your option) any later version. The latest version of this license is in
% the file:
%
% \url{http://www.latex-project.org/lppl.txt}
%
% \noindent
% This work is ``maintained'' (as per LPPL maintenance status) by
%
% \mbox{Jonathan P. Spratte}.
%
% \end{documentation}^^A=<<
%
% \begin{implementation}^^A>>=
%
% \clearpage
%
% \section{Implementation}^^A>>=
%^^A the LaTeX package >>=
% \subsection{The \LaTeX\ Package}
% First we set up the \LaTeX\ package. That one doesn't really do much except
% |\input|ting the generic code and identifying itself as a package.
% \gobbledocstriptag
%<*pkg>
%    \begin{macrocode}
\def\ekv@tmp
  {%
    \ProvidesFile{expkv.tex}%
      [\ekvDate\space v\ekvVersion\space an expandable key=val implementation]%
  }
% \iffalse meta-comment
%
% File: expkv.dtx Copyright (C) 2020 Jonathan P. Spratte
%
% This work  may be  distributed and/or  modified under  the conditions  of the
% LaTeX Project Public License (LPPL),  either version 1.3c  of this license or
% (at your option) any later version.  The latest version of this license is in
% the file:
%
%   http://www.latex-project.org/lppl.txt
%
% ------------------------------------------------------------------------------
%
%<*driver>^^A>>=
\def\nameofplainTeX{plain}
\ifx\fmtname\nameofplainTeX\else
  \expandafter\begingroup
\fi
\input l3docstrip.tex
\askforoverwritefalse
\preamble

--------------------------------------------------------------
expkv -- an expandable key=val implementation
E-mail: jspratte@yahoo.de
Released under the LaTeX Project Public License v1.3c or later
See http://www.latex-project.org/lppl.txt
--------------------------------------------------------------

Copyright (C) 2020 Jonathan P. Spratte

This  work may be  distributed and/or  modified under  the conditions  of the
LaTeX Project Public License (LPPL),  either version 1.3c  of this license or
(at your option) any later version.  The latest version of this license is in
the file:

  http://www.latex-project.org/lppl.txt

This work is "maintained" (as per LPPL maintenance status) by
  Jonathan P. Spratte.

This work consists of the file  expkv.dtx
and the derived files           expkv.pdf
                                expkv.sty
                                expkv.tex

\endpreamble
% stop docstrip adding \endinput
\postamble
\endpostamble
\generate{\file{expkv.sty}{\from{expkv.dtx}{pkg}}}
\generate{\file{expkv.tex}{\from{expkv.dtx}{tex}}}
\ifx\fmtname\nameofplainTeX
  \expandafter\endbatchfile
\else
  \expandafter\endgroup
\fi
%
\IfFileExists{expkv.tex}{\input{expkv.tex}}{}
\ProvidesFile{expkv.dtx}
  [\csname ekvDate\endcsname\ an expandable key=val implementation]
\PassOptionsToPackage{full}{textcomp}
\documentclass{l3doc}
\RequirePackage[oldstylenums,nott]{kpfonts}
\input{glyphtounicode}
\pdfgentounicode=1
\RequirePackage{xfp} ^^A required for an example
\RequirePackage{listings}
\RequirePackage{booktabs}
\RequirePackage{array}
\RequirePackage{collcell}
\RequirePackage{siunitx}
\RequirePackage{xcolor}
\RequirePackage{caption}
\RequirePackage{microtype}
\RequirePackage{accsupp}
\lstset
  {
    ,flexiblecolumns=false
    ,basewidth=.53em
    ,gobble=2
    ,basicstyle=\fontfamily{jkp}\itshape
    ,morekeywords=^^A
      {^^A
        \ekvdef,\ekvdefNoVal,\ekvlet,\ekvletNoVal,\ekvletkv,\ekvletkvNoVal,^^A
        \ekvset,\ekvparse,\ekvVersion,\ekvDate,\ekvifdefined,^^A
        \ekvifdefinedNoVal,\ekvbreak,\ekvbreakPreSneak,\ekvbreakPostSneak,^^A
        \ekvsneak,\ekvsneakPre,\ekvchangeset
      }
    ,morecomment=[l]\%
    ,commentstyle=\color[gray]{0.4}
    ,literate={\{}{{\CodeSymbol\{}}{1}
              {\}}{{\CodeSymbol\}}}{1}
    ^^A,literate=*{<key>}{\key}{4}{<set>}{\set}{4}
  }
\newcommand*\CodeSymbol[1]{\textbf{#1}}
\RequirePackage{randtext}
\let\metaORIG\meta
\protected\def\meta #1{\texttt{\metaORIG{#1}}}
\renewcommand*\thefootnote{\fnsymbol{footnote}}
\definecolor{expkvred}{HTML}{9F393D}
\colorlet{expkvgrey}{black!75}
\makeatletter
\newcommand*\expkv
  {^^A
    \texorpdfstring
      {^^A
        \mbox
          {^^A
            \BeginAccSupp{ActualText=expkv}^^A
            \href{https://github.com/Skillmon/tex_expkv}
              {^^A
                \rmfamily
                \bfseries
                {\color{expkvgrey}e\kern-.05em x\kern-.05em}^^A
                \lower.493ex
                  \hbox{{\color{expkvgrey}P}\kern-.1em{\color{expkvred}k}}^^A
                \kern-.18em{\color{expkvred}v}^^A
              }^^A
            \EndAccSupp{}^^A
          }^^A
      }
      {expkv}^^A
  }
\newcommand*\expkvd
  {^^A
    \texorpdfstring
      {^^A
        \mbox
          {^^A
            \BeginAccSupp{ActualText=expkv-def}^^A
            \rmfamily
            \bfseries
            {\color{expkvgrey}e\kern-.05em x\kern-.05em}^^A
            \lower.493ex
              \hbox{{\color{expkvgrey}P}\kern-.1em{\color{expkvred}k}}^^A
            \kern-.18em{\color{expkvred}v}^^A
            {\color{expkvgrey}^^A
              \kern.05em\rule[-.1ex]{.08em}{1.2ex}\kern.05em\textsc{def}^^A
            }^^A
            \EndAccSupp{}^^A
          }^^A
      }
      {expkv-def}^^A
  }
\newcommand\kv{\meta{key}=\meta{value}}
\newcommand\key{\meta{key}}
\newcommand\val{\meta{value}}
\newcommand\set{\meta{set}}
\hypersetup{linkcolor=red!80!black,urlcolor=purple!80!black}
\DoNotIndex{\def,\edef,\,,\=,\begingroup,\catcode,\chardef,\csname,\endcsname}
\DoNotIndex{\endgroup,\endinput,\errmessage,\expandafter,\input,\let,\long}
\DoNotIndex{\noexpand,\protected,\ProvidesFile,\ProvidesPackage,\relax,\space}
\DoNotIndex{\@,\unexpanded,\string,\expanded}
\DoNotIndex{\ifcsname}
\DoNotIndex{\ifx}
\DoNotIndex{\else}
\DoNotIndex{\fi}
\@gobble\fi ^^A ignoring \ifx and \ifcsname, but only one \fi
\@ifdefinable\gobbledocstriptag{\def\gobbledocstriptag#1>{}}
\newcommand*\pmso[1] ^^A poor man's strike out
  {^^A
    \leavevmode
    \begingroup
      \sbox0{#1}^^A
      \rlap{\vrule height .6ex depth -.5ex width \wd0\relax}^^A
      \usebox0\relax
    \endgroup
  }
\makeatother
\begin{document}
  \title
    {^^A
      \texorpdfstring
        {^^A
          \huge\expkv
          \\[\medskipamount]
          \Large an expandable \kv\ implementation^^A
        }{expkv - an expandable <key>=<value> implementation}^^A
    }
  \date{\ekvDate\space v\ekvVersion}
  \author{Jonathan P. Spratte\thanks{\protect\randomize{jspratte@yahoo.de}}}
  \DocInput{expkv.dtx}
\end{document}
%</driver>^^A=<<
% \fi
%
% \maketitle
% \renewcommand*\thefootnote{\arabic{footnote}}
%
% \begin{abstract}
% \noindent\parfillskip=0pt
% \expkv\ provides a small interface for \kv\ parsing. The parsing macro is
% fully expandable, the \meta{code} of your keys might be not. \expkv\ is pretty
% fast, but not the fastest available \kv\ solution (\pkg{keyval} for instance
% is thrice as fast, but not expandable and it might strip braces it shouldn't
% have stripped).
% \end{abstract}
%
% \tableofcontents
%
% \begin{documentation}^^A>>=
%
% \section{Documentation}
%
% \expkv\ provides an expandable \kv\ parser. The \kv\ pairs should be given as
% a comma separated list and the separator between a \key\ and the associated
% \val\ should be an equal sign. Both, the commas and the equal signs, might be
% of category 12 (other) or 13 (active). To support this is necessary as for
% example \pkg{babel} turns characters active for some languages, for instance
% the equal sign is turned active for Turkish.
%
% \expkv\ is usable as generic code or as a \LaTeX\ package. To use it, just
% use one of:
% \begin{lstlisting}
% \usepackage{expkv} % LaTeX
% \input expkv       % plainTeX
% \end{lstlisting}
% The \LaTeX\ package doesn't do more than \file{expkv.tex}, except calling
% |\ProvidesPackage| and setting things up such that \file{expkv.tex} will use
% |\ProvidesFile|.
%
% \subsection{Setting up Keys}\label{sec:define}
% \expkv\ provides a rather simple approach to setting up keys, similar to
% \pkg{keyval}. However there is an auxiliary package named
% \href{https://github.com/Skillmon/tex_expkv-def}{\expkvd} (the files use
% \file{expkv-def} instead because \string| isn't allowed in package names by
% CTAN) which provides a more sophisticated interface, similar to well
% established packages like \pkg{pgfkeys} or \pkg{l3keys}.
%
% Keys in \expkv\ (as in almost all other \kv\ implementations) belong to a
% \emph{set} such that different sets can contain keys of the same name. Unlike
% many other implementations \expkv\ doesn't provide means to set a default
% value, instead we have keys that take values and keys that don't (the latter
% are called |NoVal| keys by \expkv), but both can have the same name (on the
% user level).
%
% The following macros are available to define new keys. Those macros containing
% ``|def|'' in their name can be prefixed by anything allowed to prefix |\def|,
% prefixes allowed for |\let| can prefix those with ``|let|'' in their name,
% accordingly. Neither \set\ nor \key\ are allowed to be empty for new keys and
% must not contain a |\par| or tokens that expand to it -- they must be legal
% inside of |\csname ...\endcsname|.
%
% \begin{function}{\ekvdef}
%   \begin{syntax}
%     \cs{ekvdef}\marg{set}\marg{key}\marg{code}
%   \end{syntax}
%   Defines a \key\ taking a value in a \set\ to expand to \meta{code}. In
%   \meta{code} you can use |#1| to refer to the given value.
% \end{function}
%
% \begin{function}{\ekvdefNoVal}
%   \begin{syntax}
%     \cs{ekvdefNoVal}\marg{set}\marg{key}\marg{code}
%   \end{syntax}
%   Defines a no value taking \key\ in a \set\ to expand to \meta{code}.
% \end{function}
%
% \begin{function}{\ekvlet}
%   \begin{syntax}
%     \cs{ekvlet}\marg{set}\marg{key}\meta{cs}
%   \end{syntax}
%   Let the value taking \key\ in \set\ to \meta{cs}, there are no checks on
%   \meta{cs} enforced.
% \end{function}
%
% \begin{function}{\ekvletNoVal}
%   \begin{syntax}
%     \cs{ekvletNoVal}\marg{set}\marg{key}\meta{cs}
%   \end{syntax}
%   Let the no value taking \key\ in \set\ to \meta{cs}, it is not checked
%   whether \meta{cs} exists or that it takes no parameter.
% \end{function}
%
% \begin{function}{\ekvletkv}
%   \begin{syntax}
%     \cs{ekvletkv}\marg{set}\marg{key}\marg{set2}\marg{key2}
%   \end{syntax}
%   Let the \key\ in \set\ to \meta{key2} in \meta{set2}, it is not checked
%   whether that second key exists.
% \end{function}
%
% \begin{function}{\ekvletkvNoVal}
%   \begin{syntax}
%     \cs{ekvletkvNoVal}\marg{set}\marg{key}\marg{set2}\marg{key2}
%   \end{syntax}
%   Let the \key\ in \set\ to \meta{key2} in \meta{set2}, it is not checked
%   whether that second key exists.
% \end{function}
%
% \subsection{Parsing Keys}
%
% \begin{function}{\ekvset}
%   \begin{syntax}
%     \cs{ekvset}\marg{set}\{\kv,\ldots\}
%   \end{syntax}
%   Splits \kv\ pairs on commas. From both \key\ and \val\ up to one space is
%   stripped from both ends, if then only a braced group remains the braces are
%   stripped as well. So |\ekvset{foo}{bar=baz}| and
%   |\ekvset{foo}{ {bar}= {baz} }| will both do
%   \texttt{\cs[no-index]{\meta{foobarcode}}\{baz\}}, so you can hide commas,
%   equal signs and spaces at the ends of either \key\ or \val\ by putting
%   braces around them. If you omit the equal sign the code of the key created
%   with the |NoVal| variants described in \autoref{sec:define} will be
%   executed. If \kv\ contains more than a single unhidden equal sign, it will
%   be split at the first one and the others are considered part of the value.
%   |\ekvset| should be nestable.
% \end{function}
%
% \begin{function}{\ekvparse}
%   \begin{syntax}
%     \cs{ekvparse}\meta{cs1}\meta{cs2}\{\kv,\ldots\}
%   \end{syntax}
%   This macro parses the \kv\ pairs and provides those list elements which are
%   only keys as the argument to \meta{cs1}, and those which are a \kv\ pair to
%   \meta{cs2} as two arguments. It is fully expandable as well and returns the
%   parsed list in |\unexpanded|, which has no effect outside of an |\expanded|
%   or |\edef| context\footnotemark.
%   If you need control over the necessary steps of expansion you
%   can use |\expanded| around it.
%
%   |\ekvbreak|, |\ekvsneak|, and |\ekvchangeset| and their relatives don't work
%   in |\ekvparse|.  It is analogue to \pkg{expl3}'s |\keyval_parse:NNn|, but
%   not with the same parsing rules -- |\keyval_parse:NNn| throws an error on
%   multiple equal signs per \kv\ pair and on empty \key\ names in a \kv\ pair,
%   both of which |\ekvparse| doesn't deal with.
%
%   As a small example:
% \begin{lstlisting}
% \ekvparse\handlekey\handlekeyval{foo = bar, key, baz={zzz}}
% \end{lstlisting}
%   would expand to
% \begin{lstlisting}
% \handlekeyval{foo}{bar}\handlekey{key}\handlekeyval{baz}{zzz}
% \end{lstlisting}
%   and afterwards |\handlekey| and |\handlekeyval| would have to further handle
%   the \key. There are no macros like these two contained in \expkv, you have
%   to set them up yourself if you want to use |\ekvparse| (of course the names
%   might differ). If you need the results of |\ekvparse| as the argument for
%   another macro, you should use |\expanded| as only then the input stream will
%   contain the output above:
% \begin{lstlisting}
% \expandafter\handle\expanded{\ekvparse\k\kv{foo = bar, key, baz={zzz}}}
% \end{lstlisting}
%   would expand to
% \begin{lstlisting}
% \handle\kv{foo}{bar}\k{key}\kv{baz}{zzz}
% \end{lstlisting}
% \end{function}
% \footnotetext 
%   {^^A
%     This is a change in behaviour, previously (v0.3 and before) \cs{ekvparse}
%     would expand in exactly two steps. This isn't always necessary, but makes
%     the parsing considerably slower. If this is necessary for your application
%     you can put an \cs[no-index]{expanded} around it and will still be faster
%     since you need only a single \cs[no-index]{expandafter} this way.^^A
%   }.
%
% \subsection{Miscellaneous}
%
% \subsubsection{Other Macros}
%
% \expkv\ provides some other macros which might be of interest.
%
% \begin{function}{\ekvVersion,\ekvDate}
%   These two macros store the version and date of the package.
% \end{function}
% \begin{function}{\ekvifdefined,\ekvifdefinedNoVal}
%   \begin{syntax}
%     \cs{ekvifdefined}\marg{set}\marg{key}\marg{true}\marg{false}
%     \cs{ekvifdefinedNoVal}\marg{set}\marg{key}\marg{true}\marg{false}
%   \end{syntax}
%   These two macros test whether there is a \key\ in \set. It is false if
%   either a hash table entry doesn't exist for that key or its meaning is
%   |\relax|.
% \end{function}
%
% \begin{function}{\ekvbreak,\ekvbreakPreSneak,\ekvbreakPostSneak}
%   \begin{syntax}
%     \cs{ekvbreak}\marg{after}
%   \end{syntax}
%   Gobbles the remainder of the current |\ekvset| macro and its argument list
%   and reinserts \meta{after}. So this can be used to break out of |\ekvset|.
%   The first variant will also gobble anything that has been sneaked out using
%   |\ekvsneak| or |\ekvsneakPre|, while |\ekvbreakPreSneak| will put
%   \meta{after} before anything that has been smuggled and |\ekvbreakPostSneak|
%   will put \meta{after} after the stuff that has been sneaked out.
% \end{function}
%
% \begin{function}{\ekvsneak,\ekvsneakPre}
%   \begin{syntax}
%     \cs{ekvsneak}\marg{after}
%   \end{syntax}
%   Puts \meta{after} after the effects of |\ekvset|. The first variant will put
%   \meta{after} after any other tokens which might have been sneaked before,
%   while |\ekvsneakPre| will put \meta{after} before other smuggled stuff.
%   This reads and reinserts the remainder of the current |\ekvset| macro and
%   its argument list to do its job. A small usage example is shown in
%   \autoref{sec:sneakex}.
% \end{function}
%
% \begin{function}{\ekvchangeset}
%   \begin{syntax}
%     \cs{ekvchangeset}\marg{new-set}
%   \end{syntax}
%   Replaces the current set with \meta{new-set}, so for the rest of the current
%   |\ekvset| call, that call behaves as if it was called with
%   \texttt{\cs[no-index]{ekvset}\marg{new-set}}. Just like |\ekvsneak| this
%   reads and reinserts the remainder of the current |\ekvset| macro to do its
%   job. It is comparable to using \texttt{\meta{key}/.cd} in \pkg{pgfkeys}.
% \end{function}
%
% \bigskip
%
% \begin{function}{\ekv@name,\ekv@name@set,\ekv@name@key}
%   \begin{syntax}
%     \cs{ekv@name}\marg{set}\marg{key}\\
%     \cs{ekv@name@set}\marg{set}\\
%     \cs{ekv@name@key}\marg{key}
%   \end{syntax}
%   The names of the macros that correspond to a key in a set are build with
%   these macros. The default definition of |\ekv@name@set| is
%   ``\texttt{\csname ekv@name@set\endcsname{\set}}'' and the default of
%   |\ekv@name@key| is ``\texttt{\csname ekv@name@key\endcsname{\key}}''.
%   The complete name is build using |\ekv@name| which is equivalent to
%   \texttt{\cs[no-index]{ekv@name@set}\marg{set}\cs{ekv@name@key}\marg{key}}.
%   For |NoVal| keys an additional |N| gets appended irrespective of these
%   macros' definition, so their name is
%   \cs[no-index]{\csname ekv@name\endcsname{\set}{\key}N}. You might redefine
%   |\ekv@name@set| and |\ekv@name@key| locally but \emph{don't redefine}
%   |\ekv@name|!
% \end{function}
%
% \subsubsection{Bugs}
%
% Just like \pkg{keyval}, \expkv\ is bug free. But if you find \pmso{bugs}hidden
% features\footnote{Thanks, David!} you can tell me about them either via mail
% (see the first page) or directly on GitHub if you have an account there:
% \url{https://github.com/Skillmon/tex_expkv}
%
% \subsubsection{Comparisons}
%
% Comparisons of speed are done with a very simple test key and the help of the
% \pkg{l3benchmark} package. The key and its usage should be equivalent to
% \begin{lstlisting}
% \protected\ekvdef{test}{height}{\def\myheight{#1}}
% \ekvset{test}{ height = 6 }
% \end{lstlisting}
% and only the usage of the key, not its definition, is benchmarked. For the
% impatient, the essence of these comparisons regarding speed and buggy
% behaviour is contained in \autoref{tab:comp}.
%
% As far as I know \expkv\ is the only fully expandable \kv\ parser. I tried to
% compare \expkv\ to every \kv\ package listed on
% \href{https://ctan.org/topic/keyval}{CTAN}, however, one might notice that
% some of those are missing from this list. That's because I didn't get the
% others to work due to bugs, or because they just provide wrappers around other
% packages in this list.
%
% In this subsubsection is no benchmark of |\ekvparse| and |\keyval_parse:NNn|
% contained, as most other packages don't provide equivalent features to my
% knowledge. |\ekvparse| is slightly faster than |\ekvset|, but keep in mind
% that it does less. The same is true for |\keyval_parse:NNn| compared to
% |\keys_set:nn| of \pkg{expl3} (where the difference is much bigger).
%
% \paragraph{\pkg{keyval}} is about two times faster and has a comparable
% feature set just a slightly different way how it handles keys without values.
% That might be considered a drawback, as it limits the versatility, but also as
% an advantage, as it might reduce doubled code. Keep in mind that as soon as
% someone loads \pkg{xkeyval} the performance of \pkg{keyval} gets replaced by
% \pkg{xkeyval}'s.
%
% Also \pkg{keyval} has a bug, which unfortunately can't really be resolved
% without breaking backwards compatibility for \emph{many} documents, namely it
% strips braces from the argument before stripping spaces if the argument isn't
% surrounded by spaces, also it might strip more than one set of braces. Hence
% all of the following are equivalent in their outcome, though the last two
% lines should result in something different than the first two:
% \begin{lstlisting}[belowskip=0pt]
% \setkeys{foo}{bar=baz}
% \setkeys{foo}{bar= {baz}}
% \setkeys{foo}{bar={ baz}}
% \setkeys{foo}{bar={{baz}}}
% \end{lstlisting}
%
% \paragraph{\pkg{xkeyval}} is roughly fourteen times slower, but it provides
% more functionality, e.g., it has choice keys, boolean keys, and so on. It
% contains the same bug as \pkg{keyval} as it has to be compatible with it by
% design (it replaces \pkg{keyval}'s frontend), but also adds even more cases in
% which braces are stripped that shouldn't be stripped, worsening the situation.
%
% \paragraph{\pkg{ltxkeys}} is over 300 times slower -- which is funny, because
% it aims to be ``[\ldots] faster than these earlier packages [referring to
% \pkg{keyval} and \pkg{xkeyval}].'' Since it aims to have a bigger feature set
% than \pkg{xkeyval}, it most definitely also has a bigger feature set than
% \expkv. Also, it can't parse |\long| input, so as soon as your values contain
% a |\par|, it'll throw errors. Furthermore, \pkg{ltxkeys} doesn't strip outer
% braces at all by design, which, imho, is a weird design choice. In addition
% \pkg{ltxkeys} loads \pkg{catoptions} which is known to introduce bugs (e.g.,
% see \url{https://tex.stackexchange.com/questions/461783}).
%
% \paragraph{\pkg{l3keys}} is almost five times slower, but has an, imho, great
% interface to define keys. It strips \emph{all} outer spaces, even if somehow
% multiple spaces ended up on either end. It offers more features, but is pretty
% much bound to \pkg{expl3} code. Whether that's a drawback is up to you.
%
% \paragraph{\pkg{pgfkeys}} is a bit more than two times slower for one key,
% but has an \emph{enormous} feature set. However, since adding additional keys
% doesn't add as much needed time for \pkg{pgfkeys} compared to \expkv, it gets
% faster than \expkv\ at around eight \kv\ pairs.
% It has the same or a very similar bug \pkg{keyval} has. The brace bug (and
% also the category fragility) can be fixed by \pkg{pgfkeyx}, but this package
% was last updated in 2012 and it slows down |\pgfkeys| by factor~$8$. Also I
% don't know whether this might introduce new bugs.
%
% \paragraph{\pkg{kvsetkeys} with \pkg{kvdefinekeys}} is about three times
% slower, but it works even if commas and equals have category codes different
% from 12 (just as some other packages in this list). Else the features of the
% keys are equal to those of \pkg{keyval}, the parser has more features, though.
%
% \paragraph{\pkg{options}} is a bit slower for only a single value, but gets a
% tad faster than \expkv\ at around 10 \kv\ pairs. It has a much bigger feature
% set. Unfortunately it also suffers from the premature unbracing bug
% \pkg{keyval} has.
%
% \paragraph{\pkg{simplekv}} is hard to compare because I don't speak French (so
% I don't understand the documentation) and from what I can see, there is no
% direct way to define the equivalent test key. Nevertheless, I tested the
% closest possible equivalent of my test key while siding for \pkg{simplekv}'s
% design not forcing something into it it doesn't seem to be designed for. It is
% more than five times slower and has hard to predict behaviour regarding brace
% and space stripping, similar to \pkg{keyval}. The tested definition was:
% \begin{lstlisting}[belowskip=0pt]
% \usepackage{simplekv}
% \setKVdefault[simplekv]{height={ abc}}                     % key setup
% \setKV[simplekv]{ height = 6 }                             % benchmarked
% \end{lstlisting}
%
% \paragraph{\pkg{yax}} is over eighteen times slower. It has a pretty strange
% syntax, imho, and again a direct equivalent is hard to define. It has the
% premature unbracing bug, too. Also somehow loading \pkg{yax} broke
% \pkg{options} for me. The tested definition was:
% \begin{lstlisting}[belowskip=0pt]
% \usepackage{yax}
% \defactiveparameter yax {\storevalue\myheight yax:height } % key setup
% \setparameterlist{yax}{ height = 6 }                       % benchmarked
% \end{lstlisting}
%
% \begin{table}
%   \def\fnsym{\textcolor{red!80!black}{*}}%
%   \sisetup{round-precision=1, round-mode=places}%
%   \begingroup
%   \centering
%   \newcommand*\yes{\textcolor{red!80!black}  {yes}}^^A
%   \newcommand*\no {\textcolor{green!80!black}{no}}^^A
%   \caption[Comparison of \kv\ packages]
%     {^^A
%       Comparison of \kv\ packages. The packages are ordered from
%       fastest to slowest for one \kv\ pair. Benchmarking was done using
%       \pkg{l3benchmark} and the scripts in the \file{Benchmarks} folder
%       of the \href{https://github.com/Skillmon/tex_expkv}{git repository}.
%       The columns $p_i$ are the polynomial coefficients of a linear fit to the
%       run-time, $p_0$ can be interpreted as the overhead for initialisation
%       and $p_1$ the cost per key. The $T_0$ column is the actual mean ops
%       needed for an empty list argument, as the linear fit doesn't match that
%       point well in general. The column ``BB'' lists whether the
%       parsing is affected by some sort of brace bug, ``CF'' stands for
%       category code fragile and lists whether the parsing breaks with active
%       commas or equal signs.^^A
%       \label{tab:comp}^^A
%     }
%   \begin{tabular}
%       {>{\collectcell\pkg}l<{\endcollectcell}*3{S[table-format=4.1]}ccc}
%     \toprule
%     \rmfamily Package & {$p_1$} & {$p_0$} & {$T_0$}& BB & CF & Date \\
%     \midrule
%     keyval    &   13.539 &    1.844 &    6.969 & \yes & \yes & 2014-10-28 \\
%     \expkv    &   26.411 &    2.391 &   12.610 & \no  & \no  & 2020-02-22 \\
%     options   &   25.532 &   10.518 &   21.393 & \yes & \yes & 2015-03-01 \\
%     pgfkeys   &   26.444 &   40.939 &   55.790 & \yes & \yes & 2020-01-08 \\
%     kvsetkeys & {\fnsym} & {\fnsym} &   41.620 & \no  & \no  & 2019-12-15 \\
%     l3keys    &  114.456 &   35.357 &   52.667 & \no  & \no  & 2020-02-14 \\
%     simplekv  &  160.523 &   10.809 &    8.698 & \yes & \yes & 2017-08-08 \\
%     xkeyval   &  260.872 &  180.435 &  164.167 & \yes & \yes & 2014-12-03 \\
%     yax       &  507.643 &   62.863 &  123.533 & \yes & \yes & 2010-01-22 \\
%     ltxkeys   & 3932.759 & 4737.818 & 5883.000 & \no  & \no  & 2012-11-17 \\
%     \bottomrule
%   \end{tabular}
%   \par
%   \endgroup
%   \medskip
%   \fnsym For \pkg{kvsetkeys} the linear model used for the other
%   packages is a poor fit, \pkg{kvsetkeys} seems to have approximately
%   quadratic run-time, the coefficients of the second degree polynomial fit are
%   $p_2=\num{9.911}$, $p_1=\num{40.523}$, and $p_0=\num{61.500}$. Of course the
%   other packages might not really have linear run-time, but at least from 1~to
%   20~keys the fits don't seem too bad (the maximum ratio $p_2/p_1$
%   for the other packages is \num{9.697e-3}). If one extrapolates the fits
%   for 100 \kv\ pairs one finds that most of them match pretty well, the
%   exception being \pkg{ltxkeys}, which behaves quadratic as well with
%   $p_2=\num{31.685}$, $p_1=\num{3267.368}$, and $p_0=\num{7177.582}$.
% \end{table}
%
% \subsection{Examples}
%
% \subsubsection{Standard Use-Case}
%
% Say we have a macro for which we want to create a \kv\ interface. The macro
% has a parameter, which is stored in the dimension |\ourdim| having a default
% value from its initialization. Now we want to be able to change that dimension
% with the |width| key to some specified value. For that we'd do
% \begin{lstlisting}
% \newdimen\ourdim
% \ourdim=150pt
% \protected\ekvdef{our}{width}{\ourdim=#1\relax}
% \end{lstlisting}
% as you can see, we use the set |our| here. We want the key to behave different
% if no value is specified. In that case the key should not use its initial
% value, but be smart and determine the available space from
% \cs[no-index]{hsize}, so we also define
% \begin{lstlisting}
% \protected\ekvdefNoVal{our}{width}{\ourdim=.9\hsize}
% \end{lstlisting}
% Now we set up our macro to use this \kv\ interface
% \begin{lstlisting}
% \protected\def\ourmacro#1{\begingroup\ekvset{our}{#1}\the\ourdim\endgroup}
% \end{lstlisting}
% Finally we can use our macro like in the following\\
% \begin{minipage}{6cm}
% \begin{lstlisting}
% \ourmacro{}\par
% \ourmacro{width}\par
% \ourmacro{width=5pt}\par
% \end{lstlisting}
% \end{minipage}^^A
% \begin{minipage}{\dimexpr\linewidth-6cm\relax}
%   \newdimen\ourdim\ourdim=150pt
%   \protected\ekvdef{our}{width}{\ourdim=#1\relax}^^A
%   \protected\ekvdefNoVal{our}{width}{\ourdim=.9\hsize}^^A
%   \protected\def\ourmacro#1^^A
%     {\begingroup\ekvset{our}{#1}\the\ourdim\endgroup}^^A
%   \ourmacro{}\par
%   \ourmacro{width}\par
%   \ourmacro{width=5pt}\par
% \end{minipage}
%
% \paragraph{The same key using \protect\expkvd}
% Using \expkvd\ we can set up the equivalent key using a \kv\ interface, after
% the following we could use |\ourmacro| in the same way as above. \expkvd\ will
% allocate and initialise |\ourdim| and define the |width| key |\protected| for
% us, so the result will be exactly the same -- with the exception that the
% default will use |\ourdim=.9\hsize\relax| instead.
% \begin{lstlisting}
% \input expkv-def                 % or \usepackage{expkv-def}
% \ekvdefinekeys{our}
%   {
%     dimen    width = \ourdim,
%     qdefault width = .9\hsize,
%     initial  width = 150pt
%   }
% \end{lstlisting}
%
% \subsubsection{An Expandable \kv\ Macro Using \cs[no-index]{ekvsneak}}
% \label{sec:sneakex}
%
% Let's set up an expandable macro, that uses a \kv\ interface. The problems
% we'll face for this are:
% \begin{enumerate}
%   \item ignoring duplicate keys
%   \item default values for keys which weren't used
%   \item providing the values as the correct argument to a macro (ordered)
% \end{enumerate}
%
% First we need to decide which \kv\ parsing macro we want to do this with,
% |\ekvset| or |\ekvparse|. For this example we also want to show the
% usage of |\ekvsneak|, hence we'll choose |\ekvset|. And we'll have to use
% |\ekvset| such that it builds a parsable list for our macro internals. To
% gain back control after |\ekvset| is done we have to put an internal of our
% macro at the start of that list, so we use an internal key that uses
% |\ekvsneakPre| after any user input.
%
% To ignore duplicates will be easy if the value of the key used last will be
% put first in the list, so the following will use |\ekvsneakPre| for the
% user-level keys. If we wanted some key for which the first usage should be the
% the binding one we would use |\ekvsneak| instead for that key.
%
% Providing default values can be done in different ways, we'll use a simple
% approach in which we'll just put the outcome of our keys if they were used
% with default values before the parsing list terminator.
%
% Ordering the keys can be done simply by searching for a specific token for
% each argument which acts like a flag, so our sneaked out values will include
% specific tokens acting as markers.
%
% Now that we have answers for our technical problems, we have to decide what
% our example macro should do. How about we define a macro that calculates the
% sine of a number and rounds that to a specified precision? As a small extra
% this macro should understand input in radian and degree and the used
% trigonometric function should be selectable as well. For the hard part of this
% task (expandably evaluating trigonometric functions) we'll use the \pkg{xfp}
% package.
%
% First we set up our keys according to our earlier considerations and set up
% the user facing macro |\sine|. The end marker of the parsing list will be a
% |\sine@stop| token, which we don't need to define and we put our defaults
% right before it.
%
% \begin{lstlisting}
% \RequirePackage{xfp}
% \makeatletter
% \ekvdef{expex}{f}{\ekvsneakPre{\f{#1}}}
% \ekvdef{expex}{round}{\ekvsneakPre{\rnd{#1}}}
% \ekvdefNoVal{expex}{degree}{\ekvsneakPre{\deg{d}}}
% \ekvdefNoVal{expex}{radian}{\ekvsneakPre{\deg{}}}
% \ekvdefNoVal{expex}{internal}{\ekvsneakPre{\sine@rnd}}
% \newcommand*\sine[2]
%   {\ekvset{expex}{#1,internal}\rnd{3}\deg{d}\f{sin}\sine@stop{#2}}
% \end{lstlisting}
% For the sake of simplicity we defined the macro |\sine| with two mandatory
% arguments, the first being the \kv\ list, the second the argument to the
% trigonometric function. We could've used \pkg{xparse}'s facilities here to
% define an expandable macro which takes an optional argument instead.
%
% Now we need to define some internal macros to extract the value of each key's
% last usage (remember that this will be the group after the first special
% flag-token). For that we use one delimited macro per key.
% \begin{lstlisting}
% \def\sine@rnd#1\rnd#2#3\sine@stop{\sine@deg#1#3\sine@stop{#2}}
% \def\sine@deg#1\deg#2#3\sine@stop{\sine@f#1#3\sine@stop{#2}}
% \def\sine@f#1\f#2#3\sine@stop{\sine@final{#2}}
% \end{lstlisting}
% After the macros |\sine@rnd|, |\sine@deg|, and |\sine@f| the macro
% |\sine@final| will see
% \texttt
%   {^^A
%     \cs[no-index]{sine@final}^^A
%     \marg{f}\marg{degree/radian}\marg{round}\marg{num}^^A
%   }.
% Now |\sine@final| has to expandably deal with those arguments such that the
% |\fpeval| macro of \pkg{xfp} gets the correct input. Luckily this is pretty
% straight forward in this example. In |\fpeval| the trigonometric functions
% have names such as |sin| or |cos| and the degree taking variants |sind| or
% |cosd|. And since the |degree| key puts a |d| in |#2| and the |radian| key
% leaves |#2| empty all we have to do to get the correct function name is stick
% the two together.
% \begin{lstlisting}
% \newcommand*\sine@final[4]{\fpeval{round(#1#2(#4),#3)}}
% \makeatother
% \end{lstlisting}
% Let's test our macro:\\
% \begin{minipage}{.7\linewidth}
%   \begin{lstlisting}[gobble=4]
%   \sine{}{60}\par
%   \sine{round=10}{60}\par
%   \sine{f=cos,radian}{pi}\par
%   \edef\myval{\sine{f=tan}{1}}\texttt{\meaning\myval}
%   \end{lstlisting}
% \end{minipage}\hfill
% \begin{minipage}{.2\linewidth}
%   \makeatletter
%   \ekvdef{expex}{f}{\ekvsneakPre{\f{#1}}}^^A
%   \ekvdef{expex}{round}{\ekvsneakPre{\rnd{#1}}}^^A
%   \ekvdefNoVal{expex}{degree}{\ekvsneakPre{\deg{d}}}^^A
%   \ekvdefNoVal{expex}{radian}{\ekvsneakPre{\deg{}}}^^A
%   \ekvdefNoVal{expex}{internal}{\ekvsneakPre{\sine@rnd}}^^A
%   \newcommand*\sine[2]
%     {\ekvset{expex}{#1,internal}\rnd{3}\deg{d}\f{sin}\sine@stop{#2}}^^A
%   \def\sine@rnd#1\rnd#2#3\sine@stop{\sine@deg#1#3\sine@stop{#2}}^^A
%   \def\sine@deg#1\deg#2#3\sine@stop{\sine@f#1#3\sine@stop{#2}}^^A
%   \def\sine@f#1\f#2#3\sine@stop{\sine@final{#2}}^^A
%   \newcommand*\sine@final[4]{\fpeval{round(#1#2(#4),#3)}}^^A
%   \makeatother
%   \sine{}{60}\par
%   \sine{round=10}{60}\par
%   \sine{f=cos,radian}{pi}\par
%   \edef\myval{\sine{f=tan}{1}}\texttt{\meaning\myval}
% \end{minipage}
%
%
% \subsection{Error Messages}
%
% \expkv\ should only send messages in case of errors, there are no warnings and
% no info messages. In this subsection those errors are listed.
%
% \subsubsection{Load Time}
%
% \file{expkv.tex} checks whether \eTeX\ is available. If it isn't, an error
% will be thrown using |\errmessage|:
% \begin{lstlisting}
% ! expkv Error: e-TeX required.
% \end{lstlisting}
%
% \subsubsection{Defining Keys}
%
% If you get any error from \expkv\ while you're trying to define a key, the
% definition will be aborted and gobbled.
%
% If you try to define a key with an empty set name you'll get:
% \begin{lstlisting}
% ! expkv Error: empty set name not allowed.
% \end{lstlisting}
%
% Similarly, if you try to define a key with an empty key name:
% \begin{lstlisting}
% ! expkv Error: empty key name not allowed.
% \end{lstlisting}
%
% Both of these messages are done in a way that doesn't throw additional errors
% due to |\global|, |\long|, etc., not being used correctly if you prefixed one
% of the defining macros.
%
% \subsubsection{Using Keys}
%
% This subsubsection contains the errors thrown during |\ekvset|. The errors are
% thrown in an expandable manner by providing an undefined macro. In the
% following messages \lstinline|<key>| gets replaced with the problematic key's
% name, and \lstinline|<set>| with the corresponding set. If any errors during
% \kv\ handling are encountered, the entry in the comma separated list will be
% omitted after the error is thrown and the next \kv\ pair will be parsed.
%
% If you're using an undefined key you'll get:
% \begin{lstlisting}
% ! Undefined control sequence.
% <argument> \! expkv Error:
%                            unknown key (`<key>', set `<set>').
% \end{lstlisting}
%
% If you're using a key for which only a normal version and no |NoVal| version
% is defined, but don't provide a value, you'll get:
% \begin{lstlisting}
% ! Undefined control sequence.
% <argument> \! expkv Error:
%                            value required (`<key>', set `<set>').
% \end{lstlisting}
%
% If you're using a key for which only a |NoVal| version and no normal version
% is defined, but provide a value, you'll get:
% \begin{lstlisting}
% ! Undefined control sequence.
% <argument> \! expkv Error:
%                            value forbidden (`<key>', set `<set>').
% \end{lstlisting}
%
% If you're using a set for which you never executed one of the defining macros
% from \autoref{sec:define} you'll get a low level \TeX\ error, as that isn't
% actively tested by the parser (and hence will lead to undefined behaviour and
% not be gracefully ignored). The error will look like
% \begin{lstlisting}
% ! Missing \endcsname inserted.
% <to be read again>
%                    \! expkv Error: Set `<set>' undefined.
% \end{lstlisting}
%
% \subsection{License}
%
% Copyright \textcopyright\ 2020\unless\ifnum\year=2020--\the\year\fi\
% Jonathan P. Spratte
%
% \medskip\noindent
% This work may be distributed and/or modified under the conditions of the
% \LaTeX\ Project Public License (LPPL), either version 1.3c of this license or
% (at your option) any later version. The latest version of this license is in
% the file:
%
% \url{http://www.latex-project.org/lppl.txt}
%
% \noindent
% This work is ``maintained'' (as per LPPL maintenance status) by
%
% \mbox{Jonathan P. Spratte}.
%
% \end{documentation}^^A=<<
%
% \begin{implementation}^^A>>=
%
% \clearpage
%
% \section{Implementation}^^A>>=
%^^A the LaTeX package >>=
% \subsection{The \LaTeX\ Package}
% First we set up the \LaTeX\ package. That one doesn't really do much except
% |\input|ting the generic code and identifying itself as a package.
% \gobbledocstriptag
%<*pkg>
%    \begin{macrocode}
\def\ekv@tmp
  {%
    \ProvidesFile{expkv.tex}%
      [\ekvDate\space v\ekvVersion\space an expandable key=val implementation]%
  }
\input{expkv.tex}
\ProvidesPackage{expkv}%
  [\ekvDate\space v\ekvVersion\space an expandable key=val implementation]
%    \end{macrocode}
% \gobbledocstriptag
%</pkg>
%^^A=<<
%^^A main file >>=
% \subsection{The Generic Code}
% The rest of this implementation will be the generic code.
% \gobbledocstriptag
%<*tex>
%
% Check whether \eTeX\ is available -- \expkv\ requires \eTeX.
%    \begin{macrocode}
\begingroup\expandafter\expandafter\expandafter\endgroup
\expandafter\ifx\csname numexpr\endcsname\relax
  \errmessage{expkv requires e-TeX}
  \expandafter\endinput
\fi
%    \end{macrocode}
%
% We make sure that it's only input once:
%    \begin{macrocode}
\expandafter\ifx\csname ekvVersion\endcsname\relax
\else
  \expandafter\endinput
\fi
%    \end{macrocode}
% \begin{macro}{\ekvVersion,\ekvDate}
% We're on our first input, so lets store the version and date in a macro.
%    \begin{macrocode}
\def\ekvVersion{0.5a}
\def\ekvDate{2020-02-27}
%    \end{macrocode}
% \end{macro}
%
% If the \LaTeX\ format is loaded we want to be a good file and report back who
% we are, for this the package will have defined |\ekv@tmp| to use
% |\ProvidesFile|, else this will expand to a |\relax| and do no harm.
%    \begin{macrocode}
\csname ekv@tmp\endcsname
%    \end{macrocode}
%
% Store the category code of |@| to later be able to reset it and change it to
% 11 for now.
%    \begin{macrocode}
\expandafter\chardef\csname ekv@tmp\endcsname=\catcode`\@
\catcode`\@=11
%    \end{macrocode}
% |\ekv@tmp| might later be reused to gobble any prefixes which might be
% provided to |\ekvdef| and similar in case the names are invalid, we just
% temporarily use it here as means to store the current category code of |@| to
% restore it at the end of the file, we never care for the actual definition of
% it.
%
% \begin{macro}[internal]
%   {
%     \@gobble,\@firstofone,\@firstoftwo,\@secondoftwo,
%     \ekv@gobbleto@stop,\ekv@fi@secondoftwo,\ekv@gobble@mark
%   }
% Since branching tests are often more versatile than |\if...\else...\fi|
% constructs, we define helpers that are branching pretty fast. Also here are
% some other utility functions that just grab some tokens. The ones that are
% also contained in \LaTeX\ don't use the |ekv| prefix.
%    \begin{macrocode}
\long\def\@gobble#1{}
\long\def\@firstofone#1{#1}
\long\def\@firstoftwo#1#2{#1}
\long\def\@secondoftwo#1#2{#2}
\long\def\ekv@fi@secondoftwo\fi\@firstoftwo#1#2{\fi#2}
\long\def\ekv@gobbleto@stop#1\ekv@stop{}
\def\ekv@gobble@mark\ekv@mark{}
%    \end{macrocode}
% \end{macro}
% As you can see |\ekv@gobbleto@stop| uses a special marker |\ekv@stop|. The
% package will use three such markers, the one you've seen already, |\ekv@mark|
% and |\ekv@nil|. Contrarily to how for instance \pkg{expl3} does things, we
% don't define them, as we don't need them to have an actual meaning. This has
% the advantage that if they somehow get expanded -- which should never happen
% if things work out -- they'll throw an error directly.
%
% \begin{macro}[internal]
%   {
%     \ekv@ifempty,\ekv@ifempty@,\ekv@ifempty@true,\ekv@ifempty@false,
%     \ekv@ifempty@true@F
%   }
% We can test for a lot of things building on an if-empty test, so lets define a
% really fast one. Since some tests might have reversed logic (true if something
% is not empty) we also set up macros for the reversed branches.
%    \begin{macrocode}
\long\def\ekv@ifempty#1%
  {%
    \ekv@ifempty@\ekv@ifempty@A#1\ekv@ifempty@B\ekv@ifempty@true
      \ekv@ifempty@A\ekv@ifempty@B\@secondoftwo
  }
\long\def\ekv@ifempty@#1\ekv@ifempty@A\ekv@ifempty@B{}
\long\def\ekv@ifempty@true\ekv@ifempty@A\ekv@ifempty@B\@secondoftwo#1#2{#1}
\long\def\ekv@ifempty@false\ekv@ifempty@A\ekv@ifempty@B\@firstoftwo#1#2{#2}
\long\def\ekv@ifempty@true@F\ekv@ifempty@A\ekv@ifempty@B\@firstofone#1{}
%    \end{macrocode}
% \end{macro}
%
% \begin{macro}[internal]{\ekv@ifblank@}
% The obvious test that can be based on an if-empty is if-blank, meaning a test
% checking whether the argument is empty or consists only of spaces. Our version
% here will be tweaked a bit, as we want to check this, but with one leading
% |\ekv@mark| token that is to be ignored.
%    \begin{macrocode}
\long\def\ekv@ifblank@\ekv@mark#1{\ekv@ifempty@\ekv@ifempty@A}
%    \end{macrocode}
% \end{macro}
%
% \begin{macro}[internal]{\ekv@ifdefined}
% We'll need to check whether something is defined quite frequently, so why not
% define a macro that does this. The following test is expandable, slower than
% the typical expandable test for undefined control sequences, but faster for
% defined ones. Since we want to be as fast as possible for correct input, this
% is to be preferred.
%    \begin{macrocode}
\def\ekv@ifdefined#1%
  {%
    \expandafter
    \ifx\csname\ifcsname #1\endcsname #1\else relax\fi\endcsname\relax
      \ekv@fi@secondoftwo
    \fi
    \@firstoftwo
  }
%    \end{macrocode}
% \end{macro}
  % \begin{macro}[internal]
%   {
%     \ekv@ifdefined@pair,\ekv@ifdefined@pair@,\ekv@ifdefined@key,
%     \ekv@ifdefined@key@
%   }
% Since we can save some time if we only have to create the control sequence
% once when we know beforehand how we want to use it, we build some other macros
% for those cases (which we'll have quite often, once per key usage).
%    \begin{macrocode}
\def\ekv@ifdefined@pair#1#2%
  {%
    \expandafter\ekv@ifdefined@pair@
      \csname
        \ifcsname #1{#2}\endcsname
          #1{#2}%
        \else
          relax%
        \fi
      \endcsname
  }
\def\ekv@ifdefined@pair@#1%
  {%
    \ifx#1\relax
      \ekv@fi@secondoftwo
    \fi
    \@firstoftwo
    {\ekv@set@pair@#1\ekv@mark}%
  }
\def\ekv@ifdefined@key#1#2%
  {%
    \expandafter\ekv@ifdefined@key@
      \csname
        \ifcsname #1{#2}N\endcsname
          #1{#2}N%
        \else
          relax%
        \fi
      \endcsname
  }
\def\ekv@ifdefined@key@#1%
  {%
    \ifx#1\relax
      \ekv@fi@secondoftwo
    \fi
    \@firstoftwo#1%
  }
%    \end{macrocode}
% \end{macro}
%
% \begin{macro}{\ekv@name,\ekv@name@set,\ekv@name@key}
% The keys will all follow the same naming scheme, so we define it here.
%    \begin{macrocode}
\def\ekv@name#1#2{\ekv@name@set{#1}\ekv@name@key{#2}}
\def\ekv@name@set#1{ekv#1(}
\def\ekv@name@key#1{#1)}
%    \end{macrocode}
% \end{macro}
% \begin{macro}[internal]{\ekv@undefined@set}
% We can misuse the macro name we use to expandably store the set-name in a
% single token -- since this increases performance drastically, especially for
% long set-names -- to throw a more meaningful error message in case a set isn't
% defined. The name of |\ekv@undefined@set| is a little bit misleading, as it is
% called in either case inside of |\csname|, but the result will be a control
% sequence with meaning |\relax| if the set is undefined, hence will break the
% |\csname| building the key-macro which will throw the error message.
%    \begin{macrocode}
\def\ekv@undefined@set#1{! expkv Error: Set `#1' undefined.}
%    \end{macrocode}
% \end{macro}
% \begin{macro}[internal]{\ekv@checkvalid}
% We place some restrictions on the allowed names, though, namely sets and
% keys are not allowed to be empty -- blanks are fine (meaning \mbox{set-
% or} key-names consisting of spaces).
%    \begin{macrocode}
\protected\def\ekv@checkvalid#1#2%
  {%
    \ekv@ifempty{#1}%
      {%
        \def\ekv@tmp{}%
        \errmessage{expkv Error: empty set name not allowed}%
      }%
      {%
        \ekv@ifempty{#2}%
          {%
            \def\ekv@tmp{}%
            \errmessage{expkv Error: empty key name not allowed}%
          }%
          \@secondoftwo
      }%
    \@gobble
  }
%    \end{macrocode}
% \end{macro}
% \begin{macro}{\ekvifdefined,\ekvifdefinedNoVal}
% And provide user-level macros to test whether a key is defined.
%    \begin{macrocode}
\def\ekvifdefined#1#2{\ekv@ifdefined{\ekv@name{#1}{#2}}}
\def\ekvifdefinedNoVal#1#2{\ekv@ifdefined{\ekv@name{#1}{#2}N}}
%    \end{macrocode}
% \end{macro}
%
% \begin{macro}
%   {\ekvdef,\ekvdefNoVal,\ekvlet,\ekvletNoVal,\ekvletkv,\ekvletkvNoVal}
% Set up the key defining macros |\ekvdef| etc.
%    \begin{macrocode}
\protected\long\def\ekvdef#1#2#3%
  {%
    \ekv@checkvalid{#1}{#2}%
      {%
        \expandafter\def\csname\ekv@name{#1}{#2}\endcsname##1{#3}%
        \ekv@defset{#1}%
      }%
  }
\protected\long\def\ekvdefNoVal#1#2#3%
  {%
    \ekv@checkvalid{#1}{#2}%
      {%
        \expandafter\def\csname\ekv@name{#1}{#2}N\endcsname{#3}%
        \ekv@defset{#1}%
      }%
  }
\protected\def\ekvlet#1#2#3%
  {%
    \ekv@checkvalid{#1}{#2}%
      {%
        \expandafter\let\csname\ekv@name{#1}{#2}\endcsname#3%
        \ekv@defset{#1}%
      }%
  }
\protected\def\ekvletNoVal#1#2#3%
  {%
    \ekv@checkvalid{#1}{#2}%
      {%
        \expandafter\let\csname\ekv@name{#1}{#2}N\endcsname#3%
        \ekv@defset{#1}%
      }%
  }
\protected\def\ekvletkv#1#2#3#4%
  {%
    \ekv@checkvalid{#1}{#2}%
      {%
        \expandafter\let\csname\ekv@name{#1}{#2}\expandafter\endcsname
        \csname\ekv@name{#3}{#4}\endcsname
        \ekv@defset{#1}%
      }%
  }
\protected\def\ekvletkvNoVal#1#2#3#4%
  {%
    \ekv@checkvalid{#1}{#2}%
      {%
        \expandafter\let\csname\ekv@name{#1}{#2}N\expandafter\endcsname
        \csname\ekv@name{#3}{#4}N\endcsname
        \ekv@defset{#1}%
      }%
  }
%    \end{macrocode}
% \end{macro}
% \begin{macro}[internal]{\ekv@defset}
%   In order to enhance the speed the set name given to |\ekvset| will be turned
%   into a control sequence pretty early, so we have to define that control
%   sequence.
%    \begin{macrocode}
\protected\def\ekv@defset#1%
  {%
    \expandafter\edef\csname\ekv@undefined@set{#1}\endcsname##1%
      {\ekv@name@set{#1}\ekv@name@key{##1}}%
  }
%    \end{macrocode}
% \end{macro}
%
% \begin{macro}{\ekvset}
% Set up |\ekvset|, which should not be affected by active commas and equal
% signs. The equal signs are a bit harder to cope with and we'll do that later,
% but replacing the active commas with commas of category other can be done
% beforehand. That's why we define |\ekvset| here with a temporary meaning just
% to set up the things with two different category codes. |#1| will be a
% \texttt{,\textsubscript{13}} and |#2| will be a \texttt{=\textsubscript{13}}.
%    \begin{macrocode}
\def\ekvset#1#2{%
\endgroup
\long\def\ekvset##1##2%
  {%
    \expandafter\ekv@set\csname\ekv@undefined@set{##1}\endcsname
      \ekv@mark##2#1\ekv@stop#1{}%
  }
%    \end{macrocode}
% \end{macro}
% \begin{macro}[internal]{\ekv@set}
% |\ekv@set| will split the \kv\ list at active commas. Then it has to check
% whether there were unprotected other commas and resplit there.
%    \begin{macrocode}
\long\def\ekv@set##1##2#1%
  {%
%    \end{macrocode}
% Test whether we're at the end, if so invoke |\ekv@endset|,
%    \begin{macrocode}
    \ekv@gobbleto@markstop##2\ekv@endset\ekv@mark\ekv@stop
%    \end{macrocode}
% else go on with other commas,
%    \begin{macrocode}
    \ekv@set@other##1##2,\ekv@stop,%
%    \end{macrocode}
% and get the next active comma delimited \kv\ pair.
%    \begin{macrocode}
    \ekv@set##1\ekv@mark
  }
%    \end{macrocode}
% \end{macro}
% \begin{macro}[internal]{\ekv@endset}
% |\ekv@endset| is a hungry little macro. It will eat everything that remains
% of |\ekv@set| and unbrace the sneaked stuff.
%    \begin{macrocode}
\long\def\ekv@endset
    \ekv@mark\ekv@stop\ekv@set@other##1,\ekv@stop,\ekv@set##2\ekv@mark
    ##3%
  {##3}
%    \end{macrocode}
% \end{macro}
% \begin{macro}[internal]{\ekv@set@other}
% The macro |\ekv@set@other| is guaranteed to get only single \kv\ pairs.
% First we test whether we're done, if not split at equal signs. It is faster to
% first split at category 12 equal signs and only after that on actives.
% If there is no equal sign, we need to test whether we got a blank argument and
% if not this is a |NoVal| key.
%    \begin{macrocode}
\long\def\ekv@set@other##1##2,%
  {%
    \ekv@gobbleto@markstop##2\ekv@endset@other\ekv@mark\ekv@stop
    \ekv@ifhas@eq@other##2=\ekv@ifempty@B\ekv@ifempty@false
      \ekv@ifempty@A\ekv@ifempty@B\@firstoftwo
      {\ekv@set@eq@other##1##2\ekv@stop}%
      {%
        \ekv@ifhas@eq@active##2#2\ekv@ifempty@B\ekv@ifempty@false
          \ekv@ifempty@A\ekv@ifempty@B\@firstoftwo
          {\ekv@set@eq@active##1##2\ekv@stop}%
          {%
            \ekv@ifblank@##2\ekv@nil\ekv@ifempty@B\ekv@ifempty@true@F
              \ekv@ifempty@A\ekv@ifempty@B\@firstofone
              {\ekv@strip{##2}\ekv@set@key##1}%
          }%
      }%
    \ekv@set@other##1\ekv@mark%
  }
%    \end{macrocode}
% \end{macro}
% \begin{macro}[internal]{\ekv@set@eq@other}
% |\ekv@set@eq@other| might not be the correct break point, there might be an
% active equal sign in the currently parsed key-name. If so, we have to resplit.
% If the split is correct strip the key-name of outer spaces and braces and feed
% it to |\ekv@set@pair|.
%    \begin{macrocode}
\long\def\ekv@set@eq@other##1##2=%
  {%
    \ekv@ifhas@eq@active##2#2\ekv@ifempty@B\ekv@ifempty@false
      \ekv@ifempty@A\ekv@ifempty@B\@firstoftwo
      {\ekv@set@eq@active##1##2=}%
      {\ekv@strip{##2}\ekv@set@pair##1}%
  }
%    \end{macrocode}
% \end{macro}
% \begin{macro}[internal]{\ekv@set@eq@active}
% We need to handle the active equal signs.
%    \begin{macrocode}
\long\def\ekv@set@eq@active##1##2#2%
  {%
    \ekv@strip{##2}\ekv@set@pair##1%
  }
%    \end{macrocode}
% \end{macro}
% \begin{macro}[internal]
%   {\ekv@ifhas@eq@other,\ekv@ifhas@eq@active,\ekv@endset@other}
% And we have to set up the testing macros for our equal signs and
% |\ekv@endset@other|.
%    \begin{macrocode}
\long\def\ekv@ifhas@eq@other\ekv@mark##1={\ekv@ifempty@\ekv@ifempty@A}
\long\def\ekv@ifhas@eq@active\ekv@mark##1#2{\ekv@ifempty@\ekv@ifempty@A}
\long\def\ekv@endset@other
    \ekv@mark\ekv@stop
    \ekv@ifhas@eq@other##1=\ekv@ifempty@B\ekv@ifempty@false
    \ekv@ifempty@A\ekv@ifempty@B\@firstoftwo
    ##2%
    \ekv@set@other##3\ekv@mark
  {}
%    \end{macrocode}
% \end{macro}
% \begin{macro}{\ekvbreak,\ekvbreakPreSneak,\ekvbreakPostSneak}
% Provide macros that can completely stop the parsing of |\ekvset|, who knows
% what it'll be useful for.
%    \begin{macrocode}
\long\def\ekvbreak##1##2\ekv@stop#1##3{##1}
\long\def\ekvbreakPreSneak ##1##2\ekv@stop#1##3{##1##3}
\long\def\ekvbreakPostSneak##1##2\ekv@stop#1##3{##3##1}
%    \end{macrocode}
% \end{macro}
% \begin{macro}{\ekvsneak,\ekvsneakPre}
% One last thing we want to do for |\ekvset| is to provide macros that just
% smuggle stuff after |\ekvset|'s effects.
%    \begin{macrocode}
\long\def\ekvsneak##1##2\ekv@stop#1##3%
  {%
    ##2\ekv@stop#1{##3##1}%
  }
\long\def\ekvsneakPre##1##2\ekv@stop#1##3%
  {%
    ##2\ekv@stop#1{##1##3}%
  }
%    \end{macrocode}
% \end{macro}
%
% \begin{macro}{\ekvparse}
% Additionally to the |\ekvset| macro we also want to provide an |\ekvparse|
% macro, that has the same scope as |\keyval_parse:NNn| from \pkg{expl3}.
% This is pretty analogue to the |\ekvset| implementation, we just put an
% |\unexpanded| here and there instead of other macros to stop the |\expanded|
% on our output.
%    \begin{macrocode}
\long\def\ekvparse##1##2##3%
  {%
    \ekv@parse##1##2\ekv@mark##3#1\ekv@stop#1%
  }
%    \end{macrocode}
% \end{macro}
% \begin{macro}[internal]{\ekv@parse}
%    \begin{macrocode}
\long\def\ekv@parse##1##2##3#1%
  {%
    \ekv@gobbleto@markstop##3\ekv@endparse\ekv@mark\ekv@stop
    \ekv@parse@other##1##2##3,\ekv@stop,%
    \ekv@parse##1##2\ekv@mark
  }
%    \end{macrocode}
% \end{macro}
% \begin{macro}[internal]{\ekv@endparse}
%    \begin{macrocode}
\long\def\ekv@endparse
    \ekv@mark\ekv@stop\ekv@parse@other##1,\ekv@stop,\ekv@parse##2\ekv@mark
  {}
%    \end{macrocode}
% \end{macro}
% \begin{macro}[internal]{\ekv@parse@other}
%    \begin{macrocode}
\long\def\ekv@parse@other##1##2##3,%
  {%
    \ekv@gobbleto@markstop##3\ekv@endparse@other\ekv@mark\ekv@stop
    \ekv@ifhas@eq@other##3=\ekv@ifempty@B\ekv@ifempty@false
      \ekv@ifempty@A\ekv@ifempty@B\@firstoftwo
      {\ekv@parse@eq@other##3\ekv@stop##2}%
      {%
        \ekv@ifhas@eq@active##3#2\ekv@ifempty@B\ekv@ifempty@false
          \ekv@ifempty@A\ekv@ifempty@B\@firstoftwo
          {\ekv@parse@eq@active##3\ekv@stop##2}%
          {%
            \ekv@ifblank@##3\ekv@nil\ekv@ifempty@B\ekv@ifempty@true@F
              \ekv@ifempty@A\ekv@ifempty@B\@firstofone
              {\ekv@strip{##3}\ekv@parse@key##1}%
          }%
      }%
    \ekv@parse@other##1##2\ekv@mark
  }
%    \end{macrocode}
% \end{macro}
% \begin{macro}[internal]{\ekv@parse@eq@other}
%    \begin{macrocode}
\long\def\ekv@parse@eq@other##1=%
  {%
    \ekv@ifhas@eq@active##1#2\ekv@ifempty@B\ekv@ifempty@false
      \ekv@ifempty@A\ekv@ifempty@B\@firstoftwo
      {\ekv@parse@eq@active##1=}%
      {\ekv@strip{##1}\ekv@parse@pair\ekv@mark}%
  }
%    \end{macrocode}
% \end{macro}
% \begin{macro}[internal]{\ekv@parse@eq@active}
%    \begin{macrocode}
\long\def\ekv@parse@eq@active##1#2%
  {%
    \ekv@strip{##1}\ekv@parse@pair\ekv@mark
  }
%    \end{macrocode}
% \end{macro}
% \begin{macro}[internal]{\ekv@endparse@other}
%    \begin{macrocode}
\long\def\ekv@endparse@other
    \ekv@mark\ekv@stop
    \ekv@ifhas@eq@other##1=\ekv@ifempty@B\ekv@ifempty@false
    \ekv@ifempty@A\ekv@ifempty@B\@firstoftwo
    ##2%
    \ekv@parse@other##3\ekv@mark
  {}
%    \end{macrocode}
% \end{macro}
% \begin{macro}[internal]{\ekv@parse@pair,\ekv@parse@pair@}
%    \begin{macrocode}
\long\def\ekv@parse@pair##1##2\ekv@stop
  {%
    \ekv@strip{##2}\ekv@parse@pair@{##1}%
  }
\long\def\ekv@parse@pair@##1##2##3%
  {%
    \unexpanded{##3{##2}{##1}}%
  }
%    \end{macrocode}
% \end{macro}
% \begin{macro}[internal]{\ekv@parse@key}
%    \begin{macrocode}
\long\def\ekv@parse@key##1##2%
  {%
    \unexpanded{##2{##1}}%
  }
%    \end{macrocode}
% \end{macro}
%
% Finally really setting things up with |\ekvset|'s temporary meaning:
%    \begin{macrocode}
}
\begingroup
\catcode`\,=13
\catcode`\==13
\ekvset,=
%    \end{macrocode}
%
% \begin{macro}{\ekvchangeset}
% Provide a macro that is able to switch out the current \set\ in |\ekvset|.
% This operation is slow (by comparison, it should be slightly faster than
% |\ekvsneak|), but allows for something similar to \pkg{pgfkeys}'s
% \texttt{\meta{key}/.cd} mechanism. However this operation is more expensive
% than |/.cd| as we can't just redefine some token to reflect this, but have to
% switch out the set expandably, so this works similar to the |\ekvsneak| macros
% reading and reinserting the remainder of the \kv\ list.
%    \begin{macrocode}
\def\ekvchangeset#1%
  {%
    \expandafter\ekv@changeset\csname\ekv@undefined@set{#1}\endcsname\ekv@mark
  }
%    \end{macrocode}
% \end{macro}
% \begin{macro}[internal]{\ekv@changeset}
% This macro does the real change-out of |\ekvchangeset|. We introduced an
% |\ekv@mark| to not accidentally remove some braces which we have to remove
% again.
%    \begin{macrocode}
\long\def\ekv@changeset#1#2\ekv@set@other#3#4\ekv@set#5%
  {%
    \ekv@gobble@mark#2\ekv@set@other#1#4\ekv@set#1%
  }
%    \end{macrocode}
% \end{macro}
%
%
% \begin{macro}[internal]{\ekv@gobbleto@markstop}
% The |\ekv@gobbleto@markstop| can be used to test for |\ekv@stop| similar to
% our if-empty test, but instead of using tokens which are used nowhere else
% (|\ekv@ifempty@A| and |\ekv@ifempty@B|) it uses |\ekv@mark| and |\ekv@stop|.
%    \begin{macrocode}
\long\def\ekv@gobbleto@markstop#1\ekv@mark\ekv@stop{}
%    \end{macrocode}
% \end{macro}
%
% \begin{macro}[internal]{\ekv@set@pair,\ekv@set@pair@}
% |\ekv@set@pair| gets invoked with the space and brace stripped key-name as its
% first argument, the set-macro as the second argument, and following that is
% the key-value right delimited by an |\ekv@stop|.
%    \begin{macrocode}
\long\def\ekv@set@pair#1#2%
  {%
    \ekv@ifdefined@pair#2{#1}%
      {%
%    \end{macrocode}
% This branch will be executed if the key is not defined as an argument grabbing
% one. If so test whether there is a |NoVal| key of the same name or whether the
% key is unknown. Throw a meaningful error message and gobble the value.
%    \begin{macrocode}
        \ekv@ifdefined{#2{#1}N}%
          \ekv@err@noarg
          \ekv@err@unknown
          #2{#1}%
        \ekv@gobbleto@stop
      }%
  }
%    \end{macrocode}
% |\ekv@ifdefined@pair| will call |\ekv@set@pair@| if the key is correctly
% defined. This will then grab the value, strip outer spaces and braces from it
% and feed it to the key-macro. Afterwards |\ekv@set@other| will take control
% again.
%    \begin{macrocode}
\long\def\ekv@set@pair@#1#2\ekv@stop
  {%
    \ekv@strip{#2}#1%
  }
%    \end{macrocode}
% \end{macro}
%
% \begin{macro}[internal]{\ekv@set@key}
% Analogous to |\ekv@set@pair|, |\ekv@set@key| lets |\ekv@ifdefined@key| test
% whether a |NoVal| key is defined, else it'll throw a meaningful error message.
% Since we don't have to grab any value |\ekv@ifdefined@key| will invoke the
% key-macro and we're done here, |\ekv@set@other| will take over again.
%    \begin{macrocode}
\long\def\ekv@set@key#1#2%
  {%
    \ekv@ifdefined@key#2{#1}%
      {%
        \ekv@ifdefined{#2{#1}}%
          \ekv@err@reqval
          \ekv@err@unknown
          #2{#1}%
      }%
  }
%    \end{macrocode}
% \end{macro}
%
% \begin{macro}[internal]{\ekv@err,\ekv@err@}
% Since |\ekvset| is fully expandable as long as the code of the keys is (which
% is unlikely) we want to somehow throw expandable errors, in our case via
% undefined control sequences.
%    \begin{macrocode}
\begingroup
\edef\ekv@err
  {%
    \endgroup
    \unexpanded{\long\def\ekv@err}##1%
      {%
        \unexpanded{\expandafter\ekv@err@\@firstofone}%
        {\expandafter\noexpand\csname ! expkv Error:\endcsname ##1.}%
        \unexpanded{\ekv@stop}%
      }%
  }
\ekv@err
\def\ekv@err@{\expandafter\ekv@gobbleto@stop}
%    \end{macrocode}
% \end{macro}
% \begin{macro}[internal]
%   {
%     \ekv@err@common,\ekv@err@common@,
%     \ekv@err@unknown,\ekv@err@noarg,\ekv@err@reqval
%   }
% Now we can use |\ekv@err| to set up some error messages so that we can later
% use those instead of the full strings.
%    \begin{macrocode}
\long\def\ekv@err@common #1#2{\expandafter\ekv@err@common@\string#2{#1}}
\long\def\ekv@err@common@#1`#2' #3.#4#5{\ekv@err{#4 (`#5', set `#2')}}
\long\def\ekv@err@unknown#1#2{\ekv@err@common{unknown key}#1{#2}}
\long\def\ekv@err@noarg  #1#2{\ekv@err@common{value forbidden}#1{#2}}
\long\def\ekv@err@reqval #1#2{\ekv@err@common{value required}#1{#2}}
%    \end{macrocode}
% \end{macro}
%
% \begin{macro}[internal]{\ekv@strip,\ekv@strip@a,\ekv@strip@b,\ekv@strip@c}
% Finally we borrow some ideas of \pkg{expl3}'s \pkg{l3tl} to strip spaces
% from keys and values. This |\ekv@strip| also strips one level of outer braces
% \emph{after} stripping spaces, so an input of | {abc} | becomes |abc| after
% stripping. It should be used with |#1| prefixed by |\ekv@mark|. Also this
% implementation at most strips \emph{one} space from both sides.
%    \begin{macrocode}
\def\ekv@strip#1%
  {%
    \long\def\ekv@strip##1%
      {%
        \ekv@strip@a
          ##1%
          \ekv@nil
          \ekv@mark#1%
          #1\ekv@nil{}%
        \ekv@stop
      }%
    \long\def\ekv@strip@a##1\ekv@mark#1##2\ekv@nil##3%
      {%
        \ekv@strip@b##3##1##2\ekv@nil
      }%
    \long\def\ekv@strip@b##1#1\ekv@nil
      {%
        \ekv@strip@c##1\ekv@nil
      }%
    \long\def\ekv@strip@c\ekv@mark##1\ekv@nil##2\ekv@stop##3%
      {%
        ##3{##1}%
      }%
  }
\ekv@strip{ }
%    \end{macrocode}
% \end{macro}
%
% Now everything that's left is to reset the category code of |@|.
%    \begin{macrocode}
\catcode`\@=\ekv@tmp
%    \end{macrocode}
%
% \gobbledocstriptag
%</tex>
%^^A=<<
%
%^^A=<<
%
% \end{implementation}^^A=<<
%
% \clearpage
% \PrintIndex
%
\endinput
%
^^A vim: ft=tex fdm=marker fmr=>>=,=<<

\ProvidesPackage{expkv}%
  [\ekvDate\space v\ekvVersion\space an expandable key=val implementation]
%    \end{macrocode}
% \gobbledocstriptag
%</pkg>
%^^A=<<
%^^A main file >>=
% \subsection{The Generic Code}
% The rest of this implementation will be the generic code.
% \gobbledocstriptag
%<*tex>
%
% Check whether \eTeX\ is available -- \expkv\ requires \eTeX.
%    \begin{macrocode}
\begingroup\expandafter\expandafter\expandafter\endgroup
\expandafter\ifx\csname numexpr\endcsname\relax
  \errmessage{expkv requires e-TeX}
  \expandafter\endinput
\fi
%    \end{macrocode}
%
% We make sure that it's only input once:
%    \begin{macrocode}
\expandafter\ifx\csname ekvVersion\endcsname\relax
\else
  \expandafter\endinput
\fi
%    \end{macrocode}
% \begin{macro}{\ekvVersion,\ekvDate}
% We're on our first input, so lets store the version and date in a macro.
%    \begin{macrocode}
\def\ekvVersion{0.5a}
\def\ekvDate{2020-02-27}
%    \end{macrocode}
% \end{macro}
%
% If the \LaTeX\ format is loaded we want to be a good file and report back who
% we are, for this the package will have defined |\ekv@tmp| to use
% |\ProvidesFile|, else this will expand to a |\relax| and do no harm.
%    \begin{macrocode}
\csname ekv@tmp\endcsname
%    \end{macrocode}
%
% Store the category code of |@| to later be able to reset it and change it to
% 11 for now.
%    \begin{macrocode}
\expandafter\chardef\csname ekv@tmp\endcsname=\catcode`\@
\catcode`\@=11
%    \end{macrocode}
% |\ekv@tmp| might later be reused to gobble any prefixes which might be
% provided to |\ekvdef| and similar in case the names are invalid, we just
% temporarily use it here as means to store the current category code of |@| to
% restore it at the end of the file, we never care for the actual definition of
% it.
%
% \begin{macro}[internal]
%   {
%     \@gobble,\@firstofone,\@firstoftwo,\@secondoftwo,
%     \ekv@gobbleto@stop,\ekv@fi@secondoftwo,\ekv@gobble@mark
%   }
% Since branching tests are often more versatile than |\if...\else...\fi|
% constructs, we define helpers that are branching pretty fast. Also here are
% some other utility functions that just grab some tokens. The ones that are
% also contained in \LaTeX\ don't use the |ekv| prefix.
%    \begin{macrocode}
\long\def\@gobble#1{}
\long\def\@firstofone#1{#1}
\long\def\@firstoftwo#1#2{#1}
\long\def\@secondoftwo#1#2{#2}
\long\def\ekv@fi@secondoftwo\fi\@firstoftwo#1#2{\fi#2}
\long\def\ekv@gobbleto@stop#1\ekv@stop{}
\def\ekv@gobble@mark\ekv@mark{}
%    \end{macrocode}
% \end{macro}
% As you can see |\ekv@gobbleto@stop| uses a special marker |\ekv@stop|. The
% package will use three such markers, the one you've seen already, |\ekv@mark|
% and |\ekv@nil|. Contrarily to how for instance \pkg{expl3} does things, we
% don't define them, as we don't need them to have an actual meaning. This has
% the advantage that if they somehow get expanded -- which should never happen
% if things work out -- they'll throw an error directly.
%
% \begin{macro}[internal]
%   {
%     \ekv@ifempty,\ekv@ifempty@,\ekv@ifempty@true,\ekv@ifempty@false,
%     \ekv@ifempty@true@F
%   }
% We can test for a lot of things building on an if-empty test, so lets define a
% really fast one. Since some tests might have reversed logic (true if something
% is not empty) we also set up macros for the reversed branches.
%    \begin{macrocode}
\long\def\ekv@ifempty#1%
  {%
    \ekv@ifempty@\ekv@ifempty@A#1\ekv@ifempty@B\ekv@ifempty@true
      \ekv@ifempty@A\ekv@ifempty@B\@secondoftwo
  }
\long\def\ekv@ifempty@#1\ekv@ifempty@A\ekv@ifempty@B{}
\long\def\ekv@ifempty@true\ekv@ifempty@A\ekv@ifempty@B\@secondoftwo#1#2{#1}
\long\def\ekv@ifempty@false\ekv@ifempty@A\ekv@ifempty@B\@firstoftwo#1#2{#2}
\long\def\ekv@ifempty@true@F\ekv@ifempty@A\ekv@ifempty@B\@firstofone#1{}
%    \end{macrocode}
% \end{macro}
%
% \begin{macro}[internal]{\ekv@ifblank@}
% The obvious test that can be based on an if-empty is if-blank, meaning a test
% checking whether the argument is empty or consists only of spaces. Our version
% here will be tweaked a bit, as we want to check this, but with one leading
% |\ekv@mark| token that is to be ignored.
%    \begin{macrocode}
\long\def\ekv@ifblank@\ekv@mark#1{\ekv@ifempty@\ekv@ifempty@A}
%    \end{macrocode}
% \end{macro}
%
% \begin{macro}[internal]{\ekv@ifdefined}
% We'll need to check whether something is defined quite frequently, so why not
% define a macro that does this. The following test is expandable, slower than
% the typical expandable test for undefined control sequences, but faster for
% defined ones. Since we want to be as fast as possible for correct input, this
% is to be preferred.
%    \begin{macrocode}
\def\ekv@ifdefined#1%
  {%
    \expandafter
    \ifx\csname\ifcsname #1\endcsname #1\else relax\fi\endcsname\relax
      \ekv@fi@secondoftwo
    \fi
    \@firstoftwo
  }
%    \end{macrocode}
% \end{macro}
  % \begin{macro}[internal]
%   {
%     \ekv@ifdefined@pair,\ekv@ifdefined@pair@,\ekv@ifdefined@key,
%     \ekv@ifdefined@key@
%   }
% Since we can save some time if we only have to create the control sequence
% once when we know beforehand how we want to use it, we build some other macros
% for those cases (which we'll have quite often, once per key usage).
%    \begin{macrocode}
\def\ekv@ifdefined@pair#1#2%
  {%
    \expandafter\ekv@ifdefined@pair@
      \csname
        \ifcsname #1{#2}\endcsname
          #1{#2}%
        \else
          relax%
        \fi
      \endcsname
  }
\def\ekv@ifdefined@pair@#1%
  {%
    \ifx#1\relax
      \ekv@fi@secondoftwo
    \fi
    \@firstoftwo
    {\ekv@set@pair@#1\ekv@mark}%
  }
\def\ekv@ifdefined@key#1#2%
  {%
    \expandafter\ekv@ifdefined@key@
      \csname
        \ifcsname #1{#2}N\endcsname
          #1{#2}N%
        \else
          relax%
        \fi
      \endcsname
  }
\def\ekv@ifdefined@key@#1%
  {%
    \ifx#1\relax
      \ekv@fi@secondoftwo
    \fi
    \@firstoftwo#1%
  }
%    \end{macrocode}
% \end{macro}
%
% \begin{macro}{\ekv@name,\ekv@name@set,\ekv@name@key}
% The keys will all follow the same naming scheme, so we define it here.
%    \begin{macrocode}
\def\ekv@name#1#2{\ekv@name@set{#1}\ekv@name@key{#2}}
\def\ekv@name@set#1{ekv#1(}
\def\ekv@name@key#1{#1)}
%    \end{macrocode}
% \end{macro}
% \begin{macro}[internal]{\ekv@undefined@set}
% We can misuse the macro name we use to expandably store the set-name in a
% single token -- since this increases performance drastically, especially for
% long set-names -- to throw a more meaningful error message in case a set isn't
% defined. The name of |\ekv@undefined@set| is a little bit misleading, as it is
% called in either case inside of |\csname|, but the result will be a control
% sequence with meaning |\relax| if the set is undefined, hence will break the
% |\csname| building the key-macro which will throw the error message.
%    \begin{macrocode}
\def\ekv@undefined@set#1{! expkv Error: Set `#1' undefined.}
%    \end{macrocode}
% \end{macro}
% \begin{macro}[internal]{\ekv@checkvalid}
% We place some restrictions on the allowed names, though, namely sets and
% keys are not allowed to be empty -- blanks are fine (meaning \mbox{set-
% or} key-names consisting of spaces).
%    \begin{macrocode}
\protected\def\ekv@checkvalid#1#2%
  {%
    \ekv@ifempty{#1}%
      {%
        \def\ekv@tmp{}%
        \errmessage{expkv Error: empty set name not allowed}%
      }%
      {%
        \ekv@ifempty{#2}%
          {%
            \def\ekv@tmp{}%
            \errmessage{expkv Error: empty key name not allowed}%
          }%
          \@secondoftwo
      }%
    \@gobble
  }
%    \end{macrocode}
% \end{macro}
% \begin{macro}{\ekvifdefined,\ekvifdefinedNoVal}
% And provide user-level macros to test whether a key is defined.
%    \begin{macrocode}
\def\ekvifdefined#1#2{\ekv@ifdefined{\ekv@name{#1}{#2}}}
\def\ekvifdefinedNoVal#1#2{\ekv@ifdefined{\ekv@name{#1}{#2}N}}
%    \end{macrocode}
% \end{macro}
%
% \begin{macro}
%   {\ekvdef,\ekvdefNoVal,\ekvlet,\ekvletNoVal,\ekvletkv,\ekvletkvNoVal}
% Set up the key defining macros |\ekvdef| etc.
%    \begin{macrocode}
\protected\long\def\ekvdef#1#2#3%
  {%
    \ekv@checkvalid{#1}{#2}%
      {%
        \expandafter\def\csname\ekv@name{#1}{#2}\endcsname##1{#3}%
        \ekv@defset{#1}%
      }%
  }
\protected\long\def\ekvdefNoVal#1#2#3%
  {%
    \ekv@checkvalid{#1}{#2}%
      {%
        \expandafter\def\csname\ekv@name{#1}{#2}N\endcsname{#3}%
        \ekv@defset{#1}%
      }%
  }
\protected\def\ekvlet#1#2#3%
  {%
    \ekv@checkvalid{#1}{#2}%
      {%
        \expandafter\let\csname\ekv@name{#1}{#2}\endcsname#3%
        \ekv@defset{#1}%
      }%
  }
\protected\def\ekvletNoVal#1#2#3%
  {%
    \ekv@checkvalid{#1}{#2}%
      {%
        \expandafter\let\csname\ekv@name{#1}{#2}N\endcsname#3%
        \ekv@defset{#1}%
      }%
  }
\protected\def\ekvletkv#1#2#3#4%
  {%
    \ekv@checkvalid{#1}{#2}%
      {%
        \expandafter\let\csname\ekv@name{#1}{#2}\expandafter\endcsname
        \csname\ekv@name{#3}{#4}\endcsname
        \ekv@defset{#1}%
      }%
  }
\protected\def\ekvletkvNoVal#1#2#3#4%
  {%
    \ekv@checkvalid{#1}{#2}%
      {%
        \expandafter\let\csname\ekv@name{#1}{#2}N\expandafter\endcsname
        \csname\ekv@name{#3}{#4}N\endcsname
        \ekv@defset{#1}%
      }%
  }
%    \end{macrocode}
% \end{macro}
% \begin{macro}[internal]{\ekv@defset}
%   In order to enhance the speed the set name given to |\ekvset| will be turned
%   into a control sequence pretty early, so we have to define that control
%   sequence.
%    \begin{macrocode}
\protected\def\ekv@defset#1%
  {%
    \expandafter\edef\csname\ekv@undefined@set{#1}\endcsname##1%
      {\ekv@name@set{#1}\ekv@name@key{##1}}%
  }
%    \end{macrocode}
% \end{macro}
%
% \begin{macro}{\ekvset}
% Set up |\ekvset|, which should not be affected by active commas and equal
% signs. The equal signs are a bit harder to cope with and we'll do that later,
% but replacing the active commas with commas of category other can be done
% beforehand. That's why we define |\ekvset| here with a temporary meaning just
% to set up the things with two different category codes. |#1| will be a
% \texttt{,\textsubscript{13}} and |#2| will be a \texttt{=\textsubscript{13}}.
%    \begin{macrocode}
\def\ekvset#1#2{%
\endgroup
\long\def\ekvset##1##2%
  {%
    \expandafter\ekv@set\csname\ekv@undefined@set{##1}\endcsname
      \ekv@mark##2#1\ekv@stop#1{}%
  }
%    \end{macrocode}
% \end{macro}
% \begin{macro}[internal]{\ekv@set}
% |\ekv@set| will split the \kv\ list at active commas. Then it has to check
% whether there were unprotected other commas and resplit there.
%    \begin{macrocode}
\long\def\ekv@set##1##2#1%
  {%
%    \end{macrocode}
% Test whether we're at the end, if so invoke |\ekv@endset|,
%    \begin{macrocode}
    \ekv@gobbleto@markstop##2\ekv@endset\ekv@mark\ekv@stop
%    \end{macrocode}
% else go on with other commas,
%    \begin{macrocode}
    \ekv@set@other##1##2,\ekv@stop,%
%    \end{macrocode}
% and get the next active comma delimited \kv\ pair.
%    \begin{macrocode}
    \ekv@set##1\ekv@mark
  }
%    \end{macrocode}
% \end{macro}
% \begin{macro}[internal]{\ekv@endset}
% |\ekv@endset| is a hungry little macro. It will eat everything that remains
% of |\ekv@set| and unbrace the sneaked stuff.
%    \begin{macrocode}
\long\def\ekv@endset
    \ekv@mark\ekv@stop\ekv@set@other##1,\ekv@stop,\ekv@set##2\ekv@mark
    ##3%
  {##3}
%    \end{macrocode}
% \end{macro}
% \begin{macro}[internal]{\ekv@set@other}
% The macro |\ekv@set@other| is guaranteed to get only single \kv\ pairs.
% First we test whether we're done, if not split at equal signs. It is faster to
% first split at category 12 equal signs and only after that on actives.
% If there is no equal sign, we need to test whether we got a blank argument and
% if not this is a |NoVal| key.
%    \begin{macrocode}
\long\def\ekv@set@other##1##2,%
  {%
    \ekv@gobbleto@markstop##2\ekv@endset@other\ekv@mark\ekv@stop
    \ekv@ifhas@eq@other##2=\ekv@ifempty@B\ekv@ifempty@false
      \ekv@ifempty@A\ekv@ifempty@B\@firstoftwo
      {\ekv@set@eq@other##1##2\ekv@stop}%
      {%
        \ekv@ifhas@eq@active##2#2\ekv@ifempty@B\ekv@ifempty@false
          \ekv@ifempty@A\ekv@ifempty@B\@firstoftwo
          {\ekv@set@eq@active##1##2\ekv@stop}%
          {%
            \ekv@ifblank@##2\ekv@nil\ekv@ifempty@B\ekv@ifempty@true@F
              \ekv@ifempty@A\ekv@ifempty@B\@firstofone
              {\ekv@strip{##2}\ekv@set@key##1}%
          }%
      }%
    \ekv@set@other##1\ekv@mark%
  }
%    \end{macrocode}
% \end{macro}
% \begin{macro}[internal]{\ekv@set@eq@other}
% |\ekv@set@eq@other| might not be the correct break point, there might be an
% active equal sign in the currently parsed key-name. If so, we have to resplit.
% If the split is correct strip the key-name of outer spaces and braces and feed
% it to |\ekv@set@pair|.
%    \begin{macrocode}
\long\def\ekv@set@eq@other##1##2=%
  {%
    \ekv@ifhas@eq@active##2#2\ekv@ifempty@B\ekv@ifempty@false
      \ekv@ifempty@A\ekv@ifempty@B\@firstoftwo
      {\ekv@set@eq@active##1##2=}%
      {\ekv@strip{##2}\ekv@set@pair##1}%
  }
%    \end{macrocode}
% \end{macro}
% \begin{macro}[internal]{\ekv@set@eq@active}
% We need to handle the active equal signs.
%    \begin{macrocode}
\long\def\ekv@set@eq@active##1##2#2%
  {%
    \ekv@strip{##2}\ekv@set@pair##1%
  }
%    \end{macrocode}
% \end{macro}
% \begin{macro}[internal]
%   {\ekv@ifhas@eq@other,\ekv@ifhas@eq@active,\ekv@endset@other}
% And we have to set up the testing macros for our equal signs and
% |\ekv@endset@other|.
%    \begin{macrocode}
\long\def\ekv@ifhas@eq@other\ekv@mark##1={\ekv@ifempty@\ekv@ifempty@A}
\long\def\ekv@ifhas@eq@active\ekv@mark##1#2{\ekv@ifempty@\ekv@ifempty@A}
\long\def\ekv@endset@other
    \ekv@mark\ekv@stop
    \ekv@ifhas@eq@other##1=\ekv@ifempty@B\ekv@ifempty@false
    \ekv@ifempty@A\ekv@ifempty@B\@firstoftwo
    ##2%
    \ekv@set@other##3\ekv@mark
  {}
%    \end{macrocode}
% \end{macro}
% \begin{macro}{\ekvbreak,\ekvbreakPreSneak,\ekvbreakPostSneak}
% Provide macros that can completely stop the parsing of |\ekvset|, who knows
% what it'll be useful for.
%    \begin{macrocode}
\long\def\ekvbreak##1##2\ekv@stop#1##3{##1}
\long\def\ekvbreakPreSneak ##1##2\ekv@stop#1##3{##1##3}
\long\def\ekvbreakPostSneak##1##2\ekv@stop#1##3{##3##1}
%    \end{macrocode}
% \end{macro}
% \begin{macro}{\ekvsneak,\ekvsneakPre}
% One last thing we want to do for |\ekvset| is to provide macros that just
% smuggle stuff after |\ekvset|'s effects.
%    \begin{macrocode}
\long\def\ekvsneak##1##2\ekv@stop#1##3%
  {%
    ##2\ekv@stop#1{##3##1}%
  }
\long\def\ekvsneakPre##1##2\ekv@stop#1##3%
  {%
    ##2\ekv@stop#1{##1##3}%
  }
%    \end{macrocode}
% \end{macro}
%
% \begin{macro}{\ekvparse}
% Additionally to the |\ekvset| macro we also want to provide an |\ekvparse|
% macro, that has the same scope as |\keyval_parse:NNn| from \pkg{expl3}.
% This is pretty analogue to the |\ekvset| implementation, we just put an
% |\unexpanded| here and there instead of other macros to stop the |\expanded|
% on our output.
%    \begin{macrocode}
\long\def\ekvparse##1##2##3%
  {%
    \ekv@parse##1##2\ekv@mark##3#1\ekv@stop#1%
  }
%    \end{macrocode}
% \end{macro}
% \begin{macro}[internal]{\ekv@parse}
%    \begin{macrocode}
\long\def\ekv@parse##1##2##3#1%
  {%
    \ekv@gobbleto@markstop##3\ekv@endparse\ekv@mark\ekv@stop
    \ekv@parse@other##1##2##3,\ekv@stop,%
    \ekv@parse##1##2\ekv@mark
  }
%    \end{macrocode}
% \end{macro}
% \begin{macro}[internal]{\ekv@endparse}
%    \begin{macrocode}
\long\def\ekv@endparse
    \ekv@mark\ekv@stop\ekv@parse@other##1,\ekv@stop,\ekv@parse##2\ekv@mark
  {}
%    \end{macrocode}
% \end{macro}
% \begin{macro}[internal]{\ekv@parse@other}
%    \begin{macrocode}
\long\def\ekv@parse@other##1##2##3,%
  {%
    \ekv@gobbleto@markstop##3\ekv@endparse@other\ekv@mark\ekv@stop
    \ekv@ifhas@eq@other##3=\ekv@ifempty@B\ekv@ifempty@false
      \ekv@ifempty@A\ekv@ifempty@B\@firstoftwo
      {\ekv@parse@eq@other##3\ekv@stop##2}%
      {%
        \ekv@ifhas@eq@active##3#2\ekv@ifempty@B\ekv@ifempty@false
          \ekv@ifempty@A\ekv@ifempty@B\@firstoftwo
          {\ekv@parse@eq@active##3\ekv@stop##2}%
          {%
            \ekv@ifblank@##3\ekv@nil\ekv@ifempty@B\ekv@ifempty@true@F
              \ekv@ifempty@A\ekv@ifempty@B\@firstofone
              {\ekv@strip{##3}\ekv@parse@key##1}%
          }%
      }%
    \ekv@parse@other##1##2\ekv@mark
  }
%    \end{macrocode}
% \end{macro}
% \begin{macro}[internal]{\ekv@parse@eq@other}
%    \begin{macrocode}
\long\def\ekv@parse@eq@other##1=%
  {%
    \ekv@ifhas@eq@active##1#2\ekv@ifempty@B\ekv@ifempty@false
      \ekv@ifempty@A\ekv@ifempty@B\@firstoftwo
      {\ekv@parse@eq@active##1=}%
      {\ekv@strip{##1}\ekv@parse@pair\ekv@mark}%
  }
%    \end{macrocode}
% \end{macro}
% \begin{macro}[internal]{\ekv@parse@eq@active}
%    \begin{macrocode}
\long\def\ekv@parse@eq@active##1#2%
  {%
    \ekv@strip{##1}\ekv@parse@pair\ekv@mark
  }
%    \end{macrocode}
% \end{macro}
% \begin{macro}[internal]{\ekv@endparse@other}
%    \begin{macrocode}
\long\def\ekv@endparse@other
    \ekv@mark\ekv@stop
    \ekv@ifhas@eq@other##1=\ekv@ifempty@B\ekv@ifempty@false
    \ekv@ifempty@A\ekv@ifempty@B\@firstoftwo
    ##2%
    \ekv@parse@other##3\ekv@mark
  {}
%    \end{macrocode}
% \end{macro}
% \begin{macro}[internal]{\ekv@parse@pair,\ekv@parse@pair@}
%    \begin{macrocode}
\long\def\ekv@parse@pair##1##2\ekv@stop
  {%
    \ekv@strip{##2}\ekv@parse@pair@{##1}%
  }
\long\def\ekv@parse@pair@##1##2##3%
  {%
    \unexpanded{##3{##2}{##1}}%
  }
%    \end{macrocode}
% \end{macro}
% \begin{macro}[internal]{\ekv@parse@key}
%    \begin{macrocode}
\long\def\ekv@parse@key##1##2%
  {%
    \unexpanded{##2{##1}}%
  }
%    \end{macrocode}
% \end{macro}
%
% Finally really setting things up with |\ekvset|'s temporary meaning:
%    \begin{macrocode}
}
\begingroup
\catcode`\,=13
\catcode`\==13
\ekvset,=
%    \end{macrocode}
%
% \begin{macro}{\ekvchangeset}
% Provide a macro that is able to switch out the current \set\ in |\ekvset|.
% This operation is slow (by comparison, it should be slightly faster than
% |\ekvsneak|), but allows for something similar to \pkg{pgfkeys}'s
% \texttt{\meta{key}/.cd} mechanism. However this operation is more expensive
% than |/.cd| as we can't just redefine some token to reflect this, but have to
% switch out the set expandably, so this works similar to the |\ekvsneak| macros
% reading and reinserting the remainder of the \kv\ list.
%    \begin{macrocode}
\def\ekvchangeset#1%
  {%
    \expandafter\ekv@changeset\csname\ekv@undefined@set{#1}\endcsname\ekv@mark
  }
%    \end{macrocode}
% \end{macro}
% \begin{macro}[internal]{\ekv@changeset}
% This macro does the real change-out of |\ekvchangeset|. We introduced an
% |\ekv@mark| to not accidentally remove some braces which we have to remove
% again.
%    \begin{macrocode}
\long\def\ekv@changeset#1#2\ekv@set@other#3#4\ekv@set#5%
  {%
    \ekv@gobble@mark#2\ekv@set@other#1#4\ekv@set#1%
  }
%    \end{macrocode}
% \end{macro}
%
%
% \begin{macro}[internal]{\ekv@gobbleto@markstop}
% The |\ekv@gobbleto@markstop| can be used to test for |\ekv@stop| similar to
% our if-empty test, but instead of using tokens which are used nowhere else
% (|\ekv@ifempty@A| and |\ekv@ifempty@B|) it uses |\ekv@mark| and |\ekv@stop|.
%    \begin{macrocode}
\long\def\ekv@gobbleto@markstop#1\ekv@mark\ekv@stop{}
%    \end{macrocode}
% \end{macro}
%
% \begin{macro}[internal]{\ekv@set@pair,\ekv@set@pair@}
% |\ekv@set@pair| gets invoked with the space and brace stripped key-name as its
% first argument, the set-macro as the second argument, and following that is
% the key-value right delimited by an |\ekv@stop|.
%    \begin{macrocode}
\long\def\ekv@set@pair#1#2%
  {%
    \ekv@ifdefined@pair#2{#1}%
      {%
%    \end{macrocode}
% This branch will be executed if the key is not defined as an argument grabbing
% one. If so test whether there is a |NoVal| key of the same name or whether the
% key is unknown. Throw a meaningful error message and gobble the value.
%    \begin{macrocode}
        \ekv@ifdefined{#2{#1}N}%
          \ekv@err@noarg
          \ekv@err@unknown
          #2{#1}%
        \ekv@gobbleto@stop
      }%
  }
%    \end{macrocode}
% |\ekv@ifdefined@pair| will call |\ekv@set@pair@| if the key is correctly
% defined. This will then grab the value, strip outer spaces and braces from it
% and feed it to the key-macro. Afterwards |\ekv@set@other| will take control
% again.
%    \begin{macrocode}
\long\def\ekv@set@pair@#1#2\ekv@stop
  {%
    \ekv@strip{#2}#1%
  }
%    \end{macrocode}
% \end{macro}
%
% \begin{macro}[internal]{\ekv@set@key}
% Analogous to |\ekv@set@pair|, |\ekv@set@key| lets |\ekv@ifdefined@key| test
% whether a |NoVal| key is defined, else it'll throw a meaningful error message.
% Since we don't have to grab any value |\ekv@ifdefined@key| will invoke the
% key-macro and we're done here, |\ekv@set@other| will take over again.
%    \begin{macrocode}
\long\def\ekv@set@key#1#2%
  {%
    \ekv@ifdefined@key#2{#1}%
      {%
        \ekv@ifdefined{#2{#1}}%
          \ekv@err@reqval
          \ekv@err@unknown
          #2{#1}%
      }%
  }
%    \end{macrocode}
% \end{macro}
%
% \begin{macro}[internal]{\ekv@err,\ekv@err@}
% Since |\ekvset| is fully expandable as long as the code of the keys is (which
% is unlikely) we want to somehow throw expandable errors, in our case via
% undefined control sequences.
%    \begin{macrocode}
\begingroup
\edef\ekv@err
  {%
    \endgroup
    \unexpanded{\long\def\ekv@err}##1%
      {%
        \unexpanded{\expandafter\ekv@err@\@firstofone}%
        {\expandafter\noexpand\csname ! expkv Error:\endcsname ##1.}%
        \unexpanded{\ekv@stop}%
      }%
  }
\ekv@err
\def\ekv@err@{\expandafter\ekv@gobbleto@stop}
%    \end{macrocode}
% \end{macro}
% \begin{macro}[internal]
%   {
%     \ekv@err@common,\ekv@err@common@,
%     \ekv@err@unknown,\ekv@err@noarg,\ekv@err@reqval
%   }
% Now we can use |\ekv@err| to set up some error messages so that we can later
% use those instead of the full strings.
%    \begin{macrocode}
\long\def\ekv@err@common #1#2{\expandafter\ekv@err@common@\string#2{#1}}
\long\def\ekv@err@common@#1`#2' #3.#4#5{\ekv@err{#4 (`#5', set `#2')}}
\long\def\ekv@err@unknown#1#2{\ekv@err@common{unknown key}#1{#2}}
\long\def\ekv@err@noarg  #1#2{\ekv@err@common{value forbidden}#1{#2}}
\long\def\ekv@err@reqval #1#2{\ekv@err@common{value required}#1{#2}}
%    \end{macrocode}
% \end{macro}
%
% \begin{macro}[internal]{\ekv@strip,\ekv@strip@a,\ekv@strip@b,\ekv@strip@c}
% Finally we borrow some ideas of \pkg{expl3}'s \pkg{l3tl} to strip spaces
% from keys and values. This |\ekv@strip| also strips one level of outer braces
% \emph{after} stripping spaces, so an input of | {abc} | becomes |abc| after
% stripping. It should be used with |#1| prefixed by |\ekv@mark|. Also this
% implementation at most strips \emph{one} space from both sides.
%    \begin{macrocode}
\def\ekv@strip#1%
  {%
    \long\def\ekv@strip##1%
      {%
        \ekv@strip@a
          ##1%
          \ekv@nil
          \ekv@mark#1%
          #1\ekv@nil{}%
        \ekv@stop
      }%
    \long\def\ekv@strip@a##1\ekv@mark#1##2\ekv@nil##3%
      {%
        \ekv@strip@b##3##1##2\ekv@nil
      }%
    \long\def\ekv@strip@b##1#1\ekv@nil
      {%
        \ekv@strip@c##1\ekv@nil
      }%
    \long\def\ekv@strip@c\ekv@mark##1\ekv@nil##2\ekv@stop##3%
      {%
        ##3{##1}%
      }%
  }
\ekv@strip{ }
%    \end{macrocode}
% \end{macro}
%
% Now everything that's left is to reset the category code of |@|.
%    \begin{macrocode}
\catcode`\@=\ekv@tmp
%    \end{macrocode}
%
% \gobbledocstriptag
%</tex>
%^^A=<<
%
%^^A=<<
%
% \end{implementation}^^A=<<
%
% \clearpage
% \PrintIndex
%
\endinput
%
^^A vim: ft=tex fdm=marker fmr=>>=,=<<
}{}
\ProvidesFile{expkv.dtx}
  [\csname ekvDate\endcsname\ an expandable key=val implementation]
\PassOptionsToPackage{full}{textcomp}
\documentclass{l3doc}
\RequirePackage[oldstylenums,nott]{kpfonts}
\input{glyphtounicode}
\pdfgentounicode=1
\RequirePackage{xfp} ^^A required for an example
\RequirePackage{listings}
\RequirePackage{booktabs}
\RequirePackage{array}
\RequirePackage{collcell}
\RequirePackage{siunitx}
\RequirePackage{xcolor}
\RequirePackage{caption}
\RequirePackage{microtype}
\RequirePackage{accsupp}
\lstset
  {
    ,flexiblecolumns=false
    ,basewidth=.53em
    ,gobble=2
    ,basicstyle=\fontfamily{jkp}\itshape
    ,morekeywords=^^A
      {^^A
        \ekvdef,\ekvdefNoVal,\ekvlet,\ekvletNoVal,\ekvletkv,\ekvletkvNoVal,^^A
        \ekvset,\ekvparse,\ekvVersion,\ekvDate,\ekvifdefined,^^A
        \ekvifdefinedNoVal,\ekvbreak,\ekvbreakPreSneak,\ekvbreakPostSneak,^^A
        \ekvsneak,\ekvsneakPre,\ekvchangeset
      }
    ,morecomment=[l]\%
    ,commentstyle=\color[gray]{0.4}
    ,literate={\{}{{\CodeSymbol\{}}{1}
              {\}}{{\CodeSymbol\}}}{1}
    ^^A,literate=*{<key>}{\key}{4}{<set>}{\set}{4}
  }
\newcommand*\CodeSymbol[1]{\textbf{#1}}
\RequirePackage{randtext}
\let\metaORIG\meta
\protected\def\meta #1{\texttt{\metaORIG{#1}}}
\renewcommand*\thefootnote{\fnsymbol{footnote}}
\definecolor{expkvred}{HTML}{9F393D}
\colorlet{expkvgrey}{black!75}
\makeatletter
\newcommand*\expkv
  {^^A
    \texorpdfstring
      {^^A
        \mbox
          {^^A
            \BeginAccSupp{ActualText=expkv}^^A
            \href{https://github.com/Skillmon/tex_expkv}
              {^^A
                \rmfamily
                \bfseries
                {\color{expkvgrey}e\kern-.05em x\kern-.05em}^^A
                \lower.493ex
                  \hbox{{\color{expkvgrey}P}\kern-.1em{\color{expkvred}k}}^^A
                \kern-.18em{\color{expkvred}v}^^A
              }^^A
            \EndAccSupp{}^^A
          }^^A
      }
      {expkv}^^A
  }
\newcommand*\expkvd
  {^^A
    \texorpdfstring
      {^^A
        \mbox
          {^^A
            \BeginAccSupp{ActualText=expkv-def}^^A
            \rmfamily
            \bfseries
            {\color{expkvgrey}e\kern-.05em x\kern-.05em}^^A
            \lower.493ex
              \hbox{{\color{expkvgrey}P}\kern-.1em{\color{expkvred}k}}^^A
            \kern-.18em{\color{expkvred}v}^^A
            {\color{expkvgrey}^^A
              \kern.05em\rule[-.1ex]{.08em}{1.2ex}\kern.05em\textsc{def}^^A
            }^^A
            \EndAccSupp{}^^A
          }^^A
      }
      {expkv-def}^^A
  }
\newcommand\kv{\meta{key}=\meta{value}}
\newcommand\key{\meta{key}}
\newcommand\val{\meta{value}}
\newcommand\set{\meta{set}}
\hypersetup{linkcolor=red!80!black,urlcolor=purple!80!black}
\DoNotIndex{\def,\edef,\,,\=,\begingroup,\catcode,\chardef,\csname,\endcsname}
\DoNotIndex{\endgroup,\endinput,\errmessage,\expandafter,\input,\let,\long}
\DoNotIndex{\noexpand,\protected,\ProvidesFile,\ProvidesPackage,\relax,\space}
\DoNotIndex{\@,\unexpanded,\string,\expanded}
\DoNotIndex{\ifcsname}
\DoNotIndex{\ifx}
\DoNotIndex{\else}
\DoNotIndex{\fi}
\@gobble\fi ^^A ignoring \ifx and \ifcsname, but only one \fi
\@ifdefinable\gobbledocstriptag{\def\gobbledocstriptag#1>{}}
\newcommand*\pmso[1] ^^A poor man's strike out
  {^^A
    \leavevmode
    \begingroup
      \sbox0{#1}^^A
      \rlap{\vrule height .6ex depth -.5ex width \wd0\relax}^^A
      \usebox0\relax
    \endgroup
  }
\makeatother
\begin{document}
  \title
    {^^A
      \texorpdfstring
        {^^A
          \huge\expkv
          \\[\medskipamount]
          \Large an expandable \kv\ implementation^^A
        }{expkv - an expandable <key>=<value> implementation}^^A
    }
  \date{\ekvDate\space v\ekvVersion}
  \author{Jonathan P. Spratte\thanks{\protect\randomize{jspratte@yahoo.de}}}
  \DocInput{expkv.dtx}
\end{document}
%</driver>^^A=<<
% \fi
%
% \maketitle
% \renewcommand*\thefootnote{\arabic{footnote}}
%
% \begin{abstract}
% \noindent\parfillskip=0pt
% \expkv\ provides a small interface for \kv\ parsing. The parsing macro is
% fully expandable, the \meta{code} of your keys might be not. \expkv\ is pretty
% fast, but not the fastest available \kv\ solution (\pkg{keyval} for instance
% is thrice as fast, but not expandable and it might strip braces it shouldn't
% have stripped).
% \end{abstract}
%
% \tableofcontents
%
% \begin{documentation}^^A>>=
%
% \section{Documentation}
%
% \expkv\ provides an expandable \kv\ parser. The \kv\ pairs should be given as
% a comma separated list and the separator between a \key\ and the associated
% \val\ should be an equal sign. Both, the commas and the equal signs, might be
% of category 12 (other) or 13 (active). To support this is necessary as for
% example \pkg{babel} turns characters active for some languages, for instance
% the equal sign is turned active for Turkish.
%
% \expkv\ is usable as generic code or as a \LaTeX\ package. To use it, just
% use one of:
% \begin{lstlisting}
% \usepackage{expkv} % LaTeX
% \input expkv       % plainTeX
% \end{lstlisting}
% The \LaTeX\ package doesn't do more than \file{expkv.tex}, except calling
% |\ProvidesPackage| and setting things up such that \file{expkv.tex} will use
% |\ProvidesFile|.
%
% \subsection{Setting up Keys}\label{sec:define}
% \expkv\ provides a rather simple approach to setting up keys, similar to
% \pkg{keyval}. However there is an auxiliary package named
% \href{https://github.com/Skillmon/tex_expkv-def}{\expkvd} (the files use
% \file{expkv-def} instead because \string| isn't allowed in package names by
% CTAN) which provides a more sophisticated interface, similar to well
% established packages like \pkg{pgfkeys} or \pkg{l3keys}.
%
% Keys in \expkv\ (as in almost all other \kv\ implementations) belong to a
% \emph{set} such that different sets can contain keys of the same name. Unlike
% many other implementations \expkv\ doesn't provide means to set a default
% value, instead we have keys that take values and keys that don't (the latter
% are called |NoVal| keys by \expkv), but both can have the same name (on the
% user level).
%
% The following macros are available to define new keys. Those macros containing
% ``|def|'' in their name can be prefixed by anything allowed to prefix |\def|,
% prefixes allowed for |\let| can prefix those with ``|let|'' in their name,
% accordingly. Neither \set\ nor \key\ are allowed to be empty for new keys and
% must not contain a |\par| or tokens that expand to it -- they must be legal
% inside of |\csname ...\endcsname|.
%
% \begin{function}{\ekvdef}
%   \begin{syntax}
%     \cs{ekvdef}\marg{set}\marg{key}\marg{code}
%   \end{syntax}
%   Defines a \key\ taking a value in a \set\ to expand to \meta{code}. In
%   \meta{code} you can use |#1| to refer to the given value.
% \end{function}
%
% \begin{function}{\ekvdefNoVal}
%   \begin{syntax}
%     \cs{ekvdefNoVal}\marg{set}\marg{key}\marg{code}
%   \end{syntax}
%   Defines a no value taking \key\ in a \set\ to expand to \meta{code}.
% \end{function}
%
% \begin{function}{\ekvlet}
%   \begin{syntax}
%     \cs{ekvlet}\marg{set}\marg{key}\meta{cs}
%   \end{syntax}
%   Let the value taking \key\ in \set\ to \meta{cs}, there are no checks on
%   \meta{cs} enforced.
% \end{function}
%
% \begin{function}{\ekvletNoVal}
%   \begin{syntax}
%     \cs{ekvletNoVal}\marg{set}\marg{key}\meta{cs}
%   \end{syntax}
%   Let the no value taking \key\ in \set\ to \meta{cs}, it is not checked
%   whether \meta{cs} exists or that it takes no parameter.
% \end{function}
%
% \begin{function}{\ekvletkv}
%   \begin{syntax}
%     \cs{ekvletkv}\marg{set}\marg{key}\marg{set2}\marg{key2}
%   \end{syntax}
%   Let the \key\ in \set\ to \meta{key2} in \meta{set2}, it is not checked
%   whether that second key exists.
% \end{function}
%
% \begin{function}{\ekvletkvNoVal}
%   \begin{syntax}
%     \cs{ekvletkvNoVal}\marg{set}\marg{key}\marg{set2}\marg{key2}
%   \end{syntax}
%   Let the \key\ in \set\ to \meta{key2} in \meta{set2}, it is not checked
%   whether that second key exists.
% \end{function}
%
% \subsection{Parsing Keys}
%
% \begin{function}{\ekvset}
%   \begin{syntax}
%     \cs{ekvset}\marg{set}\{\kv,\ldots\}
%   \end{syntax}
%   Splits \kv\ pairs on commas. From both \key\ and \val\ up to one space is
%   stripped from both ends, if then only a braced group remains the braces are
%   stripped as well. So |\ekvset{foo}{bar=baz}| and
%   |\ekvset{foo}{ {bar}= {baz} }| will both do
%   \texttt{\cs[no-index]{\meta{foobarcode}}\{baz\}}, so you can hide commas,
%   equal signs and spaces at the ends of either \key\ or \val\ by putting
%   braces around them. If you omit the equal sign the code of the key created
%   with the |NoVal| variants described in \autoref{sec:define} will be
%   executed. If \kv\ contains more than a single unhidden equal sign, it will
%   be split at the first one and the others are considered part of the value.
%   |\ekvset| should be nestable.
% \end{function}
%
% \begin{function}{\ekvparse}
%   \begin{syntax}
%     \cs{ekvparse}\meta{cs1}\meta{cs2}\{\kv,\ldots\}
%   \end{syntax}
%   This macro parses the \kv\ pairs and provides those list elements which are
%   only keys as the argument to \meta{cs1}, and those which are a \kv\ pair to
%   \meta{cs2} as two arguments. It is fully expandable as well and returns the
%   parsed list in |\unexpanded|, which has no effect outside of an |\expanded|
%   or |\edef| context\footnotemark.
%   If you need control over the necessary steps of expansion you
%   can use |\expanded| around it.
%
%   |\ekvbreak|, |\ekvsneak|, and |\ekvchangeset| and their relatives don't work
%   in |\ekvparse|.  It is analogue to \pkg{expl3}'s |\keyval_parse:NNn|, but
%   not with the same parsing rules -- |\keyval_parse:NNn| throws an error on
%   multiple equal signs per \kv\ pair and on empty \key\ names in a \kv\ pair,
%   both of which |\ekvparse| doesn't deal with.
%
%   As a small example:
% \begin{lstlisting}
% \ekvparse\handlekey\handlekeyval{foo = bar, key, baz={zzz}}
% \end{lstlisting}
%   would expand to
% \begin{lstlisting}
% \handlekeyval{foo}{bar}\handlekey{key}\handlekeyval{baz}{zzz}
% \end{lstlisting}
%   and afterwards |\handlekey| and |\handlekeyval| would have to further handle
%   the \key. There are no macros like these two contained in \expkv, you have
%   to set them up yourself if you want to use |\ekvparse| (of course the names
%   might differ). If you need the results of |\ekvparse| as the argument for
%   another macro, you should use |\expanded| as only then the input stream will
%   contain the output above:
% \begin{lstlisting}
% \expandafter\handle\expanded{\ekvparse\k\kv{foo = bar, key, baz={zzz}}}
% \end{lstlisting}
%   would expand to
% \begin{lstlisting}
% \handle\kv{foo}{bar}\k{key}\kv{baz}{zzz}
% \end{lstlisting}
% \end{function}
% \footnotetext 
%   {^^A
%     This is a change in behaviour, previously (v0.3 and before) \cs{ekvparse}
%     would expand in exactly two steps. This isn't always necessary, but makes
%     the parsing considerably slower. If this is necessary for your application
%     you can put an \cs[no-index]{expanded} around it and will still be faster
%     since you need only a single \cs[no-index]{expandafter} this way.^^A
%   }.
%
% \subsection{Miscellaneous}
%
% \subsubsection{Other Macros}
%
% \expkv\ provides some other macros which might be of interest.
%
% \begin{function}{\ekvVersion,\ekvDate}
%   These two macros store the version and date of the package.
% \end{function}
% \begin{function}{\ekvifdefined,\ekvifdefinedNoVal}
%   \begin{syntax}
%     \cs{ekvifdefined}\marg{set}\marg{key}\marg{true}\marg{false}
%     \cs{ekvifdefinedNoVal}\marg{set}\marg{key}\marg{true}\marg{false}
%   \end{syntax}
%   These two macros test whether there is a \key\ in \set. It is false if
%   either a hash table entry doesn't exist for that key or its meaning is
%   |\relax|.
% \end{function}
%
% \begin{function}{\ekvbreak,\ekvbreakPreSneak,\ekvbreakPostSneak}
%   \begin{syntax}
%     \cs{ekvbreak}\marg{after}
%   \end{syntax}
%   Gobbles the remainder of the current |\ekvset| macro and its argument list
%   and reinserts \meta{after}. So this can be used to break out of |\ekvset|.
%   The first variant will also gobble anything that has been sneaked out using
%   |\ekvsneak| or |\ekvsneakPre|, while |\ekvbreakPreSneak| will put
%   \meta{after} before anything that has been smuggled and |\ekvbreakPostSneak|
%   will put \meta{after} after the stuff that has been sneaked out.
% \end{function}
%
% \begin{function}{\ekvsneak,\ekvsneakPre}
%   \begin{syntax}
%     \cs{ekvsneak}\marg{after}
%   \end{syntax}
%   Puts \meta{after} after the effects of |\ekvset|. The first variant will put
%   \meta{after} after any other tokens which might have been sneaked before,
%   while |\ekvsneakPre| will put \meta{after} before other smuggled stuff.
%   This reads and reinserts the remainder of the current |\ekvset| macro and
%   its argument list to do its job. A small usage example is shown in
%   \autoref{sec:sneakex}.
% \end{function}
%
% \begin{function}{\ekvchangeset}
%   \begin{syntax}
%     \cs{ekvchangeset}\marg{new-set}
%   \end{syntax}
%   Replaces the current set with \meta{new-set}, so for the rest of the current
%   |\ekvset| call, that call behaves as if it was called with
%   \texttt{\cs[no-index]{ekvset}\marg{new-set}}. Just like |\ekvsneak| this
%   reads and reinserts the remainder of the current |\ekvset| macro to do its
%   job. It is comparable to using \texttt{\meta{key}/.cd} in \pkg{pgfkeys}.
% \end{function}
%
% \bigskip
%
% \begin{function}{\ekv@name,\ekv@name@set,\ekv@name@key}
%   \begin{syntax}
%     \cs{ekv@name}\marg{set}\marg{key}\\
%     \cs{ekv@name@set}\marg{set}\\
%     \cs{ekv@name@key}\marg{key}
%   \end{syntax}
%   The names of the macros that correspond to a key in a set are build with
%   these macros. The default definition of |\ekv@name@set| is
%   ``\texttt{\csname ekv@name@set\endcsname{\set}}'' and the default of
%   |\ekv@name@key| is ``\texttt{\csname ekv@name@key\endcsname{\key}}''.
%   The complete name is build using |\ekv@name| which is equivalent to
%   \texttt{\cs[no-index]{ekv@name@set}\marg{set}\cs{ekv@name@key}\marg{key}}.
%   For |NoVal| keys an additional |N| gets appended irrespective of these
%   macros' definition, so their name is
%   \cs[no-index]{\csname ekv@name\endcsname{\set}{\key}N}. You might redefine
%   |\ekv@name@set| and |\ekv@name@key| locally but \emph{don't redefine}
%   |\ekv@name|!
% \end{function}
%
% \subsubsection{Bugs}
%
% Just like \pkg{keyval}, \expkv\ is bug free. But if you find \pmso{bugs}hidden
% features\footnote{Thanks, David!} you can tell me about them either via mail
% (see the first page) or directly on GitHub if you have an account there:
% \url{https://github.com/Skillmon/tex_expkv}
%
% \subsubsection{Comparisons}
%
% Comparisons of speed are done with a very simple test key and the help of the
% \pkg{l3benchmark} package. The key and its usage should be equivalent to
% \begin{lstlisting}
% \protected\ekvdef{test}{height}{\def\myheight{#1}}
% \ekvset{test}{ height = 6 }
% \end{lstlisting}
% and only the usage of the key, not its definition, is benchmarked. For the
% impatient, the essence of these comparisons regarding speed and buggy
% behaviour is contained in \autoref{tab:comp}.
%
% As far as I know \expkv\ is the only fully expandable \kv\ parser. I tried to
% compare \expkv\ to every \kv\ package listed on
% \href{https://ctan.org/topic/keyval}{CTAN}, however, one might notice that
% some of those are missing from this list. That's because I didn't get the
% others to work due to bugs, or because they just provide wrappers around other
% packages in this list.
%
% In this subsubsection is no benchmark of |\ekvparse| and |\keyval_parse:NNn|
% contained, as most other packages don't provide equivalent features to my
% knowledge. |\ekvparse| is slightly faster than |\ekvset|, but keep in mind
% that it does less. The same is true for |\keyval_parse:NNn| compared to
% |\keys_set:nn| of \pkg{expl3} (where the difference is much bigger).
%
% \paragraph{\pkg{keyval}} is about two times faster and has a comparable
% feature set just a slightly different way how it handles keys without values.
% That might be considered a drawback, as it limits the versatility, but also as
% an advantage, as it might reduce doubled code. Keep in mind that as soon as
% someone loads \pkg{xkeyval} the performance of \pkg{keyval} gets replaced by
% \pkg{xkeyval}'s.
%
% Also \pkg{keyval} has a bug, which unfortunately can't really be resolved
% without breaking backwards compatibility for \emph{many} documents, namely it
% strips braces from the argument before stripping spaces if the argument isn't
% surrounded by spaces, also it might strip more than one set of braces. Hence
% all of the following are equivalent in their outcome, though the last two
% lines should result in something different than the first two:
% \begin{lstlisting}[belowskip=0pt]
% \setkeys{foo}{bar=baz}
% \setkeys{foo}{bar= {baz}}
% \setkeys{foo}{bar={ baz}}
% \setkeys{foo}{bar={{baz}}}
% \end{lstlisting}
%
% \paragraph{\pkg{xkeyval}} is roughly fourteen times slower, but it provides
% more functionality, e.g., it has choice keys, boolean keys, and so on. It
% contains the same bug as \pkg{keyval} as it has to be compatible with it by
% design (it replaces \pkg{keyval}'s frontend), but also adds even more cases in
% which braces are stripped that shouldn't be stripped, worsening the situation.
%
% \paragraph{\pkg{ltxkeys}} is over 300 times slower -- which is funny, because
% it aims to be ``[\ldots] faster than these earlier packages [referring to
% \pkg{keyval} and \pkg{xkeyval}].'' Since it aims to have a bigger feature set
% than \pkg{xkeyval}, it most definitely also has a bigger feature set than
% \expkv. Also, it can't parse |\long| input, so as soon as your values contain
% a |\par|, it'll throw errors. Furthermore, \pkg{ltxkeys} doesn't strip outer
% braces at all by design, which, imho, is a weird design choice. In addition
% \pkg{ltxkeys} loads \pkg{catoptions} which is known to introduce bugs (e.g.,
% see \url{https://tex.stackexchange.com/questions/461783}).
%
% \paragraph{\pkg{l3keys}} is almost five times slower, but has an, imho, great
% interface to define keys. It strips \emph{all} outer spaces, even if somehow
% multiple spaces ended up on either end. It offers more features, but is pretty
% much bound to \pkg{expl3} code. Whether that's a drawback is up to you.
%
% \paragraph{\pkg{pgfkeys}} is a bit more than two times slower for one key,
% but has an \emph{enormous} feature set. However, since adding additional keys
% doesn't add as much needed time for \pkg{pgfkeys} compared to \expkv, it gets
% faster than \expkv\ at around eight \kv\ pairs.
% It has the same or a very similar bug \pkg{keyval} has. The brace bug (and
% also the category fragility) can be fixed by \pkg{pgfkeyx}, but this package
% was last updated in 2012 and it slows down |\pgfkeys| by factor~$8$. Also I
% don't know whether this might introduce new bugs.
%
% \paragraph{\pkg{kvsetkeys} with \pkg{kvdefinekeys}} is about three times
% slower, but it works even if commas and equals have category codes different
% from 12 (just as some other packages in this list). Else the features of the
% keys are equal to those of \pkg{keyval}, the parser has more features, though.
%
% \paragraph{\pkg{options}} is a bit slower for only a single value, but gets a
% tad faster than \expkv\ at around 10 \kv\ pairs. It has a much bigger feature
% set. Unfortunately it also suffers from the premature unbracing bug
% \pkg{keyval} has.
%
% \paragraph{\pkg{simplekv}} is hard to compare because I don't speak French (so
% I don't understand the documentation) and from what I can see, there is no
% direct way to define the equivalent test key. Nevertheless, I tested the
% closest possible equivalent of my test key while siding for \pkg{simplekv}'s
% design not forcing something into it it doesn't seem to be designed for. It is
% more than five times slower and has hard to predict behaviour regarding brace
% and space stripping, similar to \pkg{keyval}. The tested definition was:
% \begin{lstlisting}[belowskip=0pt]
% \usepackage{simplekv}
% \setKVdefault[simplekv]{height={ abc}}                     % key setup
% \setKV[simplekv]{ height = 6 }                             % benchmarked
% \end{lstlisting}
%
% \paragraph{\pkg{yax}} is over eighteen times slower. It has a pretty strange
% syntax, imho, and again a direct equivalent is hard to define. It has the
% premature unbracing bug, too. Also somehow loading \pkg{yax} broke
% \pkg{options} for me. The tested definition was:
% \begin{lstlisting}[belowskip=0pt]
% \usepackage{yax}
% \defactiveparameter yax {\storevalue\myheight yax:height } % key setup
% \setparameterlist{yax}{ height = 6 }                       % benchmarked
% \end{lstlisting}
%
% \begin{table}
%   \def\fnsym{\textcolor{red!80!black}{*}}%
%   \sisetup{round-precision=1, round-mode=places}%
%   \begingroup
%   \centering
%   \newcommand*\yes{\textcolor{red!80!black}  {yes}}^^A
%   \newcommand*\no {\textcolor{green!80!black}{no}}^^A
%   \caption[Comparison of \kv\ packages]
%     {^^A
%       Comparison of \kv\ packages. The packages are ordered from
%       fastest to slowest for one \kv\ pair. Benchmarking was done using
%       \pkg{l3benchmark} and the scripts in the \file{Benchmarks} folder
%       of the \href{https://github.com/Skillmon/tex_expkv}{git repository}.
%       The columns $p_i$ are the polynomial coefficients of a linear fit to the
%       run-time, $p_0$ can be interpreted as the overhead for initialisation
%       and $p_1$ the cost per key. The $T_0$ column is the actual mean ops
%       needed for an empty list argument, as the linear fit doesn't match that
%       point well in general. The column ``BB'' lists whether the
%       parsing is affected by some sort of brace bug, ``CF'' stands for
%       category code fragile and lists whether the parsing breaks with active
%       commas or equal signs.^^A
%       \label{tab:comp}^^A
%     }
%   \begin{tabular}
%       {>{\collectcell\pkg}l<{\endcollectcell}*3{S[table-format=4.1]}ccc}
%     \toprule
%     \rmfamily Package & {$p_1$} & {$p_0$} & {$T_0$}& BB & CF & Date \\
%     \midrule
%     keyval    &   13.539 &    1.844 &    6.969 & \yes & \yes & 2014-10-28 \\
%     \expkv    &   26.411 &    2.391 &   12.610 & \no  & \no  & 2020-02-22 \\
%     options   &   25.532 &   10.518 &   21.393 & \yes & \yes & 2015-03-01 \\
%     pgfkeys   &   26.444 &   40.939 &   55.790 & \yes & \yes & 2020-01-08 \\
%     kvsetkeys & {\fnsym} & {\fnsym} &   41.620 & \no  & \no  & 2019-12-15 \\
%     l3keys    &  114.456 &   35.357 &   52.667 & \no  & \no  & 2020-02-14 \\
%     simplekv  &  160.523 &   10.809 &    8.698 & \yes & \yes & 2017-08-08 \\
%     xkeyval   &  260.872 &  180.435 &  164.167 & \yes & \yes & 2014-12-03 \\
%     yax       &  507.643 &   62.863 &  123.533 & \yes & \yes & 2010-01-22 \\
%     ltxkeys   & 3932.759 & 4737.818 & 5883.000 & \no  & \no  & 2012-11-17 \\
%     \bottomrule
%   \end{tabular}
%   \par
%   \endgroup
%   \medskip
%   \fnsym For \pkg{kvsetkeys} the linear model used for the other
%   packages is a poor fit, \pkg{kvsetkeys} seems to have approximately
%   quadratic run-time, the coefficients of the second degree polynomial fit are
%   $p_2=\num{9.911}$, $p_1=\num{40.523}$, and $p_0=\num{61.500}$. Of course the
%   other packages might not really have linear run-time, but at least from 1~to
%   20~keys the fits don't seem too bad (the maximum ratio $p_2/p_1$
%   for the other packages is \num{9.697e-3}). If one extrapolates the fits
%   for 100 \kv\ pairs one finds that most of them match pretty well, the
%   exception being \pkg{ltxkeys}, which behaves quadratic as well with
%   $p_2=\num{31.685}$, $p_1=\num{3267.368}$, and $p_0=\num{7177.582}$.
% \end{table}
%
% \subsection{Examples}
%
% \subsubsection{Standard Use-Case}
%
% Say we have a macro for which we want to create a \kv\ interface. The macro
% has a parameter, which is stored in the dimension |\ourdim| having a default
% value from its initialization. Now we want to be able to change that dimension
% with the |width| key to some specified value. For that we'd do
% \begin{lstlisting}
% \newdimen\ourdim
% \ourdim=150pt
% \protected\ekvdef{our}{width}{\ourdim=#1\relax}
% \end{lstlisting}
% as you can see, we use the set |our| here. We want the key to behave different
% if no value is specified. In that case the key should not use its initial
% value, but be smart and determine the available space from
% \cs[no-index]{hsize}, so we also define
% \begin{lstlisting}
% \protected\ekvdefNoVal{our}{width}{\ourdim=.9\hsize}
% \end{lstlisting}
% Now we set up our macro to use this \kv\ interface
% \begin{lstlisting}
% \protected\def\ourmacro#1{\begingroup\ekvset{our}{#1}\the\ourdim\endgroup}
% \end{lstlisting}
% Finally we can use our macro like in the following\\
% \begin{minipage}{6cm}
% \begin{lstlisting}
% \ourmacro{}\par
% \ourmacro{width}\par
% \ourmacro{width=5pt}\par
% \end{lstlisting}
% \end{minipage}^^A
% \begin{minipage}{\dimexpr\linewidth-6cm\relax}
%   \newdimen\ourdim\ourdim=150pt
%   \protected\ekvdef{our}{width}{\ourdim=#1\relax}^^A
%   \protected\ekvdefNoVal{our}{width}{\ourdim=.9\hsize}^^A
%   \protected\def\ourmacro#1^^A
%     {\begingroup\ekvset{our}{#1}\the\ourdim\endgroup}^^A
%   \ourmacro{}\par
%   \ourmacro{width}\par
%   \ourmacro{width=5pt}\par
% \end{minipage}
%
% \paragraph{The same key using \protect\expkvd}
% Using \expkvd\ we can set up the equivalent key using a \kv\ interface, after
% the following we could use |\ourmacro| in the same way as above. \expkvd\ will
% allocate and initialise |\ourdim| and define the |width| key |\protected| for
% us, so the result will be exactly the same -- with the exception that the
% default will use |\ourdim=.9\hsize\relax| instead.
% \begin{lstlisting}
% \input expkv-def                 % or \usepackage{expkv-def}
% \ekvdefinekeys{our}
%   {
%     dimen    width = \ourdim,
%     qdefault width = .9\hsize,
%     initial  width = 150pt
%   }
% \end{lstlisting}
%
% \subsubsection{An Expandable \kv\ Macro Using \cs[no-index]{ekvsneak}}
% \label{sec:sneakex}
%
% Let's set up an expandable macro, that uses a \kv\ interface. The problems
% we'll face for this are:
% \begin{enumerate}
%   \item ignoring duplicate keys
%   \item default values for keys which weren't used
%   \item providing the values as the correct argument to a macro (ordered)
% \end{enumerate}
%
% First we need to decide which \kv\ parsing macro we want to do this with,
% |\ekvset| or |\ekvparse|. For this example we also want to show the
% usage of |\ekvsneak|, hence we'll choose |\ekvset|. And we'll have to use
% |\ekvset| such that it builds a parsable list for our macro internals. To
% gain back control after |\ekvset| is done we have to put an internal of our
% macro at the start of that list, so we use an internal key that uses
% |\ekvsneakPre| after any user input.
%
% To ignore duplicates will be easy if the value of the key used last will be
% put first in the list, so the following will use |\ekvsneakPre| for the
% user-level keys. If we wanted some key for which the first usage should be the
% the binding one we would use |\ekvsneak| instead for that key.
%
% Providing default values can be done in different ways, we'll use a simple
% approach in which we'll just put the outcome of our keys if they were used
% with default values before the parsing list terminator.
%
% Ordering the keys can be done simply by searching for a specific token for
% each argument which acts like a flag, so our sneaked out values will include
% specific tokens acting as markers.
%
% Now that we have answers for our technical problems, we have to decide what
% our example macro should do. How about we define a macro that calculates the
% sine of a number and rounds that to a specified precision? As a small extra
% this macro should understand input in radian and degree and the used
% trigonometric function should be selectable as well. For the hard part of this
% task (expandably evaluating trigonometric functions) we'll use the \pkg{xfp}
% package.
%
% First we set up our keys according to our earlier considerations and set up
% the user facing macro |\sine|. The end marker of the parsing list will be a
% |\sine@stop| token, which we don't need to define and we put our defaults
% right before it.
%
% \begin{lstlisting}
% \RequirePackage{xfp}
% \makeatletter
% \ekvdef{expex}{f}{\ekvsneakPre{\f{#1}}}
% \ekvdef{expex}{round}{\ekvsneakPre{\rnd{#1}}}
% \ekvdefNoVal{expex}{degree}{\ekvsneakPre{\deg{d}}}
% \ekvdefNoVal{expex}{radian}{\ekvsneakPre{\deg{}}}
% \ekvdefNoVal{expex}{internal}{\ekvsneakPre{\sine@rnd}}
% \newcommand*\sine[2]
%   {\ekvset{expex}{#1,internal}\rnd{3}\deg{d}\f{sin}\sine@stop{#2}}
% \end{lstlisting}
% For the sake of simplicity we defined the macro |\sine| with two mandatory
% arguments, the first being the \kv\ list, the second the argument to the
% trigonometric function. We could've used \pkg{xparse}'s facilities here to
% define an expandable macro which takes an optional argument instead.
%
% Now we need to define some internal macros to extract the value of each key's
% last usage (remember that this will be the group after the first special
% flag-token). For that we use one delimited macro per key.
% \begin{lstlisting}
% \def\sine@rnd#1\rnd#2#3\sine@stop{\sine@deg#1#3\sine@stop{#2}}
% \def\sine@deg#1\deg#2#3\sine@stop{\sine@f#1#3\sine@stop{#2}}
% \def\sine@f#1\f#2#3\sine@stop{\sine@final{#2}}
% \end{lstlisting}
% After the macros |\sine@rnd|, |\sine@deg|, and |\sine@f| the macro
% |\sine@final| will see
% \texttt
%   {^^A
%     \cs[no-index]{sine@final}^^A
%     \marg{f}\marg{degree/radian}\marg{round}\marg{num}^^A
%   }.
% Now |\sine@final| has to expandably deal with those arguments such that the
% |\fpeval| macro of \pkg{xfp} gets the correct input. Luckily this is pretty
% straight forward in this example. In |\fpeval| the trigonometric functions
% have names such as |sin| or |cos| and the degree taking variants |sind| or
% |cosd|. And since the |degree| key puts a |d| in |#2| and the |radian| key
% leaves |#2| empty all we have to do to get the correct function name is stick
% the two together.
% \begin{lstlisting}
% \newcommand*\sine@final[4]{\fpeval{round(#1#2(#4),#3)}}
% \makeatother
% \end{lstlisting}
% Let's test our macro:\\
% \begin{minipage}{.7\linewidth}
%   \begin{lstlisting}[gobble=4]
%   \sine{}{60}\par
%   \sine{round=10}{60}\par
%   \sine{f=cos,radian}{pi}\par
%   \edef\myval{\sine{f=tan}{1}}\texttt{\meaning\myval}
%   \end{lstlisting}
% \end{minipage}\hfill
% \begin{minipage}{.2\linewidth}
%   \makeatletter
%   \ekvdef{expex}{f}{\ekvsneakPre{\f{#1}}}^^A
%   \ekvdef{expex}{round}{\ekvsneakPre{\rnd{#1}}}^^A
%   \ekvdefNoVal{expex}{degree}{\ekvsneakPre{\deg{d}}}^^A
%   \ekvdefNoVal{expex}{radian}{\ekvsneakPre{\deg{}}}^^A
%   \ekvdefNoVal{expex}{internal}{\ekvsneakPre{\sine@rnd}}^^A
%   \newcommand*\sine[2]
%     {\ekvset{expex}{#1,internal}\rnd{3}\deg{d}\f{sin}\sine@stop{#2}}^^A
%   \def\sine@rnd#1\rnd#2#3\sine@stop{\sine@deg#1#3\sine@stop{#2}}^^A
%   \def\sine@deg#1\deg#2#3\sine@stop{\sine@f#1#3\sine@stop{#2}}^^A
%   \def\sine@f#1\f#2#3\sine@stop{\sine@final{#2}}^^A
%   \newcommand*\sine@final[4]{\fpeval{round(#1#2(#4),#3)}}^^A
%   \makeatother
%   \sine{}{60}\par
%   \sine{round=10}{60}\par
%   \sine{f=cos,radian}{pi}\par
%   \edef\myval{\sine{f=tan}{1}}\texttt{\meaning\myval}
% \end{minipage}
%
%
% \subsection{Error Messages}
%
% \expkv\ should only send messages in case of errors, there are no warnings and
% no info messages. In this subsection those errors are listed.
%
% \subsubsection{Load Time}
%
% \file{expkv.tex} checks whether \eTeX\ is available. If it isn't, an error
% will be thrown using |\errmessage|:
% \begin{lstlisting}
% ! expkv Error: e-TeX required.
% \end{lstlisting}
%
% \subsubsection{Defining Keys}
%
% If you get any error from \expkv\ while you're trying to define a key, the
% definition will be aborted and gobbled.
%
% If you try to define a key with an empty set name you'll get:
% \begin{lstlisting}
% ! expkv Error: empty set name not allowed.
% \end{lstlisting}
%
% Similarly, if you try to define a key with an empty key name:
% \begin{lstlisting}
% ! expkv Error: empty key name not allowed.
% \end{lstlisting}
%
% Both of these messages are done in a way that doesn't throw additional errors
% due to |\global|, |\long|, etc., not being used correctly if you prefixed one
% of the defining macros.
%
% \subsubsection{Using Keys}
%
% This subsubsection contains the errors thrown during |\ekvset|. The errors are
% thrown in an expandable manner by providing an undefined macro. In the
% following messages \lstinline|<key>| gets replaced with the problematic key's
% name, and \lstinline|<set>| with the corresponding set. If any errors during
% \kv\ handling are encountered, the entry in the comma separated list will be
% omitted after the error is thrown and the next \kv\ pair will be parsed.
%
% If you're using an undefined key you'll get:
% \begin{lstlisting}
% ! Undefined control sequence.
% <argument> \! expkv Error:
%                            unknown key (`<key>', set `<set>').
% \end{lstlisting}
%
% If you're using a key for which only a normal version and no |NoVal| version
% is defined, but don't provide a value, you'll get:
% \begin{lstlisting}
% ! Undefined control sequence.
% <argument> \! expkv Error:
%                            value required (`<key>', set `<set>').
% \end{lstlisting}
%
% If you're using a key for which only a |NoVal| version and no normal version
% is defined, but provide a value, you'll get:
% \begin{lstlisting}
% ! Undefined control sequence.
% <argument> \! expkv Error:
%                            value forbidden (`<key>', set `<set>').
% \end{lstlisting}
%
% If you're using a set for which you never executed one of the defining macros
% from \autoref{sec:define} you'll get a low level \TeX\ error, as that isn't
% actively tested by the parser (and hence will lead to undefined behaviour and
% not be gracefully ignored). The error will look like
% \begin{lstlisting}
% ! Missing \endcsname inserted.
% <to be read again>
%                    \! expkv Error: Set `<set>' undefined.
% \end{lstlisting}
%
% \subsection{License}
%
% Copyright \textcopyright\ 2020\unless\ifnum\year=2020--\the\year\fi\
% Jonathan P. Spratte
%
% \medskip\noindent
% This work may be distributed and/or modified under the conditions of the
% \LaTeX\ Project Public License (LPPL), either version 1.3c of this license or
% (at your option) any later version. The latest version of this license is in
% the file:
%
% \url{http://www.latex-project.org/lppl.txt}
%
% \noindent
% This work is ``maintained'' (as per LPPL maintenance status) by
%
% \mbox{Jonathan P. Spratte}.
%
% \end{documentation}^^A=<<
%
% \begin{implementation}^^A>>=
%
% \clearpage
%
% \section{Implementation}^^A>>=
%^^A the LaTeX package >>=
% \subsection{The \LaTeX\ Package}
% First we set up the \LaTeX\ package. That one doesn't really do much except
% |\input|ting the generic code and identifying itself as a package.
% \gobbledocstriptag
%<*pkg>
%    \begin{macrocode}
\def\ekv@tmp
  {%
    \ProvidesFile{expkv.tex}%
      [\ekvDate\space v\ekvVersion\space an expandable key=val implementation]%
  }
% \iffalse meta-comment
%
% File: expkv.dtx Copyright (C) 2020 Jonathan P. Spratte
%
% This work  may be  distributed and/or  modified under  the conditions  of the
% LaTeX Project Public License (LPPL),  either version 1.3c  of this license or
% (at your option) any later version.  The latest version of this license is in
% the file:
%
%   http://www.latex-project.org/lppl.txt
%
% ------------------------------------------------------------------------------
%
%<*driver>^^A>>=
\def\nameofplainTeX{plain}
\ifx\fmtname\nameofplainTeX\else
  \expandafter\begingroup
\fi
\input l3docstrip.tex
\askforoverwritefalse
\preamble

--------------------------------------------------------------
expkv -- an expandable key=val implementation
E-mail: jspratte@yahoo.de
Released under the LaTeX Project Public License v1.3c or later
See http://www.latex-project.org/lppl.txt
--------------------------------------------------------------

Copyright (C) 2020 Jonathan P. Spratte

This  work may be  distributed and/or  modified under  the conditions  of the
LaTeX Project Public License (LPPL),  either version 1.3c  of this license or
(at your option) any later version.  The latest version of this license is in
the file:

  http://www.latex-project.org/lppl.txt

This work is "maintained" (as per LPPL maintenance status) by
  Jonathan P. Spratte.

This work consists of the file  expkv.dtx
and the derived files           expkv.pdf
                                expkv.sty
                                expkv.tex

\endpreamble
% stop docstrip adding \endinput
\postamble
\endpostamble
\generate{\file{expkv.sty}{\from{expkv.dtx}{pkg}}}
\generate{\file{expkv.tex}{\from{expkv.dtx}{tex}}}
\ifx\fmtname\nameofplainTeX
  \expandafter\endbatchfile
\else
  \expandafter\endgroup
\fi
%
\IfFileExists{expkv.tex}{% \iffalse meta-comment
%
% File: expkv.dtx Copyright (C) 2020 Jonathan P. Spratte
%
% This work  may be  distributed and/or  modified under  the conditions  of the
% LaTeX Project Public License (LPPL),  either version 1.3c  of this license or
% (at your option) any later version.  The latest version of this license is in
% the file:
%
%   http://www.latex-project.org/lppl.txt
%
% ------------------------------------------------------------------------------
%
%<*driver>^^A>>=
\def\nameofplainTeX{plain}
\ifx\fmtname\nameofplainTeX\else
  \expandafter\begingroup
\fi
\input l3docstrip.tex
\askforoverwritefalse
\preamble

--------------------------------------------------------------
expkv -- an expandable key=val implementation
E-mail: jspratte@yahoo.de
Released under the LaTeX Project Public License v1.3c or later
See http://www.latex-project.org/lppl.txt
--------------------------------------------------------------

Copyright (C) 2020 Jonathan P. Spratte

This  work may be  distributed and/or  modified under  the conditions  of the
LaTeX Project Public License (LPPL),  either version 1.3c  of this license or
(at your option) any later version.  The latest version of this license is in
the file:

  http://www.latex-project.org/lppl.txt

This work is "maintained" (as per LPPL maintenance status) by
  Jonathan P. Spratte.

This work consists of the file  expkv.dtx
and the derived files           expkv.pdf
                                expkv.sty
                                expkv.tex

\endpreamble
% stop docstrip adding \endinput
\postamble
\endpostamble
\generate{\file{expkv.sty}{\from{expkv.dtx}{pkg}}}
\generate{\file{expkv.tex}{\from{expkv.dtx}{tex}}}
\ifx\fmtname\nameofplainTeX
  \expandafter\endbatchfile
\else
  \expandafter\endgroup
\fi
%
\IfFileExists{expkv.tex}{\input{expkv.tex}}{}
\ProvidesFile{expkv.dtx}
  [\csname ekvDate\endcsname\ an expandable key=val implementation]
\PassOptionsToPackage{full}{textcomp}
\documentclass{l3doc}
\RequirePackage[oldstylenums,nott]{kpfonts}
\input{glyphtounicode}
\pdfgentounicode=1
\RequirePackage{xfp} ^^A required for an example
\RequirePackage{listings}
\RequirePackage{booktabs}
\RequirePackage{array}
\RequirePackage{collcell}
\RequirePackage{siunitx}
\RequirePackage{xcolor}
\RequirePackage{caption}
\RequirePackage{microtype}
\RequirePackage{accsupp}
\lstset
  {
    ,flexiblecolumns=false
    ,basewidth=.53em
    ,gobble=2
    ,basicstyle=\fontfamily{jkp}\itshape
    ,morekeywords=^^A
      {^^A
        \ekvdef,\ekvdefNoVal,\ekvlet,\ekvletNoVal,\ekvletkv,\ekvletkvNoVal,^^A
        \ekvset,\ekvparse,\ekvVersion,\ekvDate,\ekvifdefined,^^A
        \ekvifdefinedNoVal,\ekvbreak,\ekvbreakPreSneak,\ekvbreakPostSneak,^^A
        \ekvsneak,\ekvsneakPre,\ekvchangeset
      }
    ,morecomment=[l]\%
    ,commentstyle=\color[gray]{0.4}
    ,literate={\{}{{\CodeSymbol\{}}{1}
              {\}}{{\CodeSymbol\}}}{1}
    ^^A,literate=*{<key>}{\key}{4}{<set>}{\set}{4}
  }
\newcommand*\CodeSymbol[1]{\textbf{#1}}
\RequirePackage{randtext}
\let\metaORIG\meta
\protected\def\meta #1{\texttt{\metaORIG{#1}}}
\renewcommand*\thefootnote{\fnsymbol{footnote}}
\definecolor{expkvred}{HTML}{9F393D}
\colorlet{expkvgrey}{black!75}
\makeatletter
\newcommand*\expkv
  {^^A
    \texorpdfstring
      {^^A
        \mbox
          {^^A
            \BeginAccSupp{ActualText=expkv}^^A
            \href{https://github.com/Skillmon/tex_expkv}
              {^^A
                \rmfamily
                \bfseries
                {\color{expkvgrey}e\kern-.05em x\kern-.05em}^^A
                \lower.493ex
                  \hbox{{\color{expkvgrey}P}\kern-.1em{\color{expkvred}k}}^^A
                \kern-.18em{\color{expkvred}v}^^A
              }^^A
            \EndAccSupp{}^^A
          }^^A
      }
      {expkv}^^A
  }
\newcommand*\expkvd
  {^^A
    \texorpdfstring
      {^^A
        \mbox
          {^^A
            \BeginAccSupp{ActualText=expkv-def}^^A
            \rmfamily
            \bfseries
            {\color{expkvgrey}e\kern-.05em x\kern-.05em}^^A
            \lower.493ex
              \hbox{{\color{expkvgrey}P}\kern-.1em{\color{expkvred}k}}^^A
            \kern-.18em{\color{expkvred}v}^^A
            {\color{expkvgrey}^^A
              \kern.05em\rule[-.1ex]{.08em}{1.2ex}\kern.05em\textsc{def}^^A
            }^^A
            \EndAccSupp{}^^A
          }^^A
      }
      {expkv-def}^^A
  }
\newcommand\kv{\meta{key}=\meta{value}}
\newcommand\key{\meta{key}}
\newcommand\val{\meta{value}}
\newcommand\set{\meta{set}}
\hypersetup{linkcolor=red!80!black,urlcolor=purple!80!black}
\DoNotIndex{\def,\edef,\,,\=,\begingroup,\catcode,\chardef,\csname,\endcsname}
\DoNotIndex{\endgroup,\endinput,\errmessage,\expandafter,\input,\let,\long}
\DoNotIndex{\noexpand,\protected,\ProvidesFile,\ProvidesPackage,\relax,\space}
\DoNotIndex{\@,\unexpanded,\string,\expanded}
\DoNotIndex{\ifcsname}
\DoNotIndex{\ifx}
\DoNotIndex{\else}
\DoNotIndex{\fi}
\@gobble\fi ^^A ignoring \ifx and \ifcsname, but only one \fi
\@ifdefinable\gobbledocstriptag{\def\gobbledocstriptag#1>{}}
\newcommand*\pmso[1] ^^A poor man's strike out
  {^^A
    \leavevmode
    \begingroup
      \sbox0{#1}^^A
      \rlap{\vrule height .6ex depth -.5ex width \wd0\relax}^^A
      \usebox0\relax
    \endgroup
  }
\makeatother
\begin{document}
  \title
    {^^A
      \texorpdfstring
        {^^A
          \huge\expkv
          \\[\medskipamount]
          \Large an expandable \kv\ implementation^^A
        }{expkv - an expandable <key>=<value> implementation}^^A
    }
  \date{\ekvDate\space v\ekvVersion}
  \author{Jonathan P. Spratte\thanks{\protect\randomize{jspratte@yahoo.de}}}
  \DocInput{expkv.dtx}
\end{document}
%</driver>^^A=<<
% \fi
%
% \maketitle
% \renewcommand*\thefootnote{\arabic{footnote}}
%
% \begin{abstract}
% \noindent\parfillskip=0pt
% \expkv\ provides a small interface for \kv\ parsing. The parsing macro is
% fully expandable, the \meta{code} of your keys might be not. \expkv\ is pretty
% fast, but not the fastest available \kv\ solution (\pkg{keyval} for instance
% is thrice as fast, but not expandable and it might strip braces it shouldn't
% have stripped).
% \end{abstract}
%
% \tableofcontents
%
% \begin{documentation}^^A>>=
%
% \section{Documentation}
%
% \expkv\ provides an expandable \kv\ parser. The \kv\ pairs should be given as
% a comma separated list and the separator between a \key\ and the associated
% \val\ should be an equal sign. Both, the commas and the equal signs, might be
% of category 12 (other) or 13 (active). To support this is necessary as for
% example \pkg{babel} turns characters active for some languages, for instance
% the equal sign is turned active for Turkish.
%
% \expkv\ is usable as generic code or as a \LaTeX\ package. To use it, just
% use one of:
% \begin{lstlisting}
% \usepackage{expkv} % LaTeX
% \input expkv       % plainTeX
% \end{lstlisting}
% The \LaTeX\ package doesn't do more than \file{expkv.tex}, except calling
% |\ProvidesPackage| and setting things up such that \file{expkv.tex} will use
% |\ProvidesFile|.
%
% \subsection{Setting up Keys}\label{sec:define}
% \expkv\ provides a rather simple approach to setting up keys, similar to
% \pkg{keyval}. However there is an auxiliary package named
% \href{https://github.com/Skillmon/tex_expkv-def}{\expkvd} (the files use
% \file{expkv-def} instead because \string| isn't allowed in package names by
% CTAN) which provides a more sophisticated interface, similar to well
% established packages like \pkg{pgfkeys} or \pkg{l3keys}.
%
% Keys in \expkv\ (as in almost all other \kv\ implementations) belong to a
% \emph{set} such that different sets can contain keys of the same name. Unlike
% many other implementations \expkv\ doesn't provide means to set a default
% value, instead we have keys that take values and keys that don't (the latter
% are called |NoVal| keys by \expkv), but both can have the same name (on the
% user level).
%
% The following macros are available to define new keys. Those macros containing
% ``|def|'' in their name can be prefixed by anything allowed to prefix |\def|,
% prefixes allowed for |\let| can prefix those with ``|let|'' in their name,
% accordingly. Neither \set\ nor \key\ are allowed to be empty for new keys and
% must not contain a |\par| or tokens that expand to it -- they must be legal
% inside of |\csname ...\endcsname|.
%
% \begin{function}{\ekvdef}
%   \begin{syntax}
%     \cs{ekvdef}\marg{set}\marg{key}\marg{code}
%   \end{syntax}
%   Defines a \key\ taking a value in a \set\ to expand to \meta{code}. In
%   \meta{code} you can use |#1| to refer to the given value.
% \end{function}
%
% \begin{function}{\ekvdefNoVal}
%   \begin{syntax}
%     \cs{ekvdefNoVal}\marg{set}\marg{key}\marg{code}
%   \end{syntax}
%   Defines a no value taking \key\ in a \set\ to expand to \meta{code}.
% \end{function}
%
% \begin{function}{\ekvlet}
%   \begin{syntax}
%     \cs{ekvlet}\marg{set}\marg{key}\meta{cs}
%   \end{syntax}
%   Let the value taking \key\ in \set\ to \meta{cs}, there are no checks on
%   \meta{cs} enforced.
% \end{function}
%
% \begin{function}{\ekvletNoVal}
%   \begin{syntax}
%     \cs{ekvletNoVal}\marg{set}\marg{key}\meta{cs}
%   \end{syntax}
%   Let the no value taking \key\ in \set\ to \meta{cs}, it is not checked
%   whether \meta{cs} exists or that it takes no parameter.
% \end{function}
%
% \begin{function}{\ekvletkv}
%   \begin{syntax}
%     \cs{ekvletkv}\marg{set}\marg{key}\marg{set2}\marg{key2}
%   \end{syntax}
%   Let the \key\ in \set\ to \meta{key2} in \meta{set2}, it is not checked
%   whether that second key exists.
% \end{function}
%
% \begin{function}{\ekvletkvNoVal}
%   \begin{syntax}
%     \cs{ekvletkvNoVal}\marg{set}\marg{key}\marg{set2}\marg{key2}
%   \end{syntax}
%   Let the \key\ in \set\ to \meta{key2} in \meta{set2}, it is not checked
%   whether that second key exists.
% \end{function}
%
% \subsection{Parsing Keys}
%
% \begin{function}{\ekvset}
%   \begin{syntax}
%     \cs{ekvset}\marg{set}\{\kv,\ldots\}
%   \end{syntax}
%   Splits \kv\ pairs on commas. From both \key\ and \val\ up to one space is
%   stripped from both ends, if then only a braced group remains the braces are
%   stripped as well. So |\ekvset{foo}{bar=baz}| and
%   |\ekvset{foo}{ {bar}= {baz} }| will both do
%   \texttt{\cs[no-index]{\meta{foobarcode}}\{baz\}}, so you can hide commas,
%   equal signs and spaces at the ends of either \key\ or \val\ by putting
%   braces around them. If you omit the equal sign the code of the key created
%   with the |NoVal| variants described in \autoref{sec:define} will be
%   executed. If \kv\ contains more than a single unhidden equal sign, it will
%   be split at the first one and the others are considered part of the value.
%   |\ekvset| should be nestable.
% \end{function}
%
% \begin{function}{\ekvparse}
%   \begin{syntax}
%     \cs{ekvparse}\meta{cs1}\meta{cs2}\{\kv,\ldots\}
%   \end{syntax}
%   This macro parses the \kv\ pairs and provides those list elements which are
%   only keys as the argument to \meta{cs1}, and those which are a \kv\ pair to
%   \meta{cs2} as two arguments. It is fully expandable as well and returns the
%   parsed list in |\unexpanded|, which has no effect outside of an |\expanded|
%   or |\edef| context\footnotemark.
%   If you need control over the necessary steps of expansion you
%   can use |\expanded| around it.
%
%   |\ekvbreak|, |\ekvsneak|, and |\ekvchangeset| and their relatives don't work
%   in |\ekvparse|.  It is analogue to \pkg{expl3}'s |\keyval_parse:NNn|, but
%   not with the same parsing rules -- |\keyval_parse:NNn| throws an error on
%   multiple equal signs per \kv\ pair and on empty \key\ names in a \kv\ pair,
%   both of which |\ekvparse| doesn't deal with.
%
%   As a small example:
% \begin{lstlisting}
% \ekvparse\handlekey\handlekeyval{foo = bar, key, baz={zzz}}
% \end{lstlisting}
%   would expand to
% \begin{lstlisting}
% \handlekeyval{foo}{bar}\handlekey{key}\handlekeyval{baz}{zzz}
% \end{lstlisting}
%   and afterwards |\handlekey| and |\handlekeyval| would have to further handle
%   the \key. There are no macros like these two contained in \expkv, you have
%   to set them up yourself if you want to use |\ekvparse| (of course the names
%   might differ). If you need the results of |\ekvparse| as the argument for
%   another macro, you should use |\expanded| as only then the input stream will
%   contain the output above:
% \begin{lstlisting}
% \expandafter\handle\expanded{\ekvparse\k\kv{foo = bar, key, baz={zzz}}}
% \end{lstlisting}
%   would expand to
% \begin{lstlisting}
% \handle\kv{foo}{bar}\k{key}\kv{baz}{zzz}
% \end{lstlisting}
% \end{function}
% \footnotetext 
%   {^^A
%     This is a change in behaviour, previously (v0.3 and before) \cs{ekvparse}
%     would expand in exactly two steps. This isn't always necessary, but makes
%     the parsing considerably slower. If this is necessary for your application
%     you can put an \cs[no-index]{expanded} around it and will still be faster
%     since you need only a single \cs[no-index]{expandafter} this way.^^A
%   }.
%
% \subsection{Miscellaneous}
%
% \subsubsection{Other Macros}
%
% \expkv\ provides some other macros which might be of interest.
%
% \begin{function}{\ekvVersion,\ekvDate}
%   These two macros store the version and date of the package.
% \end{function}
% \begin{function}{\ekvifdefined,\ekvifdefinedNoVal}
%   \begin{syntax}
%     \cs{ekvifdefined}\marg{set}\marg{key}\marg{true}\marg{false}
%     \cs{ekvifdefinedNoVal}\marg{set}\marg{key}\marg{true}\marg{false}
%   \end{syntax}
%   These two macros test whether there is a \key\ in \set. It is false if
%   either a hash table entry doesn't exist for that key or its meaning is
%   |\relax|.
% \end{function}
%
% \begin{function}{\ekvbreak,\ekvbreakPreSneak,\ekvbreakPostSneak}
%   \begin{syntax}
%     \cs{ekvbreak}\marg{after}
%   \end{syntax}
%   Gobbles the remainder of the current |\ekvset| macro and its argument list
%   and reinserts \meta{after}. So this can be used to break out of |\ekvset|.
%   The first variant will also gobble anything that has been sneaked out using
%   |\ekvsneak| or |\ekvsneakPre|, while |\ekvbreakPreSneak| will put
%   \meta{after} before anything that has been smuggled and |\ekvbreakPostSneak|
%   will put \meta{after} after the stuff that has been sneaked out.
% \end{function}
%
% \begin{function}{\ekvsneak,\ekvsneakPre}
%   \begin{syntax}
%     \cs{ekvsneak}\marg{after}
%   \end{syntax}
%   Puts \meta{after} after the effects of |\ekvset|. The first variant will put
%   \meta{after} after any other tokens which might have been sneaked before,
%   while |\ekvsneakPre| will put \meta{after} before other smuggled stuff.
%   This reads and reinserts the remainder of the current |\ekvset| macro and
%   its argument list to do its job. A small usage example is shown in
%   \autoref{sec:sneakex}.
% \end{function}
%
% \begin{function}{\ekvchangeset}
%   \begin{syntax}
%     \cs{ekvchangeset}\marg{new-set}
%   \end{syntax}
%   Replaces the current set with \meta{new-set}, so for the rest of the current
%   |\ekvset| call, that call behaves as if it was called with
%   \texttt{\cs[no-index]{ekvset}\marg{new-set}}. Just like |\ekvsneak| this
%   reads and reinserts the remainder of the current |\ekvset| macro to do its
%   job. It is comparable to using \texttt{\meta{key}/.cd} in \pkg{pgfkeys}.
% \end{function}
%
% \bigskip
%
% \begin{function}{\ekv@name,\ekv@name@set,\ekv@name@key}
%   \begin{syntax}
%     \cs{ekv@name}\marg{set}\marg{key}\\
%     \cs{ekv@name@set}\marg{set}\\
%     \cs{ekv@name@key}\marg{key}
%   \end{syntax}
%   The names of the macros that correspond to a key in a set are build with
%   these macros. The default definition of |\ekv@name@set| is
%   ``\texttt{\csname ekv@name@set\endcsname{\set}}'' and the default of
%   |\ekv@name@key| is ``\texttt{\csname ekv@name@key\endcsname{\key}}''.
%   The complete name is build using |\ekv@name| which is equivalent to
%   \texttt{\cs[no-index]{ekv@name@set}\marg{set}\cs{ekv@name@key}\marg{key}}.
%   For |NoVal| keys an additional |N| gets appended irrespective of these
%   macros' definition, so their name is
%   \cs[no-index]{\csname ekv@name\endcsname{\set}{\key}N}. You might redefine
%   |\ekv@name@set| and |\ekv@name@key| locally but \emph{don't redefine}
%   |\ekv@name|!
% \end{function}
%
% \subsubsection{Bugs}
%
% Just like \pkg{keyval}, \expkv\ is bug free. But if you find \pmso{bugs}hidden
% features\footnote{Thanks, David!} you can tell me about them either via mail
% (see the first page) or directly on GitHub if you have an account there:
% \url{https://github.com/Skillmon/tex_expkv}
%
% \subsubsection{Comparisons}
%
% Comparisons of speed are done with a very simple test key and the help of the
% \pkg{l3benchmark} package. The key and its usage should be equivalent to
% \begin{lstlisting}
% \protected\ekvdef{test}{height}{\def\myheight{#1}}
% \ekvset{test}{ height = 6 }
% \end{lstlisting}
% and only the usage of the key, not its definition, is benchmarked. For the
% impatient, the essence of these comparisons regarding speed and buggy
% behaviour is contained in \autoref{tab:comp}.
%
% As far as I know \expkv\ is the only fully expandable \kv\ parser. I tried to
% compare \expkv\ to every \kv\ package listed on
% \href{https://ctan.org/topic/keyval}{CTAN}, however, one might notice that
% some of those are missing from this list. That's because I didn't get the
% others to work due to bugs, or because they just provide wrappers around other
% packages in this list.
%
% In this subsubsection is no benchmark of |\ekvparse| and |\keyval_parse:NNn|
% contained, as most other packages don't provide equivalent features to my
% knowledge. |\ekvparse| is slightly faster than |\ekvset|, but keep in mind
% that it does less. The same is true for |\keyval_parse:NNn| compared to
% |\keys_set:nn| of \pkg{expl3} (where the difference is much bigger).
%
% \paragraph{\pkg{keyval}} is about two times faster and has a comparable
% feature set just a slightly different way how it handles keys without values.
% That might be considered a drawback, as it limits the versatility, but also as
% an advantage, as it might reduce doubled code. Keep in mind that as soon as
% someone loads \pkg{xkeyval} the performance of \pkg{keyval} gets replaced by
% \pkg{xkeyval}'s.
%
% Also \pkg{keyval} has a bug, which unfortunately can't really be resolved
% without breaking backwards compatibility for \emph{many} documents, namely it
% strips braces from the argument before stripping spaces if the argument isn't
% surrounded by spaces, also it might strip more than one set of braces. Hence
% all of the following are equivalent in their outcome, though the last two
% lines should result in something different than the first two:
% \begin{lstlisting}[belowskip=0pt]
% \setkeys{foo}{bar=baz}
% \setkeys{foo}{bar= {baz}}
% \setkeys{foo}{bar={ baz}}
% \setkeys{foo}{bar={{baz}}}
% \end{lstlisting}
%
% \paragraph{\pkg{xkeyval}} is roughly fourteen times slower, but it provides
% more functionality, e.g., it has choice keys, boolean keys, and so on. It
% contains the same bug as \pkg{keyval} as it has to be compatible with it by
% design (it replaces \pkg{keyval}'s frontend), but also adds even more cases in
% which braces are stripped that shouldn't be stripped, worsening the situation.
%
% \paragraph{\pkg{ltxkeys}} is over 300 times slower -- which is funny, because
% it aims to be ``[\ldots] faster than these earlier packages [referring to
% \pkg{keyval} and \pkg{xkeyval}].'' Since it aims to have a bigger feature set
% than \pkg{xkeyval}, it most definitely also has a bigger feature set than
% \expkv. Also, it can't parse |\long| input, so as soon as your values contain
% a |\par|, it'll throw errors. Furthermore, \pkg{ltxkeys} doesn't strip outer
% braces at all by design, which, imho, is a weird design choice. In addition
% \pkg{ltxkeys} loads \pkg{catoptions} which is known to introduce bugs (e.g.,
% see \url{https://tex.stackexchange.com/questions/461783}).
%
% \paragraph{\pkg{l3keys}} is almost five times slower, but has an, imho, great
% interface to define keys. It strips \emph{all} outer spaces, even if somehow
% multiple spaces ended up on either end. It offers more features, but is pretty
% much bound to \pkg{expl3} code. Whether that's a drawback is up to you.
%
% \paragraph{\pkg{pgfkeys}} is a bit more than two times slower for one key,
% but has an \emph{enormous} feature set. However, since adding additional keys
% doesn't add as much needed time for \pkg{pgfkeys} compared to \expkv, it gets
% faster than \expkv\ at around eight \kv\ pairs.
% It has the same or a very similar bug \pkg{keyval} has. The brace bug (and
% also the category fragility) can be fixed by \pkg{pgfkeyx}, but this package
% was last updated in 2012 and it slows down |\pgfkeys| by factor~$8$. Also I
% don't know whether this might introduce new bugs.
%
% \paragraph{\pkg{kvsetkeys} with \pkg{kvdefinekeys}} is about three times
% slower, but it works even if commas and equals have category codes different
% from 12 (just as some other packages in this list). Else the features of the
% keys are equal to those of \pkg{keyval}, the parser has more features, though.
%
% \paragraph{\pkg{options}} is a bit slower for only a single value, but gets a
% tad faster than \expkv\ at around 10 \kv\ pairs. It has a much bigger feature
% set. Unfortunately it also suffers from the premature unbracing bug
% \pkg{keyval} has.
%
% \paragraph{\pkg{simplekv}} is hard to compare because I don't speak French (so
% I don't understand the documentation) and from what I can see, there is no
% direct way to define the equivalent test key. Nevertheless, I tested the
% closest possible equivalent of my test key while siding for \pkg{simplekv}'s
% design not forcing something into it it doesn't seem to be designed for. It is
% more than five times slower and has hard to predict behaviour regarding brace
% and space stripping, similar to \pkg{keyval}. The tested definition was:
% \begin{lstlisting}[belowskip=0pt]
% \usepackage{simplekv}
% \setKVdefault[simplekv]{height={ abc}}                     % key setup
% \setKV[simplekv]{ height = 6 }                             % benchmarked
% \end{lstlisting}
%
% \paragraph{\pkg{yax}} is over eighteen times slower. It has a pretty strange
% syntax, imho, and again a direct equivalent is hard to define. It has the
% premature unbracing bug, too. Also somehow loading \pkg{yax} broke
% \pkg{options} for me. The tested definition was:
% \begin{lstlisting}[belowskip=0pt]
% \usepackage{yax}
% \defactiveparameter yax {\storevalue\myheight yax:height } % key setup
% \setparameterlist{yax}{ height = 6 }                       % benchmarked
% \end{lstlisting}
%
% \begin{table}
%   \def\fnsym{\textcolor{red!80!black}{*}}%
%   \sisetup{round-precision=1, round-mode=places}%
%   \begingroup
%   \centering
%   \newcommand*\yes{\textcolor{red!80!black}  {yes}}^^A
%   \newcommand*\no {\textcolor{green!80!black}{no}}^^A
%   \caption[Comparison of \kv\ packages]
%     {^^A
%       Comparison of \kv\ packages. The packages are ordered from
%       fastest to slowest for one \kv\ pair. Benchmarking was done using
%       \pkg{l3benchmark} and the scripts in the \file{Benchmarks} folder
%       of the \href{https://github.com/Skillmon/tex_expkv}{git repository}.
%       The columns $p_i$ are the polynomial coefficients of a linear fit to the
%       run-time, $p_0$ can be interpreted as the overhead for initialisation
%       and $p_1$ the cost per key. The $T_0$ column is the actual mean ops
%       needed for an empty list argument, as the linear fit doesn't match that
%       point well in general. The column ``BB'' lists whether the
%       parsing is affected by some sort of brace bug, ``CF'' stands for
%       category code fragile and lists whether the parsing breaks with active
%       commas or equal signs.^^A
%       \label{tab:comp}^^A
%     }
%   \begin{tabular}
%       {>{\collectcell\pkg}l<{\endcollectcell}*3{S[table-format=4.1]}ccc}
%     \toprule
%     \rmfamily Package & {$p_1$} & {$p_0$} & {$T_0$}& BB & CF & Date \\
%     \midrule
%     keyval    &   13.539 &    1.844 &    6.969 & \yes & \yes & 2014-10-28 \\
%     \expkv    &   26.411 &    2.391 &   12.610 & \no  & \no  & 2020-02-22 \\
%     options   &   25.532 &   10.518 &   21.393 & \yes & \yes & 2015-03-01 \\
%     pgfkeys   &   26.444 &   40.939 &   55.790 & \yes & \yes & 2020-01-08 \\
%     kvsetkeys & {\fnsym} & {\fnsym} &   41.620 & \no  & \no  & 2019-12-15 \\
%     l3keys    &  114.456 &   35.357 &   52.667 & \no  & \no  & 2020-02-14 \\
%     simplekv  &  160.523 &   10.809 &    8.698 & \yes & \yes & 2017-08-08 \\
%     xkeyval   &  260.872 &  180.435 &  164.167 & \yes & \yes & 2014-12-03 \\
%     yax       &  507.643 &   62.863 &  123.533 & \yes & \yes & 2010-01-22 \\
%     ltxkeys   & 3932.759 & 4737.818 & 5883.000 & \no  & \no  & 2012-11-17 \\
%     \bottomrule
%   \end{tabular}
%   \par
%   \endgroup
%   \medskip
%   \fnsym For \pkg{kvsetkeys} the linear model used for the other
%   packages is a poor fit, \pkg{kvsetkeys} seems to have approximately
%   quadratic run-time, the coefficients of the second degree polynomial fit are
%   $p_2=\num{9.911}$, $p_1=\num{40.523}$, and $p_0=\num{61.500}$. Of course the
%   other packages might not really have linear run-time, but at least from 1~to
%   20~keys the fits don't seem too bad (the maximum ratio $p_2/p_1$
%   for the other packages is \num{9.697e-3}). If one extrapolates the fits
%   for 100 \kv\ pairs one finds that most of them match pretty well, the
%   exception being \pkg{ltxkeys}, which behaves quadratic as well with
%   $p_2=\num{31.685}$, $p_1=\num{3267.368}$, and $p_0=\num{7177.582}$.
% \end{table}
%
% \subsection{Examples}
%
% \subsubsection{Standard Use-Case}
%
% Say we have a macro for which we want to create a \kv\ interface. The macro
% has a parameter, which is stored in the dimension |\ourdim| having a default
% value from its initialization. Now we want to be able to change that dimension
% with the |width| key to some specified value. For that we'd do
% \begin{lstlisting}
% \newdimen\ourdim
% \ourdim=150pt
% \protected\ekvdef{our}{width}{\ourdim=#1\relax}
% \end{lstlisting}
% as you can see, we use the set |our| here. We want the key to behave different
% if no value is specified. In that case the key should not use its initial
% value, but be smart and determine the available space from
% \cs[no-index]{hsize}, so we also define
% \begin{lstlisting}
% \protected\ekvdefNoVal{our}{width}{\ourdim=.9\hsize}
% \end{lstlisting}
% Now we set up our macro to use this \kv\ interface
% \begin{lstlisting}
% \protected\def\ourmacro#1{\begingroup\ekvset{our}{#1}\the\ourdim\endgroup}
% \end{lstlisting}
% Finally we can use our macro like in the following\\
% \begin{minipage}{6cm}
% \begin{lstlisting}
% \ourmacro{}\par
% \ourmacro{width}\par
% \ourmacro{width=5pt}\par
% \end{lstlisting}
% \end{minipage}^^A
% \begin{minipage}{\dimexpr\linewidth-6cm\relax}
%   \newdimen\ourdim\ourdim=150pt
%   \protected\ekvdef{our}{width}{\ourdim=#1\relax}^^A
%   \protected\ekvdefNoVal{our}{width}{\ourdim=.9\hsize}^^A
%   \protected\def\ourmacro#1^^A
%     {\begingroup\ekvset{our}{#1}\the\ourdim\endgroup}^^A
%   \ourmacro{}\par
%   \ourmacro{width}\par
%   \ourmacro{width=5pt}\par
% \end{minipage}
%
% \paragraph{The same key using \protect\expkvd}
% Using \expkvd\ we can set up the equivalent key using a \kv\ interface, after
% the following we could use |\ourmacro| in the same way as above. \expkvd\ will
% allocate and initialise |\ourdim| and define the |width| key |\protected| for
% us, so the result will be exactly the same -- with the exception that the
% default will use |\ourdim=.9\hsize\relax| instead.
% \begin{lstlisting}
% \input expkv-def                 % or \usepackage{expkv-def}
% \ekvdefinekeys{our}
%   {
%     dimen    width = \ourdim,
%     qdefault width = .9\hsize,
%     initial  width = 150pt
%   }
% \end{lstlisting}
%
% \subsubsection{An Expandable \kv\ Macro Using \cs[no-index]{ekvsneak}}
% \label{sec:sneakex}
%
% Let's set up an expandable macro, that uses a \kv\ interface. The problems
% we'll face for this are:
% \begin{enumerate}
%   \item ignoring duplicate keys
%   \item default values for keys which weren't used
%   \item providing the values as the correct argument to a macro (ordered)
% \end{enumerate}
%
% First we need to decide which \kv\ parsing macro we want to do this with,
% |\ekvset| or |\ekvparse|. For this example we also want to show the
% usage of |\ekvsneak|, hence we'll choose |\ekvset|. And we'll have to use
% |\ekvset| such that it builds a parsable list for our macro internals. To
% gain back control after |\ekvset| is done we have to put an internal of our
% macro at the start of that list, so we use an internal key that uses
% |\ekvsneakPre| after any user input.
%
% To ignore duplicates will be easy if the value of the key used last will be
% put first in the list, so the following will use |\ekvsneakPre| for the
% user-level keys. If we wanted some key for which the first usage should be the
% the binding one we would use |\ekvsneak| instead for that key.
%
% Providing default values can be done in different ways, we'll use a simple
% approach in which we'll just put the outcome of our keys if they were used
% with default values before the parsing list terminator.
%
% Ordering the keys can be done simply by searching for a specific token for
% each argument which acts like a flag, so our sneaked out values will include
% specific tokens acting as markers.
%
% Now that we have answers for our technical problems, we have to decide what
% our example macro should do. How about we define a macro that calculates the
% sine of a number and rounds that to a specified precision? As a small extra
% this macro should understand input in radian and degree and the used
% trigonometric function should be selectable as well. For the hard part of this
% task (expandably evaluating trigonometric functions) we'll use the \pkg{xfp}
% package.
%
% First we set up our keys according to our earlier considerations and set up
% the user facing macro |\sine|. The end marker of the parsing list will be a
% |\sine@stop| token, which we don't need to define and we put our defaults
% right before it.
%
% \begin{lstlisting}
% \RequirePackage{xfp}
% \makeatletter
% \ekvdef{expex}{f}{\ekvsneakPre{\f{#1}}}
% \ekvdef{expex}{round}{\ekvsneakPre{\rnd{#1}}}
% \ekvdefNoVal{expex}{degree}{\ekvsneakPre{\deg{d}}}
% \ekvdefNoVal{expex}{radian}{\ekvsneakPre{\deg{}}}
% \ekvdefNoVal{expex}{internal}{\ekvsneakPre{\sine@rnd}}
% \newcommand*\sine[2]
%   {\ekvset{expex}{#1,internal}\rnd{3}\deg{d}\f{sin}\sine@stop{#2}}
% \end{lstlisting}
% For the sake of simplicity we defined the macro |\sine| with two mandatory
% arguments, the first being the \kv\ list, the second the argument to the
% trigonometric function. We could've used \pkg{xparse}'s facilities here to
% define an expandable macro which takes an optional argument instead.
%
% Now we need to define some internal macros to extract the value of each key's
% last usage (remember that this will be the group after the first special
% flag-token). For that we use one delimited macro per key.
% \begin{lstlisting}
% \def\sine@rnd#1\rnd#2#3\sine@stop{\sine@deg#1#3\sine@stop{#2}}
% \def\sine@deg#1\deg#2#3\sine@stop{\sine@f#1#3\sine@stop{#2}}
% \def\sine@f#1\f#2#3\sine@stop{\sine@final{#2}}
% \end{lstlisting}
% After the macros |\sine@rnd|, |\sine@deg|, and |\sine@f| the macro
% |\sine@final| will see
% \texttt
%   {^^A
%     \cs[no-index]{sine@final}^^A
%     \marg{f}\marg{degree/radian}\marg{round}\marg{num}^^A
%   }.
% Now |\sine@final| has to expandably deal with those arguments such that the
% |\fpeval| macro of \pkg{xfp} gets the correct input. Luckily this is pretty
% straight forward in this example. In |\fpeval| the trigonometric functions
% have names such as |sin| or |cos| and the degree taking variants |sind| or
% |cosd|. And since the |degree| key puts a |d| in |#2| and the |radian| key
% leaves |#2| empty all we have to do to get the correct function name is stick
% the two together.
% \begin{lstlisting}
% \newcommand*\sine@final[4]{\fpeval{round(#1#2(#4),#3)}}
% \makeatother
% \end{lstlisting}
% Let's test our macro:\\
% \begin{minipage}{.7\linewidth}
%   \begin{lstlisting}[gobble=4]
%   \sine{}{60}\par
%   \sine{round=10}{60}\par
%   \sine{f=cos,radian}{pi}\par
%   \edef\myval{\sine{f=tan}{1}}\texttt{\meaning\myval}
%   \end{lstlisting}
% \end{minipage}\hfill
% \begin{minipage}{.2\linewidth}
%   \makeatletter
%   \ekvdef{expex}{f}{\ekvsneakPre{\f{#1}}}^^A
%   \ekvdef{expex}{round}{\ekvsneakPre{\rnd{#1}}}^^A
%   \ekvdefNoVal{expex}{degree}{\ekvsneakPre{\deg{d}}}^^A
%   \ekvdefNoVal{expex}{radian}{\ekvsneakPre{\deg{}}}^^A
%   \ekvdefNoVal{expex}{internal}{\ekvsneakPre{\sine@rnd}}^^A
%   \newcommand*\sine[2]
%     {\ekvset{expex}{#1,internal}\rnd{3}\deg{d}\f{sin}\sine@stop{#2}}^^A
%   \def\sine@rnd#1\rnd#2#3\sine@stop{\sine@deg#1#3\sine@stop{#2}}^^A
%   \def\sine@deg#1\deg#2#3\sine@stop{\sine@f#1#3\sine@stop{#2}}^^A
%   \def\sine@f#1\f#2#3\sine@stop{\sine@final{#2}}^^A
%   \newcommand*\sine@final[4]{\fpeval{round(#1#2(#4),#3)}}^^A
%   \makeatother
%   \sine{}{60}\par
%   \sine{round=10}{60}\par
%   \sine{f=cos,radian}{pi}\par
%   \edef\myval{\sine{f=tan}{1}}\texttt{\meaning\myval}
% \end{minipage}
%
%
% \subsection{Error Messages}
%
% \expkv\ should only send messages in case of errors, there are no warnings and
% no info messages. In this subsection those errors are listed.
%
% \subsubsection{Load Time}
%
% \file{expkv.tex} checks whether \eTeX\ is available. If it isn't, an error
% will be thrown using |\errmessage|:
% \begin{lstlisting}
% ! expkv Error: e-TeX required.
% \end{lstlisting}
%
% \subsubsection{Defining Keys}
%
% If you get any error from \expkv\ while you're trying to define a key, the
% definition will be aborted and gobbled.
%
% If you try to define a key with an empty set name you'll get:
% \begin{lstlisting}
% ! expkv Error: empty set name not allowed.
% \end{lstlisting}
%
% Similarly, if you try to define a key with an empty key name:
% \begin{lstlisting}
% ! expkv Error: empty key name not allowed.
% \end{lstlisting}
%
% Both of these messages are done in a way that doesn't throw additional errors
% due to |\global|, |\long|, etc., not being used correctly if you prefixed one
% of the defining macros.
%
% \subsubsection{Using Keys}
%
% This subsubsection contains the errors thrown during |\ekvset|. The errors are
% thrown in an expandable manner by providing an undefined macro. In the
% following messages \lstinline|<key>| gets replaced with the problematic key's
% name, and \lstinline|<set>| with the corresponding set. If any errors during
% \kv\ handling are encountered, the entry in the comma separated list will be
% omitted after the error is thrown and the next \kv\ pair will be parsed.
%
% If you're using an undefined key you'll get:
% \begin{lstlisting}
% ! Undefined control sequence.
% <argument> \! expkv Error:
%                            unknown key (`<key>', set `<set>').
% \end{lstlisting}
%
% If you're using a key for which only a normal version and no |NoVal| version
% is defined, but don't provide a value, you'll get:
% \begin{lstlisting}
% ! Undefined control sequence.
% <argument> \! expkv Error:
%                            value required (`<key>', set `<set>').
% \end{lstlisting}
%
% If you're using a key for which only a |NoVal| version and no normal version
% is defined, but provide a value, you'll get:
% \begin{lstlisting}
% ! Undefined control sequence.
% <argument> \! expkv Error:
%                            value forbidden (`<key>', set `<set>').
% \end{lstlisting}
%
% If you're using a set for which you never executed one of the defining macros
% from \autoref{sec:define} you'll get a low level \TeX\ error, as that isn't
% actively tested by the parser (and hence will lead to undefined behaviour and
% not be gracefully ignored). The error will look like
% \begin{lstlisting}
% ! Missing \endcsname inserted.
% <to be read again>
%                    \! expkv Error: Set `<set>' undefined.
% \end{lstlisting}
%
% \subsection{License}
%
% Copyright \textcopyright\ 2020\unless\ifnum\year=2020--\the\year\fi\
% Jonathan P. Spratte
%
% \medskip\noindent
% This work may be distributed and/or modified under the conditions of the
% \LaTeX\ Project Public License (LPPL), either version 1.3c of this license or
% (at your option) any later version. The latest version of this license is in
% the file:
%
% \url{http://www.latex-project.org/lppl.txt}
%
% \noindent
% This work is ``maintained'' (as per LPPL maintenance status) by
%
% \mbox{Jonathan P. Spratte}.
%
% \end{documentation}^^A=<<
%
% \begin{implementation}^^A>>=
%
% \clearpage
%
% \section{Implementation}^^A>>=
%^^A the LaTeX package >>=
% \subsection{The \LaTeX\ Package}
% First we set up the \LaTeX\ package. That one doesn't really do much except
% |\input|ting the generic code and identifying itself as a package.
% \gobbledocstriptag
%<*pkg>
%    \begin{macrocode}
\def\ekv@tmp
  {%
    \ProvidesFile{expkv.tex}%
      [\ekvDate\space v\ekvVersion\space an expandable key=val implementation]%
  }
\input{expkv.tex}
\ProvidesPackage{expkv}%
  [\ekvDate\space v\ekvVersion\space an expandable key=val implementation]
%    \end{macrocode}
% \gobbledocstriptag
%</pkg>
%^^A=<<
%^^A main file >>=
% \subsection{The Generic Code}
% The rest of this implementation will be the generic code.
% \gobbledocstriptag
%<*tex>
%
% Check whether \eTeX\ is available -- \expkv\ requires \eTeX.
%    \begin{macrocode}
\begingroup\expandafter\expandafter\expandafter\endgroup
\expandafter\ifx\csname numexpr\endcsname\relax
  \errmessage{expkv requires e-TeX}
  \expandafter\endinput
\fi
%    \end{macrocode}
%
% We make sure that it's only input once:
%    \begin{macrocode}
\expandafter\ifx\csname ekvVersion\endcsname\relax
\else
  \expandafter\endinput
\fi
%    \end{macrocode}
% \begin{macro}{\ekvVersion,\ekvDate}
% We're on our first input, so lets store the version and date in a macro.
%    \begin{macrocode}
\def\ekvVersion{0.5a}
\def\ekvDate{2020-02-27}
%    \end{macrocode}
% \end{macro}
%
% If the \LaTeX\ format is loaded we want to be a good file and report back who
% we are, for this the package will have defined |\ekv@tmp| to use
% |\ProvidesFile|, else this will expand to a |\relax| and do no harm.
%    \begin{macrocode}
\csname ekv@tmp\endcsname
%    \end{macrocode}
%
% Store the category code of |@| to later be able to reset it and change it to
% 11 for now.
%    \begin{macrocode}
\expandafter\chardef\csname ekv@tmp\endcsname=\catcode`\@
\catcode`\@=11
%    \end{macrocode}
% |\ekv@tmp| might later be reused to gobble any prefixes which might be
% provided to |\ekvdef| and similar in case the names are invalid, we just
% temporarily use it here as means to store the current category code of |@| to
% restore it at the end of the file, we never care for the actual definition of
% it.
%
% \begin{macro}[internal]
%   {
%     \@gobble,\@firstofone,\@firstoftwo,\@secondoftwo,
%     \ekv@gobbleto@stop,\ekv@fi@secondoftwo,\ekv@gobble@mark
%   }
% Since branching tests are often more versatile than |\if...\else...\fi|
% constructs, we define helpers that are branching pretty fast. Also here are
% some other utility functions that just grab some tokens. The ones that are
% also contained in \LaTeX\ don't use the |ekv| prefix.
%    \begin{macrocode}
\long\def\@gobble#1{}
\long\def\@firstofone#1{#1}
\long\def\@firstoftwo#1#2{#1}
\long\def\@secondoftwo#1#2{#2}
\long\def\ekv@fi@secondoftwo\fi\@firstoftwo#1#2{\fi#2}
\long\def\ekv@gobbleto@stop#1\ekv@stop{}
\def\ekv@gobble@mark\ekv@mark{}
%    \end{macrocode}
% \end{macro}
% As you can see |\ekv@gobbleto@stop| uses a special marker |\ekv@stop|. The
% package will use three such markers, the one you've seen already, |\ekv@mark|
% and |\ekv@nil|. Contrarily to how for instance \pkg{expl3} does things, we
% don't define them, as we don't need them to have an actual meaning. This has
% the advantage that if they somehow get expanded -- which should never happen
% if things work out -- they'll throw an error directly.
%
% \begin{macro}[internal]
%   {
%     \ekv@ifempty,\ekv@ifempty@,\ekv@ifempty@true,\ekv@ifempty@false,
%     \ekv@ifempty@true@F
%   }
% We can test for a lot of things building on an if-empty test, so lets define a
% really fast one. Since some tests might have reversed logic (true if something
% is not empty) we also set up macros for the reversed branches.
%    \begin{macrocode}
\long\def\ekv@ifempty#1%
  {%
    \ekv@ifempty@\ekv@ifempty@A#1\ekv@ifempty@B\ekv@ifempty@true
      \ekv@ifempty@A\ekv@ifempty@B\@secondoftwo
  }
\long\def\ekv@ifempty@#1\ekv@ifempty@A\ekv@ifempty@B{}
\long\def\ekv@ifempty@true\ekv@ifempty@A\ekv@ifempty@B\@secondoftwo#1#2{#1}
\long\def\ekv@ifempty@false\ekv@ifempty@A\ekv@ifempty@B\@firstoftwo#1#2{#2}
\long\def\ekv@ifempty@true@F\ekv@ifempty@A\ekv@ifempty@B\@firstofone#1{}
%    \end{macrocode}
% \end{macro}
%
% \begin{macro}[internal]{\ekv@ifblank@}
% The obvious test that can be based on an if-empty is if-blank, meaning a test
% checking whether the argument is empty or consists only of spaces. Our version
% here will be tweaked a bit, as we want to check this, but with one leading
% |\ekv@mark| token that is to be ignored.
%    \begin{macrocode}
\long\def\ekv@ifblank@\ekv@mark#1{\ekv@ifempty@\ekv@ifempty@A}
%    \end{macrocode}
% \end{macro}
%
% \begin{macro}[internal]{\ekv@ifdefined}
% We'll need to check whether something is defined quite frequently, so why not
% define a macro that does this. The following test is expandable, slower than
% the typical expandable test for undefined control sequences, but faster for
% defined ones. Since we want to be as fast as possible for correct input, this
% is to be preferred.
%    \begin{macrocode}
\def\ekv@ifdefined#1%
  {%
    \expandafter
    \ifx\csname\ifcsname #1\endcsname #1\else relax\fi\endcsname\relax
      \ekv@fi@secondoftwo
    \fi
    \@firstoftwo
  }
%    \end{macrocode}
% \end{macro}
  % \begin{macro}[internal]
%   {
%     \ekv@ifdefined@pair,\ekv@ifdefined@pair@,\ekv@ifdefined@key,
%     \ekv@ifdefined@key@
%   }
% Since we can save some time if we only have to create the control sequence
% once when we know beforehand how we want to use it, we build some other macros
% for those cases (which we'll have quite often, once per key usage).
%    \begin{macrocode}
\def\ekv@ifdefined@pair#1#2%
  {%
    \expandafter\ekv@ifdefined@pair@
      \csname
        \ifcsname #1{#2}\endcsname
          #1{#2}%
        \else
          relax%
        \fi
      \endcsname
  }
\def\ekv@ifdefined@pair@#1%
  {%
    \ifx#1\relax
      \ekv@fi@secondoftwo
    \fi
    \@firstoftwo
    {\ekv@set@pair@#1\ekv@mark}%
  }
\def\ekv@ifdefined@key#1#2%
  {%
    \expandafter\ekv@ifdefined@key@
      \csname
        \ifcsname #1{#2}N\endcsname
          #1{#2}N%
        \else
          relax%
        \fi
      \endcsname
  }
\def\ekv@ifdefined@key@#1%
  {%
    \ifx#1\relax
      \ekv@fi@secondoftwo
    \fi
    \@firstoftwo#1%
  }
%    \end{macrocode}
% \end{macro}
%
% \begin{macro}{\ekv@name,\ekv@name@set,\ekv@name@key}
% The keys will all follow the same naming scheme, so we define it here.
%    \begin{macrocode}
\def\ekv@name#1#2{\ekv@name@set{#1}\ekv@name@key{#2}}
\def\ekv@name@set#1{ekv#1(}
\def\ekv@name@key#1{#1)}
%    \end{macrocode}
% \end{macro}
% \begin{macro}[internal]{\ekv@undefined@set}
% We can misuse the macro name we use to expandably store the set-name in a
% single token -- since this increases performance drastically, especially for
% long set-names -- to throw a more meaningful error message in case a set isn't
% defined. The name of |\ekv@undefined@set| is a little bit misleading, as it is
% called in either case inside of |\csname|, but the result will be a control
% sequence with meaning |\relax| if the set is undefined, hence will break the
% |\csname| building the key-macro which will throw the error message.
%    \begin{macrocode}
\def\ekv@undefined@set#1{! expkv Error: Set `#1' undefined.}
%    \end{macrocode}
% \end{macro}
% \begin{macro}[internal]{\ekv@checkvalid}
% We place some restrictions on the allowed names, though, namely sets and
% keys are not allowed to be empty -- blanks are fine (meaning \mbox{set-
% or} key-names consisting of spaces).
%    \begin{macrocode}
\protected\def\ekv@checkvalid#1#2%
  {%
    \ekv@ifempty{#1}%
      {%
        \def\ekv@tmp{}%
        \errmessage{expkv Error: empty set name not allowed}%
      }%
      {%
        \ekv@ifempty{#2}%
          {%
            \def\ekv@tmp{}%
            \errmessage{expkv Error: empty key name not allowed}%
          }%
          \@secondoftwo
      }%
    \@gobble
  }
%    \end{macrocode}
% \end{macro}
% \begin{macro}{\ekvifdefined,\ekvifdefinedNoVal}
% And provide user-level macros to test whether a key is defined.
%    \begin{macrocode}
\def\ekvifdefined#1#2{\ekv@ifdefined{\ekv@name{#1}{#2}}}
\def\ekvifdefinedNoVal#1#2{\ekv@ifdefined{\ekv@name{#1}{#2}N}}
%    \end{macrocode}
% \end{macro}
%
% \begin{macro}
%   {\ekvdef,\ekvdefNoVal,\ekvlet,\ekvletNoVal,\ekvletkv,\ekvletkvNoVal}
% Set up the key defining macros |\ekvdef| etc.
%    \begin{macrocode}
\protected\long\def\ekvdef#1#2#3%
  {%
    \ekv@checkvalid{#1}{#2}%
      {%
        \expandafter\def\csname\ekv@name{#1}{#2}\endcsname##1{#3}%
        \ekv@defset{#1}%
      }%
  }
\protected\long\def\ekvdefNoVal#1#2#3%
  {%
    \ekv@checkvalid{#1}{#2}%
      {%
        \expandafter\def\csname\ekv@name{#1}{#2}N\endcsname{#3}%
        \ekv@defset{#1}%
      }%
  }
\protected\def\ekvlet#1#2#3%
  {%
    \ekv@checkvalid{#1}{#2}%
      {%
        \expandafter\let\csname\ekv@name{#1}{#2}\endcsname#3%
        \ekv@defset{#1}%
      }%
  }
\protected\def\ekvletNoVal#1#2#3%
  {%
    \ekv@checkvalid{#1}{#2}%
      {%
        \expandafter\let\csname\ekv@name{#1}{#2}N\endcsname#3%
        \ekv@defset{#1}%
      }%
  }
\protected\def\ekvletkv#1#2#3#4%
  {%
    \ekv@checkvalid{#1}{#2}%
      {%
        \expandafter\let\csname\ekv@name{#1}{#2}\expandafter\endcsname
        \csname\ekv@name{#3}{#4}\endcsname
        \ekv@defset{#1}%
      }%
  }
\protected\def\ekvletkvNoVal#1#2#3#4%
  {%
    \ekv@checkvalid{#1}{#2}%
      {%
        \expandafter\let\csname\ekv@name{#1}{#2}N\expandafter\endcsname
        \csname\ekv@name{#3}{#4}N\endcsname
        \ekv@defset{#1}%
      }%
  }
%    \end{macrocode}
% \end{macro}
% \begin{macro}[internal]{\ekv@defset}
%   In order to enhance the speed the set name given to |\ekvset| will be turned
%   into a control sequence pretty early, so we have to define that control
%   sequence.
%    \begin{macrocode}
\protected\def\ekv@defset#1%
  {%
    \expandafter\edef\csname\ekv@undefined@set{#1}\endcsname##1%
      {\ekv@name@set{#1}\ekv@name@key{##1}}%
  }
%    \end{macrocode}
% \end{macro}
%
% \begin{macro}{\ekvset}
% Set up |\ekvset|, which should not be affected by active commas and equal
% signs. The equal signs are a bit harder to cope with and we'll do that later,
% but replacing the active commas with commas of category other can be done
% beforehand. That's why we define |\ekvset| here with a temporary meaning just
% to set up the things with two different category codes. |#1| will be a
% \texttt{,\textsubscript{13}} and |#2| will be a \texttt{=\textsubscript{13}}.
%    \begin{macrocode}
\def\ekvset#1#2{%
\endgroup
\long\def\ekvset##1##2%
  {%
    \expandafter\ekv@set\csname\ekv@undefined@set{##1}\endcsname
      \ekv@mark##2#1\ekv@stop#1{}%
  }
%    \end{macrocode}
% \end{macro}
% \begin{macro}[internal]{\ekv@set}
% |\ekv@set| will split the \kv\ list at active commas. Then it has to check
% whether there were unprotected other commas and resplit there.
%    \begin{macrocode}
\long\def\ekv@set##1##2#1%
  {%
%    \end{macrocode}
% Test whether we're at the end, if so invoke |\ekv@endset|,
%    \begin{macrocode}
    \ekv@gobbleto@markstop##2\ekv@endset\ekv@mark\ekv@stop
%    \end{macrocode}
% else go on with other commas,
%    \begin{macrocode}
    \ekv@set@other##1##2,\ekv@stop,%
%    \end{macrocode}
% and get the next active comma delimited \kv\ pair.
%    \begin{macrocode}
    \ekv@set##1\ekv@mark
  }
%    \end{macrocode}
% \end{macro}
% \begin{macro}[internal]{\ekv@endset}
% |\ekv@endset| is a hungry little macro. It will eat everything that remains
% of |\ekv@set| and unbrace the sneaked stuff.
%    \begin{macrocode}
\long\def\ekv@endset
    \ekv@mark\ekv@stop\ekv@set@other##1,\ekv@stop,\ekv@set##2\ekv@mark
    ##3%
  {##3}
%    \end{macrocode}
% \end{macro}
% \begin{macro}[internal]{\ekv@set@other}
% The macro |\ekv@set@other| is guaranteed to get only single \kv\ pairs.
% First we test whether we're done, if not split at equal signs. It is faster to
% first split at category 12 equal signs and only after that on actives.
% If there is no equal sign, we need to test whether we got a blank argument and
% if not this is a |NoVal| key.
%    \begin{macrocode}
\long\def\ekv@set@other##1##2,%
  {%
    \ekv@gobbleto@markstop##2\ekv@endset@other\ekv@mark\ekv@stop
    \ekv@ifhas@eq@other##2=\ekv@ifempty@B\ekv@ifempty@false
      \ekv@ifempty@A\ekv@ifempty@B\@firstoftwo
      {\ekv@set@eq@other##1##2\ekv@stop}%
      {%
        \ekv@ifhas@eq@active##2#2\ekv@ifempty@B\ekv@ifempty@false
          \ekv@ifempty@A\ekv@ifempty@B\@firstoftwo
          {\ekv@set@eq@active##1##2\ekv@stop}%
          {%
            \ekv@ifblank@##2\ekv@nil\ekv@ifempty@B\ekv@ifempty@true@F
              \ekv@ifempty@A\ekv@ifempty@B\@firstofone
              {\ekv@strip{##2}\ekv@set@key##1}%
          }%
      }%
    \ekv@set@other##1\ekv@mark%
  }
%    \end{macrocode}
% \end{macro}
% \begin{macro}[internal]{\ekv@set@eq@other}
% |\ekv@set@eq@other| might not be the correct break point, there might be an
% active equal sign in the currently parsed key-name. If so, we have to resplit.
% If the split is correct strip the key-name of outer spaces and braces and feed
% it to |\ekv@set@pair|.
%    \begin{macrocode}
\long\def\ekv@set@eq@other##1##2=%
  {%
    \ekv@ifhas@eq@active##2#2\ekv@ifempty@B\ekv@ifempty@false
      \ekv@ifempty@A\ekv@ifempty@B\@firstoftwo
      {\ekv@set@eq@active##1##2=}%
      {\ekv@strip{##2}\ekv@set@pair##1}%
  }
%    \end{macrocode}
% \end{macro}
% \begin{macro}[internal]{\ekv@set@eq@active}
% We need to handle the active equal signs.
%    \begin{macrocode}
\long\def\ekv@set@eq@active##1##2#2%
  {%
    \ekv@strip{##2}\ekv@set@pair##1%
  }
%    \end{macrocode}
% \end{macro}
% \begin{macro}[internal]
%   {\ekv@ifhas@eq@other,\ekv@ifhas@eq@active,\ekv@endset@other}
% And we have to set up the testing macros for our equal signs and
% |\ekv@endset@other|.
%    \begin{macrocode}
\long\def\ekv@ifhas@eq@other\ekv@mark##1={\ekv@ifempty@\ekv@ifempty@A}
\long\def\ekv@ifhas@eq@active\ekv@mark##1#2{\ekv@ifempty@\ekv@ifempty@A}
\long\def\ekv@endset@other
    \ekv@mark\ekv@stop
    \ekv@ifhas@eq@other##1=\ekv@ifempty@B\ekv@ifempty@false
    \ekv@ifempty@A\ekv@ifempty@B\@firstoftwo
    ##2%
    \ekv@set@other##3\ekv@mark
  {}
%    \end{macrocode}
% \end{macro}
% \begin{macro}{\ekvbreak,\ekvbreakPreSneak,\ekvbreakPostSneak}
% Provide macros that can completely stop the parsing of |\ekvset|, who knows
% what it'll be useful for.
%    \begin{macrocode}
\long\def\ekvbreak##1##2\ekv@stop#1##3{##1}
\long\def\ekvbreakPreSneak ##1##2\ekv@stop#1##3{##1##3}
\long\def\ekvbreakPostSneak##1##2\ekv@stop#1##3{##3##1}
%    \end{macrocode}
% \end{macro}
% \begin{macro}{\ekvsneak,\ekvsneakPre}
% One last thing we want to do for |\ekvset| is to provide macros that just
% smuggle stuff after |\ekvset|'s effects.
%    \begin{macrocode}
\long\def\ekvsneak##1##2\ekv@stop#1##3%
  {%
    ##2\ekv@stop#1{##3##1}%
  }
\long\def\ekvsneakPre##1##2\ekv@stop#1##3%
  {%
    ##2\ekv@stop#1{##1##3}%
  }
%    \end{macrocode}
% \end{macro}
%
% \begin{macro}{\ekvparse}
% Additionally to the |\ekvset| macro we also want to provide an |\ekvparse|
% macro, that has the same scope as |\keyval_parse:NNn| from \pkg{expl3}.
% This is pretty analogue to the |\ekvset| implementation, we just put an
% |\unexpanded| here and there instead of other macros to stop the |\expanded|
% on our output.
%    \begin{macrocode}
\long\def\ekvparse##1##2##3%
  {%
    \ekv@parse##1##2\ekv@mark##3#1\ekv@stop#1%
  }
%    \end{macrocode}
% \end{macro}
% \begin{macro}[internal]{\ekv@parse}
%    \begin{macrocode}
\long\def\ekv@parse##1##2##3#1%
  {%
    \ekv@gobbleto@markstop##3\ekv@endparse\ekv@mark\ekv@stop
    \ekv@parse@other##1##2##3,\ekv@stop,%
    \ekv@parse##1##2\ekv@mark
  }
%    \end{macrocode}
% \end{macro}
% \begin{macro}[internal]{\ekv@endparse}
%    \begin{macrocode}
\long\def\ekv@endparse
    \ekv@mark\ekv@stop\ekv@parse@other##1,\ekv@stop,\ekv@parse##2\ekv@mark
  {}
%    \end{macrocode}
% \end{macro}
% \begin{macro}[internal]{\ekv@parse@other}
%    \begin{macrocode}
\long\def\ekv@parse@other##1##2##3,%
  {%
    \ekv@gobbleto@markstop##3\ekv@endparse@other\ekv@mark\ekv@stop
    \ekv@ifhas@eq@other##3=\ekv@ifempty@B\ekv@ifempty@false
      \ekv@ifempty@A\ekv@ifempty@B\@firstoftwo
      {\ekv@parse@eq@other##3\ekv@stop##2}%
      {%
        \ekv@ifhas@eq@active##3#2\ekv@ifempty@B\ekv@ifempty@false
          \ekv@ifempty@A\ekv@ifempty@B\@firstoftwo
          {\ekv@parse@eq@active##3\ekv@stop##2}%
          {%
            \ekv@ifblank@##3\ekv@nil\ekv@ifempty@B\ekv@ifempty@true@F
              \ekv@ifempty@A\ekv@ifempty@B\@firstofone
              {\ekv@strip{##3}\ekv@parse@key##1}%
          }%
      }%
    \ekv@parse@other##1##2\ekv@mark
  }
%    \end{macrocode}
% \end{macro}
% \begin{macro}[internal]{\ekv@parse@eq@other}
%    \begin{macrocode}
\long\def\ekv@parse@eq@other##1=%
  {%
    \ekv@ifhas@eq@active##1#2\ekv@ifempty@B\ekv@ifempty@false
      \ekv@ifempty@A\ekv@ifempty@B\@firstoftwo
      {\ekv@parse@eq@active##1=}%
      {\ekv@strip{##1}\ekv@parse@pair\ekv@mark}%
  }
%    \end{macrocode}
% \end{macro}
% \begin{macro}[internal]{\ekv@parse@eq@active}
%    \begin{macrocode}
\long\def\ekv@parse@eq@active##1#2%
  {%
    \ekv@strip{##1}\ekv@parse@pair\ekv@mark
  }
%    \end{macrocode}
% \end{macro}
% \begin{macro}[internal]{\ekv@endparse@other}
%    \begin{macrocode}
\long\def\ekv@endparse@other
    \ekv@mark\ekv@stop
    \ekv@ifhas@eq@other##1=\ekv@ifempty@B\ekv@ifempty@false
    \ekv@ifempty@A\ekv@ifempty@B\@firstoftwo
    ##2%
    \ekv@parse@other##3\ekv@mark
  {}
%    \end{macrocode}
% \end{macro}
% \begin{macro}[internal]{\ekv@parse@pair,\ekv@parse@pair@}
%    \begin{macrocode}
\long\def\ekv@parse@pair##1##2\ekv@stop
  {%
    \ekv@strip{##2}\ekv@parse@pair@{##1}%
  }
\long\def\ekv@parse@pair@##1##2##3%
  {%
    \unexpanded{##3{##2}{##1}}%
  }
%    \end{macrocode}
% \end{macro}
% \begin{macro}[internal]{\ekv@parse@key}
%    \begin{macrocode}
\long\def\ekv@parse@key##1##2%
  {%
    \unexpanded{##2{##1}}%
  }
%    \end{macrocode}
% \end{macro}
%
% Finally really setting things up with |\ekvset|'s temporary meaning:
%    \begin{macrocode}
}
\begingroup
\catcode`\,=13
\catcode`\==13
\ekvset,=
%    \end{macrocode}
%
% \begin{macro}{\ekvchangeset}
% Provide a macro that is able to switch out the current \set\ in |\ekvset|.
% This operation is slow (by comparison, it should be slightly faster than
% |\ekvsneak|), but allows for something similar to \pkg{pgfkeys}'s
% \texttt{\meta{key}/.cd} mechanism. However this operation is more expensive
% than |/.cd| as we can't just redefine some token to reflect this, but have to
% switch out the set expandably, so this works similar to the |\ekvsneak| macros
% reading and reinserting the remainder of the \kv\ list.
%    \begin{macrocode}
\def\ekvchangeset#1%
  {%
    \expandafter\ekv@changeset\csname\ekv@undefined@set{#1}\endcsname\ekv@mark
  }
%    \end{macrocode}
% \end{macro}
% \begin{macro}[internal]{\ekv@changeset}
% This macro does the real change-out of |\ekvchangeset|. We introduced an
% |\ekv@mark| to not accidentally remove some braces which we have to remove
% again.
%    \begin{macrocode}
\long\def\ekv@changeset#1#2\ekv@set@other#3#4\ekv@set#5%
  {%
    \ekv@gobble@mark#2\ekv@set@other#1#4\ekv@set#1%
  }
%    \end{macrocode}
% \end{macro}
%
%
% \begin{macro}[internal]{\ekv@gobbleto@markstop}
% The |\ekv@gobbleto@markstop| can be used to test for |\ekv@stop| similar to
% our if-empty test, but instead of using tokens which are used nowhere else
% (|\ekv@ifempty@A| and |\ekv@ifempty@B|) it uses |\ekv@mark| and |\ekv@stop|.
%    \begin{macrocode}
\long\def\ekv@gobbleto@markstop#1\ekv@mark\ekv@stop{}
%    \end{macrocode}
% \end{macro}
%
% \begin{macro}[internal]{\ekv@set@pair,\ekv@set@pair@}
% |\ekv@set@pair| gets invoked with the space and brace stripped key-name as its
% first argument, the set-macro as the second argument, and following that is
% the key-value right delimited by an |\ekv@stop|.
%    \begin{macrocode}
\long\def\ekv@set@pair#1#2%
  {%
    \ekv@ifdefined@pair#2{#1}%
      {%
%    \end{macrocode}
% This branch will be executed if the key is not defined as an argument grabbing
% one. If so test whether there is a |NoVal| key of the same name or whether the
% key is unknown. Throw a meaningful error message and gobble the value.
%    \begin{macrocode}
        \ekv@ifdefined{#2{#1}N}%
          \ekv@err@noarg
          \ekv@err@unknown
          #2{#1}%
        \ekv@gobbleto@stop
      }%
  }
%    \end{macrocode}
% |\ekv@ifdefined@pair| will call |\ekv@set@pair@| if the key is correctly
% defined. This will then grab the value, strip outer spaces and braces from it
% and feed it to the key-macro. Afterwards |\ekv@set@other| will take control
% again.
%    \begin{macrocode}
\long\def\ekv@set@pair@#1#2\ekv@stop
  {%
    \ekv@strip{#2}#1%
  }
%    \end{macrocode}
% \end{macro}
%
% \begin{macro}[internal]{\ekv@set@key}
% Analogous to |\ekv@set@pair|, |\ekv@set@key| lets |\ekv@ifdefined@key| test
% whether a |NoVal| key is defined, else it'll throw a meaningful error message.
% Since we don't have to grab any value |\ekv@ifdefined@key| will invoke the
% key-macro and we're done here, |\ekv@set@other| will take over again.
%    \begin{macrocode}
\long\def\ekv@set@key#1#2%
  {%
    \ekv@ifdefined@key#2{#1}%
      {%
        \ekv@ifdefined{#2{#1}}%
          \ekv@err@reqval
          \ekv@err@unknown
          #2{#1}%
      }%
  }
%    \end{macrocode}
% \end{macro}
%
% \begin{macro}[internal]{\ekv@err,\ekv@err@}
% Since |\ekvset| is fully expandable as long as the code of the keys is (which
% is unlikely) we want to somehow throw expandable errors, in our case via
% undefined control sequences.
%    \begin{macrocode}
\begingroup
\edef\ekv@err
  {%
    \endgroup
    \unexpanded{\long\def\ekv@err}##1%
      {%
        \unexpanded{\expandafter\ekv@err@\@firstofone}%
        {\expandafter\noexpand\csname ! expkv Error:\endcsname ##1.}%
        \unexpanded{\ekv@stop}%
      }%
  }
\ekv@err
\def\ekv@err@{\expandafter\ekv@gobbleto@stop}
%    \end{macrocode}
% \end{macro}
% \begin{macro}[internal]
%   {
%     \ekv@err@common,\ekv@err@common@,
%     \ekv@err@unknown,\ekv@err@noarg,\ekv@err@reqval
%   }
% Now we can use |\ekv@err| to set up some error messages so that we can later
% use those instead of the full strings.
%    \begin{macrocode}
\long\def\ekv@err@common #1#2{\expandafter\ekv@err@common@\string#2{#1}}
\long\def\ekv@err@common@#1`#2' #3.#4#5{\ekv@err{#4 (`#5', set `#2')}}
\long\def\ekv@err@unknown#1#2{\ekv@err@common{unknown key}#1{#2}}
\long\def\ekv@err@noarg  #1#2{\ekv@err@common{value forbidden}#1{#2}}
\long\def\ekv@err@reqval #1#2{\ekv@err@common{value required}#1{#2}}
%    \end{macrocode}
% \end{macro}
%
% \begin{macro}[internal]{\ekv@strip,\ekv@strip@a,\ekv@strip@b,\ekv@strip@c}
% Finally we borrow some ideas of \pkg{expl3}'s \pkg{l3tl} to strip spaces
% from keys and values. This |\ekv@strip| also strips one level of outer braces
% \emph{after} stripping spaces, so an input of | {abc} | becomes |abc| after
% stripping. It should be used with |#1| prefixed by |\ekv@mark|. Also this
% implementation at most strips \emph{one} space from both sides.
%    \begin{macrocode}
\def\ekv@strip#1%
  {%
    \long\def\ekv@strip##1%
      {%
        \ekv@strip@a
          ##1%
          \ekv@nil
          \ekv@mark#1%
          #1\ekv@nil{}%
        \ekv@stop
      }%
    \long\def\ekv@strip@a##1\ekv@mark#1##2\ekv@nil##3%
      {%
        \ekv@strip@b##3##1##2\ekv@nil
      }%
    \long\def\ekv@strip@b##1#1\ekv@nil
      {%
        \ekv@strip@c##1\ekv@nil
      }%
    \long\def\ekv@strip@c\ekv@mark##1\ekv@nil##2\ekv@stop##3%
      {%
        ##3{##1}%
      }%
  }
\ekv@strip{ }
%    \end{macrocode}
% \end{macro}
%
% Now everything that's left is to reset the category code of |@|.
%    \begin{macrocode}
\catcode`\@=\ekv@tmp
%    \end{macrocode}
%
% \gobbledocstriptag
%</tex>
%^^A=<<
%
%^^A=<<
%
% \end{implementation}^^A=<<
%
% \clearpage
% \PrintIndex
%
\endinput
%
^^A vim: ft=tex fdm=marker fmr=>>=,=<<
}{}
\ProvidesFile{expkv.dtx}
  [\csname ekvDate\endcsname\ an expandable key=val implementation]
\PassOptionsToPackage{full}{textcomp}
\documentclass{l3doc}
\RequirePackage[oldstylenums,nott]{kpfonts}
\input{glyphtounicode}
\pdfgentounicode=1
\RequirePackage{xfp} ^^A required for an example
\RequirePackage{listings}
\RequirePackage{booktabs}
\RequirePackage{array}
\RequirePackage{collcell}
\RequirePackage{siunitx}
\RequirePackage{xcolor}
\RequirePackage{caption}
\RequirePackage{microtype}
\RequirePackage{accsupp}
\lstset
  {
    ,flexiblecolumns=false
    ,basewidth=.53em
    ,gobble=2
    ,basicstyle=\fontfamily{jkp}\itshape
    ,morekeywords=^^A
      {^^A
        \ekvdef,\ekvdefNoVal,\ekvlet,\ekvletNoVal,\ekvletkv,\ekvletkvNoVal,^^A
        \ekvset,\ekvparse,\ekvVersion,\ekvDate,\ekvifdefined,^^A
        \ekvifdefinedNoVal,\ekvbreak,\ekvbreakPreSneak,\ekvbreakPostSneak,^^A
        \ekvsneak,\ekvsneakPre,\ekvchangeset
      }
    ,morecomment=[l]\%
    ,commentstyle=\color[gray]{0.4}
    ,literate={\{}{{\CodeSymbol\{}}{1}
              {\}}{{\CodeSymbol\}}}{1}
    ^^A,literate=*{<key>}{\key}{4}{<set>}{\set}{4}
  }
\newcommand*\CodeSymbol[1]{\textbf{#1}}
\RequirePackage{randtext}
\let\metaORIG\meta
\protected\def\meta #1{\texttt{\metaORIG{#1}}}
\renewcommand*\thefootnote{\fnsymbol{footnote}}
\definecolor{expkvred}{HTML}{9F393D}
\colorlet{expkvgrey}{black!75}
\makeatletter
\newcommand*\expkv
  {^^A
    \texorpdfstring
      {^^A
        \mbox
          {^^A
            \BeginAccSupp{ActualText=expkv}^^A
            \href{https://github.com/Skillmon/tex_expkv}
              {^^A
                \rmfamily
                \bfseries
                {\color{expkvgrey}e\kern-.05em x\kern-.05em}^^A
                \lower.493ex
                  \hbox{{\color{expkvgrey}P}\kern-.1em{\color{expkvred}k}}^^A
                \kern-.18em{\color{expkvred}v}^^A
              }^^A
            \EndAccSupp{}^^A
          }^^A
      }
      {expkv}^^A
  }
\newcommand*\expkvd
  {^^A
    \texorpdfstring
      {^^A
        \mbox
          {^^A
            \BeginAccSupp{ActualText=expkv-def}^^A
            \rmfamily
            \bfseries
            {\color{expkvgrey}e\kern-.05em x\kern-.05em}^^A
            \lower.493ex
              \hbox{{\color{expkvgrey}P}\kern-.1em{\color{expkvred}k}}^^A
            \kern-.18em{\color{expkvred}v}^^A
            {\color{expkvgrey}^^A
              \kern.05em\rule[-.1ex]{.08em}{1.2ex}\kern.05em\textsc{def}^^A
            }^^A
            \EndAccSupp{}^^A
          }^^A
      }
      {expkv-def}^^A
  }
\newcommand\kv{\meta{key}=\meta{value}}
\newcommand\key{\meta{key}}
\newcommand\val{\meta{value}}
\newcommand\set{\meta{set}}
\hypersetup{linkcolor=red!80!black,urlcolor=purple!80!black}
\DoNotIndex{\def,\edef,\,,\=,\begingroup,\catcode,\chardef,\csname,\endcsname}
\DoNotIndex{\endgroup,\endinput,\errmessage,\expandafter,\input,\let,\long}
\DoNotIndex{\noexpand,\protected,\ProvidesFile,\ProvidesPackage,\relax,\space}
\DoNotIndex{\@,\unexpanded,\string,\expanded}
\DoNotIndex{\ifcsname}
\DoNotIndex{\ifx}
\DoNotIndex{\else}
\DoNotIndex{\fi}
\@gobble\fi ^^A ignoring \ifx and \ifcsname, but only one \fi
\@ifdefinable\gobbledocstriptag{\def\gobbledocstriptag#1>{}}
\newcommand*\pmso[1] ^^A poor man's strike out
  {^^A
    \leavevmode
    \begingroup
      \sbox0{#1}^^A
      \rlap{\vrule height .6ex depth -.5ex width \wd0\relax}^^A
      \usebox0\relax
    \endgroup
  }
\makeatother
\begin{document}
  \title
    {^^A
      \texorpdfstring
        {^^A
          \huge\expkv
          \\[\medskipamount]
          \Large an expandable \kv\ implementation^^A
        }{expkv - an expandable <key>=<value> implementation}^^A
    }
  \date{\ekvDate\space v\ekvVersion}
  \author{Jonathan P. Spratte\thanks{\protect\randomize{jspratte@yahoo.de}}}
  \DocInput{expkv.dtx}
\end{document}
%</driver>^^A=<<
% \fi
%
% \maketitle
% \renewcommand*\thefootnote{\arabic{footnote}}
%
% \begin{abstract}
% \noindent\parfillskip=0pt
% \expkv\ provides a small interface for \kv\ parsing. The parsing macro is
% fully expandable, the \meta{code} of your keys might be not. \expkv\ is pretty
% fast, but not the fastest available \kv\ solution (\pkg{keyval} for instance
% is thrice as fast, but not expandable and it might strip braces it shouldn't
% have stripped).
% \end{abstract}
%
% \tableofcontents
%
% \begin{documentation}^^A>>=
%
% \section{Documentation}
%
% \expkv\ provides an expandable \kv\ parser. The \kv\ pairs should be given as
% a comma separated list and the separator between a \key\ and the associated
% \val\ should be an equal sign. Both, the commas and the equal signs, might be
% of category 12 (other) or 13 (active). To support this is necessary as for
% example \pkg{babel} turns characters active for some languages, for instance
% the equal sign is turned active for Turkish.
%
% \expkv\ is usable as generic code or as a \LaTeX\ package. To use it, just
% use one of:
% \begin{lstlisting}
% \usepackage{expkv} % LaTeX
% \input expkv       % plainTeX
% \end{lstlisting}
% The \LaTeX\ package doesn't do more than \file{expkv.tex}, except calling
% |\ProvidesPackage| and setting things up such that \file{expkv.tex} will use
% |\ProvidesFile|.
%
% \subsection{Setting up Keys}\label{sec:define}
% \expkv\ provides a rather simple approach to setting up keys, similar to
% \pkg{keyval}. However there is an auxiliary package named
% \href{https://github.com/Skillmon/tex_expkv-def}{\expkvd} (the files use
% \file{expkv-def} instead because \string| isn't allowed in package names by
% CTAN) which provides a more sophisticated interface, similar to well
% established packages like \pkg{pgfkeys} or \pkg{l3keys}.
%
% Keys in \expkv\ (as in almost all other \kv\ implementations) belong to a
% \emph{set} such that different sets can contain keys of the same name. Unlike
% many other implementations \expkv\ doesn't provide means to set a default
% value, instead we have keys that take values and keys that don't (the latter
% are called |NoVal| keys by \expkv), but both can have the same name (on the
% user level).
%
% The following macros are available to define new keys. Those macros containing
% ``|def|'' in their name can be prefixed by anything allowed to prefix |\def|,
% prefixes allowed for |\let| can prefix those with ``|let|'' in their name,
% accordingly. Neither \set\ nor \key\ are allowed to be empty for new keys and
% must not contain a |\par| or tokens that expand to it -- they must be legal
% inside of |\csname ...\endcsname|.
%
% \begin{function}{\ekvdef}
%   \begin{syntax}
%     \cs{ekvdef}\marg{set}\marg{key}\marg{code}
%   \end{syntax}
%   Defines a \key\ taking a value in a \set\ to expand to \meta{code}. In
%   \meta{code} you can use |#1| to refer to the given value.
% \end{function}
%
% \begin{function}{\ekvdefNoVal}
%   \begin{syntax}
%     \cs{ekvdefNoVal}\marg{set}\marg{key}\marg{code}
%   \end{syntax}
%   Defines a no value taking \key\ in a \set\ to expand to \meta{code}.
% \end{function}
%
% \begin{function}{\ekvlet}
%   \begin{syntax}
%     \cs{ekvlet}\marg{set}\marg{key}\meta{cs}
%   \end{syntax}
%   Let the value taking \key\ in \set\ to \meta{cs}, there are no checks on
%   \meta{cs} enforced.
% \end{function}
%
% \begin{function}{\ekvletNoVal}
%   \begin{syntax}
%     \cs{ekvletNoVal}\marg{set}\marg{key}\meta{cs}
%   \end{syntax}
%   Let the no value taking \key\ in \set\ to \meta{cs}, it is not checked
%   whether \meta{cs} exists or that it takes no parameter.
% \end{function}
%
% \begin{function}{\ekvletkv}
%   \begin{syntax}
%     \cs{ekvletkv}\marg{set}\marg{key}\marg{set2}\marg{key2}
%   \end{syntax}
%   Let the \key\ in \set\ to \meta{key2} in \meta{set2}, it is not checked
%   whether that second key exists.
% \end{function}
%
% \begin{function}{\ekvletkvNoVal}
%   \begin{syntax}
%     \cs{ekvletkvNoVal}\marg{set}\marg{key}\marg{set2}\marg{key2}
%   \end{syntax}
%   Let the \key\ in \set\ to \meta{key2} in \meta{set2}, it is not checked
%   whether that second key exists.
% \end{function}
%
% \subsection{Parsing Keys}
%
% \begin{function}{\ekvset}
%   \begin{syntax}
%     \cs{ekvset}\marg{set}\{\kv,\ldots\}
%   \end{syntax}
%   Splits \kv\ pairs on commas. From both \key\ and \val\ up to one space is
%   stripped from both ends, if then only a braced group remains the braces are
%   stripped as well. So |\ekvset{foo}{bar=baz}| and
%   |\ekvset{foo}{ {bar}= {baz} }| will both do
%   \texttt{\cs[no-index]{\meta{foobarcode}}\{baz\}}, so you can hide commas,
%   equal signs and spaces at the ends of either \key\ or \val\ by putting
%   braces around them. If you omit the equal sign the code of the key created
%   with the |NoVal| variants described in \autoref{sec:define} will be
%   executed. If \kv\ contains more than a single unhidden equal sign, it will
%   be split at the first one and the others are considered part of the value.
%   |\ekvset| should be nestable.
% \end{function}
%
% \begin{function}{\ekvparse}
%   \begin{syntax}
%     \cs{ekvparse}\meta{cs1}\meta{cs2}\{\kv,\ldots\}
%   \end{syntax}
%   This macro parses the \kv\ pairs and provides those list elements which are
%   only keys as the argument to \meta{cs1}, and those which are a \kv\ pair to
%   \meta{cs2} as two arguments. It is fully expandable as well and returns the
%   parsed list in |\unexpanded|, which has no effect outside of an |\expanded|
%   or |\edef| context\footnotemark.
%   If you need control over the necessary steps of expansion you
%   can use |\expanded| around it.
%
%   |\ekvbreak|, |\ekvsneak|, and |\ekvchangeset| and their relatives don't work
%   in |\ekvparse|.  It is analogue to \pkg{expl3}'s |\keyval_parse:NNn|, but
%   not with the same parsing rules -- |\keyval_parse:NNn| throws an error on
%   multiple equal signs per \kv\ pair and on empty \key\ names in a \kv\ pair,
%   both of which |\ekvparse| doesn't deal with.
%
%   As a small example:
% \begin{lstlisting}
% \ekvparse\handlekey\handlekeyval{foo = bar, key, baz={zzz}}
% \end{lstlisting}
%   would expand to
% \begin{lstlisting}
% \handlekeyval{foo}{bar}\handlekey{key}\handlekeyval{baz}{zzz}
% \end{lstlisting}
%   and afterwards |\handlekey| and |\handlekeyval| would have to further handle
%   the \key. There are no macros like these two contained in \expkv, you have
%   to set them up yourself if you want to use |\ekvparse| (of course the names
%   might differ). If you need the results of |\ekvparse| as the argument for
%   another macro, you should use |\expanded| as only then the input stream will
%   contain the output above:
% \begin{lstlisting}
% \expandafter\handle\expanded{\ekvparse\k\kv{foo = bar, key, baz={zzz}}}
% \end{lstlisting}
%   would expand to
% \begin{lstlisting}
% \handle\kv{foo}{bar}\k{key}\kv{baz}{zzz}
% \end{lstlisting}
% \end{function}
% \footnotetext 
%   {^^A
%     This is a change in behaviour, previously (v0.3 and before) \cs{ekvparse}
%     would expand in exactly two steps. This isn't always necessary, but makes
%     the parsing considerably slower. If this is necessary for your application
%     you can put an \cs[no-index]{expanded} around it and will still be faster
%     since you need only a single \cs[no-index]{expandafter} this way.^^A
%   }.
%
% \subsection{Miscellaneous}
%
% \subsubsection{Other Macros}
%
% \expkv\ provides some other macros which might be of interest.
%
% \begin{function}{\ekvVersion,\ekvDate}
%   These two macros store the version and date of the package.
% \end{function}
% \begin{function}{\ekvifdefined,\ekvifdefinedNoVal}
%   \begin{syntax}
%     \cs{ekvifdefined}\marg{set}\marg{key}\marg{true}\marg{false}
%     \cs{ekvifdefinedNoVal}\marg{set}\marg{key}\marg{true}\marg{false}
%   \end{syntax}
%   These two macros test whether there is a \key\ in \set. It is false if
%   either a hash table entry doesn't exist for that key or its meaning is
%   |\relax|.
% \end{function}
%
% \begin{function}{\ekvbreak,\ekvbreakPreSneak,\ekvbreakPostSneak}
%   \begin{syntax}
%     \cs{ekvbreak}\marg{after}
%   \end{syntax}
%   Gobbles the remainder of the current |\ekvset| macro and its argument list
%   and reinserts \meta{after}. So this can be used to break out of |\ekvset|.
%   The first variant will also gobble anything that has been sneaked out using
%   |\ekvsneak| or |\ekvsneakPre|, while |\ekvbreakPreSneak| will put
%   \meta{after} before anything that has been smuggled and |\ekvbreakPostSneak|
%   will put \meta{after} after the stuff that has been sneaked out.
% \end{function}
%
% \begin{function}{\ekvsneak,\ekvsneakPre}
%   \begin{syntax}
%     \cs{ekvsneak}\marg{after}
%   \end{syntax}
%   Puts \meta{after} after the effects of |\ekvset|. The first variant will put
%   \meta{after} after any other tokens which might have been sneaked before,
%   while |\ekvsneakPre| will put \meta{after} before other smuggled stuff.
%   This reads and reinserts the remainder of the current |\ekvset| macro and
%   its argument list to do its job. A small usage example is shown in
%   \autoref{sec:sneakex}.
% \end{function}
%
% \begin{function}{\ekvchangeset}
%   \begin{syntax}
%     \cs{ekvchangeset}\marg{new-set}
%   \end{syntax}
%   Replaces the current set with \meta{new-set}, so for the rest of the current
%   |\ekvset| call, that call behaves as if it was called with
%   \texttt{\cs[no-index]{ekvset}\marg{new-set}}. Just like |\ekvsneak| this
%   reads and reinserts the remainder of the current |\ekvset| macro to do its
%   job. It is comparable to using \texttt{\meta{key}/.cd} in \pkg{pgfkeys}.
% \end{function}
%
% \bigskip
%
% \begin{function}{\ekv@name,\ekv@name@set,\ekv@name@key}
%   \begin{syntax}
%     \cs{ekv@name}\marg{set}\marg{key}\\
%     \cs{ekv@name@set}\marg{set}\\
%     \cs{ekv@name@key}\marg{key}
%   \end{syntax}
%   The names of the macros that correspond to a key in a set are build with
%   these macros. The default definition of |\ekv@name@set| is
%   ``\texttt{\csname ekv@name@set\endcsname{\set}}'' and the default of
%   |\ekv@name@key| is ``\texttt{\csname ekv@name@key\endcsname{\key}}''.
%   The complete name is build using |\ekv@name| which is equivalent to
%   \texttt{\cs[no-index]{ekv@name@set}\marg{set}\cs{ekv@name@key}\marg{key}}.
%   For |NoVal| keys an additional |N| gets appended irrespective of these
%   macros' definition, so their name is
%   \cs[no-index]{\csname ekv@name\endcsname{\set}{\key}N}. You might redefine
%   |\ekv@name@set| and |\ekv@name@key| locally but \emph{don't redefine}
%   |\ekv@name|!
% \end{function}
%
% \subsubsection{Bugs}
%
% Just like \pkg{keyval}, \expkv\ is bug free. But if you find \pmso{bugs}hidden
% features\footnote{Thanks, David!} you can tell me about them either via mail
% (see the first page) or directly on GitHub if you have an account there:
% \url{https://github.com/Skillmon/tex_expkv}
%
% \subsubsection{Comparisons}
%
% Comparisons of speed are done with a very simple test key and the help of the
% \pkg{l3benchmark} package. The key and its usage should be equivalent to
% \begin{lstlisting}
% \protected\ekvdef{test}{height}{\def\myheight{#1}}
% \ekvset{test}{ height = 6 }
% \end{lstlisting}
% and only the usage of the key, not its definition, is benchmarked. For the
% impatient, the essence of these comparisons regarding speed and buggy
% behaviour is contained in \autoref{tab:comp}.
%
% As far as I know \expkv\ is the only fully expandable \kv\ parser. I tried to
% compare \expkv\ to every \kv\ package listed on
% \href{https://ctan.org/topic/keyval}{CTAN}, however, one might notice that
% some of those are missing from this list. That's because I didn't get the
% others to work due to bugs, or because they just provide wrappers around other
% packages in this list.
%
% In this subsubsection is no benchmark of |\ekvparse| and |\keyval_parse:NNn|
% contained, as most other packages don't provide equivalent features to my
% knowledge. |\ekvparse| is slightly faster than |\ekvset|, but keep in mind
% that it does less. The same is true for |\keyval_parse:NNn| compared to
% |\keys_set:nn| of \pkg{expl3} (where the difference is much bigger).
%
% \paragraph{\pkg{keyval}} is about two times faster and has a comparable
% feature set just a slightly different way how it handles keys without values.
% That might be considered a drawback, as it limits the versatility, but also as
% an advantage, as it might reduce doubled code. Keep in mind that as soon as
% someone loads \pkg{xkeyval} the performance of \pkg{keyval} gets replaced by
% \pkg{xkeyval}'s.
%
% Also \pkg{keyval} has a bug, which unfortunately can't really be resolved
% without breaking backwards compatibility for \emph{many} documents, namely it
% strips braces from the argument before stripping spaces if the argument isn't
% surrounded by spaces, also it might strip more than one set of braces. Hence
% all of the following are equivalent in their outcome, though the last two
% lines should result in something different than the first two:
% \begin{lstlisting}[belowskip=0pt]
% \setkeys{foo}{bar=baz}
% \setkeys{foo}{bar= {baz}}
% \setkeys{foo}{bar={ baz}}
% \setkeys{foo}{bar={{baz}}}
% \end{lstlisting}
%
% \paragraph{\pkg{xkeyval}} is roughly fourteen times slower, but it provides
% more functionality, e.g., it has choice keys, boolean keys, and so on. It
% contains the same bug as \pkg{keyval} as it has to be compatible with it by
% design (it replaces \pkg{keyval}'s frontend), but also adds even more cases in
% which braces are stripped that shouldn't be stripped, worsening the situation.
%
% \paragraph{\pkg{ltxkeys}} is over 300 times slower -- which is funny, because
% it aims to be ``[\ldots] faster than these earlier packages [referring to
% \pkg{keyval} and \pkg{xkeyval}].'' Since it aims to have a bigger feature set
% than \pkg{xkeyval}, it most definitely also has a bigger feature set than
% \expkv. Also, it can't parse |\long| input, so as soon as your values contain
% a |\par|, it'll throw errors. Furthermore, \pkg{ltxkeys} doesn't strip outer
% braces at all by design, which, imho, is a weird design choice. In addition
% \pkg{ltxkeys} loads \pkg{catoptions} which is known to introduce bugs (e.g.,
% see \url{https://tex.stackexchange.com/questions/461783}).
%
% \paragraph{\pkg{l3keys}} is almost five times slower, but has an, imho, great
% interface to define keys. It strips \emph{all} outer spaces, even if somehow
% multiple spaces ended up on either end. It offers more features, but is pretty
% much bound to \pkg{expl3} code. Whether that's a drawback is up to you.
%
% \paragraph{\pkg{pgfkeys}} is a bit more than two times slower for one key,
% but has an \emph{enormous} feature set. However, since adding additional keys
% doesn't add as much needed time for \pkg{pgfkeys} compared to \expkv, it gets
% faster than \expkv\ at around eight \kv\ pairs.
% It has the same or a very similar bug \pkg{keyval} has. The brace bug (and
% also the category fragility) can be fixed by \pkg{pgfkeyx}, but this package
% was last updated in 2012 and it slows down |\pgfkeys| by factor~$8$. Also I
% don't know whether this might introduce new bugs.
%
% \paragraph{\pkg{kvsetkeys} with \pkg{kvdefinekeys}} is about three times
% slower, but it works even if commas and equals have category codes different
% from 12 (just as some other packages in this list). Else the features of the
% keys are equal to those of \pkg{keyval}, the parser has more features, though.
%
% \paragraph{\pkg{options}} is a bit slower for only a single value, but gets a
% tad faster than \expkv\ at around 10 \kv\ pairs. It has a much bigger feature
% set. Unfortunately it also suffers from the premature unbracing bug
% \pkg{keyval} has.
%
% \paragraph{\pkg{simplekv}} is hard to compare because I don't speak French (so
% I don't understand the documentation) and from what I can see, there is no
% direct way to define the equivalent test key. Nevertheless, I tested the
% closest possible equivalent of my test key while siding for \pkg{simplekv}'s
% design not forcing something into it it doesn't seem to be designed for. It is
% more than five times slower and has hard to predict behaviour regarding brace
% and space stripping, similar to \pkg{keyval}. The tested definition was:
% \begin{lstlisting}[belowskip=0pt]
% \usepackage{simplekv}
% \setKVdefault[simplekv]{height={ abc}}                     % key setup
% \setKV[simplekv]{ height = 6 }                             % benchmarked
% \end{lstlisting}
%
% \paragraph{\pkg{yax}} is over eighteen times slower. It has a pretty strange
% syntax, imho, and again a direct equivalent is hard to define. It has the
% premature unbracing bug, too. Also somehow loading \pkg{yax} broke
% \pkg{options} for me. The tested definition was:
% \begin{lstlisting}[belowskip=0pt]
% \usepackage{yax}
% \defactiveparameter yax {\storevalue\myheight yax:height } % key setup
% \setparameterlist{yax}{ height = 6 }                       % benchmarked
% \end{lstlisting}
%
% \begin{table}
%   \def\fnsym{\textcolor{red!80!black}{*}}%
%   \sisetup{round-precision=1, round-mode=places}%
%   \begingroup
%   \centering
%   \newcommand*\yes{\textcolor{red!80!black}  {yes}}^^A
%   \newcommand*\no {\textcolor{green!80!black}{no}}^^A
%   \caption[Comparison of \kv\ packages]
%     {^^A
%       Comparison of \kv\ packages. The packages are ordered from
%       fastest to slowest for one \kv\ pair. Benchmarking was done using
%       \pkg{l3benchmark} and the scripts in the \file{Benchmarks} folder
%       of the \href{https://github.com/Skillmon/tex_expkv}{git repository}.
%       The columns $p_i$ are the polynomial coefficients of a linear fit to the
%       run-time, $p_0$ can be interpreted as the overhead for initialisation
%       and $p_1$ the cost per key. The $T_0$ column is the actual mean ops
%       needed for an empty list argument, as the linear fit doesn't match that
%       point well in general. The column ``BB'' lists whether the
%       parsing is affected by some sort of brace bug, ``CF'' stands for
%       category code fragile and lists whether the parsing breaks with active
%       commas or equal signs.^^A
%       \label{tab:comp}^^A
%     }
%   \begin{tabular}
%       {>{\collectcell\pkg}l<{\endcollectcell}*3{S[table-format=4.1]}ccc}
%     \toprule
%     \rmfamily Package & {$p_1$} & {$p_0$} & {$T_0$}& BB & CF & Date \\
%     \midrule
%     keyval    &   13.539 &    1.844 &    6.969 & \yes & \yes & 2014-10-28 \\
%     \expkv    &   26.411 &    2.391 &   12.610 & \no  & \no  & 2020-02-22 \\
%     options   &   25.532 &   10.518 &   21.393 & \yes & \yes & 2015-03-01 \\
%     pgfkeys   &   26.444 &   40.939 &   55.790 & \yes & \yes & 2020-01-08 \\
%     kvsetkeys & {\fnsym} & {\fnsym} &   41.620 & \no  & \no  & 2019-12-15 \\
%     l3keys    &  114.456 &   35.357 &   52.667 & \no  & \no  & 2020-02-14 \\
%     simplekv  &  160.523 &   10.809 &    8.698 & \yes & \yes & 2017-08-08 \\
%     xkeyval   &  260.872 &  180.435 &  164.167 & \yes & \yes & 2014-12-03 \\
%     yax       &  507.643 &   62.863 &  123.533 & \yes & \yes & 2010-01-22 \\
%     ltxkeys   & 3932.759 & 4737.818 & 5883.000 & \no  & \no  & 2012-11-17 \\
%     \bottomrule
%   \end{tabular}
%   \par
%   \endgroup
%   \medskip
%   \fnsym For \pkg{kvsetkeys} the linear model used for the other
%   packages is a poor fit, \pkg{kvsetkeys} seems to have approximately
%   quadratic run-time, the coefficients of the second degree polynomial fit are
%   $p_2=\num{9.911}$, $p_1=\num{40.523}$, and $p_0=\num{61.500}$. Of course the
%   other packages might not really have linear run-time, but at least from 1~to
%   20~keys the fits don't seem too bad (the maximum ratio $p_2/p_1$
%   for the other packages is \num{9.697e-3}). If one extrapolates the fits
%   for 100 \kv\ pairs one finds that most of them match pretty well, the
%   exception being \pkg{ltxkeys}, which behaves quadratic as well with
%   $p_2=\num{31.685}$, $p_1=\num{3267.368}$, and $p_0=\num{7177.582}$.
% \end{table}
%
% \subsection{Examples}
%
% \subsubsection{Standard Use-Case}
%
% Say we have a macro for which we want to create a \kv\ interface. The macro
% has a parameter, which is stored in the dimension |\ourdim| having a default
% value from its initialization. Now we want to be able to change that dimension
% with the |width| key to some specified value. For that we'd do
% \begin{lstlisting}
% \newdimen\ourdim
% \ourdim=150pt
% \protected\ekvdef{our}{width}{\ourdim=#1\relax}
% \end{lstlisting}
% as you can see, we use the set |our| here. We want the key to behave different
% if no value is specified. In that case the key should not use its initial
% value, but be smart and determine the available space from
% \cs[no-index]{hsize}, so we also define
% \begin{lstlisting}
% \protected\ekvdefNoVal{our}{width}{\ourdim=.9\hsize}
% \end{lstlisting}
% Now we set up our macro to use this \kv\ interface
% \begin{lstlisting}
% \protected\def\ourmacro#1{\begingroup\ekvset{our}{#1}\the\ourdim\endgroup}
% \end{lstlisting}
% Finally we can use our macro like in the following\\
% \begin{minipage}{6cm}
% \begin{lstlisting}
% \ourmacro{}\par
% \ourmacro{width}\par
% \ourmacro{width=5pt}\par
% \end{lstlisting}
% \end{minipage}^^A
% \begin{minipage}{\dimexpr\linewidth-6cm\relax}
%   \newdimen\ourdim\ourdim=150pt
%   \protected\ekvdef{our}{width}{\ourdim=#1\relax}^^A
%   \protected\ekvdefNoVal{our}{width}{\ourdim=.9\hsize}^^A
%   \protected\def\ourmacro#1^^A
%     {\begingroup\ekvset{our}{#1}\the\ourdim\endgroup}^^A
%   \ourmacro{}\par
%   \ourmacro{width}\par
%   \ourmacro{width=5pt}\par
% \end{minipage}
%
% \paragraph{The same key using \protect\expkvd}
% Using \expkvd\ we can set up the equivalent key using a \kv\ interface, after
% the following we could use |\ourmacro| in the same way as above. \expkvd\ will
% allocate and initialise |\ourdim| and define the |width| key |\protected| for
% us, so the result will be exactly the same -- with the exception that the
% default will use |\ourdim=.9\hsize\relax| instead.
% \begin{lstlisting}
% \input expkv-def                 % or \usepackage{expkv-def}
% \ekvdefinekeys{our}
%   {
%     dimen    width = \ourdim,
%     qdefault width = .9\hsize,
%     initial  width = 150pt
%   }
% \end{lstlisting}
%
% \subsubsection{An Expandable \kv\ Macro Using \cs[no-index]{ekvsneak}}
% \label{sec:sneakex}
%
% Let's set up an expandable macro, that uses a \kv\ interface. The problems
% we'll face for this are:
% \begin{enumerate}
%   \item ignoring duplicate keys
%   \item default values for keys which weren't used
%   \item providing the values as the correct argument to a macro (ordered)
% \end{enumerate}
%
% First we need to decide which \kv\ parsing macro we want to do this with,
% |\ekvset| or |\ekvparse|. For this example we also want to show the
% usage of |\ekvsneak|, hence we'll choose |\ekvset|. And we'll have to use
% |\ekvset| such that it builds a parsable list for our macro internals. To
% gain back control after |\ekvset| is done we have to put an internal of our
% macro at the start of that list, so we use an internal key that uses
% |\ekvsneakPre| after any user input.
%
% To ignore duplicates will be easy if the value of the key used last will be
% put first in the list, so the following will use |\ekvsneakPre| for the
% user-level keys. If we wanted some key for which the first usage should be the
% the binding one we would use |\ekvsneak| instead for that key.
%
% Providing default values can be done in different ways, we'll use a simple
% approach in which we'll just put the outcome of our keys if they were used
% with default values before the parsing list terminator.
%
% Ordering the keys can be done simply by searching for a specific token for
% each argument which acts like a flag, so our sneaked out values will include
% specific tokens acting as markers.
%
% Now that we have answers for our technical problems, we have to decide what
% our example macro should do. How about we define a macro that calculates the
% sine of a number and rounds that to a specified precision? As a small extra
% this macro should understand input in radian and degree and the used
% trigonometric function should be selectable as well. For the hard part of this
% task (expandably evaluating trigonometric functions) we'll use the \pkg{xfp}
% package.
%
% First we set up our keys according to our earlier considerations and set up
% the user facing macro |\sine|. The end marker of the parsing list will be a
% |\sine@stop| token, which we don't need to define and we put our defaults
% right before it.
%
% \begin{lstlisting}
% \RequirePackage{xfp}
% \makeatletter
% \ekvdef{expex}{f}{\ekvsneakPre{\f{#1}}}
% \ekvdef{expex}{round}{\ekvsneakPre{\rnd{#1}}}
% \ekvdefNoVal{expex}{degree}{\ekvsneakPre{\deg{d}}}
% \ekvdefNoVal{expex}{radian}{\ekvsneakPre{\deg{}}}
% \ekvdefNoVal{expex}{internal}{\ekvsneakPre{\sine@rnd}}
% \newcommand*\sine[2]
%   {\ekvset{expex}{#1,internal}\rnd{3}\deg{d}\f{sin}\sine@stop{#2}}
% \end{lstlisting}
% For the sake of simplicity we defined the macro |\sine| with two mandatory
% arguments, the first being the \kv\ list, the second the argument to the
% trigonometric function. We could've used \pkg{xparse}'s facilities here to
% define an expandable macro which takes an optional argument instead.
%
% Now we need to define some internal macros to extract the value of each key's
% last usage (remember that this will be the group after the first special
% flag-token). For that we use one delimited macro per key.
% \begin{lstlisting}
% \def\sine@rnd#1\rnd#2#3\sine@stop{\sine@deg#1#3\sine@stop{#2}}
% \def\sine@deg#1\deg#2#3\sine@stop{\sine@f#1#3\sine@stop{#2}}
% \def\sine@f#1\f#2#3\sine@stop{\sine@final{#2}}
% \end{lstlisting}
% After the macros |\sine@rnd|, |\sine@deg|, and |\sine@f| the macro
% |\sine@final| will see
% \texttt
%   {^^A
%     \cs[no-index]{sine@final}^^A
%     \marg{f}\marg{degree/radian}\marg{round}\marg{num}^^A
%   }.
% Now |\sine@final| has to expandably deal with those arguments such that the
% |\fpeval| macro of \pkg{xfp} gets the correct input. Luckily this is pretty
% straight forward in this example. In |\fpeval| the trigonometric functions
% have names such as |sin| or |cos| and the degree taking variants |sind| or
% |cosd|. And since the |degree| key puts a |d| in |#2| and the |radian| key
% leaves |#2| empty all we have to do to get the correct function name is stick
% the two together.
% \begin{lstlisting}
% \newcommand*\sine@final[4]{\fpeval{round(#1#2(#4),#3)}}
% \makeatother
% \end{lstlisting}
% Let's test our macro:\\
% \begin{minipage}{.7\linewidth}
%   \begin{lstlisting}[gobble=4]
%   \sine{}{60}\par
%   \sine{round=10}{60}\par
%   \sine{f=cos,radian}{pi}\par
%   \edef\myval{\sine{f=tan}{1}}\texttt{\meaning\myval}
%   \end{lstlisting}
% \end{minipage}\hfill
% \begin{minipage}{.2\linewidth}
%   \makeatletter
%   \ekvdef{expex}{f}{\ekvsneakPre{\f{#1}}}^^A
%   \ekvdef{expex}{round}{\ekvsneakPre{\rnd{#1}}}^^A
%   \ekvdefNoVal{expex}{degree}{\ekvsneakPre{\deg{d}}}^^A
%   \ekvdefNoVal{expex}{radian}{\ekvsneakPre{\deg{}}}^^A
%   \ekvdefNoVal{expex}{internal}{\ekvsneakPre{\sine@rnd}}^^A
%   \newcommand*\sine[2]
%     {\ekvset{expex}{#1,internal}\rnd{3}\deg{d}\f{sin}\sine@stop{#2}}^^A
%   \def\sine@rnd#1\rnd#2#3\sine@stop{\sine@deg#1#3\sine@stop{#2}}^^A
%   \def\sine@deg#1\deg#2#3\sine@stop{\sine@f#1#3\sine@stop{#2}}^^A
%   \def\sine@f#1\f#2#3\sine@stop{\sine@final{#2}}^^A
%   \newcommand*\sine@final[4]{\fpeval{round(#1#2(#4),#3)}}^^A
%   \makeatother
%   \sine{}{60}\par
%   \sine{round=10}{60}\par
%   \sine{f=cos,radian}{pi}\par
%   \edef\myval{\sine{f=tan}{1}}\texttt{\meaning\myval}
% \end{minipage}
%
%
% \subsection{Error Messages}
%
% \expkv\ should only send messages in case of errors, there are no warnings and
% no info messages. In this subsection those errors are listed.
%
% \subsubsection{Load Time}
%
% \file{expkv.tex} checks whether \eTeX\ is available. If it isn't, an error
% will be thrown using |\errmessage|:
% \begin{lstlisting}
% ! expkv Error: e-TeX required.
% \end{lstlisting}
%
% \subsubsection{Defining Keys}
%
% If you get any error from \expkv\ while you're trying to define a key, the
% definition will be aborted and gobbled.
%
% If you try to define a key with an empty set name you'll get:
% \begin{lstlisting}
% ! expkv Error: empty set name not allowed.
% \end{lstlisting}
%
% Similarly, if you try to define a key with an empty key name:
% \begin{lstlisting}
% ! expkv Error: empty key name not allowed.
% \end{lstlisting}
%
% Both of these messages are done in a way that doesn't throw additional errors
% due to |\global|, |\long|, etc., not being used correctly if you prefixed one
% of the defining macros.
%
% \subsubsection{Using Keys}
%
% This subsubsection contains the errors thrown during |\ekvset|. The errors are
% thrown in an expandable manner by providing an undefined macro. In the
% following messages \lstinline|<key>| gets replaced with the problematic key's
% name, and \lstinline|<set>| with the corresponding set. If any errors during
% \kv\ handling are encountered, the entry in the comma separated list will be
% omitted after the error is thrown and the next \kv\ pair will be parsed.
%
% If you're using an undefined key you'll get:
% \begin{lstlisting}
% ! Undefined control sequence.
% <argument> \! expkv Error:
%                            unknown key (`<key>', set `<set>').
% \end{lstlisting}
%
% If you're using a key for which only a normal version and no |NoVal| version
% is defined, but don't provide a value, you'll get:
% \begin{lstlisting}
% ! Undefined control sequence.
% <argument> \! expkv Error:
%                            value required (`<key>', set `<set>').
% \end{lstlisting}
%
% If you're using a key for which only a |NoVal| version and no normal version
% is defined, but provide a value, you'll get:
% \begin{lstlisting}
% ! Undefined control sequence.
% <argument> \! expkv Error:
%                            value forbidden (`<key>', set `<set>').
% \end{lstlisting}
%
% If you're using a set for which you never executed one of the defining macros
% from \autoref{sec:define} you'll get a low level \TeX\ error, as that isn't
% actively tested by the parser (and hence will lead to undefined behaviour and
% not be gracefully ignored). The error will look like
% \begin{lstlisting}
% ! Missing \endcsname inserted.
% <to be read again>
%                    \! expkv Error: Set `<set>' undefined.
% \end{lstlisting}
%
% \subsection{License}
%
% Copyright \textcopyright\ 2020\unless\ifnum\year=2020--\the\year\fi\
% Jonathan P. Spratte
%
% \medskip\noindent
% This work may be distributed and/or modified under the conditions of the
% \LaTeX\ Project Public License (LPPL), either version 1.3c of this license or
% (at your option) any later version. The latest version of this license is in
% the file:
%
% \url{http://www.latex-project.org/lppl.txt}
%
% \noindent
% This work is ``maintained'' (as per LPPL maintenance status) by
%
% \mbox{Jonathan P. Spratte}.
%
% \end{documentation}^^A=<<
%
% \begin{implementation}^^A>>=
%
% \clearpage
%
% \section{Implementation}^^A>>=
%^^A the LaTeX package >>=
% \subsection{The \LaTeX\ Package}
% First we set up the \LaTeX\ package. That one doesn't really do much except
% |\input|ting the generic code and identifying itself as a package.
% \gobbledocstriptag
%<*pkg>
%    \begin{macrocode}
\def\ekv@tmp
  {%
    \ProvidesFile{expkv.tex}%
      [\ekvDate\space v\ekvVersion\space an expandable key=val implementation]%
  }
% \iffalse meta-comment
%
% File: expkv.dtx Copyright (C) 2020 Jonathan P. Spratte
%
% This work  may be  distributed and/or  modified under  the conditions  of the
% LaTeX Project Public License (LPPL),  either version 1.3c  of this license or
% (at your option) any later version.  The latest version of this license is in
% the file:
%
%   http://www.latex-project.org/lppl.txt
%
% ------------------------------------------------------------------------------
%
%<*driver>^^A>>=
\def\nameofplainTeX{plain}
\ifx\fmtname\nameofplainTeX\else
  \expandafter\begingroup
\fi
\input l3docstrip.tex
\askforoverwritefalse
\preamble

--------------------------------------------------------------
expkv -- an expandable key=val implementation
E-mail: jspratte@yahoo.de
Released under the LaTeX Project Public License v1.3c or later
See http://www.latex-project.org/lppl.txt
--------------------------------------------------------------

Copyright (C) 2020 Jonathan P. Spratte

This  work may be  distributed and/or  modified under  the conditions  of the
LaTeX Project Public License (LPPL),  either version 1.3c  of this license or
(at your option) any later version.  The latest version of this license is in
the file:

  http://www.latex-project.org/lppl.txt

This work is "maintained" (as per LPPL maintenance status) by
  Jonathan P. Spratte.

This work consists of the file  expkv.dtx
and the derived files           expkv.pdf
                                expkv.sty
                                expkv.tex

\endpreamble
% stop docstrip adding \endinput
\postamble
\endpostamble
\generate{\file{expkv.sty}{\from{expkv.dtx}{pkg}}}
\generate{\file{expkv.tex}{\from{expkv.dtx}{tex}}}
\ifx\fmtname\nameofplainTeX
  \expandafter\endbatchfile
\else
  \expandafter\endgroup
\fi
%
\IfFileExists{expkv.tex}{\input{expkv.tex}}{}
\ProvidesFile{expkv.dtx}
  [\csname ekvDate\endcsname\ an expandable key=val implementation]
\PassOptionsToPackage{full}{textcomp}
\documentclass{l3doc}
\RequirePackage[oldstylenums,nott]{kpfonts}
\input{glyphtounicode}
\pdfgentounicode=1
\RequirePackage{xfp} ^^A required for an example
\RequirePackage{listings}
\RequirePackage{booktabs}
\RequirePackage{array}
\RequirePackage{collcell}
\RequirePackage{siunitx}
\RequirePackage{xcolor}
\RequirePackage{caption}
\RequirePackage{microtype}
\RequirePackage{accsupp}
\lstset
  {
    ,flexiblecolumns=false
    ,basewidth=.53em
    ,gobble=2
    ,basicstyle=\fontfamily{jkp}\itshape
    ,morekeywords=^^A
      {^^A
        \ekvdef,\ekvdefNoVal,\ekvlet,\ekvletNoVal,\ekvletkv,\ekvletkvNoVal,^^A
        \ekvset,\ekvparse,\ekvVersion,\ekvDate,\ekvifdefined,^^A
        \ekvifdefinedNoVal,\ekvbreak,\ekvbreakPreSneak,\ekvbreakPostSneak,^^A
        \ekvsneak,\ekvsneakPre,\ekvchangeset
      }
    ,morecomment=[l]\%
    ,commentstyle=\color[gray]{0.4}
    ,literate={\{}{{\CodeSymbol\{}}{1}
              {\}}{{\CodeSymbol\}}}{1}
    ^^A,literate=*{<key>}{\key}{4}{<set>}{\set}{4}
  }
\newcommand*\CodeSymbol[1]{\textbf{#1}}
\RequirePackage{randtext}
\let\metaORIG\meta
\protected\def\meta #1{\texttt{\metaORIG{#1}}}
\renewcommand*\thefootnote{\fnsymbol{footnote}}
\definecolor{expkvred}{HTML}{9F393D}
\colorlet{expkvgrey}{black!75}
\makeatletter
\newcommand*\expkv
  {^^A
    \texorpdfstring
      {^^A
        \mbox
          {^^A
            \BeginAccSupp{ActualText=expkv}^^A
            \href{https://github.com/Skillmon/tex_expkv}
              {^^A
                \rmfamily
                \bfseries
                {\color{expkvgrey}e\kern-.05em x\kern-.05em}^^A
                \lower.493ex
                  \hbox{{\color{expkvgrey}P}\kern-.1em{\color{expkvred}k}}^^A
                \kern-.18em{\color{expkvred}v}^^A
              }^^A
            \EndAccSupp{}^^A
          }^^A
      }
      {expkv}^^A
  }
\newcommand*\expkvd
  {^^A
    \texorpdfstring
      {^^A
        \mbox
          {^^A
            \BeginAccSupp{ActualText=expkv-def}^^A
            \rmfamily
            \bfseries
            {\color{expkvgrey}e\kern-.05em x\kern-.05em}^^A
            \lower.493ex
              \hbox{{\color{expkvgrey}P}\kern-.1em{\color{expkvred}k}}^^A
            \kern-.18em{\color{expkvred}v}^^A
            {\color{expkvgrey}^^A
              \kern.05em\rule[-.1ex]{.08em}{1.2ex}\kern.05em\textsc{def}^^A
            }^^A
            \EndAccSupp{}^^A
          }^^A
      }
      {expkv-def}^^A
  }
\newcommand\kv{\meta{key}=\meta{value}}
\newcommand\key{\meta{key}}
\newcommand\val{\meta{value}}
\newcommand\set{\meta{set}}
\hypersetup{linkcolor=red!80!black,urlcolor=purple!80!black}
\DoNotIndex{\def,\edef,\,,\=,\begingroup,\catcode,\chardef,\csname,\endcsname}
\DoNotIndex{\endgroup,\endinput,\errmessage,\expandafter,\input,\let,\long}
\DoNotIndex{\noexpand,\protected,\ProvidesFile,\ProvidesPackage,\relax,\space}
\DoNotIndex{\@,\unexpanded,\string,\expanded}
\DoNotIndex{\ifcsname}
\DoNotIndex{\ifx}
\DoNotIndex{\else}
\DoNotIndex{\fi}
\@gobble\fi ^^A ignoring \ifx and \ifcsname, but only one \fi
\@ifdefinable\gobbledocstriptag{\def\gobbledocstriptag#1>{}}
\newcommand*\pmso[1] ^^A poor man's strike out
  {^^A
    \leavevmode
    \begingroup
      \sbox0{#1}^^A
      \rlap{\vrule height .6ex depth -.5ex width \wd0\relax}^^A
      \usebox0\relax
    \endgroup
  }
\makeatother
\begin{document}
  \title
    {^^A
      \texorpdfstring
        {^^A
          \huge\expkv
          \\[\medskipamount]
          \Large an expandable \kv\ implementation^^A
        }{expkv - an expandable <key>=<value> implementation}^^A
    }
  \date{\ekvDate\space v\ekvVersion}
  \author{Jonathan P. Spratte\thanks{\protect\randomize{jspratte@yahoo.de}}}
  \DocInput{expkv.dtx}
\end{document}
%</driver>^^A=<<
% \fi
%
% \maketitle
% \renewcommand*\thefootnote{\arabic{footnote}}
%
% \begin{abstract}
% \noindent\parfillskip=0pt
% \expkv\ provides a small interface for \kv\ parsing. The parsing macro is
% fully expandable, the \meta{code} of your keys might be not. \expkv\ is pretty
% fast, but not the fastest available \kv\ solution (\pkg{keyval} for instance
% is thrice as fast, but not expandable and it might strip braces it shouldn't
% have stripped).
% \end{abstract}
%
% \tableofcontents
%
% \begin{documentation}^^A>>=
%
% \section{Documentation}
%
% \expkv\ provides an expandable \kv\ parser. The \kv\ pairs should be given as
% a comma separated list and the separator between a \key\ and the associated
% \val\ should be an equal sign. Both, the commas and the equal signs, might be
% of category 12 (other) or 13 (active). To support this is necessary as for
% example \pkg{babel} turns characters active for some languages, for instance
% the equal sign is turned active for Turkish.
%
% \expkv\ is usable as generic code or as a \LaTeX\ package. To use it, just
% use one of:
% \begin{lstlisting}
% \usepackage{expkv} % LaTeX
% \input expkv       % plainTeX
% \end{lstlisting}
% The \LaTeX\ package doesn't do more than \file{expkv.tex}, except calling
% |\ProvidesPackage| and setting things up such that \file{expkv.tex} will use
% |\ProvidesFile|.
%
% \subsection{Setting up Keys}\label{sec:define}
% \expkv\ provides a rather simple approach to setting up keys, similar to
% \pkg{keyval}. However there is an auxiliary package named
% \href{https://github.com/Skillmon/tex_expkv-def}{\expkvd} (the files use
% \file{expkv-def} instead because \string| isn't allowed in package names by
% CTAN) which provides a more sophisticated interface, similar to well
% established packages like \pkg{pgfkeys} or \pkg{l3keys}.
%
% Keys in \expkv\ (as in almost all other \kv\ implementations) belong to a
% \emph{set} such that different sets can contain keys of the same name. Unlike
% many other implementations \expkv\ doesn't provide means to set a default
% value, instead we have keys that take values and keys that don't (the latter
% are called |NoVal| keys by \expkv), but both can have the same name (on the
% user level).
%
% The following macros are available to define new keys. Those macros containing
% ``|def|'' in their name can be prefixed by anything allowed to prefix |\def|,
% prefixes allowed for |\let| can prefix those with ``|let|'' in their name,
% accordingly. Neither \set\ nor \key\ are allowed to be empty for new keys and
% must not contain a |\par| or tokens that expand to it -- they must be legal
% inside of |\csname ...\endcsname|.
%
% \begin{function}{\ekvdef}
%   \begin{syntax}
%     \cs{ekvdef}\marg{set}\marg{key}\marg{code}
%   \end{syntax}
%   Defines a \key\ taking a value in a \set\ to expand to \meta{code}. In
%   \meta{code} you can use |#1| to refer to the given value.
% \end{function}
%
% \begin{function}{\ekvdefNoVal}
%   \begin{syntax}
%     \cs{ekvdefNoVal}\marg{set}\marg{key}\marg{code}
%   \end{syntax}
%   Defines a no value taking \key\ in a \set\ to expand to \meta{code}.
% \end{function}
%
% \begin{function}{\ekvlet}
%   \begin{syntax}
%     \cs{ekvlet}\marg{set}\marg{key}\meta{cs}
%   \end{syntax}
%   Let the value taking \key\ in \set\ to \meta{cs}, there are no checks on
%   \meta{cs} enforced.
% \end{function}
%
% \begin{function}{\ekvletNoVal}
%   \begin{syntax}
%     \cs{ekvletNoVal}\marg{set}\marg{key}\meta{cs}
%   \end{syntax}
%   Let the no value taking \key\ in \set\ to \meta{cs}, it is not checked
%   whether \meta{cs} exists or that it takes no parameter.
% \end{function}
%
% \begin{function}{\ekvletkv}
%   \begin{syntax}
%     \cs{ekvletkv}\marg{set}\marg{key}\marg{set2}\marg{key2}
%   \end{syntax}
%   Let the \key\ in \set\ to \meta{key2} in \meta{set2}, it is not checked
%   whether that second key exists.
% \end{function}
%
% \begin{function}{\ekvletkvNoVal}
%   \begin{syntax}
%     \cs{ekvletkvNoVal}\marg{set}\marg{key}\marg{set2}\marg{key2}
%   \end{syntax}
%   Let the \key\ in \set\ to \meta{key2} in \meta{set2}, it is not checked
%   whether that second key exists.
% \end{function}
%
% \subsection{Parsing Keys}
%
% \begin{function}{\ekvset}
%   \begin{syntax}
%     \cs{ekvset}\marg{set}\{\kv,\ldots\}
%   \end{syntax}
%   Splits \kv\ pairs on commas. From both \key\ and \val\ up to one space is
%   stripped from both ends, if then only a braced group remains the braces are
%   stripped as well. So |\ekvset{foo}{bar=baz}| and
%   |\ekvset{foo}{ {bar}= {baz} }| will both do
%   \texttt{\cs[no-index]{\meta{foobarcode}}\{baz\}}, so you can hide commas,
%   equal signs and spaces at the ends of either \key\ or \val\ by putting
%   braces around them. If you omit the equal sign the code of the key created
%   with the |NoVal| variants described in \autoref{sec:define} will be
%   executed. If \kv\ contains more than a single unhidden equal sign, it will
%   be split at the first one and the others are considered part of the value.
%   |\ekvset| should be nestable.
% \end{function}
%
% \begin{function}{\ekvparse}
%   \begin{syntax}
%     \cs{ekvparse}\meta{cs1}\meta{cs2}\{\kv,\ldots\}
%   \end{syntax}
%   This macro parses the \kv\ pairs and provides those list elements which are
%   only keys as the argument to \meta{cs1}, and those which are a \kv\ pair to
%   \meta{cs2} as two arguments. It is fully expandable as well and returns the
%   parsed list in |\unexpanded|, which has no effect outside of an |\expanded|
%   or |\edef| context\footnotemark.
%   If you need control over the necessary steps of expansion you
%   can use |\expanded| around it.
%
%   |\ekvbreak|, |\ekvsneak|, and |\ekvchangeset| and their relatives don't work
%   in |\ekvparse|.  It is analogue to \pkg{expl3}'s |\keyval_parse:NNn|, but
%   not with the same parsing rules -- |\keyval_parse:NNn| throws an error on
%   multiple equal signs per \kv\ pair and on empty \key\ names in a \kv\ pair,
%   both of which |\ekvparse| doesn't deal with.
%
%   As a small example:
% \begin{lstlisting}
% \ekvparse\handlekey\handlekeyval{foo = bar, key, baz={zzz}}
% \end{lstlisting}
%   would expand to
% \begin{lstlisting}
% \handlekeyval{foo}{bar}\handlekey{key}\handlekeyval{baz}{zzz}
% \end{lstlisting}
%   and afterwards |\handlekey| and |\handlekeyval| would have to further handle
%   the \key. There are no macros like these two contained in \expkv, you have
%   to set them up yourself if you want to use |\ekvparse| (of course the names
%   might differ). If you need the results of |\ekvparse| as the argument for
%   another macro, you should use |\expanded| as only then the input stream will
%   contain the output above:
% \begin{lstlisting}
% \expandafter\handle\expanded{\ekvparse\k\kv{foo = bar, key, baz={zzz}}}
% \end{lstlisting}
%   would expand to
% \begin{lstlisting}
% \handle\kv{foo}{bar}\k{key}\kv{baz}{zzz}
% \end{lstlisting}
% \end{function}
% \footnotetext 
%   {^^A
%     This is a change in behaviour, previously (v0.3 and before) \cs{ekvparse}
%     would expand in exactly two steps. This isn't always necessary, but makes
%     the parsing considerably slower. If this is necessary for your application
%     you can put an \cs[no-index]{expanded} around it and will still be faster
%     since you need only a single \cs[no-index]{expandafter} this way.^^A
%   }.
%
% \subsection{Miscellaneous}
%
% \subsubsection{Other Macros}
%
% \expkv\ provides some other macros which might be of interest.
%
% \begin{function}{\ekvVersion,\ekvDate}
%   These two macros store the version and date of the package.
% \end{function}
% \begin{function}{\ekvifdefined,\ekvifdefinedNoVal}
%   \begin{syntax}
%     \cs{ekvifdefined}\marg{set}\marg{key}\marg{true}\marg{false}
%     \cs{ekvifdefinedNoVal}\marg{set}\marg{key}\marg{true}\marg{false}
%   \end{syntax}
%   These two macros test whether there is a \key\ in \set. It is false if
%   either a hash table entry doesn't exist for that key or its meaning is
%   |\relax|.
% \end{function}
%
% \begin{function}{\ekvbreak,\ekvbreakPreSneak,\ekvbreakPostSneak}
%   \begin{syntax}
%     \cs{ekvbreak}\marg{after}
%   \end{syntax}
%   Gobbles the remainder of the current |\ekvset| macro and its argument list
%   and reinserts \meta{after}. So this can be used to break out of |\ekvset|.
%   The first variant will also gobble anything that has been sneaked out using
%   |\ekvsneak| or |\ekvsneakPre|, while |\ekvbreakPreSneak| will put
%   \meta{after} before anything that has been smuggled and |\ekvbreakPostSneak|
%   will put \meta{after} after the stuff that has been sneaked out.
% \end{function}
%
% \begin{function}{\ekvsneak,\ekvsneakPre}
%   \begin{syntax}
%     \cs{ekvsneak}\marg{after}
%   \end{syntax}
%   Puts \meta{after} after the effects of |\ekvset|. The first variant will put
%   \meta{after} after any other tokens which might have been sneaked before,
%   while |\ekvsneakPre| will put \meta{after} before other smuggled stuff.
%   This reads and reinserts the remainder of the current |\ekvset| macro and
%   its argument list to do its job. A small usage example is shown in
%   \autoref{sec:sneakex}.
% \end{function}
%
% \begin{function}{\ekvchangeset}
%   \begin{syntax}
%     \cs{ekvchangeset}\marg{new-set}
%   \end{syntax}
%   Replaces the current set with \meta{new-set}, so for the rest of the current
%   |\ekvset| call, that call behaves as if it was called with
%   \texttt{\cs[no-index]{ekvset}\marg{new-set}}. Just like |\ekvsneak| this
%   reads and reinserts the remainder of the current |\ekvset| macro to do its
%   job. It is comparable to using \texttt{\meta{key}/.cd} in \pkg{pgfkeys}.
% \end{function}
%
% \bigskip
%
% \begin{function}{\ekv@name,\ekv@name@set,\ekv@name@key}
%   \begin{syntax}
%     \cs{ekv@name}\marg{set}\marg{key}\\
%     \cs{ekv@name@set}\marg{set}\\
%     \cs{ekv@name@key}\marg{key}
%   \end{syntax}
%   The names of the macros that correspond to a key in a set are build with
%   these macros. The default definition of |\ekv@name@set| is
%   ``\texttt{\csname ekv@name@set\endcsname{\set}}'' and the default of
%   |\ekv@name@key| is ``\texttt{\csname ekv@name@key\endcsname{\key}}''.
%   The complete name is build using |\ekv@name| which is equivalent to
%   \texttt{\cs[no-index]{ekv@name@set}\marg{set}\cs{ekv@name@key}\marg{key}}.
%   For |NoVal| keys an additional |N| gets appended irrespective of these
%   macros' definition, so their name is
%   \cs[no-index]{\csname ekv@name\endcsname{\set}{\key}N}. You might redefine
%   |\ekv@name@set| and |\ekv@name@key| locally but \emph{don't redefine}
%   |\ekv@name|!
% \end{function}
%
% \subsubsection{Bugs}
%
% Just like \pkg{keyval}, \expkv\ is bug free. But if you find \pmso{bugs}hidden
% features\footnote{Thanks, David!} you can tell me about them either via mail
% (see the first page) or directly on GitHub if you have an account there:
% \url{https://github.com/Skillmon/tex_expkv}
%
% \subsubsection{Comparisons}
%
% Comparisons of speed are done with a very simple test key and the help of the
% \pkg{l3benchmark} package. The key and its usage should be equivalent to
% \begin{lstlisting}
% \protected\ekvdef{test}{height}{\def\myheight{#1}}
% \ekvset{test}{ height = 6 }
% \end{lstlisting}
% and only the usage of the key, not its definition, is benchmarked. For the
% impatient, the essence of these comparisons regarding speed and buggy
% behaviour is contained in \autoref{tab:comp}.
%
% As far as I know \expkv\ is the only fully expandable \kv\ parser. I tried to
% compare \expkv\ to every \kv\ package listed on
% \href{https://ctan.org/topic/keyval}{CTAN}, however, one might notice that
% some of those are missing from this list. That's because I didn't get the
% others to work due to bugs, or because they just provide wrappers around other
% packages in this list.
%
% In this subsubsection is no benchmark of |\ekvparse| and |\keyval_parse:NNn|
% contained, as most other packages don't provide equivalent features to my
% knowledge. |\ekvparse| is slightly faster than |\ekvset|, but keep in mind
% that it does less. The same is true for |\keyval_parse:NNn| compared to
% |\keys_set:nn| of \pkg{expl3} (where the difference is much bigger).
%
% \paragraph{\pkg{keyval}} is about two times faster and has a comparable
% feature set just a slightly different way how it handles keys without values.
% That might be considered a drawback, as it limits the versatility, but also as
% an advantage, as it might reduce doubled code. Keep in mind that as soon as
% someone loads \pkg{xkeyval} the performance of \pkg{keyval} gets replaced by
% \pkg{xkeyval}'s.
%
% Also \pkg{keyval} has a bug, which unfortunately can't really be resolved
% without breaking backwards compatibility for \emph{many} documents, namely it
% strips braces from the argument before stripping spaces if the argument isn't
% surrounded by spaces, also it might strip more than one set of braces. Hence
% all of the following are equivalent in their outcome, though the last two
% lines should result in something different than the first two:
% \begin{lstlisting}[belowskip=0pt]
% \setkeys{foo}{bar=baz}
% \setkeys{foo}{bar= {baz}}
% \setkeys{foo}{bar={ baz}}
% \setkeys{foo}{bar={{baz}}}
% \end{lstlisting}
%
% \paragraph{\pkg{xkeyval}} is roughly fourteen times slower, but it provides
% more functionality, e.g., it has choice keys, boolean keys, and so on. It
% contains the same bug as \pkg{keyval} as it has to be compatible with it by
% design (it replaces \pkg{keyval}'s frontend), but also adds even more cases in
% which braces are stripped that shouldn't be stripped, worsening the situation.
%
% \paragraph{\pkg{ltxkeys}} is over 300 times slower -- which is funny, because
% it aims to be ``[\ldots] faster than these earlier packages [referring to
% \pkg{keyval} and \pkg{xkeyval}].'' Since it aims to have a bigger feature set
% than \pkg{xkeyval}, it most definitely also has a bigger feature set than
% \expkv. Also, it can't parse |\long| input, so as soon as your values contain
% a |\par|, it'll throw errors. Furthermore, \pkg{ltxkeys} doesn't strip outer
% braces at all by design, which, imho, is a weird design choice. In addition
% \pkg{ltxkeys} loads \pkg{catoptions} which is known to introduce bugs (e.g.,
% see \url{https://tex.stackexchange.com/questions/461783}).
%
% \paragraph{\pkg{l3keys}} is almost five times slower, but has an, imho, great
% interface to define keys. It strips \emph{all} outer spaces, even if somehow
% multiple spaces ended up on either end. It offers more features, but is pretty
% much bound to \pkg{expl3} code. Whether that's a drawback is up to you.
%
% \paragraph{\pkg{pgfkeys}} is a bit more than two times slower for one key,
% but has an \emph{enormous} feature set. However, since adding additional keys
% doesn't add as much needed time for \pkg{pgfkeys} compared to \expkv, it gets
% faster than \expkv\ at around eight \kv\ pairs.
% It has the same or a very similar bug \pkg{keyval} has. The brace bug (and
% also the category fragility) can be fixed by \pkg{pgfkeyx}, but this package
% was last updated in 2012 and it slows down |\pgfkeys| by factor~$8$. Also I
% don't know whether this might introduce new bugs.
%
% \paragraph{\pkg{kvsetkeys} with \pkg{kvdefinekeys}} is about three times
% slower, but it works even if commas and equals have category codes different
% from 12 (just as some other packages in this list). Else the features of the
% keys are equal to those of \pkg{keyval}, the parser has more features, though.
%
% \paragraph{\pkg{options}} is a bit slower for only a single value, but gets a
% tad faster than \expkv\ at around 10 \kv\ pairs. It has a much bigger feature
% set. Unfortunately it also suffers from the premature unbracing bug
% \pkg{keyval} has.
%
% \paragraph{\pkg{simplekv}} is hard to compare because I don't speak French (so
% I don't understand the documentation) and from what I can see, there is no
% direct way to define the equivalent test key. Nevertheless, I tested the
% closest possible equivalent of my test key while siding for \pkg{simplekv}'s
% design not forcing something into it it doesn't seem to be designed for. It is
% more than five times slower and has hard to predict behaviour regarding brace
% and space stripping, similar to \pkg{keyval}. The tested definition was:
% \begin{lstlisting}[belowskip=0pt]
% \usepackage{simplekv}
% \setKVdefault[simplekv]{height={ abc}}                     % key setup
% \setKV[simplekv]{ height = 6 }                             % benchmarked
% \end{lstlisting}
%
% \paragraph{\pkg{yax}} is over eighteen times slower. It has a pretty strange
% syntax, imho, and again a direct equivalent is hard to define. It has the
% premature unbracing bug, too. Also somehow loading \pkg{yax} broke
% \pkg{options} for me. The tested definition was:
% \begin{lstlisting}[belowskip=0pt]
% \usepackage{yax}
% \defactiveparameter yax {\storevalue\myheight yax:height } % key setup
% \setparameterlist{yax}{ height = 6 }                       % benchmarked
% \end{lstlisting}
%
% \begin{table}
%   \def\fnsym{\textcolor{red!80!black}{*}}%
%   \sisetup{round-precision=1, round-mode=places}%
%   \begingroup
%   \centering
%   \newcommand*\yes{\textcolor{red!80!black}  {yes}}^^A
%   \newcommand*\no {\textcolor{green!80!black}{no}}^^A
%   \caption[Comparison of \kv\ packages]
%     {^^A
%       Comparison of \kv\ packages. The packages are ordered from
%       fastest to slowest for one \kv\ pair. Benchmarking was done using
%       \pkg{l3benchmark} and the scripts in the \file{Benchmarks} folder
%       of the \href{https://github.com/Skillmon/tex_expkv}{git repository}.
%       The columns $p_i$ are the polynomial coefficients of a linear fit to the
%       run-time, $p_0$ can be interpreted as the overhead for initialisation
%       and $p_1$ the cost per key. The $T_0$ column is the actual mean ops
%       needed for an empty list argument, as the linear fit doesn't match that
%       point well in general. The column ``BB'' lists whether the
%       parsing is affected by some sort of brace bug, ``CF'' stands for
%       category code fragile and lists whether the parsing breaks with active
%       commas or equal signs.^^A
%       \label{tab:comp}^^A
%     }
%   \begin{tabular}
%       {>{\collectcell\pkg}l<{\endcollectcell}*3{S[table-format=4.1]}ccc}
%     \toprule
%     \rmfamily Package & {$p_1$} & {$p_0$} & {$T_0$}& BB & CF & Date \\
%     \midrule
%     keyval    &   13.539 &    1.844 &    6.969 & \yes & \yes & 2014-10-28 \\
%     \expkv    &   26.411 &    2.391 &   12.610 & \no  & \no  & 2020-02-22 \\
%     options   &   25.532 &   10.518 &   21.393 & \yes & \yes & 2015-03-01 \\
%     pgfkeys   &   26.444 &   40.939 &   55.790 & \yes & \yes & 2020-01-08 \\
%     kvsetkeys & {\fnsym} & {\fnsym} &   41.620 & \no  & \no  & 2019-12-15 \\
%     l3keys    &  114.456 &   35.357 &   52.667 & \no  & \no  & 2020-02-14 \\
%     simplekv  &  160.523 &   10.809 &    8.698 & \yes & \yes & 2017-08-08 \\
%     xkeyval   &  260.872 &  180.435 &  164.167 & \yes & \yes & 2014-12-03 \\
%     yax       &  507.643 &   62.863 &  123.533 & \yes & \yes & 2010-01-22 \\
%     ltxkeys   & 3932.759 & 4737.818 & 5883.000 & \no  & \no  & 2012-11-17 \\
%     \bottomrule
%   \end{tabular}
%   \par
%   \endgroup
%   \medskip
%   \fnsym For \pkg{kvsetkeys} the linear model used for the other
%   packages is a poor fit, \pkg{kvsetkeys} seems to have approximately
%   quadratic run-time, the coefficients of the second degree polynomial fit are
%   $p_2=\num{9.911}$, $p_1=\num{40.523}$, and $p_0=\num{61.500}$. Of course the
%   other packages might not really have linear run-time, but at least from 1~to
%   20~keys the fits don't seem too bad (the maximum ratio $p_2/p_1$
%   for the other packages is \num{9.697e-3}). If one extrapolates the fits
%   for 100 \kv\ pairs one finds that most of them match pretty well, the
%   exception being \pkg{ltxkeys}, which behaves quadratic as well with
%   $p_2=\num{31.685}$, $p_1=\num{3267.368}$, and $p_0=\num{7177.582}$.
% \end{table}
%
% \subsection{Examples}
%
% \subsubsection{Standard Use-Case}
%
% Say we have a macro for which we want to create a \kv\ interface. The macro
% has a parameter, which is stored in the dimension |\ourdim| having a default
% value from its initialization. Now we want to be able to change that dimension
% with the |width| key to some specified value. For that we'd do
% \begin{lstlisting}
% \newdimen\ourdim
% \ourdim=150pt
% \protected\ekvdef{our}{width}{\ourdim=#1\relax}
% \end{lstlisting}
% as you can see, we use the set |our| here. We want the key to behave different
% if no value is specified. In that case the key should not use its initial
% value, but be smart and determine the available space from
% \cs[no-index]{hsize}, so we also define
% \begin{lstlisting}
% \protected\ekvdefNoVal{our}{width}{\ourdim=.9\hsize}
% \end{lstlisting}
% Now we set up our macro to use this \kv\ interface
% \begin{lstlisting}
% \protected\def\ourmacro#1{\begingroup\ekvset{our}{#1}\the\ourdim\endgroup}
% \end{lstlisting}
% Finally we can use our macro like in the following\\
% \begin{minipage}{6cm}
% \begin{lstlisting}
% \ourmacro{}\par
% \ourmacro{width}\par
% \ourmacro{width=5pt}\par
% \end{lstlisting}
% \end{minipage}^^A
% \begin{minipage}{\dimexpr\linewidth-6cm\relax}
%   \newdimen\ourdim\ourdim=150pt
%   \protected\ekvdef{our}{width}{\ourdim=#1\relax}^^A
%   \protected\ekvdefNoVal{our}{width}{\ourdim=.9\hsize}^^A
%   \protected\def\ourmacro#1^^A
%     {\begingroup\ekvset{our}{#1}\the\ourdim\endgroup}^^A
%   \ourmacro{}\par
%   \ourmacro{width}\par
%   \ourmacro{width=5pt}\par
% \end{minipage}
%
% \paragraph{The same key using \protect\expkvd}
% Using \expkvd\ we can set up the equivalent key using a \kv\ interface, after
% the following we could use |\ourmacro| in the same way as above. \expkvd\ will
% allocate and initialise |\ourdim| and define the |width| key |\protected| for
% us, so the result will be exactly the same -- with the exception that the
% default will use |\ourdim=.9\hsize\relax| instead.
% \begin{lstlisting}
% \input expkv-def                 % or \usepackage{expkv-def}
% \ekvdefinekeys{our}
%   {
%     dimen    width = \ourdim,
%     qdefault width = .9\hsize,
%     initial  width = 150pt
%   }
% \end{lstlisting}
%
% \subsubsection{An Expandable \kv\ Macro Using \cs[no-index]{ekvsneak}}
% \label{sec:sneakex}
%
% Let's set up an expandable macro, that uses a \kv\ interface. The problems
% we'll face for this are:
% \begin{enumerate}
%   \item ignoring duplicate keys
%   \item default values for keys which weren't used
%   \item providing the values as the correct argument to a macro (ordered)
% \end{enumerate}
%
% First we need to decide which \kv\ parsing macro we want to do this with,
% |\ekvset| or |\ekvparse|. For this example we also want to show the
% usage of |\ekvsneak|, hence we'll choose |\ekvset|. And we'll have to use
% |\ekvset| such that it builds a parsable list for our macro internals. To
% gain back control after |\ekvset| is done we have to put an internal of our
% macro at the start of that list, so we use an internal key that uses
% |\ekvsneakPre| after any user input.
%
% To ignore duplicates will be easy if the value of the key used last will be
% put first in the list, so the following will use |\ekvsneakPre| for the
% user-level keys. If we wanted some key for which the first usage should be the
% the binding one we would use |\ekvsneak| instead for that key.
%
% Providing default values can be done in different ways, we'll use a simple
% approach in which we'll just put the outcome of our keys if they were used
% with default values before the parsing list terminator.
%
% Ordering the keys can be done simply by searching for a specific token for
% each argument which acts like a flag, so our sneaked out values will include
% specific tokens acting as markers.
%
% Now that we have answers for our technical problems, we have to decide what
% our example macro should do. How about we define a macro that calculates the
% sine of a number and rounds that to a specified precision? As a small extra
% this macro should understand input in radian and degree and the used
% trigonometric function should be selectable as well. For the hard part of this
% task (expandably evaluating trigonometric functions) we'll use the \pkg{xfp}
% package.
%
% First we set up our keys according to our earlier considerations and set up
% the user facing macro |\sine|. The end marker of the parsing list will be a
% |\sine@stop| token, which we don't need to define and we put our defaults
% right before it.
%
% \begin{lstlisting}
% \RequirePackage{xfp}
% \makeatletter
% \ekvdef{expex}{f}{\ekvsneakPre{\f{#1}}}
% \ekvdef{expex}{round}{\ekvsneakPre{\rnd{#1}}}
% \ekvdefNoVal{expex}{degree}{\ekvsneakPre{\deg{d}}}
% \ekvdefNoVal{expex}{radian}{\ekvsneakPre{\deg{}}}
% \ekvdefNoVal{expex}{internal}{\ekvsneakPre{\sine@rnd}}
% \newcommand*\sine[2]
%   {\ekvset{expex}{#1,internal}\rnd{3}\deg{d}\f{sin}\sine@stop{#2}}
% \end{lstlisting}
% For the sake of simplicity we defined the macro |\sine| with two mandatory
% arguments, the first being the \kv\ list, the second the argument to the
% trigonometric function. We could've used \pkg{xparse}'s facilities here to
% define an expandable macro which takes an optional argument instead.
%
% Now we need to define some internal macros to extract the value of each key's
% last usage (remember that this will be the group after the first special
% flag-token). For that we use one delimited macro per key.
% \begin{lstlisting}
% \def\sine@rnd#1\rnd#2#3\sine@stop{\sine@deg#1#3\sine@stop{#2}}
% \def\sine@deg#1\deg#2#3\sine@stop{\sine@f#1#3\sine@stop{#2}}
% \def\sine@f#1\f#2#3\sine@stop{\sine@final{#2}}
% \end{lstlisting}
% After the macros |\sine@rnd|, |\sine@deg|, and |\sine@f| the macro
% |\sine@final| will see
% \texttt
%   {^^A
%     \cs[no-index]{sine@final}^^A
%     \marg{f}\marg{degree/radian}\marg{round}\marg{num}^^A
%   }.
% Now |\sine@final| has to expandably deal with those arguments such that the
% |\fpeval| macro of \pkg{xfp} gets the correct input. Luckily this is pretty
% straight forward in this example. In |\fpeval| the trigonometric functions
% have names such as |sin| or |cos| and the degree taking variants |sind| or
% |cosd|. And since the |degree| key puts a |d| in |#2| and the |radian| key
% leaves |#2| empty all we have to do to get the correct function name is stick
% the two together.
% \begin{lstlisting}
% \newcommand*\sine@final[4]{\fpeval{round(#1#2(#4),#3)}}
% \makeatother
% \end{lstlisting}
% Let's test our macro:\\
% \begin{minipage}{.7\linewidth}
%   \begin{lstlisting}[gobble=4]
%   \sine{}{60}\par
%   \sine{round=10}{60}\par
%   \sine{f=cos,radian}{pi}\par
%   \edef\myval{\sine{f=tan}{1}}\texttt{\meaning\myval}
%   \end{lstlisting}
% \end{minipage}\hfill
% \begin{minipage}{.2\linewidth}
%   \makeatletter
%   \ekvdef{expex}{f}{\ekvsneakPre{\f{#1}}}^^A
%   \ekvdef{expex}{round}{\ekvsneakPre{\rnd{#1}}}^^A
%   \ekvdefNoVal{expex}{degree}{\ekvsneakPre{\deg{d}}}^^A
%   \ekvdefNoVal{expex}{radian}{\ekvsneakPre{\deg{}}}^^A
%   \ekvdefNoVal{expex}{internal}{\ekvsneakPre{\sine@rnd}}^^A
%   \newcommand*\sine[2]
%     {\ekvset{expex}{#1,internal}\rnd{3}\deg{d}\f{sin}\sine@stop{#2}}^^A
%   \def\sine@rnd#1\rnd#2#3\sine@stop{\sine@deg#1#3\sine@stop{#2}}^^A
%   \def\sine@deg#1\deg#2#3\sine@stop{\sine@f#1#3\sine@stop{#2}}^^A
%   \def\sine@f#1\f#2#3\sine@stop{\sine@final{#2}}^^A
%   \newcommand*\sine@final[4]{\fpeval{round(#1#2(#4),#3)}}^^A
%   \makeatother
%   \sine{}{60}\par
%   \sine{round=10}{60}\par
%   \sine{f=cos,radian}{pi}\par
%   \edef\myval{\sine{f=tan}{1}}\texttt{\meaning\myval}
% \end{minipage}
%
%
% \subsection{Error Messages}
%
% \expkv\ should only send messages in case of errors, there are no warnings and
% no info messages. In this subsection those errors are listed.
%
% \subsubsection{Load Time}
%
% \file{expkv.tex} checks whether \eTeX\ is available. If it isn't, an error
% will be thrown using |\errmessage|:
% \begin{lstlisting}
% ! expkv Error: e-TeX required.
% \end{lstlisting}
%
% \subsubsection{Defining Keys}
%
% If you get any error from \expkv\ while you're trying to define a key, the
% definition will be aborted and gobbled.
%
% If you try to define a key with an empty set name you'll get:
% \begin{lstlisting}
% ! expkv Error: empty set name not allowed.
% \end{lstlisting}
%
% Similarly, if you try to define a key with an empty key name:
% \begin{lstlisting}
% ! expkv Error: empty key name not allowed.
% \end{lstlisting}
%
% Both of these messages are done in a way that doesn't throw additional errors
% due to |\global|, |\long|, etc., not being used correctly if you prefixed one
% of the defining macros.
%
% \subsubsection{Using Keys}
%
% This subsubsection contains the errors thrown during |\ekvset|. The errors are
% thrown in an expandable manner by providing an undefined macro. In the
% following messages \lstinline|<key>| gets replaced with the problematic key's
% name, and \lstinline|<set>| with the corresponding set. If any errors during
% \kv\ handling are encountered, the entry in the comma separated list will be
% omitted after the error is thrown and the next \kv\ pair will be parsed.
%
% If you're using an undefined key you'll get:
% \begin{lstlisting}
% ! Undefined control sequence.
% <argument> \! expkv Error:
%                            unknown key (`<key>', set `<set>').
% \end{lstlisting}
%
% If you're using a key for which only a normal version and no |NoVal| version
% is defined, but don't provide a value, you'll get:
% \begin{lstlisting}
% ! Undefined control sequence.
% <argument> \! expkv Error:
%                            value required (`<key>', set `<set>').
% \end{lstlisting}
%
% If you're using a key for which only a |NoVal| version and no normal version
% is defined, but provide a value, you'll get:
% \begin{lstlisting}
% ! Undefined control sequence.
% <argument> \! expkv Error:
%                            value forbidden (`<key>', set `<set>').
% \end{lstlisting}
%
% If you're using a set for which you never executed one of the defining macros
% from \autoref{sec:define} you'll get a low level \TeX\ error, as that isn't
% actively tested by the parser (and hence will lead to undefined behaviour and
% not be gracefully ignored). The error will look like
% \begin{lstlisting}
% ! Missing \endcsname inserted.
% <to be read again>
%                    \! expkv Error: Set `<set>' undefined.
% \end{lstlisting}
%
% \subsection{License}
%
% Copyright \textcopyright\ 2020\unless\ifnum\year=2020--\the\year\fi\
% Jonathan P. Spratte
%
% \medskip\noindent
% This work may be distributed and/or modified under the conditions of the
% \LaTeX\ Project Public License (LPPL), either version 1.3c of this license or
% (at your option) any later version. The latest version of this license is in
% the file:
%
% \url{http://www.latex-project.org/lppl.txt}
%
% \noindent
% This work is ``maintained'' (as per LPPL maintenance status) by
%
% \mbox{Jonathan P. Spratte}.
%
% \end{documentation}^^A=<<
%
% \begin{implementation}^^A>>=
%
% \clearpage
%
% \section{Implementation}^^A>>=
%^^A the LaTeX package >>=
% \subsection{The \LaTeX\ Package}
% First we set up the \LaTeX\ package. That one doesn't really do much except
% |\input|ting the generic code and identifying itself as a package.
% \gobbledocstriptag
%<*pkg>
%    \begin{macrocode}
\def\ekv@tmp
  {%
    \ProvidesFile{expkv.tex}%
      [\ekvDate\space v\ekvVersion\space an expandable key=val implementation]%
  }
\input{expkv.tex}
\ProvidesPackage{expkv}%
  [\ekvDate\space v\ekvVersion\space an expandable key=val implementation]
%    \end{macrocode}
% \gobbledocstriptag
%</pkg>
%^^A=<<
%^^A main file >>=
% \subsection{The Generic Code}
% The rest of this implementation will be the generic code.
% \gobbledocstriptag
%<*tex>
%
% Check whether \eTeX\ is available -- \expkv\ requires \eTeX.
%    \begin{macrocode}
\begingroup\expandafter\expandafter\expandafter\endgroup
\expandafter\ifx\csname numexpr\endcsname\relax
  \errmessage{expkv requires e-TeX}
  \expandafter\endinput
\fi
%    \end{macrocode}
%
% We make sure that it's only input once:
%    \begin{macrocode}
\expandafter\ifx\csname ekvVersion\endcsname\relax
\else
  \expandafter\endinput
\fi
%    \end{macrocode}
% \begin{macro}{\ekvVersion,\ekvDate}
% We're on our first input, so lets store the version and date in a macro.
%    \begin{macrocode}
\def\ekvVersion{0.5a}
\def\ekvDate{2020-02-27}
%    \end{macrocode}
% \end{macro}
%
% If the \LaTeX\ format is loaded we want to be a good file and report back who
% we are, for this the package will have defined |\ekv@tmp| to use
% |\ProvidesFile|, else this will expand to a |\relax| and do no harm.
%    \begin{macrocode}
\csname ekv@tmp\endcsname
%    \end{macrocode}
%
% Store the category code of |@| to later be able to reset it and change it to
% 11 for now.
%    \begin{macrocode}
\expandafter\chardef\csname ekv@tmp\endcsname=\catcode`\@
\catcode`\@=11
%    \end{macrocode}
% |\ekv@tmp| might later be reused to gobble any prefixes which might be
% provided to |\ekvdef| and similar in case the names are invalid, we just
% temporarily use it here as means to store the current category code of |@| to
% restore it at the end of the file, we never care for the actual definition of
% it.
%
% \begin{macro}[internal]
%   {
%     \@gobble,\@firstofone,\@firstoftwo,\@secondoftwo,
%     \ekv@gobbleto@stop,\ekv@fi@secondoftwo,\ekv@gobble@mark
%   }
% Since branching tests are often more versatile than |\if...\else...\fi|
% constructs, we define helpers that are branching pretty fast. Also here are
% some other utility functions that just grab some tokens. The ones that are
% also contained in \LaTeX\ don't use the |ekv| prefix.
%    \begin{macrocode}
\long\def\@gobble#1{}
\long\def\@firstofone#1{#1}
\long\def\@firstoftwo#1#2{#1}
\long\def\@secondoftwo#1#2{#2}
\long\def\ekv@fi@secondoftwo\fi\@firstoftwo#1#2{\fi#2}
\long\def\ekv@gobbleto@stop#1\ekv@stop{}
\def\ekv@gobble@mark\ekv@mark{}
%    \end{macrocode}
% \end{macro}
% As you can see |\ekv@gobbleto@stop| uses a special marker |\ekv@stop|. The
% package will use three such markers, the one you've seen already, |\ekv@mark|
% and |\ekv@nil|. Contrarily to how for instance \pkg{expl3} does things, we
% don't define them, as we don't need them to have an actual meaning. This has
% the advantage that if they somehow get expanded -- which should never happen
% if things work out -- they'll throw an error directly.
%
% \begin{macro}[internal]
%   {
%     \ekv@ifempty,\ekv@ifempty@,\ekv@ifempty@true,\ekv@ifempty@false,
%     \ekv@ifempty@true@F
%   }
% We can test for a lot of things building on an if-empty test, so lets define a
% really fast one. Since some tests might have reversed logic (true if something
% is not empty) we also set up macros for the reversed branches.
%    \begin{macrocode}
\long\def\ekv@ifempty#1%
  {%
    \ekv@ifempty@\ekv@ifempty@A#1\ekv@ifempty@B\ekv@ifempty@true
      \ekv@ifempty@A\ekv@ifempty@B\@secondoftwo
  }
\long\def\ekv@ifempty@#1\ekv@ifempty@A\ekv@ifempty@B{}
\long\def\ekv@ifempty@true\ekv@ifempty@A\ekv@ifempty@B\@secondoftwo#1#2{#1}
\long\def\ekv@ifempty@false\ekv@ifempty@A\ekv@ifempty@B\@firstoftwo#1#2{#2}
\long\def\ekv@ifempty@true@F\ekv@ifempty@A\ekv@ifempty@B\@firstofone#1{}
%    \end{macrocode}
% \end{macro}
%
% \begin{macro}[internal]{\ekv@ifblank@}
% The obvious test that can be based on an if-empty is if-blank, meaning a test
% checking whether the argument is empty or consists only of spaces. Our version
% here will be tweaked a bit, as we want to check this, but with one leading
% |\ekv@mark| token that is to be ignored.
%    \begin{macrocode}
\long\def\ekv@ifblank@\ekv@mark#1{\ekv@ifempty@\ekv@ifempty@A}
%    \end{macrocode}
% \end{macro}
%
% \begin{macro}[internal]{\ekv@ifdefined}
% We'll need to check whether something is defined quite frequently, so why not
% define a macro that does this. The following test is expandable, slower than
% the typical expandable test for undefined control sequences, but faster for
% defined ones. Since we want to be as fast as possible for correct input, this
% is to be preferred.
%    \begin{macrocode}
\def\ekv@ifdefined#1%
  {%
    \expandafter
    \ifx\csname\ifcsname #1\endcsname #1\else relax\fi\endcsname\relax
      \ekv@fi@secondoftwo
    \fi
    \@firstoftwo
  }
%    \end{macrocode}
% \end{macro}
  % \begin{macro}[internal]
%   {
%     \ekv@ifdefined@pair,\ekv@ifdefined@pair@,\ekv@ifdefined@key,
%     \ekv@ifdefined@key@
%   }
% Since we can save some time if we only have to create the control sequence
% once when we know beforehand how we want to use it, we build some other macros
% for those cases (which we'll have quite often, once per key usage).
%    \begin{macrocode}
\def\ekv@ifdefined@pair#1#2%
  {%
    \expandafter\ekv@ifdefined@pair@
      \csname
        \ifcsname #1{#2}\endcsname
          #1{#2}%
        \else
          relax%
        \fi
      \endcsname
  }
\def\ekv@ifdefined@pair@#1%
  {%
    \ifx#1\relax
      \ekv@fi@secondoftwo
    \fi
    \@firstoftwo
    {\ekv@set@pair@#1\ekv@mark}%
  }
\def\ekv@ifdefined@key#1#2%
  {%
    \expandafter\ekv@ifdefined@key@
      \csname
        \ifcsname #1{#2}N\endcsname
          #1{#2}N%
        \else
          relax%
        \fi
      \endcsname
  }
\def\ekv@ifdefined@key@#1%
  {%
    \ifx#1\relax
      \ekv@fi@secondoftwo
    \fi
    \@firstoftwo#1%
  }
%    \end{macrocode}
% \end{macro}
%
% \begin{macro}{\ekv@name,\ekv@name@set,\ekv@name@key}
% The keys will all follow the same naming scheme, so we define it here.
%    \begin{macrocode}
\def\ekv@name#1#2{\ekv@name@set{#1}\ekv@name@key{#2}}
\def\ekv@name@set#1{ekv#1(}
\def\ekv@name@key#1{#1)}
%    \end{macrocode}
% \end{macro}
% \begin{macro}[internal]{\ekv@undefined@set}
% We can misuse the macro name we use to expandably store the set-name in a
% single token -- since this increases performance drastically, especially for
% long set-names -- to throw a more meaningful error message in case a set isn't
% defined. The name of |\ekv@undefined@set| is a little bit misleading, as it is
% called in either case inside of |\csname|, but the result will be a control
% sequence with meaning |\relax| if the set is undefined, hence will break the
% |\csname| building the key-macro which will throw the error message.
%    \begin{macrocode}
\def\ekv@undefined@set#1{! expkv Error: Set `#1' undefined.}
%    \end{macrocode}
% \end{macro}
% \begin{macro}[internal]{\ekv@checkvalid}
% We place some restrictions on the allowed names, though, namely sets and
% keys are not allowed to be empty -- blanks are fine (meaning \mbox{set-
% or} key-names consisting of spaces).
%    \begin{macrocode}
\protected\def\ekv@checkvalid#1#2%
  {%
    \ekv@ifempty{#1}%
      {%
        \def\ekv@tmp{}%
        \errmessage{expkv Error: empty set name not allowed}%
      }%
      {%
        \ekv@ifempty{#2}%
          {%
            \def\ekv@tmp{}%
            \errmessage{expkv Error: empty key name not allowed}%
          }%
          \@secondoftwo
      }%
    \@gobble
  }
%    \end{macrocode}
% \end{macro}
% \begin{macro}{\ekvifdefined,\ekvifdefinedNoVal}
% And provide user-level macros to test whether a key is defined.
%    \begin{macrocode}
\def\ekvifdefined#1#2{\ekv@ifdefined{\ekv@name{#1}{#2}}}
\def\ekvifdefinedNoVal#1#2{\ekv@ifdefined{\ekv@name{#1}{#2}N}}
%    \end{macrocode}
% \end{macro}
%
% \begin{macro}
%   {\ekvdef,\ekvdefNoVal,\ekvlet,\ekvletNoVal,\ekvletkv,\ekvletkvNoVal}
% Set up the key defining macros |\ekvdef| etc.
%    \begin{macrocode}
\protected\long\def\ekvdef#1#2#3%
  {%
    \ekv@checkvalid{#1}{#2}%
      {%
        \expandafter\def\csname\ekv@name{#1}{#2}\endcsname##1{#3}%
        \ekv@defset{#1}%
      }%
  }
\protected\long\def\ekvdefNoVal#1#2#3%
  {%
    \ekv@checkvalid{#1}{#2}%
      {%
        \expandafter\def\csname\ekv@name{#1}{#2}N\endcsname{#3}%
        \ekv@defset{#1}%
      }%
  }
\protected\def\ekvlet#1#2#3%
  {%
    \ekv@checkvalid{#1}{#2}%
      {%
        \expandafter\let\csname\ekv@name{#1}{#2}\endcsname#3%
        \ekv@defset{#1}%
      }%
  }
\protected\def\ekvletNoVal#1#2#3%
  {%
    \ekv@checkvalid{#1}{#2}%
      {%
        \expandafter\let\csname\ekv@name{#1}{#2}N\endcsname#3%
        \ekv@defset{#1}%
      }%
  }
\protected\def\ekvletkv#1#2#3#4%
  {%
    \ekv@checkvalid{#1}{#2}%
      {%
        \expandafter\let\csname\ekv@name{#1}{#2}\expandafter\endcsname
        \csname\ekv@name{#3}{#4}\endcsname
        \ekv@defset{#1}%
      }%
  }
\protected\def\ekvletkvNoVal#1#2#3#4%
  {%
    \ekv@checkvalid{#1}{#2}%
      {%
        \expandafter\let\csname\ekv@name{#1}{#2}N\expandafter\endcsname
        \csname\ekv@name{#3}{#4}N\endcsname
        \ekv@defset{#1}%
      }%
  }
%    \end{macrocode}
% \end{macro}
% \begin{macro}[internal]{\ekv@defset}
%   In order to enhance the speed the set name given to |\ekvset| will be turned
%   into a control sequence pretty early, so we have to define that control
%   sequence.
%    \begin{macrocode}
\protected\def\ekv@defset#1%
  {%
    \expandafter\edef\csname\ekv@undefined@set{#1}\endcsname##1%
      {\ekv@name@set{#1}\ekv@name@key{##1}}%
  }
%    \end{macrocode}
% \end{macro}
%
% \begin{macro}{\ekvset}
% Set up |\ekvset|, which should not be affected by active commas and equal
% signs. The equal signs are a bit harder to cope with and we'll do that later,
% but replacing the active commas with commas of category other can be done
% beforehand. That's why we define |\ekvset| here with a temporary meaning just
% to set up the things with two different category codes. |#1| will be a
% \texttt{,\textsubscript{13}} and |#2| will be a \texttt{=\textsubscript{13}}.
%    \begin{macrocode}
\def\ekvset#1#2{%
\endgroup
\long\def\ekvset##1##2%
  {%
    \expandafter\ekv@set\csname\ekv@undefined@set{##1}\endcsname
      \ekv@mark##2#1\ekv@stop#1{}%
  }
%    \end{macrocode}
% \end{macro}
% \begin{macro}[internal]{\ekv@set}
% |\ekv@set| will split the \kv\ list at active commas. Then it has to check
% whether there were unprotected other commas and resplit there.
%    \begin{macrocode}
\long\def\ekv@set##1##2#1%
  {%
%    \end{macrocode}
% Test whether we're at the end, if so invoke |\ekv@endset|,
%    \begin{macrocode}
    \ekv@gobbleto@markstop##2\ekv@endset\ekv@mark\ekv@stop
%    \end{macrocode}
% else go on with other commas,
%    \begin{macrocode}
    \ekv@set@other##1##2,\ekv@stop,%
%    \end{macrocode}
% and get the next active comma delimited \kv\ pair.
%    \begin{macrocode}
    \ekv@set##1\ekv@mark
  }
%    \end{macrocode}
% \end{macro}
% \begin{macro}[internal]{\ekv@endset}
% |\ekv@endset| is a hungry little macro. It will eat everything that remains
% of |\ekv@set| and unbrace the sneaked stuff.
%    \begin{macrocode}
\long\def\ekv@endset
    \ekv@mark\ekv@stop\ekv@set@other##1,\ekv@stop,\ekv@set##2\ekv@mark
    ##3%
  {##3}
%    \end{macrocode}
% \end{macro}
% \begin{macro}[internal]{\ekv@set@other}
% The macro |\ekv@set@other| is guaranteed to get only single \kv\ pairs.
% First we test whether we're done, if not split at equal signs. It is faster to
% first split at category 12 equal signs and only after that on actives.
% If there is no equal sign, we need to test whether we got a blank argument and
% if not this is a |NoVal| key.
%    \begin{macrocode}
\long\def\ekv@set@other##1##2,%
  {%
    \ekv@gobbleto@markstop##2\ekv@endset@other\ekv@mark\ekv@stop
    \ekv@ifhas@eq@other##2=\ekv@ifempty@B\ekv@ifempty@false
      \ekv@ifempty@A\ekv@ifempty@B\@firstoftwo
      {\ekv@set@eq@other##1##2\ekv@stop}%
      {%
        \ekv@ifhas@eq@active##2#2\ekv@ifempty@B\ekv@ifempty@false
          \ekv@ifempty@A\ekv@ifempty@B\@firstoftwo
          {\ekv@set@eq@active##1##2\ekv@stop}%
          {%
            \ekv@ifblank@##2\ekv@nil\ekv@ifempty@B\ekv@ifempty@true@F
              \ekv@ifempty@A\ekv@ifempty@B\@firstofone
              {\ekv@strip{##2}\ekv@set@key##1}%
          }%
      }%
    \ekv@set@other##1\ekv@mark%
  }
%    \end{macrocode}
% \end{macro}
% \begin{macro}[internal]{\ekv@set@eq@other}
% |\ekv@set@eq@other| might not be the correct break point, there might be an
% active equal sign in the currently parsed key-name. If so, we have to resplit.
% If the split is correct strip the key-name of outer spaces and braces and feed
% it to |\ekv@set@pair|.
%    \begin{macrocode}
\long\def\ekv@set@eq@other##1##2=%
  {%
    \ekv@ifhas@eq@active##2#2\ekv@ifempty@B\ekv@ifempty@false
      \ekv@ifempty@A\ekv@ifempty@B\@firstoftwo
      {\ekv@set@eq@active##1##2=}%
      {\ekv@strip{##2}\ekv@set@pair##1}%
  }
%    \end{macrocode}
% \end{macro}
% \begin{macro}[internal]{\ekv@set@eq@active}
% We need to handle the active equal signs.
%    \begin{macrocode}
\long\def\ekv@set@eq@active##1##2#2%
  {%
    \ekv@strip{##2}\ekv@set@pair##1%
  }
%    \end{macrocode}
% \end{macro}
% \begin{macro}[internal]
%   {\ekv@ifhas@eq@other,\ekv@ifhas@eq@active,\ekv@endset@other}
% And we have to set up the testing macros for our equal signs and
% |\ekv@endset@other|.
%    \begin{macrocode}
\long\def\ekv@ifhas@eq@other\ekv@mark##1={\ekv@ifempty@\ekv@ifempty@A}
\long\def\ekv@ifhas@eq@active\ekv@mark##1#2{\ekv@ifempty@\ekv@ifempty@A}
\long\def\ekv@endset@other
    \ekv@mark\ekv@stop
    \ekv@ifhas@eq@other##1=\ekv@ifempty@B\ekv@ifempty@false
    \ekv@ifempty@A\ekv@ifempty@B\@firstoftwo
    ##2%
    \ekv@set@other##3\ekv@mark
  {}
%    \end{macrocode}
% \end{macro}
% \begin{macro}{\ekvbreak,\ekvbreakPreSneak,\ekvbreakPostSneak}
% Provide macros that can completely stop the parsing of |\ekvset|, who knows
% what it'll be useful for.
%    \begin{macrocode}
\long\def\ekvbreak##1##2\ekv@stop#1##3{##1}
\long\def\ekvbreakPreSneak ##1##2\ekv@stop#1##3{##1##3}
\long\def\ekvbreakPostSneak##1##2\ekv@stop#1##3{##3##1}
%    \end{macrocode}
% \end{macro}
% \begin{macro}{\ekvsneak,\ekvsneakPre}
% One last thing we want to do for |\ekvset| is to provide macros that just
% smuggle stuff after |\ekvset|'s effects.
%    \begin{macrocode}
\long\def\ekvsneak##1##2\ekv@stop#1##3%
  {%
    ##2\ekv@stop#1{##3##1}%
  }
\long\def\ekvsneakPre##1##2\ekv@stop#1##3%
  {%
    ##2\ekv@stop#1{##1##3}%
  }
%    \end{macrocode}
% \end{macro}
%
% \begin{macro}{\ekvparse}
% Additionally to the |\ekvset| macro we also want to provide an |\ekvparse|
% macro, that has the same scope as |\keyval_parse:NNn| from \pkg{expl3}.
% This is pretty analogue to the |\ekvset| implementation, we just put an
% |\unexpanded| here and there instead of other macros to stop the |\expanded|
% on our output.
%    \begin{macrocode}
\long\def\ekvparse##1##2##3%
  {%
    \ekv@parse##1##2\ekv@mark##3#1\ekv@stop#1%
  }
%    \end{macrocode}
% \end{macro}
% \begin{macro}[internal]{\ekv@parse}
%    \begin{macrocode}
\long\def\ekv@parse##1##2##3#1%
  {%
    \ekv@gobbleto@markstop##3\ekv@endparse\ekv@mark\ekv@stop
    \ekv@parse@other##1##2##3,\ekv@stop,%
    \ekv@parse##1##2\ekv@mark
  }
%    \end{macrocode}
% \end{macro}
% \begin{macro}[internal]{\ekv@endparse}
%    \begin{macrocode}
\long\def\ekv@endparse
    \ekv@mark\ekv@stop\ekv@parse@other##1,\ekv@stop,\ekv@parse##2\ekv@mark
  {}
%    \end{macrocode}
% \end{macro}
% \begin{macro}[internal]{\ekv@parse@other}
%    \begin{macrocode}
\long\def\ekv@parse@other##1##2##3,%
  {%
    \ekv@gobbleto@markstop##3\ekv@endparse@other\ekv@mark\ekv@stop
    \ekv@ifhas@eq@other##3=\ekv@ifempty@B\ekv@ifempty@false
      \ekv@ifempty@A\ekv@ifempty@B\@firstoftwo
      {\ekv@parse@eq@other##3\ekv@stop##2}%
      {%
        \ekv@ifhas@eq@active##3#2\ekv@ifempty@B\ekv@ifempty@false
          \ekv@ifempty@A\ekv@ifempty@B\@firstoftwo
          {\ekv@parse@eq@active##3\ekv@stop##2}%
          {%
            \ekv@ifblank@##3\ekv@nil\ekv@ifempty@B\ekv@ifempty@true@F
              \ekv@ifempty@A\ekv@ifempty@B\@firstofone
              {\ekv@strip{##3}\ekv@parse@key##1}%
          }%
      }%
    \ekv@parse@other##1##2\ekv@mark
  }
%    \end{macrocode}
% \end{macro}
% \begin{macro}[internal]{\ekv@parse@eq@other}
%    \begin{macrocode}
\long\def\ekv@parse@eq@other##1=%
  {%
    \ekv@ifhas@eq@active##1#2\ekv@ifempty@B\ekv@ifempty@false
      \ekv@ifempty@A\ekv@ifempty@B\@firstoftwo
      {\ekv@parse@eq@active##1=}%
      {\ekv@strip{##1}\ekv@parse@pair\ekv@mark}%
  }
%    \end{macrocode}
% \end{macro}
% \begin{macro}[internal]{\ekv@parse@eq@active}
%    \begin{macrocode}
\long\def\ekv@parse@eq@active##1#2%
  {%
    \ekv@strip{##1}\ekv@parse@pair\ekv@mark
  }
%    \end{macrocode}
% \end{macro}
% \begin{macro}[internal]{\ekv@endparse@other}
%    \begin{macrocode}
\long\def\ekv@endparse@other
    \ekv@mark\ekv@stop
    \ekv@ifhas@eq@other##1=\ekv@ifempty@B\ekv@ifempty@false
    \ekv@ifempty@A\ekv@ifempty@B\@firstoftwo
    ##2%
    \ekv@parse@other##3\ekv@mark
  {}
%    \end{macrocode}
% \end{macro}
% \begin{macro}[internal]{\ekv@parse@pair,\ekv@parse@pair@}
%    \begin{macrocode}
\long\def\ekv@parse@pair##1##2\ekv@stop
  {%
    \ekv@strip{##2}\ekv@parse@pair@{##1}%
  }
\long\def\ekv@parse@pair@##1##2##3%
  {%
    \unexpanded{##3{##2}{##1}}%
  }
%    \end{macrocode}
% \end{macro}
% \begin{macro}[internal]{\ekv@parse@key}
%    \begin{macrocode}
\long\def\ekv@parse@key##1##2%
  {%
    \unexpanded{##2{##1}}%
  }
%    \end{macrocode}
% \end{macro}
%
% Finally really setting things up with |\ekvset|'s temporary meaning:
%    \begin{macrocode}
}
\begingroup
\catcode`\,=13
\catcode`\==13
\ekvset,=
%    \end{macrocode}
%
% \begin{macro}{\ekvchangeset}
% Provide a macro that is able to switch out the current \set\ in |\ekvset|.
% This operation is slow (by comparison, it should be slightly faster than
% |\ekvsneak|), but allows for something similar to \pkg{pgfkeys}'s
% \texttt{\meta{key}/.cd} mechanism. However this operation is more expensive
% than |/.cd| as we can't just redefine some token to reflect this, but have to
% switch out the set expandably, so this works similar to the |\ekvsneak| macros
% reading and reinserting the remainder of the \kv\ list.
%    \begin{macrocode}
\def\ekvchangeset#1%
  {%
    \expandafter\ekv@changeset\csname\ekv@undefined@set{#1}\endcsname\ekv@mark
  }
%    \end{macrocode}
% \end{macro}
% \begin{macro}[internal]{\ekv@changeset}
% This macro does the real change-out of |\ekvchangeset|. We introduced an
% |\ekv@mark| to not accidentally remove some braces which we have to remove
% again.
%    \begin{macrocode}
\long\def\ekv@changeset#1#2\ekv@set@other#3#4\ekv@set#5%
  {%
    \ekv@gobble@mark#2\ekv@set@other#1#4\ekv@set#1%
  }
%    \end{macrocode}
% \end{macro}
%
%
% \begin{macro}[internal]{\ekv@gobbleto@markstop}
% The |\ekv@gobbleto@markstop| can be used to test for |\ekv@stop| similar to
% our if-empty test, but instead of using tokens which are used nowhere else
% (|\ekv@ifempty@A| and |\ekv@ifempty@B|) it uses |\ekv@mark| and |\ekv@stop|.
%    \begin{macrocode}
\long\def\ekv@gobbleto@markstop#1\ekv@mark\ekv@stop{}
%    \end{macrocode}
% \end{macro}
%
% \begin{macro}[internal]{\ekv@set@pair,\ekv@set@pair@}
% |\ekv@set@pair| gets invoked with the space and brace stripped key-name as its
% first argument, the set-macro as the second argument, and following that is
% the key-value right delimited by an |\ekv@stop|.
%    \begin{macrocode}
\long\def\ekv@set@pair#1#2%
  {%
    \ekv@ifdefined@pair#2{#1}%
      {%
%    \end{macrocode}
% This branch will be executed if the key is not defined as an argument grabbing
% one. If so test whether there is a |NoVal| key of the same name or whether the
% key is unknown. Throw a meaningful error message and gobble the value.
%    \begin{macrocode}
        \ekv@ifdefined{#2{#1}N}%
          \ekv@err@noarg
          \ekv@err@unknown
          #2{#1}%
        \ekv@gobbleto@stop
      }%
  }
%    \end{macrocode}
% |\ekv@ifdefined@pair| will call |\ekv@set@pair@| if the key is correctly
% defined. This will then grab the value, strip outer spaces and braces from it
% and feed it to the key-macro. Afterwards |\ekv@set@other| will take control
% again.
%    \begin{macrocode}
\long\def\ekv@set@pair@#1#2\ekv@stop
  {%
    \ekv@strip{#2}#1%
  }
%    \end{macrocode}
% \end{macro}
%
% \begin{macro}[internal]{\ekv@set@key}
% Analogous to |\ekv@set@pair|, |\ekv@set@key| lets |\ekv@ifdefined@key| test
% whether a |NoVal| key is defined, else it'll throw a meaningful error message.
% Since we don't have to grab any value |\ekv@ifdefined@key| will invoke the
% key-macro and we're done here, |\ekv@set@other| will take over again.
%    \begin{macrocode}
\long\def\ekv@set@key#1#2%
  {%
    \ekv@ifdefined@key#2{#1}%
      {%
        \ekv@ifdefined{#2{#1}}%
          \ekv@err@reqval
          \ekv@err@unknown
          #2{#1}%
      }%
  }
%    \end{macrocode}
% \end{macro}
%
% \begin{macro}[internal]{\ekv@err,\ekv@err@}
% Since |\ekvset| is fully expandable as long as the code of the keys is (which
% is unlikely) we want to somehow throw expandable errors, in our case via
% undefined control sequences.
%    \begin{macrocode}
\begingroup
\edef\ekv@err
  {%
    \endgroup
    \unexpanded{\long\def\ekv@err}##1%
      {%
        \unexpanded{\expandafter\ekv@err@\@firstofone}%
        {\expandafter\noexpand\csname ! expkv Error:\endcsname ##1.}%
        \unexpanded{\ekv@stop}%
      }%
  }
\ekv@err
\def\ekv@err@{\expandafter\ekv@gobbleto@stop}
%    \end{macrocode}
% \end{macro}
% \begin{macro}[internal]
%   {
%     \ekv@err@common,\ekv@err@common@,
%     \ekv@err@unknown,\ekv@err@noarg,\ekv@err@reqval
%   }
% Now we can use |\ekv@err| to set up some error messages so that we can later
% use those instead of the full strings.
%    \begin{macrocode}
\long\def\ekv@err@common #1#2{\expandafter\ekv@err@common@\string#2{#1}}
\long\def\ekv@err@common@#1`#2' #3.#4#5{\ekv@err{#4 (`#5', set `#2')}}
\long\def\ekv@err@unknown#1#2{\ekv@err@common{unknown key}#1{#2}}
\long\def\ekv@err@noarg  #1#2{\ekv@err@common{value forbidden}#1{#2}}
\long\def\ekv@err@reqval #1#2{\ekv@err@common{value required}#1{#2}}
%    \end{macrocode}
% \end{macro}
%
% \begin{macro}[internal]{\ekv@strip,\ekv@strip@a,\ekv@strip@b,\ekv@strip@c}
% Finally we borrow some ideas of \pkg{expl3}'s \pkg{l3tl} to strip spaces
% from keys and values. This |\ekv@strip| also strips one level of outer braces
% \emph{after} stripping spaces, so an input of | {abc} | becomes |abc| after
% stripping. It should be used with |#1| prefixed by |\ekv@mark|. Also this
% implementation at most strips \emph{one} space from both sides.
%    \begin{macrocode}
\def\ekv@strip#1%
  {%
    \long\def\ekv@strip##1%
      {%
        \ekv@strip@a
          ##1%
          \ekv@nil
          \ekv@mark#1%
          #1\ekv@nil{}%
        \ekv@stop
      }%
    \long\def\ekv@strip@a##1\ekv@mark#1##2\ekv@nil##3%
      {%
        \ekv@strip@b##3##1##2\ekv@nil
      }%
    \long\def\ekv@strip@b##1#1\ekv@nil
      {%
        \ekv@strip@c##1\ekv@nil
      }%
    \long\def\ekv@strip@c\ekv@mark##1\ekv@nil##2\ekv@stop##3%
      {%
        ##3{##1}%
      }%
  }
\ekv@strip{ }
%    \end{macrocode}
% \end{macro}
%
% Now everything that's left is to reset the category code of |@|.
%    \begin{macrocode}
\catcode`\@=\ekv@tmp
%    \end{macrocode}
%
% \gobbledocstriptag
%</tex>
%^^A=<<
%
%^^A=<<
%
% \end{implementation}^^A=<<
%
% \clearpage
% \PrintIndex
%
\endinput
%
^^A vim: ft=tex fdm=marker fmr=>>=,=<<

\ProvidesPackage{expkv}%
  [\ekvDate\space v\ekvVersion\space an expandable key=val implementation]
%    \end{macrocode}
% \gobbledocstriptag
%</pkg>
%^^A=<<
%^^A main file >>=
% \subsection{The Generic Code}
% The rest of this implementation will be the generic code.
% \gobbledocstriptag
%<*tex>
%
% Check whether \eTeX\ is available -- \expkv\ requires \eTeX.
%    \begin{macrocode}
\begingroup\expandafter\expandafter\expandafter\endgroup
\expandafter\ifx\csname numexpr\endcsname\relax
  \errmessage{expkv requires e-TeX}
  \expandafter\endinput
\fi
%    \end{macrocode}
%
% We make sure that it's only input once:
%    \begin{macrocode}
\expandafter\ifx\csname ekvVersion\endcsname\relax
\else
  \expandafter\endinput
\fi
%    \end{macrocode}
% \begin{macro}{\ekvVersion,\ekvDate}
% We're on our first input, so lets store the version and date in a macro.
%    \begin{macrocode}
\def\ekvVersion{0.5a}
\def\ekvDate{2020-02-27}
%    \end{macrocode}
% \end{macro}
%
% If the \LaTeX\ format is loaded we want to be a good file and report back who
% we are, for this the package will have defined |\ekv@tmp| to use
% |\ProvidesFile|, else this will expand to a |\relax| and do no harm.
%    \begin{macrocode}
\csname ekv@tmp\endcsname
%    \end{macrocode}
%
% Store the category code of |@| to later be able to reset it and change it to
% 11 for now.
%    \begin{macrocode}
\expandafter\chardef\csname ekv@tmp\endcsname=\catcode`\@
\catcode`\@=11
%    \end{macrocode}
% |\ekv@tmp| might later be reused to gobble any prefixes which might be
% provided to |\ekvdef| and similar in case the names are invalid, we just
% temporarily use it here as means to store the current category code of |@| to
% restore it at the end of the file, we never care for the actual definition of
% it.
%
% \begin{macro}[internal]
%   {
%     \@gobble,\@firstofone,\@firstoftwo,\@secondoftwo,
%     \ekv@gobbleto@stop,\ekv@fi@secondoftwo,\ekv@gobble@mark
%   }
% Since branching tests are often more versatile than |\if...\else...\fi|
% constructs, we define helpers that are branching pretty fast. Also here are
% some other utility functions that just grab some tokens. The ones that are
% also contained in \LaTeX\ don't use the |ekv| prefix.
%    \begin{macrocode}
\long\def\@gobble#1{}
\long\def\@firstofone#1{#1}
\long\def\@firstoftwo#1#2{#1}
\long\def\@secondoftwo#1#2{#2}
\long\def\ekv@fi@secondoftwo\fi\@firstoftwo#1#2{\fi#2}
\long\def\ekv@gobbleto@stop#1\ekv@stop{}
\def\ekv@gobble@mark\ekv@mark{}
%    \end{macrocode}
% \end{macro}
% As you can see |\ekv@gobbleto@stop| uses a special marker |\ekv@stop|. The
% package will use three such markers, the one you've seen already, |\ekv@mark|
% and |\ekv@nil|. Contrarily to how for instance \pkg{expl3} does things, we
% don't define them, as we don't need them to have an actual meaning. This has
% the advantage that if they somehow get expanded -- which should never happen
% if things work out -- they'll throw an error directly.
%
% \begin{macro}[internal]
%   {
%     \ekv@ifempty,\ekv@ifempty@,\ekv@ifempty@true,\ekv@ifempty@false,
%     \ekv@ifempty@true@F
%   }
% We can test for a lot of things building on an if-empty test, so lets define a
% really fast one. Since some tests might have reversed logic (true if something
% is not empty) we also set up macros for the reversed branches.
%    \begin{macrocode}
\long\def\ekv@ifempty#1%
  {%
    \ekv@ifempty@\ekv@ifempty@A#1\ekv@ifempty@B\ekv@ifempty@true
      \ekv@ifempty@A\ekv@ifempty@B\@secondoftwo
  }
\long\def\ekv@ifempty@#1\ekv@ifempty@A\ekv@ifempty@B{}
\long\def\ekv@ifempty@true\ekv@ifempty@A\ekv@ifempty@B\@secondoftwo#1#2{#1}
\long\def\ekv@ifempty@false\ekv@ifempty@A\ekv@ifempty@B\@firstoftwo#1#2{#2}
\long\def\ekv@ifempty@true@F\ekv@ifempty@A\ekv@ifempty@B\@firstofone#1{}
%    \end{macrocode}
% \end{macro}
%
% \begin{macro}[internal]{\ekv@ifblank@}
% The obvious test that can be based on an if-empty is if-blank, meaning a test
% checking whether the argument is empty or consists only of spaces. Our version
% here will be tweaked a bit, as we want to check this, but with one leading
% |\ekv@mark| token that is to be ignored.
%    \begin{macrocode}
\long\def\ekv@ifblank@\ekv@mark#1{\ekv@ifempty@\ekv@ifempty@A}
%    \end{macrocode}
% \end{macro}
%
% \begin{macro}[internal]{\ekv@ifdefined}
% We'll need to check whether something is defined quite frequently, so why not
% define a macro that does this. The following test is expandable, slower than
% the typical expandable test for undefined control sequences, but faster for
% defined ones. Since we want to be as fast as possible for correct input, this
% is to be preferred.
%    \begin{macrocode}
\def\ekv@ifdefined#1%
  {%
    \expandafter
    \ifx\csname\ifcsname #1\endcsname #1\else relax\fi\endcsname\relax
      \ekv@fi@secondoftwo
    \fi
    \@firstoftwo
  }
%    \end{macrocode}
% \end{macro}
  % \begin{macro}[internal]
%   {
%     \ekv@ifdefined@pair,\ekv@ifdefined@pair@,\ekv@ifdefined@key,
%     \ekv@ifdefined@key@
%   }
% Since we can save some time if we only have to create the control sequence
% once when we know beforehand how we want to use it, we build some other macros
% for those cases (which we'll have quite often, once per key usage).
%    \begin{macrocode}
\def\ekv@ifdefined@pair#1#2%
  {%
    \expandafter\ekv@ifdefined@pair@
      \csname
        \ifcsname #1{#2}\endcsname
          #1{#2}%
        \else
          relax%
        \fi
      \endcsname
  }
\def\ekv@ifdefined@pair@#1%
  {%
    \ifx#1\relax
      \ekv@fi@secondoftwo
    \fi
    \@firstoftwo
    {\ekv@set@pair@#1\ekv@mark}%
  }
\def\ekv@ifdefined@key#1#2%
  {%
    \expandafter\ekv@ifdefined@key@
      \csname
        \ifcsname #1{#2}N\endcsname
          #1{#2}N%
        \else
          relax%
        \fi
      \endcsname
  }
\def\ekv@ifdefined@key@#1%
  {%
    \ifx#1\relax
      \ekv@fi@secondoftwo
    \fi
    \@firstoftwo#1%
  }
%    \end{macrocode}
% \end{macro}
%
% \begin{macro}{\ekv@name,\ekv@name@set,\ekv@name@key}
% The keys will all follow the same naming scheme, so we define it here.
%    \begin{macrocode}
\def\ekv@name#1#2{\ekv@name@set{#1}\ekv@name@key{#2}}
\def\ekv@name@set#1{ekv#1(}
\def\ekv@name@key#1{#1)}
%    \end{macrocode}
% \end{macro}
% \begin{macro}[internal]{\ekv@undefined@set}
% We can misuse the macro name we use to expandably store the set-name in a
% single token -- since this increases performance drastically, especially for
% long set-names -- to throw a more meaningful error message in case a set isn't
% defined. The name of |\ekv@undefined@set| is a little bit misleading, as it is
% called in either case inside of |\csname|, but the result will be a control
% sequence with meaning |\relax| if the set is undefined, hence will break the
% |\csname| building the key-macro which will throw the error message.
%    \begin{macrocode}
\def\ekv@undefined@set#1{! expkv Error: Set `#1' undefined.}
%    \end{macrocode}
% \end{macro}
% \begin{macro}[internal]{\ekv@checkvalid}
% We place some restrictions on the allowed names, though, namely sets and
% keys are not allowed to be empty -- blanks are fine (meaning \mbox{set-
% or} key-names consisting of spaces).
%    \begin{macrocode}
\protected\def\ekv@checkvalid#1#2%
  {%
    \ekv@ifempty{#1}%
      {%
        \def\ekv@tmp{}%
        \errmessage{expkv Error: empty set name not allowed}%
      }%
      {%
        \ekv@ifempty{#2}%
          {%
            \def\ekv@tmp{}%
            \errmessage{expkv Error: empty key name not allowed}%
          }%
          \@secondoftwo
      }%
    \@gobble
  }
%    \end{macrocode}
% \end{macro}
% \begin{macro}{\ekvifdefined,\ekvifdefinedNoVal}
% And provide user-level macros to test whether a key is defined.
%    \begin{macrocode}
\def\ekvifdefined#1#2{\ekv@ifdefined{\ekv@name{#1}{#2}}}
\def\ekvifdefinedNoVal#1#2{\ekv@ifdefined{\ekv@name{#1}{#2}N}}
%    \end{macrocode}
% \end{macro}
%
% \begin{macro}
%   {\ekvdef,\ekvdefNoVal,\ekvlet,\ekvletNoVal,\ekvletkv,\ekvletkvNoVal}
% Set up the key defining macros |\ekvdef| etc.
%    \begin{macrocode}
\protected\long\def\ekvdef#1#2#3%
  {%
    \ekv@checkvalid{#1}{#2}%
      {%
        \expandafter\def\csname\ekv@name{#1}{#2}\endcsname##1{#3}%
        \ekv@defset{#1}%
      }%
  }
\protected\long\def\ekvdefNoVal#1#2#3%
  {%
    \ekv@checkvalid{#1}{#2}%
      {%
        \expandafter\def\csname\ekv@name{#1}{#2}N\endcsname{#3}%
        \ekv@defset{#1}%
      }%
  }
\protected\def\ekvlet#1#2#3%
  {%
    \ekv@checkvalid{#1}{#2}%
      {%
        \expandafter\let\csname\ekv@name{#1}{#2}\endcsname#3%
        \ekv@defset{#1}%
      }%
  }
\protected\def\ekvletNoVal#1#2#3%
  {%
    \ekv@checkvalid{#1}{#2}%
      {%
        \expandafter\let\csname\ekv@name{#1}{#2}N\endcsname#3%
        \ekv@defset{#1}%
      }%
  }
\protected\def\ekvletkv#1#2#3#4%
  {%
    \ekv@checkvalid{#1}{#2}%
      {%
        \expandafter\let\csname\ekv@name{#1}{#2}\expandafter\endcsname
        \csname\ekv@name{#3}{#4}\endcsname
        \ekv@defset{#1}%
      }%
  }
\protected\def\ekvletkvNoVal#1#2#3#4%
  {%
    \ekv@checkvalid{#1}{#2}%
      {%
        \expandafter\let\csname\ekv@name{#1}{#2}N\expandafter\endcsname
        \csname\ekv@name{#3}{#4}N\endcsname
        \ekv@defset{#1}%
      }%
  }
%    \end{macrocode}
% \end{macro}
% \begin{macro}[internal]{\ekv@defset}
%   In order to enhance the speed the set name given to |\ekvset| will be turned
%   into a control sequence pretty early, so we have to define that control
%   sequence.
%    \begin{macrocode}
\protected\def\ekv@defset#1%
  {%
    \expandafter\edef\csname\ekv@undefined@set{#1}\endcsname##1%
      {\ekv@name@set{#1}\ekv@name@key{##1}}%
  }
%    \end{macrocode}
% \end{macro}
%
% \begin{macro}{\ekvset}
% Set up |\ekvset|, which should not be affected by active commas and equal
% signs. The equal signs are a bit harder to cope with and we'll do that later,
% but replacing the active commas with commas of category other can be done
% beforehand. That's why we define |\ekvset| here with a temporary meaning just
% to set up the things with two different category codes. |#1| will be a
% \texttt{,\textsubscript{13}} and |#2| will be a \texttt{=\textsubscript{13}}.
%    \begin{macrocode}
\def\ekvset#1#2{%
\endgroup
\long\def\ekvset##1##2%
  {%
    \expandafter\ekv@set\csname\ekv@undefined@set{##1}\endcsname
      \ekv@mark##2#1\ekv@stop#1{}%
  }
%    \end{macrocode}
% \end{macro}
% \begin{macro}[internal]{\ekv@set}
% |\ekv@set| will split the \kv\ list at active commas. Then it has to check
% whether there were unprotected other commas and resplit there.
%    \begin{macrocode}
\long\def\ekv@set##1##2#1%
  {%
%    \end{macrocode}
% Test whether we're at the end, if so invoke |\ekv@endset|,
%    \begin{macrocode}
    \ekv@gobbleto@markstop##2\ekv@endset\ekv@mark\ekv@stop
%    \end{macrocode}
% else go on with other commas,
%    \begin{macrocode}
    \ekv@set@other##1##2,\ekv@stop,%
%    \end{macrocode}
% and get the next active comma delimited \kv\ pair.
%    \begin{macrocode}
    \ekv@set##1\ekv@mark
  }
%    \end{macrocode}
% \end{macro}
% \begin{macro}[internal]{\ekv@endset}
% |\ekv@endset| is a hungry little macro. It will eat everything that remains
% of |\ekv@set| and unbrace the sneaked stuff.
%    \begin{macrocode}
\long\def\ekv@endset
    \ekv@mark\ekv@stop\ekv@set@other##1,\ekv@stop,\ekv@set##2\ekv@mark
    ##3%
  {##3}
%    \end{macrocode}
% \end{macro}
% \begin{macro}[internal]{\ekv@set@other}
% The macro |\ekv@set@other| is guaranteed to get only single \kv\ pairs.
% First we test whether we're done, if not split at equal signs. It is faster to
% first split at category 12 equal signs and only after that on actives.
% If there is no equal sign, we need to test whether we got a blank argument and
% if not this is a |NoVal| key.
%    \begin{macrocode}
\long\def\ekv@set@other##1##2,%
  {%
    \ekv@gobbleto@markstop##2\ekv@endset@other\ekv@mark\ekv@stop
    \ekv@ifhas@eq@other##2=\ekv@ifempty@B\ekv@ifempty@false
      \ekv@ifempty@A\ekv@ifempty@B\@firstoftwo
      {\ekv@set@eq@other##1##2\ekv@stop}%
      {%
        \ekv@ifhas@eq@active##2#2\ekv@ifempty@B\ekv@ifempty@false
          \ekv@ifempty@A\ekv@ifempty@B\@firstoftwo
          {\ekv@set@eq@active##1##2\ekv@stop}%
          {%
            \ekv@ifblank@##2\ekv@nil\ekv@ifempty@B\ekv@ifempty@true@F
              \ekv@ifempty@A\ekv@ifempty@B\@firstofone
              {\ekv@strip{##2}\ekv@set@key##1}%
          }%
      }%
    \ekv@set@other##1\ekv@mark%
  }
%    \end{macrocode}
% \end{macro}
% \begin{macro}[internal]{\ekv@set@eq@other}
% |\ekv@set@eq@other| might not be the correct break point, there might be an
% active equal sign in the currently parsed key-name. If so, we have to resplit.
% If the split is correct strip the key-name of outer spaces and braces and feed
% it to |\ekv@set@pair|.
%    \begin{macrocode}
\long\def\ekv@set@eq@other##1##2=%
  {%
    \ekv@ifhas@eq@active##2#2\ekv@ifempty@B\ekv@ifempty@false
      \ekv@ifempty@A\ekv@ifempty@B\@firstoftwo
      {\ekv@set@eq@active##1##2=}%
      {\ekv@strip{##2}\ekv@set@pair##1}%
  }
%    \end{macrocode}
% \end{macro}
% \begin{macro}[internal]{\ekv@set@eq@active}
% We need to handle the active equal signs.
%    \begin{macrocode}
\long\def\ekv@set@eq@active##1##2#2%
  {%
    \ekv@strip{##2}\ekv@set@pair##1%
  }
%    \end{macrocode}
% \end{macro}
% \begin{macro}[internal]
%   {\ekv@ifhas@eq@other,\ekv@ifhas@eq@active,\ekv@endset@other}
% And we have to set up the testing macros for our equal signs and
% |\ekv@endset@other|.
%    \begin{macrocode}
\long\def\ekv@ifhas@eq@other\ekv@mark##1={\ekv@ifempty@\ekv@ifempty@A}
\long\def\ekv@ifhas@eq@active\ekv@mark##1#2{\ekv@ifempty@\ekv@ifempty@A}
\long\def\ekv@endset@other
    \ekv@mark\ekv@stop
    \ekv@ifhas@eq@other##1=\ekv@ifempty@B\ekv@ifempty@false
    \ekv@ifempty@A\ekv@ifempty@B\@firstoftwo
    ##2%
    \ekv@set@other##3\ekv@mark
  {}
%    \end{macrocode}
% \end{macro}
% \begin{macro}{\ekvbreak,\ekvbreakPreSneak,\ekvbreakPostSneak}
% Provide macros that can completely stop the parsing of |\ekvset|, who knows
% what it'll be useful for.
%    \begin{macrocode}
\long\def\ekvbreak##1##2\ekv@stop#1##3{##1}
\long\def\ekvbreakPreSneak ##1##2\ekv@stop#1##3{##1##3}
\long\def\ekvbreakPostSneak##1##2\ekv@stop#1##3{##3##1}
%    \end{macrocode}
% \end{macro}
% \begin{macro}{\ekvsneak,\ekvsneakPre}
% One last thing we want to do for |\ekvset| is to provide macros that just
% smuggle stuff after |\ekvset|'s effects.
%    \begin{macrocode}
\long\def\ekvsneak##1##2\ekv@stop#1##3%
  {%
    ##2\ekv@stop#1{##3##1}%
  }
\long\def\ekvsneakPre##1##2\ekv@stop#1##3%
  {%
    ##2\ekv@stop#1{##1##3}%
  }
%    \end{macrocode}
% \end{macro}
%
% \begin{macro}{\ekvparse}
% Additionally to the |\ekvset| macro we also want to provide an |\ekvparse|
% macro, that has the same scope as |\keyval_parse:NNn| from \pkg{expl3}.
% This is pretty analogue to the |\ekvset| implementation, we just put an
% |\unexpanded| here and there instead of other macros to stop the |\expanded|
% on our output.
%    \begin{macrocode}
\long\def\ekvparse##1##2##3%
  {%
    \ekv@parse##1##2\ekv@mark##3#1\ekv@stop#1%
  }
%    \end{macrocode}
% \end{macro}
% \begin{macro}[internal]{\ekv@parse}
%    \begin{macrocode}
\long\def\ekv@parse##1##2##3#1%
  {%
    \ekv@gobbleto@markstop##3\ekv@endparse\ekv@mark\ekv@stop
    \ekv@parse@other##1##2##3,\ekv@stop,%
    \ekv@parse##1##2\ekv@mark
  }
%    \end{macrocode}
% \end{macro}
% \begin{macro}[internal]{\ekv@endparse}
%    \begin{macrocode}
\long\def\ekv@endparse
    \ekv@mark\ekv@stop\ekv@parse@other##1,\ekv@stop,\ekv@parse##2\ekv@mark
  {}
%    \end{macrocode}
% \end{macro}
% \begin{macro}[internal]{\ekv@parse@other}
%    \begin{macrocode}
\long\def\ekv@parse@other##1##2##3,%
  {%
    \ekv@gobbleto@markstop##3\ekv@endparse@other\ekv@mark\ekv@stop
    \ekv@ifhas@eq@other##3=\ekv@ifempty@B\ekv@ifempty@false
      \ekv@ifempty@A\ekv@ifempty@B\@firstoftwo
      {\ekv@parse@eq@other##3\ekv@stop##2}%
      {%
        \ekv@ifhas@eq@active##3#2\ekv@ifempty@B\ekv@ifempty@false
          \ekv@ifempty@A\ekv@ifempty@B\@firstoftwo
          {\ekv@parse@eq@active##3\ekv@stop##2}%
          {%
            \ekv@ifblank@##3\ekv@nil\ekv@ifempty@B\ekv@ifempty@true@F
              \ekv@ifempty@A\ekv@ifempty@B\@firstofone
              {\ekv@strip{##3}\ekv@parse@key##1}%
          }%
      }%
    \ekv@parse@other##1##2\ekv@mark
  }
%    \end{macrocode}
% \end{macro}
% \begin{macro}[internal]{\ekv@parse@eq@other}
%    \begin{macrocode}
\long\def\ekv@parse@eq@other##1=%
  {%
    \ekv@ifhas@eq@active##1#2\ekv@ifempty@B\ekv@ifempty@false
      \ekv@ifempty@A\ekv@ifempty@B\@firstoftwo
      {\ekv@parse@eq@active##1=}%
      {\ekv@strip{##1}\ekv@parse@pair\ekv@mark}%
  }
%    \end{macrocode}
% \end{macro}
% \begin{macro}[internal]{\ekv@parse@eq@active}
%    \begin{macrocode}
\long\def\ekv@parse@eq@active##1#2%
  {%
    \ekv@strip{##1}\ekv@parse@pair\ekv@mark
  }
%    \end{macrocode}
% \end{macro}
% \begin{macro}[internal]{\ekv@endparse@other}
%    \begin{macrocode}
\long\def\ekv@endparse@other
    \ekv@mark\ekv@stop
    \ekv@ifhas@eq@other##1=\ekv@ifempty@B\ekv@ifempty@false
    \ekv@ifempty@A\ekv@ifempty@B\@firstoftwo
    ##2%
    \ekv@parse@other##3\ekv@mark
  {}
%    \end{macrocode}
% \end{macro}
% \begin{macro}[internal]{\ekv@parse@pair,\ekv@parse@pair@}
%    \begin{macrocode}
\long\def\ekv@parse@pair##1##2\ekv@stop
  {%
    \ekv@strip{##2}\ekv@parse@pair@{##1}%
  }
\long\def\ekv@parse@pair@##1##2##3%
  {%
    \unexpanded{##3{##2}{##1}}%
  }
%    \end{macrocode}
% \end{macro}
% \begin{macro}[internal]{\ekv@parse@key}
%    \begin{macrocode}
\long\def\ekv@parse@key##1##2%
  {%
    \unexpanded{##2{##1}}%
  }
%    \end{macrocode}
% \end{macro}
%
% Finally really setting things up with |\ekvset|'s temporary meaning:
%    \begin{macrocode}
}
\begingroup
\catcode`\,=13
\catcode`\==13
\ekvset,=
%    \end{macrocode}
%
% \begin{macro}{\ekvchangeset}
% Provide a macro that is able to switch out the current \set\ in |\ekvset|.
% This operation is slow (by comparison, it should be slightly faster than
% |\ekvsneak|), but allows for something similar to \pkg{pgfkeys}'s
% \texttt{\meta{key}/.cd} mechanism. However this operation is more expensive
% than |/.cd| as we can't just redefine some token to reflect this, but have to
% switch out the set expandably, so this works similar to the |\ekvsneak| macros
% reading and reinserting the remainder of the \kv\ list.
%    \begin{macrocode}
\def\ekvchangeset#1%
  {%
    \expandafter\ekv@changeset\csname\ekv@undefined@set{#1}\endcsname\ekv@mark
  }
%    \end{macrocode}
% \end{macro}
% \begin{macro}[internal]{\ekv@changeset}
% This macro does the real change-out of |\ekvchangeset|. We introduced an
% |\ekv@mark| to not accidentally remove some braces which we have to remove
% again.
%    \begin{macrocode}
\long\def\ekv@changeset#1#2\ekv@set@other#3#4\ekv@set#5%
  {%
    \ekv@gobble@mark#2\ekv@set@other#1#4\ekv@set#1%
  }
%    \end{macrocode}
% \end{macro}
%
%
% \begin{macro}[internal]{\ekv@gobbleto@markstop}
% The |\ekv@gobbleto@markstop| can be used to test for |\ekv@stop| similar to
% our if-empty test, but instead of using tokens which are used nowhere else
% (|\ekv@ifempty@A| and |\ekv@ifempty@B|) it uses |\ekv@mark| and |\ekv@stop|.
%    \begin{macrocode}
\long\def\ekv@gobbleto@markstop#1\ekv@mark\ekv@stop{}
%    \end{macrocode}
% \end{macro}
%
% \begin{macro}[internal]{\ekv@set@pair,\ekv@set@pair@}
% |\ekv@set@pair| gets invoked with the space and brace stripped key-name as its
% first argument, the set-macro as the second argument, and following that is
% the key-value right delimited by an |\ekv@stop|.
%    \begin{macrocode}
\long\def\ekv@set@pair#1#2%
  {%
    \ekv@ifdefined@pair#2{#1}%
      {%
%    \end{macrocode}
% This branch will be executed if the key is not defined as an argument grabbing
% one. If so test whether there is a |NoVal| key of the same name or whether the
% key is unknown. Throw a meaningful error message and gobble the value.
%    \begin{macrocode}
        \ekv@ifdefined{#2{#1}N}%
          \ekv@err@noarg
          \ekv@err@unknown
          #2{#1}%
        \ekv@gobbleto@stop
      }%
  }
%    \end{macrocode}
% |\ekv@ifdefined@pair| will call |\ekv@set@pair@| if the key is correctly
% defined. This will then grab the value, strip outer spaces and braces from it
% and feed it to the key-macro. Afterwards |\ekv@set@other| will take control
% again.
%    \begin{macrocode}
\long\def\ekv@set@pair@#1#2\ekv@stop
  {%
    \ekv@strip{#2}#1%
  }
%    \end{macrocode}
% \end{macro}
%
% \begin{macro}[internal]{\ekv@set@key}
% Analogous to |\ekv@set@pair|, |\ekv@set@key| lets |\ekv@ifdefined@key| test
% whether a |NoVal| key is defined, else it'll throw a meaningful error message.
% Since we don't have to grab any value |\ekv@ifdefined@key| will invoke the
% key-macro and we're done here, |\ekv@set@other| will take over again.
%    \begin{macrocode}
\long\def\ekv@set@key#1#2%
  {%
    \ekv@ifdefined@key#2{#1}%
      {%
        \ekv@ifdefined{#2{#1}}%
          \ekv@err@reqval
          \ekv@err@unknown
          #2{#1}%
      }%
  }
%    \end{macrocode}
% \end{macro}
%
% \begin{macro}[internal]{\ekv@err,\ekv@err@}
% Since |\ekvset| is fully expandable as long as the code of the keys is (which
% is unlikely) we want to somehow throw expandable errors, in our case via
% undefined control sequences.
%    \begin{macrocode}
\begingroup
\edef\ekv@err
  {%
    \endgroup
    \unexpanded{\long\def\ekv@err}##1%
      {%
        \unexpanded{\expandafter\ekv@err@\@firstofone}%
        {\expandafter\noexpand\csname ! expkv Error:\endcsname ##1.}%
        \unexpanded{\ekv@stop}%
      }%
  }
\ekv@err
\def\ekv@err@{\expandafter\ekv@gobbleto@stop}
%    \end{macrocode}
% \end{macro}
% \begin{macro}[internal]
%   {
%     \ekv@err@common,\ekv@err@common@,
%     \ekv@err@unknown,\ekv@err@noarg,\ekv@err@reqval
%   }
% Now we can use |\ekv@err| to set up some error messages so that we can later
% use those instead of the full strings.
%    \begin{macrocode}
\long\def\ekv@err@common #1#2{\expandafter\ekv@err@common@\string#2{#1}}
\long\def\ekv@err@common@#1`#2' #3.#4#5{\ekv@err{#4 (`#5', set `#2')}}
\long\def\ekv@err@unknown#1#2{\ekv@err@common{unknown key}#1{#2}}
\long\def\ekv@err@noarg  #1#2{\ekv@err@common{value forbidden}#1{#2}}
\long\def\ekv@err@reqval #1#2{\ekv@err@common{value required}#1{#2}}
%    \end{macrocode}
% \end{macro}
%
% \begin{macro}[internal]{\ekv@strip,\ekv@strip@a,\ekv@strip@b,\ekv@strip@c}
% Finally we borrow some ideas of \pkg{expl3}'s \pkg{l3tl} to strip spaces
% from keys and values. This |\ekv@strip| also strips one level of outer braces
% \emph{after} stripping spaces, so an input of | {abc} | becomes |abc| after
% stripping. It should be used with |#1| prefixed by |\ekv@mark|. Also this
% implementation at most strips \emph{one} space from both sides.
%    \begin{macrocode}
\def\ekv@strip#1%
  {%
    \long\def\ekv@strip##1%
      {%
        \ekv@strip@a
          ##1%
          \ekv@nil
          \ekv@mark#1%
          #1\ekv@nil{}%
        \ekv@stop
      }%
    \long\def\ekv@strip@a##1\ekv@mark#1##2\ekv@nil##3%
      {%
        \ekv@strip@b##3##1##2\ekv@nil
      }%
    \long\def\ekv@strip@b##1#1\ekv@nil
      {%
        \ekv@strip@c##1\ekv@nil
      }%
    \long\def\ekv@strip@c\ekv@mark##1\ekv@nil##2\ekv@stop##3%
      {%
        ##3{##1}%
      }%
  }
\ekv@strip{ }
%    \end{macrocode}
% \end{macro}
%
% Now everything that's left is to reset the category code of |@|.
%    \begin{macrocode}
\catcode`\@=\ekv@tmp
%    \end{macrocode}
%
% \gobbledocstriptag
%</tex>
%^^A=<<
%
%^^A=<<
%
% \end{implementation}^^A=<<
%
% \clearpage
% \PrintIndex
%
\endinput
%
^^A vim: ft=tex fdm=marker fmr=>>=,=<<

\ProvidesPackage{expkv}%
  [\ekvDate\space v\ekvVersion\space an expandable key=val implementation]
%    \end{macrocode}
% \gobbledocstriptag
%</pkg>
%^^A=<<
%^^A main file >>=
% \subsection{The Generic Code}
% The rest of this implementation will be the generic code.
% \gobbledocstriptag
%<*tex>
%
% Check whether \eTeX\ is available -- \expkv\ requires \eTeX.
%    \begin{macrocode}
\begingroup\expandafter\expandafter\expandafter\endgroup
\expandafter\ifx\csname numexpr\endcsname\relax
  \errmessage{expkv requires e-TeX}
  \expandafter\endinput
\fi
%    \end{macrocode}
%
% We make sure that it's only input once:
%    \begin{macrocode}
\expandafter\ifx\csname ekvVersion\endcsname\relax
\else
  \expandafter\endinput
\fi
%    \end{macrocode}
% \begin{macro}{\ekvVersion,\ekvDate}
% We're on our first input, so lets store the version and date in a macro.
%    \begin{macrocode}
\def\ekvVersion{0.5a}
\def\ekvDate{2020-02-27}
%    \end{macrocode}
% \end{macro}
%
% If the \LaTeX\ format is loaded we want to be a good file and report back who
% we are, for this the package will have defined |\ekv@tmp| to use
% |\ProvidesFile|, else this will expand to a |\relax| and do no harm.
%    \begin{macrocode}
\csname ekv@tmp\endcsname
%    \end{macrocode}
%
% Store the category code of |@| to later be able to reset it and change it to
% 11 for now.
%    \begin{macrocode}
\expandafter\chardef\csname ekv@tmp\endcsname=\catcode`\@
\catcode`\@=11
%    \end{macrocode}
% |\ekv@tmp| might later be reused to gobble any prefixes which might be
% provided to |\ekvdef| and similar in case the names are invalid, we just
% temporarily use it here as means to store the current category code of |@| to
% restore it at the end of the file, we never care for the actual definition of
% it.
%
% \begin{macro}[internal]
%   {
%     \@gobble,\@firstofone,\@firstoftwo,\@secondoftwo,
%     \ekv@gobbleto@stop,\ekv@fi@secondoftwo,\ekv@gobble@mark
%   }
% Since branching tests are often more versatile than |\if...\else...\fi|
% constructs, we define helpers that are branching pretty fast. Also here are
% some other utility functions that just grab some tokens. The ones that are
% also contained in \LaTeX\ don't use the |ekv| prefix.
%    \begin{macrocode}
\long\def\@gobble#1{}
\long\def\@firstofone#1{#1}
\long\def\@firstoftwo#1#2{#1}
\long\def\@secondoftwo#1#2{#2}
\long\def\ekv@fi@secondoftwo\fi\@firstoftwo#1#2{\fi#2}
\long\def\ekv@gobbleto@stop#1\ekv@stop{}
\def\ekv@gobble@mark\ekv@mark{}
%    \end{macrocode}
% \end{macro}
% As you can see |\ekv@gobbleto@stop| uses a special marker |\ekv@stop|. The
% package will use three such markers, the one you've seen already, |\ekv@mark|
% and |\ekv@nil|. Contrarily to how for instance \pkg{expl3} does things, we
% don't define them, as we don't need them to have an actual meaning. This has
% the advantage that if they somehow get expanded -- which should never happen
% if things work out -- they'll throw an error directly.
%
% \begin{macro}[internal]
%   {
%     \ekv@ifempty,\ekv@ifempty@,\ekv@ifempty@true,\ekv@ifempty@false,
%     \ekv@ifempty@true@F
%   }
% We can test for a lot of things building on an if-empty test, so lets define a
% really fast one. Since some tests might have reversed logic (true if something
% is not empty) we also set up macros for the reversed branches.
%    \begin{macrocode}
\long\def\ekv@ifempty#1%
  {%
    \ekv@ifempty@\ekv@ifempty@A#1\ekv@ifempty@B\ekv@ifempty@true
      \ekv@ifempty@A\ekv@ifempty@B\@secondoftwo
  }
\long\def\ekv@ifempty@#1\ekv@ifempty@A\ekv@ifempty@B{}
\long\def\ekv@ifempty@true\ekv@ifempty@A\ekv@ifempty@B\@secondoftwo#1#2{#1}
\long\def\ekv@ifempty@false\ekv@ifempty@A\ekv@ifempty@B\@firstoftwo#1#2{#2}
\long\def\ekv@ifempty@true@F\ekv@ifempty@A\ekv@ifempty@B\@firstofone#1{}
%    \end{macrocode}
% \end{macro}
%
% \begin{macro}[internal]{\ekv@ifblank@}
% The obvious test that can be based on an if-empty is if-blank, meaning a test
% checking whether the argument is empty or consists only of spaces. Our version
% here will be tweaked a bit, as we want to check this, but with one leading
% |\ekv@mark| token that is to be ignored.
%    \begin{macrocode}
\long\def\ekv@ifblank@\ekv@mark#1{\ekv@ifempty@\ekv@ifempty@A}
%    \end{macrocode}
% \end{macro}
%
% \begin{macro}[internal]{\ekv@ifdefined}
% We'll need to check whether something is defined quite frequently, so why not
% define a macro that does this. The following test is expandable, slower than
% the typical expandable test for undefined control sequences, but faster for
% defined ones. Since we want to be as fast as possible for correct input, this
% is to be preferred.
%    \begin{macrocode}
\def\ekv@ifdefined#1%
  {%
    \expandafter
    \ifx\csname\ifcsname #1\endcsname #1\else relax\fi\endcsname\relax
      \ekv@fi@secondoftwo
    \fi
    \@firstoftwo
  }
%    \end{macrocode}
% \end{macro}
  % \begin{macro}[internal]
%   {
%     \ekv@ifdefined@pair,\ekv@ifdefined@pair@,\ekv@ifdefined@key,
%     \ekv@ifdefined@key@
%   }
% Since we can save some time if we only have to create the control sequence
% once when we know beforehand how we want to use it, we build some other macros
% for those cases (which we'll have quite often, once per key usage).
%    \begin{macrocode}
\def\ekv@ifdefined@pair#1#2%
  {%
    \expandafter\ekv@ifdefined@pair@
      \csname
        \ifcsname #1{#2}\endcsname
          #1{#2}%
        \else
          relax%
        \fi
      \endcsname
  }
\def\ekv@ifdefined@pair@#1%
  {%
    \ifx#1\relax
      \ekv@fi@secondoftwo
    \fi
    \@firstoftwo
    {\ekv@set@pair@#1\ekv@mark}%
  }
\def\ekv@ifdefined@key#1#2%
  {%
    \expandafter\ekv@ifdefined@key@
      \csname
        \ifcsname #1{#2}N\endcsname
          #1{#2}N%
        \else
          relax%
        \fi
      \endcsname
  }
\def\ekv@ifdefined@key@#1%
  {%
    \ifx#1\relax
      \ekv@fi@secondoftwo
    \fi
    \@firstoftwo#1%
  }
%    \end{macrocode}
% \end{macro}
%
% \begin{macro}{\ekv@name,\ekv@name@set,\ekv@name@key}
% The keys will all follow the same naming scheme, so we define it here.
%    \begin{macrocode}
\def\ekv@name#1#2{\ekv@name@set{#1}\ekv@name@key{#2}}
\def\ekv@name@set#1{ekv#1(}
\def\ekv@name@key#1{#1)}
%    \end{macrocode}
% \end{macro}
% \begin{macro}[internal]{\ekv@undefined@set}
% We can misuse the macro name we use to expandably store the set-name in a
% single token -- since this increases performance drastically, especially for
% long set-names -- to throw a more meaningful error message in case a set isn't
% defined. The name of |\ekv@undefined@set| is a little bit misleading, as it is
% called in either case inside of |\csname|, but the result will be a control
% sequence with meaning |\relax| if the set is undefined, hence will break the
% |\csname| building the key-macro which will throw the error message.
%    \begin{macrocode}
\def\ekv@undefined@set#1{! expkv Error: Set `#1' undefined.}
%    \end{macrocode}
% \end{macro}
% \begin{macro}[internal]{\ekv@checkvalid}
% We place some restrictions on the allowed names, though, namely sets and
% keys are not allowed to be empty -- blanks are fine (meaning \mbox{set-
% or} key-names consisting of spaces).
%    \begin{macrocode}
\protected\def\ekv@checkvalid#1#2%
  {%
    \ekv@ifempty{#1}%
      {%
        \def\ekv@tmp{}%
        \errmessage{expkv Error: empty set name not allowed}%
      }%
      {%
        \ekv@ifempty{#2}%
          {%
            \def\ekv@tmp{}%
            \errmessage{expkv Error: empty key name not allowed}%
          }%
          \@secondoftwo
      }%
    \@gobble
  }
%    \end{macrocode}
% \end{macro}
% \begin{macro}{\ekvifdefined,\ekvifdefinedNoVal}
% And provide user-level macros to test whether a key is defined.
%    \begin{macrocode}
\def\ekvifdefined#1#2{\ekv@ifdefined{\ekv@name{#1}{#2}}}
\def\ekvifdefinedNoVal#1#2{\ekv@ifdefined{\ekv@name{#1}{#2}N}}
%    \end{macrocode}
% \end{macro}
%
% \begin{macro}
%   {\ekvdef,\ekvdefNoVal,\ekvlet,\ekvletNoVal,\ekvletkv,\ekvletkvNoVal}
% Set up the key defining macros |\ekvdef| etc.
%    \begin{macrocode}
\protected\long\def\ekvdef#1#2#3%
  {%
    \ekv@checkvalid{#1}{#2}%
      {%
        \expandafter\def\csname\ekv@name{#1}{#2}\endcsname##1{#3}%
        \ekv@defset{#1}%
      }%
  }
\protected\long\def\ekvdefNoVal#1#2#3%
  {%
    \ekv@checkvalid{#1}{#2}%
      {%
        \expandafter\def\csname\ekv@name{#1}{#2}N\endcsname{#3}%
        \ekv@defset{#1}%
      }%
  }
\protected\def\ekvlet#1#2#3%
  {%
    \ekv@checkvalid{#1}{#2}%
      {%
        \expandafter\let\csname\ekv@name{#1}{#2}\endcsname#3%
        \ekv@defset{#1}%
      }%
  }
\protected\def\ekvletNoVal#1#2#3%
  {%
    \ekv@checkvalid{#1}{#2}%
      {%
        \expandafter\let\csname\ekv@name{#1}{#2}N\endcsname#3%
        \ekv@defset{#1}%
      }%
  }
\protected\def\ekvletkv#1#2#3#4%
  {%
    \ekv@checkvalid{#1}{#2}%
      {%
        \expandafter\let\csname\ekv@name{#1}{#2}\expandafter\endcsname
        \csname\ekv@name{#3}{#4}\endcsname
        \ekv@defset{#1}%
      }%
  }
\protected\def\ekvletkvNoVal#1#2#3#4%
  {%
    \ekv@checkvalid{#1}{#2}%
      {%
        \expandafter\let\csname\ekv@name{#1}{#2}N\expandafter\endcsname
        \csname\ekv@name{#3}{#4}N\endcsname
        \ekv@defset{#1}%
      }%
  }
%    \end{macrocode}
% \end{macro}
% \begin{macro}[internal]{\ekv@defset}
%   In order to enhance the speed the set name given to |\ekvset| will be turned
%   into a control sequence pretty early, so we have to define that control
%   sequence.
%    \begin{macrocode}
\protected\def\ekv@defset#1%
  {%
    \expandafter\edef\csname\ekv@undefined@set{#1}\endcsname##1%
      {\ekv@name@set{#1}\ekv@name@key{##1}}%
  }
%    \end{macrocode}
% \end{macro}
%
% \begin{macro}{\ekvset}
% Set up |\ekvset|, which should not be affected by active commas and equal
% signs. The equal signs are a bit harder to cope with and we'll do that later,
% but replacing the active commas with commas of category other can be done
% beforehand. That's why we define |\ekvset| here with a temporary meaning just
% to set up the things with two different category codes. |#1| will be a
% \texttt{,\textsubscript{13}} and |#2| will be a \texttt{=\textsubscript{13}}.
%    \begin{macrocode}
\def\ekvset#1#2{%
\endgroup
\long\def\ekvset##1##2%
  {%
    \expandafter\ekv@set\csname\ekv@undefined@set{##1}\endcsname
      \ekv@mark##2#1\ekv@stop#1{}%
  }
%    \end{macrocode}
% \end{macro}
% \begin{macro}[internal]{\ekv@set}
% |\ekv@set| will split the \kv\ list at active commas. Then it has to check
% whether there were unprotected other commas and resplit there.
%    \begin{macrocode}
\long\def\ekv@set##1##2#1%
  {%
%    \end{macrocode}
% Test whether we're at the end, if so invoke |\ekv@endset|,
%    \begin{macrocode}
    \ekv@gobbleto@markstop##2\ekv@endset\ekv@mark\ekv@stop
%    \end{macrocode}
% else go on with other commas,
%    \begin{macrocode}
    \ekv@set@other##1##2,\ekv@stop,%
%    \end{macrocode}
% and get the next active comma delimited \kv\ pair.
%    \begin{macrocode}
    \ekv@set##1\ekv@mark
  }
%    \end{macrocode}
% \end{macro}
% \begin{macro}[internal]{\ekv@endset}
% |\ekv@endset| is a hungry little macro. It will eat everything that remains
% of |\ekv@set| and unbrace the sneaked stuff.
%    \begin{macrocode}
\long\def\ekv@endset
    \ekv@mark\ekv@stop\ekv@set@other##1,\ekv@stop,\ekv@set##2\ekv@mark
    ##3%
  {##3}
%    \end{macrocode}
% \end{macro}
% \begin{macro}[internal]{\ekv@set@other}
% The macro |\ekv@set@other| is guaranteed to get only single \kv\ pairs.
% First we test whether we're done, if not split at equal signs. It is faster to
% first split at category 12 equal signs and only after that on actives.
% If there is no equal sign, we need to test whether we got a blank argument and
% if not this is a |NoVal| key.
%    \begin{macrocode}
\long\def\ekv@set@other##1##2,%
  {%
    \ekv@gobbleto@markstop##2\ekv@endset@other\ekv@mark\ekv@stop
    \ekv@ifhas@eq@other##2=\ekv@ifempty@B\ekv@ifempty@false
      \ekv@ifempty@A\ekv@ifempty@B\@firstoftwo
      {\ekv@set@eq@other##1##2\ekv@stop}%
      {%
        \ekv@ifhas@eq@active##2#2\ekv@ifempty@B\ekv@ifempty@false
          \ekv@ifempty@A\ekv@ifempty@B\@firstoftwo
          {\ekv@set@eq@active##1##2\ekv@stop}%
          {%
            \ekv@ifblank@##2\ekv@nil\ekv@ifempty@B\ekv@ifempty@true@F
              \ekv@ifempty@A\ekv@ifempty@B\@firstofone
              {\ekv@strip{##2}\ekv@set@key##1}%
          }%
      }%
    \ekv@set@other##1\ekv@mark%
  }
%    \end{macrocode}
% \end{macro}
% \begin{macro}[internal]{\ekv@set@eq@other}
% |\ekv@set@eq@other| might not be the correct break point, there might be an
% active equal sign in the currently parsed key-name. If so, we have to resplit.
% If the split is correct strip the key-name of outer spaces and braces and feed
% it to |\ekv@set@pair|.
%    \begin{macrocode}
\long\def\ekv@set@eq@other##1##2=%
  {%
    \ekv@ifhas@eq@active##2#2\ekv@ifempty@B\ekv@ifempty@false
      \ekv@ifempty@A\ekv@ifempty@B\@firstoftwo
      {\ekv@set@eq@active##1##2=}%
      {\ekv@strip{##2}\ekv@set@pair##1}%
  }
%    \end{macrocode}
% \end{macro}
% \begin{macro}[internal]{\ekv@set@eq@active}
% We need to handle the active equal signs.
%    \begin{macrocode}
\long\def\ekv@set@eq@active##1##2#2%
  {%
    \ekv@strip{##2}\ekv@set@pair##1%
  }
%    \end{macrocode}
% \end{macro}
% \begin{macro}[internal]
%   {\ekv@ifhas@eq@other,\ekv@ifhas@eq@active,\ekv@endset@other}
% And we have to set up the testing macros for our equal signs and
% |\ekv@endset@other|.
%    \begin{macrocode}
\long\def\ekv@ifhas@eq@other\ekv@mark##1={\ekv@ifempty@\ekv@ifempty@A}
\long\def\ekv@ifhas@eq@active\ekv@mark##1#2{\ekv@ifempty@\ekv@ifempty@A}
\long\def\ekv@endset@other
    \ekv@mark\ekv@stop
    \ekv@ifhas@eq@other##1=\ekv@ifempty@B\ekv@ifempty@false
    \ekv@ifempty@A\ekv@ifempty@B\@firstoftwo
    ##2%
    \ekv@set@other##3\ekv@mark
  {}
%    \end{macrocode}
% \end{macro}
% \begin{macro}{\ekvbreak,\ekvbreakPreSneak,\ekvbreakPostSneak}
% Provide macros that can completely stop the parsing of |\ekvset|, who knows
% what it'll be useful for.
%    \begin{macrocode}
\long\def\ekvbreak##1##2\ekv@stop#1##3{##1}
\long\def\ekvbreakPreSneak ##1##2\ekv@stop#1##3{##1##3}
\long\def\ekvbreakPostSneak##1##2\ekv@stop#1##3{##3##1}
%    \end{macrocode}
% \end{macro}
% \begin{macro}{\ekvsneak,\ekvsneakPre}
% One last thing we want to do for |\ekvset| is to provide macros that just
% smuggle stuff after |\ekvset|'s effects.
%    \begin{macrocode}
\long\def\ekvsneak##1##2\ekv@stop#1##3%
  {%
    ##2\ekv@stop#1{##3##1}%
  }
\long\def\ekvsneakPre##1##2\ekv@stop#1##3%
  {%
    ##2\ekv@stop#1{##1##3}%
  }
%    \end{macrocode}
% \end{macro}
%
% \begin{macro}{\ekvparse}
% Additionally to the |\ekvset| macro we also want to provide an |\ekvparse|
% macro, that has the same scope as |\keyval_parse:NNn| from \pkg{expl3}.
% This is pretty analogue to the |\ekvset| implementation, we just put an
% |\unexpanded| here and there instead of other macros to stop the |\expanded|
% on our output.
%    \begin{macrocode}
\long\def\ekvparse##1##2##3%
  {%
    \ekv@parse##1##2\ekv@mark##3#1\ekv@stop#1%
  }
%    \end{macrocode}
% \end{macro}
% \begin{macro}[internal]{\ekv@parse}
%    \begin{macrocode}
\long\def\ekv@parse##1##2##3#1%
  {%
    \ekv@gobbleto@markstop##3\ekv@endparse\ekv@mark\ekv@stop
    \ekv@parse@other##1##2##3,\ekv@stop,%
    \ekv@parse##1##2\ekv@mark
  }
%    \end{macrocode}
% \end{macro}
% \begin{macro}[internal]{\ekv@endparse}
%    \begin{macrocode}
\long\def\ekv@endparse
    \ekv@mark\ekv@stop\ekv@parse@other##1,\ekv@stop,\ekv@parse##2\ekv@mark
  {}
%    \end{macrocode}
% \end{macro}
% \begin{macro}[internal]{\ekv@parse@other}
%    \begin{macrocode}
\long\def\ekv@parse@other##1##2##3,%
  {%
    \ekv@gobbleto@markstop##3\ekv@endparse@other\ekv@mark\ekv@stop
    \ekv@ifhas@eq@other##3=\ekv@ifempty@B\ekv@ifempty@false
      \ekv@ifempty@A\ekv@ifempty@B\@firstoftwo
      {\ekv@parse@eq@other##3\ekv@stop##2}%
      {%
        \ekv@ifhas@eq@active##3#2\ekv@ifempty@B\ekv@ifempty@false
          \ekv@ifempty@A\ekv@ifempty@B\@firstoftwo
          {\ekv@parse@eq@active##3\ekv@stop##2}%
          {%
            \ekv@ifblank@##3\ekv@nil\ekv@ifempty@B\ekv@ifempty@true@F
              \ekv@ifempty@A\ekv@ifempty@B\@firstofone
              {\ekv@strip{##3}\ekv@parse@key##1}%
          }%
      }%
    \ekv@parse@other##1##2\ekv@mark
  }
%    \end{macrocode}
% \end{macro}
% \begin{macro}[internal]{\ekv@parse@eq@other}
%    \begin{macrocode}
\long\def\ekv@parse@eq@other##1=%
  {%
    \ekv@ifhas@eq@active##1#2\ekv@ifempty@B\ekv@ifempty@false
      \ekv@ifempty@A\ekv@ifempty@B\@firstoftwo
      {\ekv@parse@eq@active##1=}%
      {\ekv@strip{##1}\ekv@parse@pair\ekv@mark}%
  }
%    \end{macrocode}
% \end{macro}
% \begin{macro}[internal]{\ekv@parse@eq@active}
%    \begin{macrocode}
\long\def\ekv@parse@eq@active##1#2%
  {%
    \ekv@strip{##1}\ekv@parse@pair\ekv@mark
  }
%    \end{macrocode}
% \end{macro}
% \begin{macro}[internal]{\ekv@endparse@other}
%    \begin{macrocode}
\long\def\ekv@endparse@other
    \ekv@mark\ekv@stop
    \ekv@ifhas@eq@other##1=\ekv@ifempty@B\ekv@ifempty@false
    \ekv@ifempty@A\ekv@ifempty@B\@firstoftwo
    ##2%
    \ekv@parse@other##3\ekv@mark
  {}
%    \end{macrocode}
% \end{macro}
% \begin{macro}[internal]{\ekv@parse@pair,\ekv@parse@pair@}
%    \begin{macrocode}
\long\def\ekv@parse@pair##1##2\ekv@stop
  {%
    \ekv@strip{##2}\ekv@parse@pair@{##1}%
  }
\long\def\ekv@parse@pair@##1##2##3%
  {%
    \unexpanded{##3{##2}{##1}}%
  }
%    \end{macrocode}
% \end{macro}
% \begin{macro}[internal]{\ekv@parse@key}
%    \begin{macrocode}
\long\def\ekv@parse@key##1##2%
  {%
    \unexpanded{##2{##1}}%
  }
%    \end{macrocode}
% \end{macro}
%
% Finally really setting things up with |\ekvset|'s temporary meaning:
%    \begin{macrocode}
}
\begingroup
\catcode`\,=13
\catcode`\==13
\ekvset,=
%    \end{macrocode}
%
% \begin{macro}{\ekvchangeset}
% Provide a macro that is able to switch out the current \set\ in |\ekvset|.
% This operation is slow (by comparison, it should be slightly faster than
% |\ekvsneak|), but allows for something similar to \pkg{pgfkeys}'s
% \texttt{\meta{key}/.cd} mechanism. However this operation is more expensive
% than |/.cd| as we can't just redefine some token to reflect this, but have to
% switch out the set expandably, so this works similar to the |\ekvsneak| macros
% reading and reinserting the remainder of the \kv\ list.
%    \begin{macrocode}
\def\ekvchangeset#1%
  {%
    \expandafter\ekv@changeset\csname\ekv@undefined@set{#1}\endcsname\ekv@mark
  }
%    \end{macrocode}
% \end{macro}
% \begin{macro}[internal]{\ekv@changeset}
% This macro does the real change-out of |\ekvchangeset|. We introduced an
% |\ekv@mark| to not accidentally remove some braces which we have to remove
% again.
%    \begin{macrocode}
\long\def\ekv@changeset#1#2\ekv@set@other#3#4\ekv@set#5%
  {%
    \ekv@gobble@mark#2\ekv@set@other#1#4\ekv@set#1%
  }
%    \end{macrocode}
% \end{macro}
%
%
% \begin{macro}[internal]{\ekv@gobbleto@markstop}
% The |\ekv@gobbleto@markstop| can be used to test for |\ekv@stop| similar to
% our if-empty test, but instead of using tokens which are used nowhere else
% (|\ekv@ifempty@A| and |\ekv@ifempty@B|) it uses |\ekv@mark| and |\ekv@stop|.
%    \begin{macrocode}
\long\def\ekv@gobbleto@markstop#1\ekv@mark\ekv@stop{}
%    \end{macrocode}
% \end{macro}
%
% \begin{macro}[internal]{\ekv@set@pair,\ekv@set@pair@}
% |\ekv@set@pair| gets invoked with the space and brace stripped key-name as its
% first argument, the set-macro as the second argument, and following that is
% the key-value right delimited by an |\ekv@stop|.
%    \begin{macrocode}
\long\def\ekv@set@pair#1#2%
  {%
    \ekv@ifdefined@pair#2{#1}%
      {%
%    \end{macrocode}
% This branch will be executed if the key is not defined as an argument grabbing
% one. If so test whether there is a |NoVal| key of the same name or whether the
% key is unknown. Throw a meaningful error message and gobble the value.
%    \begin{macrocode}
        \ekv@ifdefined{#2{#1}N}%
          \ekv@err@noarg
          \ekv@err@unknown
          #2{#1}%
        \ekv@gobbleto@stop
      }%
  }
%    \end{macrocode}
% |\ekv@ifdefined@pair| will call |\ekv@set@pair@| if the key is correctly
% defined. This will then grab the value, strip outer spaces and braces from it
% and feed it to the key-macro. Afterwards |\ekv@set@other| will take control
% again.
%    \begin{macrocode}
\long\def\ekv@set@pair@#1#2\ekv@stop
  {%
    \ekv@strip{#2}#1%
  }
%    \end{macrocode}
% \end{macro}
%
% \begin{macro}[internal]{\ekv@set@key}
% Analogous to |\ekv@set@pair|, |\ekv@set@key| lets |\ekv@ifdefined@key| test
% whether a |NoVal| key is defined, else it'll throw a meaningful error message.
% Since we don't have to grab any value |\ekv@ifdefined@key| will invoke the
% key-macro and we're done here, |\ekv@set@other| will take over again.
%    \begin{macrocode}
\long\def\ekv@set@key#1#2%
  {%
    \ekv@ifdefined@key#2{#1}%
      {%
        \ekv@ifdefined{#2{#1}}%
          \ekv@err@reqval
          \ekv@err@unknown
          #2{#1}%
      }%
  }
%    \end{macrocode}
% \end{macro}
%
% \begin{macro}[internal]{\ekv@err,\ekv@err@}
% Since |\ekvset| is fully expandable as long as the code of the keys is (which
% is unlikely) we want to somehow throw expandable errors, in our case via
% undefined control sequences.
%    \begin{macrocode}
\begingroup
\edef\ekv@err
  {%
    \endgroup
    \unexpanded{\long\def\ekv@err}##1%
      {%
        \unexpanded{\expandafter\ekv@err@\@firstofone}%
        {\expandafter\noexpand\csname ! expkv Error:\endcsname ##1.}%
        \unexpanded{\ekv@stop}%
      }%
  }
\ekv@err
\def\ekv@err@{\expandafter\ekv@gobbleto@stop}
%    \end{macrocode}
% \end{macro}
% \begin{macro}[internal]
%   {
%     \ekv@err@common,\ekv@err@common@,
%     \ekv@err@unknown,\ekv@err@noarg,\ekv@err@reqval
%   }
% Now we can use |\ekv@err| to set up some error messages so that we can later
% use those instead of the full strings.
%    \begin{macrocode}
\long\def\ekv@err@common #1#2{\expandafter\ekv@err@common@\string#2{#1}}
\long\def\ekv@err@common@#1`#2' #3.#4#5{\ekv@err{#4 (`#5', set `#2')}}
\long\def\ekv@err@unknown#1#2{\ekv@err@common{unknown key}#1{#2}}
\long\def\ekv@err@noarg  #1#2{\ekv@err@common{value forbidden}#1{#2}}
\long\def\ekv@err@reqval #1#2{\ekv@err@common{value required}#1{#2}}
%    \end{macrocode}
% \end{macro}
%
% \begin{macro}[internal]{\ekv@strip,\ekv@strip@a,\ekv@strip@b,\ekv@strip@c}
% Finally we borrow some ideas of \pkg{expl3}'s \pkg{l3tl} to strip spaces
% from keys and values. This |\ekv@strip| also strips one level of outer braces
% \emph{after} stripping spaces, so an input of | {abc} | becomes |abc| after
% stripping. It should be used with |#1| prefixed by |\ekv@mark|. Also this
% implementation at most strips \emph{one} space from both sides.
%    \begin{macrocode}
\def\ekv@strip#1%
  {%
    \long\def\ekv@strip##1%
      {%
        \ekv@strip@a
          ##1%
          \ekv@nil
          \ekv@mark#1%
          #1\ekv@nil{}%
        \ekv@stop
      }%
    \long\def\ekv@strip@a##1\ekv@mark#1##2\ekv@nil##3%
      {%
        \ekv@strip@b##3##1##2\ekv@nil
      }%
    \long\def\ekv@strip@b##1#1\ekv@nil
      {%
        \ekv@strip@c##1\ekv@nil
      }%
    \long\def\ekv@strip@c\ekv@mark##1\ekv@nil##2\ekv@stop##3%
      {%
        ##3{##1}%
      }%
  }
\ekv@strip{ }
%    \end{macrocode}
% \end{macro}
%
% Now everything that's left is to reset the category code of |@|.
%    \begin{macrocode}
\catcode`\@=\ekv@tmp
%    \end{macrocode}
%
% \gobbledocstriptag
%</tex>
%^^A=<<
%
%^^A=<<
%
% \end{implementation}^^A=<<
%
% \clearpage
% \PrintIndex
%
\endinput
%
^^A vim: ft=tex fdm=marker fmr=>>=,=<<
}{}
\let\metaORIG\meta
\protected\def\meta #1{\texttt{\metaORIG{#1}}}
\renewcommand*\thefootnote{\fnsymbol{footnote}}
\colorlet{expkvred}{red!65!black}
\colorlet{expkvgrey}{black!75}
\makeatletter
^^A unfortunately, the vertical offset doesn't work well in huge bold, so we
^^A need a second version for the title
\newcommand*\expkv
  {^^A
    \texorpdfstring
      {^^A
        \begingroup
        \bfseries
        {\color{expkvgrey}e\kern-.05em x\kern-.05em}^^A
        \lower.493ex\hbox{{\color{expkvgrey}P}\kern-.1em{\color{expkvred}k}}^^A
        \kern-.18em{\color{expkvred}v}^^A
        \endgroup
      }
      {expkv}^^A
  }
\newcommand\kv{\meta{key}=\meta{value}}
\newcommand\key{\meta{key}}
\newcommand\val{\meta{value}}
\newcommand\set{\meta{set}}
\hypersetup{linkcolor=red!80!black,urlcolor=purple!80!black}
\makeatother
\begin{document}
  \title
    {^^A
      \texorpdfstring
        {^^A
          \huge
          \href{https://github.com/Skillmon/tex_expkv}
            {\expkv}^^A
          \\[\medskipamount]
          \Large an expandable \kv\ implementation^^A
        }{expkv - an expandable <key>=<value> implementation}^^A
    }
  \date{\csname ekv@date\endcsname\space v\csname ekv@version\endcsname}
  \author{Jonathan P. Spratte\thanks{\protect\randomize{jspratte@yahoo.de}}}
  \DocInput{expkv.dtx}
\end{document}
%</driver>^^A=<<
% \fi
%
% \maketitle
%
% \begin{abstract}
% \noindent\parfillskip=0pt
% \expkv\ provides a small interface for \kv\ parsing. The parsing macro is
% fully expandable, the \meta{code} of your keys might not. \expkv\ is pretty
% fast, but not the fastest available \kv\ solution (\pkg{keyval} for instance
% is twice as fast, but not expandable and it might strip braces it shouldn't
% have stripped).
% \end{abstract}
%
% \tableofcontents
%
% \begin{documentation}^^A>>=
%
% \section{Documentation}
%
% \expkv\ is usable as generic code or as a \LaTeX\ package. To use it, just
% use:
% \begin{lstlisting}
% \usepackage{expkv} % LaTeX
% \input expkv       % plainTeX
% \end{lstlisting}
% The \LaTeX\ package doesn't do more than \file{expkv.tex}, except calling
% |\ProvidesPackage| and setting things up such that \file{expkv.tex} will use
% |\ProvidesFile|.
%
% \subsection{Setting up Keys}\label{sec:define}
%
% Keys in \expkv\ (as in almost all other \kv\ implementations) belong to a
% \emph{set} such that different sets can contain keys of the same name. Unlike
% many other implementations \expkv\ doesn't provide means to set a default
% value, instead we have keys that take values and keys that don't, but both can
% have the same name (on the user level).
%
% The following macros are available to define new keys. Those macros containing
% ``|def|'' in their name can be prefixed by anything allowed to prefix |\def|,
% prefixes allowed for |\let| can prefix those with ``|let|'' in their name,
% accordingly. Neither \set\ nor \key\ are allowed to be empty for new keys and
% must not contain a |\par| or tokens that expand to it.
%
% \begin{function}{\ekvdef}
%   \begin{syntax}
%     \cs{ekvdef}\marg{set}\marg{key}\marg{code}
%   \end{syntax}
%   Defines a \key\ taking a value in a \set\ to expand to \meta{code}. In
%   \meta{code} you can use |#1| to refer to the given value.
% \end{function}
%
% \begin{function}{\ekvdefNoVal}
%   \begin{syntax}
%     \cs{ekvdefNoVal}\marg{set}\marg{key}\marg{code}
%   \end{syntax}
%   Defines a \key\ not taking a value in a \set\ to expand to \meta{code}
%   \key.
% \end{function}
%
% \begin{function}{\ekvlet}
%   \begin{syntax}
%     \cs{ekvlet}\marg{set}\marg{key}\meta{cs}
%   \end{syntax}
%   Let the value taking \key\ in \set\ to \meta{cs}, it is not checked whether
%   \meta{cs} exists or that it takes exactly one parameter.
% \end{function}
%
% \begin{function}{\ekvletNoVal}
%   \begin{syntax}
%     \cs{ekvletNoVal}\marg{set}\marg{key}\meta{cs}
%   \end{syntax}
%   Let the no value taking \key\ in \set\ to \meta{cs}, it is not checked
%   whether \meta{cs} exists or that it takes no parameter.
% \end{function}
%
% \begin{function}{\ekvletkv}
%   \begin{syntax}
%     \cs{ekvletkv}\marg{set}\marg{key}\marg{set2}\marg{key2}
%   \end{syntax}
%   Let the \key\ in \set\ to \meta{key2} in \meta{set2}, it is not checked
%   whether that second key exists.
% \end{function}
%
% \begin{function}{\ekvletkvNoVal}
%   \begin{syntax}
%     \cs{ekvletkvNoVal}\marg{set}\marg{key}\marg{set2}\marg{key2}
%   \end{syntax}
%   Let the \key\ in \set\ to \meta{key2} in \meta{set2}, it is not checked
%   whether that second key exists.
% \end{function}
%
% \subsection{Parsing Keys}
%
% \begin{function}{\ekvset}
%   \begin{syntax}
%     \cs{ekvset}\marg{set}\{\kv,\kv,\ldots\}
%   \end{syntax}
%   Splits \kv\ pairs on commas. From both \key\ and \val\ up to one space is
%   stripped from both ends, if then only a braced group remains the braces are
%   stripped as well. So |\ekvset{foo}{bar=baz}| and
%   |\ekvset{foo}{ {bar}= {baz} }| will both do
%   \texttt{\cs{\meta{foobarkey}}\{baz\}}, so you can hide commas, equal signs
%   and spaces at the ends of either \key\ or \val\ by putting braces around
%   them. If you omit the equal sign the code of the key created with the
%   |NoVal| variants described in \autoref{sec:define} will be executed.
%   |\ekvset| should be nestable.
% \end{function}
%
% \subsection{Example}
%
% Say we have a macro for which we want to create a \kv\ interface. The macro
% has a parameter, which is stored in the dimension \cs{ourdim} having a default
% value from its initialization. Now we want to be able to change that dimension
% with the |width| key to some specified value. For that we'd do
% \begin{lstlisting}
% \newdimen\ourdim
% \ourdim=150pt
% \protected\ekvdef{our}{width}{\ourdim=#1\relax}
% \end{lstlisting}
% as you can see, we use the set |our| here. We want the key to behave different
% if no value is specified. In that case the key should not use the its initial
% value, but be smart and determine the available space from \cs{hsize}, so we
% also define
% \begin{lstlisting}
% \protected\ekvdefNoVal{our}{width}{\ourdim=.9\hsize}
% \end{lstlisting}
% Now we set up our macro to use this \kv\ interface
% \begin{lstlisting}
% \protected\def\ourmacro#1{\begingroup\ekvset{our}{#1}\the\ourdim\endgroup}
% \end{lstlisting}
% Finally we can use our macro like in the following\\
% \begin{minipage}{6cm}
% \begin{lstlisting}
% \ourmacro{}\par
% \ourmacro{width}\par
% \ourmacro{width=5pt}\par
% \end{lstlisting}
% \end{minipage}^^A
% \begin{minipage}{\dimexpr\linewidth-6cm\relax}
%   \newdimen\ourdim\ourdim=150pt
%   \protected\ekvdef{our}{width}{\ourdim=#1\relax}^^A
%   \protected\ekvdefNoVal{our}{width}{\ourdim=.9\hsize}^^A
%   \protected\def\ourmacro#1^^A
%     {\begingroup\ekvset{our}{#1}\the\ourdim\endgroup}^^A
%   \ourmacro{}\par
%   \ourmacro{width}\par
%   \ourmacro{width=5pt}\par
% \end{minipage}
%
% \subsection{Miscellaneous}
%
% \subsubsection{Other Macros}
%
% \expkv\ provides some other macros which might be of interest.
%
% \begin{function}{\ekvifdefined,\ekvifNoValdefined}
%   \begin{syntax}
%     \cs{ekvifdefined}\marg{set}\marg{key}\marg{true}\marg{false}
%     \cs{ekvifNoValdefined}\marg{set}\marg{key}\marg{true}\marg{false}
%   \end{syntax}
%   These two macros test whether there is a \key\ in \set. It is false if
%   either a hash table entry doesn't exist for that key or its meaning is
%   |\relax|.
% \end{function}
%
% \begin{function}{\ekvbreak,\ekvbreakPreSneak,\ekvbreakPostSneak}
%   \begin{syntax}
%     \cs{ekvbreak}\marg{after}
%   \end{syntax}
%   Gobbles the remainder of the current |\ekvset| macro and its argument list
%   and reinserts \meta{after}. So this can be used to break out of |\ekvset|.
%   The first variant will also gobble anything that has been sneaked out using
%   |\ekvsneak| or |\ekvsneakPre|, while |\ekvbreakPreSneak| will put
%   \meta{after} before anything that has been smuggled and |\ekvbreakPostSneak|
%   will put \meta{after} after the stuff that has been sneaked out.
% \end{function}
%
% \begin{function}{\ekvsneak,\ekvsneakPre}
%   \begin{syntax}
%     \cs{ekvsneak}\marg{after}
%   \end{syntax}
%   Puts \meta{after} after the effects of |\ekvset|. The first variant will put
%   \meta{after} after any other tokens which might have been sneaked before,
%   while |\ekvsneakPre| will put \meta{after} before other smuggled stuff.
%   This reads and reinserts the remainder of the current |\ekvset| macro and
%   its argument list to do its job.
% \end{function}
%
% \bigskip
%
% \begin{function}{\ekv@name}
%   \begin{syntax}
%     \cs{ekv@name}\marg{set}\marg{key}
%   \end{syntax}
%   The names of the macros that correspond to a key in a set are build with
%   this macro. The default definition is
%   \texttt{\csname ekv@name\endcsname{\set}{\key}}, for |NoVal| keys an
%   additional |:n| gets appended irrespective of this macro's definition, so
%   their name is \texttt{\csname ekv@name\endcsname{\set}{\key}:n}.
% \end{function}
%
% \subsubsection{Bugs}
%
% You can either report bugs via mail (see the first page), or directly on
% github if you own an account there:
% \url{https://github.com/Skillmon/tex_expkv}
%
% \subsubsection{Comparisons}
%
% Comparisons of speed are done with a very simple test key and the help of the
% \pkg{l3benchmark} package. The key and its usage should be equivalent to
% \begin{lstlisting}
% \protected\ekvdef{test}{height}{\def\myheight{#1}}
% \ekvset{test}{ height = 6 }
% \end{lstlisting}
% and only the usage of the key, not its definition, is benchmarked. As far as I
% know \expkv\ is the only fully expandable \kv\ parser. I tried to compare
% \expkv\ to every \kv\ package listed on
% \href{https://ctan.org/topic/keyval}{CTAN}, however, one might notice that
% some of those are missing from this list. That's because I didn't get the
% others to work due to bugs, or because they just provide wrappers around other
% packages in this list. For the impatient, the essence of these comparisons
% regarding speed and buggy behaviour is contained in \autoref{tab:comp}.
%
% \paragraph{\pkg{keyval}} is almost 2.5 times faster and has a comparable
% feature set just a slightly different way how it handles keys without values.
% That might be considered a drawback, as it limits the versatility, but also as
% an advantage, as it might reduce doubled code. Keep in mind that as soon as
% someone loads \pkg{xkeyval} the performance of \pkg{keyval} gets replaced by
% \pkg{xkeyval}'s.
%
% Also \pkg{keyval} has a bug, which unfortunately can't really be resolved
% without breaking backwards compatibility for \emph{many} documents, namely it
% strips braces from the argument before stripping spaces if the argument isn't
% surrounded by spaces, also it might strip more than one set of braces. Hence
% all of the following are equivalent in their outcome, though the last two
% lines should result in something different than the first two:
% \begin{lstlisting}[belowskip=0pt]
% \setkeys{foo}{bar=baz}
% \setkeys{foo}{bar= {baz}}
% \setkeys{foo}{bar={ baz}}
% \setkeys{foo}{bar={{baz}}}
% \end{lstlisting}
%
% \paragraph{\pkg{xkeyval}} is ten times slower, but it provides more
% functionality, e.g., it has choice keys, boolean keys, and so on. It contains
% the same bug as \pkg{keyval} as it has to be compatible with it by design (it
% replaces \pkg{keyval}'s frontend), but also adds even more cases in which
% braces are stripped that shouldn't be stripped, worsening the situation.
%
% \paragraph{\pkg{ltxkeys}} is about 250 times slower -- which is funny, because
% it aims to be ``[\ldots] faster than these earlier packages [referring to
% \pkg{keyval} and \pkg{xkeyval}].'' Since it aims to have a bigger feature set
% than \pkg{xkeyval}, it most definitely also has a bigger feature set than
% \expkv. Also, it can't parse |\long| input, so as soon as your values contain
% a |\par|, it'll throw errors. Furthermore, \pkg{ltxkeys} doesn't strip outer
% braces at all by design, which, imho, is a weird design choice. In addition
% \pkg{ltxkeys} loads \pkg{catoptions} which is known to introduce bugs (e.g.,
% see \url{https://tex.stackexchange.com/questions/461783}).
%
% \paragraph{\pkg{l3keys}} is about four times slower, but has an, imho, great
% interface to define keys. It strips \emph{all} outer spaces, even if somehow
% multiple spaces ended up on either end. It offers more features, but is pretty
% much bound to \pkg{expl3} code. Whether that's a drawback is up to you.
%
% \paragraph{\pkg{pgfkeys}} is about two times slower, but has an
% \emph{enormous} feature set. It has the same or a very similar bug
% \pkg{keyval} has.
%
% \paragraph{\pkg{kvsetkeys} with \pkg{kvdefinekeys}} is about two and a half
% times slower, but it works even if commas and equals have category codes
% different from 12. Else the features of the keys are equal to those of
% \pkg{keyval}, the parser has more features, though.
%
% \paragraph{\pkg{options}} is almost as fast as \expkv\ (about
% \SI{15}{\percent} slower) and has a much bigger feature set. Unfortunately it
% also suffers from the premature unbracing bug \pkg{keyval} has.
%
% \paragraph{\pkg{simplekv}} is hard to compare because I don't speak French (so
% I don't understand the documentation) and from what I can see, there is no
% direct way to define the equivalent test key. Nevertheless, I tested the
% closest possible equivalent of my test key without too much extra code, it is
% five times slower and has hard to predict behaviour regarding brace and space
% stripping, similar to \pkg{keyval}. The tested definition was:
% \begin{lstlisting}[belowskip=0pt]
% \usepackage{simplekv}
% \setKVdefault[simplekv]{height={ abc}}                     % key setup 
% \setKV[simplekv]{ height = 6 }                             % benchmarked
% \end{lstlisting}
%
% \paragraph{\pkg{yax}} is 14 to 15 times slower. It has a pretty strange
% syntax, imho, and again a direct equivalent is hard to define. It has the
% premature unbracing bug, too. Also somehow loading \pkg{yax} broke
% \pkg{options} for me. The tested definition was:
% \begin{lstlisting}[belowskip=0pt]
% \usepackage{yax}
% \defactiveparameter yax {\storevalue\myheight yax:height } % key setup
% \setparameterlist{yax}{ height = 6 }                       % benchmarked
% \end{lstlisting}
%
% \begin{table}
%   \centering
%   \newcommand*\yes{\textcolor{red!80!black}  {yes}}^^A
%   \newcommand*\no {\textcolor{green!80!black}{no}}^^A
%   \caption[Comparison of \kv\ packages]
%     {^^A
%       Comparison of \kv\ packages. Operations are the required operations to
%       evaluate one key, measured with the \pkg{l3benchmark} package (and a
%       target time of \SI{20}{\second}), which only returns the three most
%       significant digits.^^A
%       \label{tab:comp}^^A
%     }
%   \begin{tabular}{lS[table-format=4.1]cc}
%     \toprule
%     Package & {Operations} & Brace-Bug & Date \\
%     \midrule
%     \pkg{keyval}    &   15.2 & \yes    & 2014-10-28 \\
%     \expkv          &   37.2 & \no     & 2020-01-12 \\
%     \pkg{options}   &   41.9 & \yes    & 2015-03-01 \\
%     \pkg{pgfkeys}   &   72.3 & \yes    & 2019-12-21 \\
%     \pkg{kvsetkeys} &   99.2 & \no     & 2019-12-15 \\
%     \pkg{l3keys}    &  143   & \no     & 2019-11-07 \\
%     \pkg{simplekv}  &  185   & \yes    & 2017-08-08 \\
%     \pkg{xkeyval}   &  446   & \yes    & 2014-12-03 \\
%     \pkg{yax}       &  543   & \yes    & 2010-01-22 \\
%     \pkg{ltxkeys}   & 9260   & \no     & 2012-11-17 \\
%     \bottomrule
%   \end{tabular}
% \end{table}
%
% \subsection{Error Messages}
%
% \expkv\ should only send messages in case of errors, there are no warnings and
% no info messages. In this subsection those errors are listed.
%
% \subsubsection{Load Time}
%
% \file{expkv.tex} checks whether \eTeX\ is available. If it isn't, an error
% will be thrown using |\errmessage|:
% \begin{lstlisting}
% ! expkv Error: e-TeX required.
% \end{lstlisting}
%
% \subsubsection{Defining Keys}
%
% If you get any error from \expkv\ while you're trying to define a key, the
% definition will be aborted and gobbled.
%
% If you try to define a key with an empty set name you'll get:
% \begin{lstlisting}
% ! expkv Error: empty set name not allowed.
% \end{lstlisting}
%
% Similarly, if you try to define a key with an empty key name:
% \begin{lstlisting}
% ! expkv Error: empty key name not allowed.
% \end{lstlisting}
%
% Both of these messages are done in a way that doesn't throw additional errors
% due to |\global|, |\long|, etc., not being used correctly if you prefixed one
% of the defining macros.
%
% \subsubsection{Using Keys}
%
% This subsubsection contains the errors thrown during |\ekvset|. The errors are
% thrown in an expandable manner by providing an undefined macro. In the
% following messages \lstinline|<key>| gets replaced with the problematic key's
% name, and \lstinline|<set>| with the corresponding set. If any errors during
% \kv\ handling are encountered, the entry in the comma separated list will be
% omitted after the error is thrown and the next \kv\ pair will be parsed.
%
% If you're using an undefined key you'll get:
% \begin{lstlisting}
% ! Undefined control sequence.
% <argument> \! expkv Error:
%                            unknown key (`<key>', set `<set>').
% \end{lstlisting}
%
% If you're using a key for which only a normal version and no |NoVal| version
% is defined, but don't provide a value, you'll get:
% \begin{lstlisting}
% ! Undefined control sequence.
% <argument> \! expkv Error:
%                            value required (`<key>', set `<set>').
% \end{lstlisting}
%
% If you're using a key for which only a |NoVal| version and no normal version
% is defined, but provide a value, you'll get:
% \begin{lstlisting}
% ! Undefined control sequence.
% <argument> \! expkv Error:
%                            value forbidden (`<key>', set `<set>').
% \end{lstlisting}
%
% \end{documentation}^^A=<<
%
% \begin{implementation}^^A>>=
%
% \clearpage
%
% \section{Implementation}^^A>>=
%^^A the LaTeX package >>=
% First we set up the \LaTeX\ package. That one doesn't really do much except
% |\input|ting the generic code and identifying itself as a package.
%    \begin{macrocode}
%<*pkg>
\def\ekv@stop
  {%
    \ProvidesFile{expkv.tex}%
      [%
        \ekv@date\space v\ekv@version\space
        an expandable key=val implementation%
      ]%
  }
% \iffalse meta-comment
%
% File: expkv.dtx Copyright (C) 2020 Jonathan P. Spratte
%
% This work  may be  distributed and/or  modified under  the conditions  of the
% LaTeX Project Public License (LPPL),  either version 1.3c  of this license or
% (at your option) any later version.  The latest version of this license is in
% the file:
%
%   http://www.latex-project.org/lppl.txt
%
% ------------------------------------------------------------------------------
%
%<*driver>^^A>>=
\def\nameofplainTeX{plain}
\ifx\fmtname\nameofplainTeX\else
  \expandafter\begingroup
\fi
\input l3docstrip.tex
\askforoverwritefalse
\preamble

--------------------------------------------------------------
expkv -- an expandable key=val implementation
E-mail: jspratte@yahoo.de
Released under the LaTeX Project Public License v1.3c or later
See http://www.latex-project.org/lppl.txt
--------------------------------------------------------------

Copyright (C) 2020 Jonathan P. Spratte

This  work may be  distributed and/or  modified under  the conditions  of the
LaTeX Project Public License (LPPL),  either version 1.3c  of this license or
(at your option) any later version.  The latest version of this license is in
the file:

  http://www.latex-project.org/lppl.txt

This work is "maintained" (as per LPPL maintenance status) by
  Jonathan P. Spratte.

This work consists of the file  expkv.dtx
and the derived files           expkv.pdf
                                expkv.sty
                                expkv.tex

\endpreamble
% stop docstrip adding \endinput
\postamble
\endpostamble
\generate{\file{expkv.sty}{\from{expkv.dtx}{pkg}}}
\generate{\file{expkv.tex}{\from{expkv.dtx}{tex}}}
\ifx\fmtname\nameofplainTeX
  \expandafter\endbatchfile
\else
  \expandafter\endgroup
\fi
%
\IfFileExists{expkv.tex}{% \iffalse meta-comment
%
% File: expkv.dtx Copyright (C) 2020 Jonathan P. Spratte
%
% This work  may be  distributed and/or  modified under  the conditions  of the
% LaTeX Project Public License (LPPL),  either version 1.3c  of this license or
% (at your option) any later version.  The latest version of this license is in
% the file:
%
%   http://www.latex-project.org/lppl.txt
%
% ------------------------------------------------------------------------------
%
%<*driver>^^A>>=
\def\nameofplainTeX{plain}
\ifx\fmtname\nameofplainTeX\else
  \expandafter\begingroup
\fi
\input l3docstrip.tex
\askforoverwritefalse
\preamble

--------------------------------------------------------------
expkv -- an expandable key=val implementation
E-mail: jspratte@yahoo.de
Released under the LaTeX Project Public License v1.3c or later
See http://www.latex-project.org/lppl.txt
--------------------------------------------------------------

Copyright (C) 2020 Jonathan P. Spratte

This  work may be  distributed and/or  modified under  the conditions  of the
LaTeX Project Public License (LPPL),  either version 1.3c  of this license or
(at your option) any later version.  The latest version of this license is in
the file:

  http://www.latex-project.org/lppl.txt

This work is "maintained" (as per LPPL maintenance status) by
  Jonathan P. Spratte.

This work consists of the file  expkv.dtx
and the derived files           expkv.pdf
                                expkv.sty
                                expkv.tex

\endpreamble
% stop docstrip adding \endinput
\postamble
\endpostamble
\generate{\file{expkv.sty}{\from{expkv.dtx}{pkg}}}
\generate{\file{expkv.tex}{\from{expkv.dtx}{tex}}}
\ifx\fmtname\nameofplainTeX
  \expandafter\endbatchfile
\else
  \expandafter\endgroup
\fi
%
\IfFileExists{expkv.tex}{% \iffalse meta-comment
%
% File: expkv.dtx Copyright (C) 2020 Jonathan P. Spratte
%
% This work  may be  distributed and/or  modified under  the conditions  of the
% LaTeX Project Public License (LPPL),  either version 1.3c  of this license or
% (at your option) any later version.  The latest version of this license is in
% the file:
%
%   http://www.latex-project.org/lppl.txt
%
% ------------------------------------------------------------------------------
%
%<*driver>^^A>>=
\def\nameofplainTeX{plain}
\ifx\fmtname\nameofplainTeX\else
  \expandafter\begingroup
\fi
\input l3docstrip.tex
\askforoverwritefalse
\preamble

--------------------------------------------------------------
expkv -- an expandable key=val implementation
E-mail: jspratte@yahoo.de
Released under the LaTeX Project Public License v1.3c or later
See http://www.latex-project.org/lppl.txt
--------------------------------------------------------------

Copyright (C) 2020 Jonathan P. Spratte

This  work may be  distributed and/or  modified under  the conditions  of the
LaTeX Project Public License (LPPL),  either version 1.3c  of this license or
(at your option) any later version.  The latest version of this license is in
the file:

  http://www.latex-project.org/lppl.txt

This work is "maintained" (as per LPPL maintenance status) by
  Jonathan P. Spratte.

This work consists of the file  expkv.dtx
and the derived files           expkv.pdf
                                expkv.sty
                                expkv.tex

\endpreamble
% stop docstrip adding \endinput
\postamble
\endpostamble
\generate{\file{expkv.sty}{\from{expkv.dtx}{pkg}}}
\generate{\file{expkv.tex}{\from{expkv.dtx}{tex}}}
\ifx\fmtname\nameofplainTeX
  \expandafter\endbatchfile
\else
  \expandafter\endgroup
\fi
%
\IfFileExists{expkv.tex}{\input{expkv.tex}}{}
\ProvidesFile{expkv.dtx}
  [\csname ekvDate\endcsname\ an expandable key=val implementation]
\PassOptionsToPackage{full}{textcomp}
\documentclass{l3doc}
\RequirePackage[oldstylenums,nott]{kpfonts}
\input{glyphtounicode}
\pdfgentounicode=1
\RequirePackage{xfp} ^^A required for an example
\RequirePackage{listings}
\RequirePackage{booktabs}
\RequirePackage{array}
\RequirePackage{collcell}
\RequirePackage{siunitx}
\RequirePackage{xcolor}
\RequirePackage{caption}
\RequirePackage{microtype}
\RequirePackage{accsupp}
\lstset
  {
    ,flexiblecolumns=false
    ,basewidth=.53em
    ,gobble=2
    ,basicstyle=\fontfamily{jkp}\itshape
    ,morekeywords=^^A
      {^^A
        \ekvdef,\ekvdefNoVal,\ekvlet,\ekvletNoVal,\ekvletkv,\ekvletkvNoVal,^^A
        \ekvset,\ekvparse,\ekvVersion,\ekvDate,\ekvifdefined,^^A
        \ekvifdefinedNoVal,\ekvbreak,\ekvbreakPreSneak,\ekvbreakPostSneak,^^A
        \ekvsneak,\ekvsneakPre,\ekvchangeset
      }
    ,morecomment=[l]\%
    ,commentstyle=\color[gray]{0.4}
    ,literate={\{}{{\CodeSymbol\{}}{1}
              {\}}{{\CodeSymbol\}}}{1}
    ^^A,literate=*{<key>}{\key}{4}{<set>}{\set}{4}
  }
\newcommand*\CodeSymbol[1]{\textbf{#1}}
\RequirePackage{randtext}
\let\metaORIG\meta
\protected\def\meta #1{\texttt{\metaORIG{#1}}}
\renewcommand*\thefootnote{\fnsymbol{footnote}}
\definecolor{expkvred}{HTML}{9F393D}
\colorlet{expkvgrey}{black!75}
\makeatletter
\newcommand*\expkv
  {^^A
    \texorpdfstring
      {^^A
        \mbox
          {^^A
            \BeginAccSupp{ActualText=expkv}^^A
            \href{https://github.com/Skillmon/tex_expkv}
              {^^A
                \rmfamily
                \bfseries
                {\color{expkvgrey}e\kern-.05em x\kern-.05em}^^A
                \lower.493ex
                  \hbox{{\color{expkvgrey}P}\kern-.1em{\color{expkvred}k}}^^A
                \kern-.18em{\color{expkvred}v}^^A
              }^^A
            \EndAccSupp{}^^A
          }^^A
      }
      {expkv}^^A
  }
\newcommand*\expkvd
  {^^A
    \texorpdfstring
      {^^A
        \mbox
          {^^A
            \BeginAccSupp{ActualText=expkv-def}^^A
            \rmfamily
            \bfseries
            {\color{expkvgrey}e\kern-.05em x\kern-.05em}^^A
            \lower.493ex
              \hbox{{\color{expkvgrey}P}\kern-.1em{\color{expkvred}k}}^^A
            \kern-.18em{\color{expkvred}v}^^A
            {\color{expkvgrey}^^A
              \kern.05em\rule[-.1ex]{.08em}{1.2ex}\kern.05em\textsc{def}^^A
            }^^A
            \EndAccSupp{}^^A
          }^^A
      }
      {expkv-def}^^A
  }
\newcommand\kv{\meta{key}=\meta{value}}
\newcommand\key{\meta{key}}
\newcommand\val{\meta{value}}
\newcommand\set{\meta{set}}
\hypersetup{linkcolor=red!80!black,urlcolor=purple!80!black}
\DoNotIndex{\def,\edef,\,,\=,\begingroup,\catcode,\chardef,\csname,\endcsname}
\DoNotIndex{\endgroup,\endinput,\errmessage,\expandafter,\input,\let,\long}
\DoNotIndex{\noexpand,\protected,\ProvidesFile,\ProvidesPackage,\relax,\space}
\DoNotIndex{\@,\unexpanded,\string,\expanded}
\DoNotIndex{\ifcsname}
\DoNotIndex{\ifx}
\DoNotIndex{\else}
\DoNotIndex{\fi}
\@gobble\fi ^^A ignoring \ifx and \ifcsname, but only one \fi
\@ifdefinable\gobbledocstriptag{\def\gobbledocstriptag#1>{}}
\newcommand*\pmso[1] ^^A poor man's strike out
  {^^A
    \leavevmode
    \begingroup
      \sbox0{#1}^^A
      \rlap{\vrule height .6ex depth -.5ex width \wd0\relax}^^A
      \usebox0\relax
    \endgroup
  }
\makeatother
\begin{document}
  \title
    {^^A
      \texorpdfstring
        {^^A
          \huge\expkv
          \\[\medskipamount]
          \Large an expandable \kv\ implementation^^A
        }{expkv - an expandable <key>=<value> implementation}^^A
    }
  \date{\ekvDate\space v\ekvVersion}
  \author{Jonathan P. Spratte\thanks{\protect\randomize{jspratte@yahoo.de}}}
  \DocInput{expkv.dtx}
\end{document}
%</driver>^^A=<<
% \fi
%
% \maketitle
% \renewcommand*\thefootnote{\arabic{footnote}}
%
% \begin{abstract}
% \noindent\parfillskip=0pt
% \expkv\ provides a small interface for \kv\ parsing. The parsing macro is
% fully expandable, the \meta{code} of your keys might be not. \expkv\ is pretty
% fast, but not the fastest available \kv\ solution (\pkg{keyval} for instance
% is thrice as fast, but not expandable and it might strip braces it shouldn't
% have stripped).
% \end{abstract}
%
% \tableofcontents
%
% \begin{documentation}^^A>>=
%
% \section{Documentation}
%
% \expkv\ provides an expandable \kv\ parser. The \kv\ pairs should be given as
% a comma separated list and the separator between a \key\ and the associated
% \val\ should be an equal sign. Both, the commas and the equal signs, might be
% of category 12 (other) or 13 (active). To support this is necessary as for
% example \pkg{babel} turns characters active for some languages, for instance
% the equal sign is turned active for Turkish.
%
% \expkv\ is usable as generic code or as a \LaTeX\ package. To use it, just
% use one of:
% \begin{lstlisting}
% \usepackage{expkv} % LaTeX
% \input expkv       % plainTeX
% \end{lstlisting}
% The \LaTeX\ package doesn't do more than \file{expkv.tex}, except calling
% |\ProvidesPackage| and setting things up such that \file{expkv.tex} will use
% |\ProvidesFile|.
%
% \subsection{Setting up Keys}\label{sec:define}
% \expkv\ provides a rather simple approach to setting up keys, similar to
% \pkg{keyval}. However there is an auxiliary package named
% \href{https://github.com/Skillmon/tex_expkv-def}{\expkvd} (the files use
% \file{expkv-def} instead because \string| isn't allowed in package names by
% CTAN) which provides a more sophisticated interface, similar to well
% established packages like \pkg{pgfkeys} or \pkg{l3keys}.
%
% Keys in \expkv\ (as in almost all other \kv\ implementations) belong to a
% \emph{set} such that different sets can contain keys of the same name. Unlike
% many other implementations \expkv\ doesn't provide means to set a default
% value, instead we have keys that take values and keys that don't (the latter
% are called |NoVal| keys by \expkv), but both can have the same name (on the
% user level).
%
% The following macros are available to define new keys. Those macros containing
% ``|def|'' in their name can be prefixed by anything allowed to prefix |\def|,
% prefixes allowed for |\let| can prefix those with ``|let|'' in their name,
% accordingly. Neither \set\ nor \key\ are allowed to be empty for new keys and
% must not contain a |\par| or tokens that expand to it -- they must be legal
% inside of |\csname ...\endcsname|.
%
% \begin{function}{\ekvdef}
%   \begin{syntax}
%     \cs{ekvdef}\marg{set}\marg{key}\marg{code}
%   \end{syntax}
%   Defines a \key\ taking a value in a \set\ to expand to \meta{code}. In
%   \meta{code} you can use |#1| to refer to the given value.
% \end{function}
%
% \begin{function}{\ekvdefNoVal}
%   \begin{syntax}
%     \cs{ekvdefNoVal}\marg{set}\marg{key}\marg{code}
%   \end{syntax}
%   Defines a no value taking \key\ in a \set\ to expand to \meta{code}.
% \end{function}
%
% \begin{function}{\ekvlet}
%   \begin{syntax}
%     \cs{ekvlet}\marg{set}\marg{key}\meta{cs}
%   \end{syntax}
%   Let the value taking \key\ in \set\ to \meta{cs}, there are no checks on
%   \meta{cs} enforced.
% \end{function}
%
% \begin{function}{\ekvletNoVal}
%   \begin{syntax}
%     \cs{ekvletNoVal}\marg{set}\marg{key}\meta{cs}
%   \end{syntax}
%   Let the no value taking \key\ in \set\ to \meta{cs}, it is not checked
%   whether \meta{cs} exists or that it takes no parameter.
% \end{function}
%
% \begin{function}{\ekvletkv}
%   \begin{syntax}
%     \cs{ekvletkv}\marg{set}\marg{key}\marg{set2}\marg{key2}
%   \end{syntax}
%   Let the \key\ in \set\ to \meta{key2} in \meta{set2}, it is not checked
%   whether that second key exists.
% \end{function}
%
% \begin{function}{\ekvletkvNoVal}
%   \begin{syntax}
%     \cs{ekvletkvNoVal}\marg{set}\marg{key}\marg{set2}\marg{key2}
%   \end{syntax}
%   Let the \key\ in \set\ to \meta{key2} in \meta{set2}, it is not checked
%   whether that second key exists.
% \end{function}
%
% \subsection{Parsing Keys}
%
% \begin{function}{\ekvset}
%   \begin{syntax}
%     \cs{ekvset}\marg{set}\{\kv,\ldots\}
%   \end{syntax}
%   Splits \kv\ pairs on commas. From both \key\ and \val\ up to one space is
%   stripped from both ends, if then only a braced group remains the braces are
%   stripped as well. So |\ekvset{foo}{bar=baz}| and
%   |\ekvset{foo}{ {bar}= {baz} }| will both do
%   \texttt{\cs[no-index]{\meta{foobarcode}}\{baz\}}, so you can hide commas,
%   equal signs and spaces at the ends of either \key\ or \val\ by putting
%   braces around them. If you omit the equal sign the code of the key created
%   with the |NoVal| variants described in \autoref{sec:define} will be
%   executed. If \kv\ contains more than a single unhidden equal sign, it will
%   be split at the first one and the others are considered part of the value.
%   |\ekvset| should be nestable.
% \end{function}
%
% \begin{function}{\ekvparse}
%   \begin{syntax}
%     \cs{ekvparse}\meta{cs1}\meta{cs2}\{\kv,\ldots\}
%   \end{syntax}
%   This macro parses the \kv\ pairs and provides those list elements which are
%   only keys as the argument to \meta{cs1}, and those which are a \kv\ pair to
%   \meta{cs2} as two arguments. It is fully expandable as well and returns the
%   parsed list in |\unexpanded|, which has no effect outside of an |\expanded|
%   or |\edef| context\footnotemark.
%   If you need control over the necessary steps of expansion you
%   can use |\expanded| around it.
%
%   |\ekvbreak|, |\ekvsneak|, and |\ekvchangeset| and their relatives don't work
%   in |\ekvparse|.  It is analogue to \pkg{expl3}'s |\keyval_parse:NNn|, but
%   not with the same parsing rules -- |\keyval_parse:NNn| throws an error on
%   multiple equal signs per \kv\ pair and on empty \key\ names in a \kv\ pair,
%   both of which |\ekvparse| doesn't deal with.
%
%   As a small example:
% \begin{lstlisting}
% \ekvparse\handlekey\handlekeyval{foo = bar, key, baz={zzz}}
% \end{lstlisting}
%   would expand to
% \begin{lstlisting}
% \handlekeyval{foo}{bar}\handlekey{key}\handlekeyval{baz}{zzz}
% \end{lstlisting}
%   and afterwards |\handlekey| and |\handlekeyval| would have to further handle
%   the \key. There are no macros like these two contained in \expkv, you have
%   to set them up yourself if you want to use |\ekvparse| (of course the names
%   might differ). If you need the results of |\ekvparse| as the argument for
%   another macro, you should use |\expanded| as only then the input stream will
%   contain the output above:
% \begin{lstlisting}
% \expandafter\handle\expanded{\ekvparse\k\kv{foo = bar, key, baz={zzz}}}
% \end{lstlisting}
%   would expand to
% \begin{lstlisting}
% \handle\kv{foo}{bar}\k{key}\kv{baz}{zzz}
% \end{lstlisting}
% \end{function}
% \footnotetext 
%   {^^A
%     This is a change in behaviour, previously (v0.3 and before) \cs{ekvparse}
%     would expand in exactly two steps. This isn't always necessary, but makes
%     the parsing considerably slower. If this is necessary for your application
%     you can put an \cs[no-index]{expanded} around it and will still be faster
%     since you need only a single \cs[no-index]{expandafter} this way.^^A
%   }.
%
% \subsection{Miscellaneous}
%
% \subsubsection{Other Macros}
%
% \expkv\ provides some other macros which might be of interest.
%
% \begin{function}{\ekvVersion,\ekvDate}
%   These two macros store the version and date of the package.
% \end{function}
% \begin{function}{\ekvifdefined,\ekvifdefinedNoVal}
%   \begin{syntax}
%     \cs{ekvifdefined}\marg{set}\marg{key}\marg{true}\marg{false}
%     \cs{ekvifdefinedNoVal}\marg{set}\marg{key}\marg{true}\marg{false}
%   \end{syntax}
%   These two macros test whether there is a \key\ in \set. It is false if
%   either a hash table entry doesn't exist for that key or its meaning is
%   |\relax|.
% \end{function}
%
% \begin{function}{\ekvbreak,\ekvbreakPreSneak,\ekvbreakPostSneak}
%   \begin{syntax}
%     \cs{ekvbreak}\marg{after}
%   \end{syntax}
%   Gobbles the remainder of the current |\ekvset| macro and its argument list
%   and reinserts \meta{after}. So this can be used to break out of |\ekvset|.
%   The first variant will also gobble anything that has been sneaked out using
%   |\ekvsneak| or |\ekvsneakPre|, while |\ekvbreakPreSneak| will put
%   \meta{after} before anything that has been smuggled and |\ekvbreakPostSneak|
%   will put \meta{after} after the stuff that has been sneaked out.
% \end{function}
%
% \begin{function}{\ekvsneak,\ekvsneakPre}
%   \begin{syntax}
%     \cs{ekvsneak}\marg{after}
%   \end{syntax}
%   Puts \meta{after} after the effects of |\ekvset|. The first variant will put
%   \meta{after} after any other tokens which might have been sneaked before,
%   while |\ekvsneakPre| will put \meta{after} before other smuggled stuff.
%   This reads and reinserts the remainder of the current |\ekvset| macro and
%   its argument list to do its job. A small usage example is shown in
%   \autoref{sec:sneakex}.
% \end{function}
%
% \begin{function}{\ekvchangeset}
%   \begin{syntax}
%     \cs{ekvchangeset}\marg{new-set}
%   \end{syntax}
%   Replaces the current set with \meta{new-set}, so for the rest of the current
%   |\ekvset| call, that call behaves as if it was called with
%   \texttt{\cs[no-index]{ekvset}\marg{new-set}}. Just like |\ekvsneak| this
%   reads and reinserts the remainder of the current |\ekvset| macro to do its
%   job. It is comparable to using \texttt{\meta{key}/.cd} in \pkg{pgfkeys}.
% \end{function}
%
% \bigskip
%
% \begin{function}{\ekv@name,\ekv@name@set,\ekv@name@key}
%   \begin{syntax}
%     \cs{ekv@name}\marg{set}\marg{key}\\
%     \cs{ekv@name@set}\marg{set}\\
%     \cs{ekv@name@key}\marg{key}
%   \end{syntax}
%   The names of the macros that correspond to a key in a set are build with
%   these macros. The default definition of |\ekv@name@set| is
%   ``\texttt{\csname ekv@name@set\endcsname{\set}}'' and the default of
%   |\ekv@name@key| is ``\texttt{\csname ekv@name@key\endcsname{\key}}''.
%   The complete name is build using |\ekv@name| which is equivalent to
%   \texttt{\cs[no-index]{ekv@name@set}\marg{set}\cs{ekv@name@key}\marg{key}}.
%   For |NoVal| keys an additional |N| gets appended irrespective of these
%   macros' definition, so their name is
%   \cs[no-index]{\csname ekv@name\endcsname{\set}{\key}N}. You might redefine
%   |\ekv@name@set| and |\ekv@name@key| locally but \emph{don't redefine}
%   |\ekv@name|!
% \end{function}
%
% \subsubsection{Bugs}
%
% Just like \pkg{keyval}, \expkv\ is bug free. But if you find \pmso{bugs}hidden
% features\footnote{Thanks, David!} you can tell me about them either via mail
% (see the first page) or directly on GitHub if you have an account there:
% \url{https://github.com/Skillmon/tex_expkv}
%
% \subsubsection{Comparisons}
%
% Comparisons of speed are done with a very simple test key and the help of the
% \pkg{l3benchmark} package. The key and its usage should be equivalent to
% \begin{lstlisting}
% \protected\ekvdef{test}{height}{\def\myheight{#1}}
% \ekvset{test}{ height = 6 }
% \end{lstlisting}
% and only the usage of the key, not its definition, is benchmarked. For the
% impatient, the essence of these comparisons regarding speed and buggy
% behaviour is contained in \autoref{tab:comp}.
%
% As far as I know \expkv\ is the only fully expandable \kv\ parser. I tried to
% compare \expkv\ to every \kv\ package listed on
% \href{https://ctan.org/topic/keyval}{CTAN}, however, one might notice that
% some of those are missing from this list. That's because I didn't get the
% others to work due to bugs, or because they just provide wrappers around other
% packages in this list.
%
% In this subsubsection is no benchmark of |\ekvparse| and |\keyval_parse:NNn|
% contained, as most other packages don't provide equivalent features to my
% knowledge. |\ekvparse| is slightly faster than |\ekvset|, but keep in mind
% that it does less. The same is true for |\keyval_parse:NNn| compared to
% |\keys_set:nn| of \pkg{expl3} (where the difference is much bigger).
%
% \paragraph{\pkg{keyval}} is about two times faster and has a comparable
% feature set just a slightly different way how it handles keys without values.
% That might be considered a drawback, as it limits the versatility, but also as
% an advantage, as it might reduce doubled code. Keep in mind that as soon as
% someone loads \pkg{xkeyval} the performance of \pkg{keyval} gets replaced by
% \pkg{xkeyval}'s.
%
% Also \pkg{keyval} has a bug, which unfortunately can't really be resolved
% without breaking backwards compatibility for \emph{many} documents, namely it
% strips braces from the argument before stripping spaces if the argument isn't
% surrounded by spaces, also it might strip more than one set of braces. Hence
% all of the following are equivalent in their outcome, though the last two
% lines should result in something different than the first two:
% \begin{lstlisting}[belowskip=0pt]
% \setkeys{foo}{bar=baz}
% \setkeys{foo}{bar= {baz}}
% \setkeys{foo}{bar={ baz}}
% \setkeys{foo}{bar={{baz}}}
% \end{lstlisting}
%
% \paragraph{\pkg{xkeyval}} is roughly fourteen times slower, but it provides
% more functionality, e.g., it has choice keys, boolean keys, and so on. It
% contains the same bug as \pkg{keyval} as it has to be compatible with it by
% design (it replaces \pkg{keyval}'s frontend), but also adds even more cases in
% which braces are stripped that shouldn't be stripped, worsening the situation.
%
% \paragraph{\pkg{ltxkeys}} is over 300 times slower -- which is funny, because
% it aims to be ``[\ldots] faster than these earlier packages [referring to
% \pkg{keyval} and \pkg{xkeyval}].'' Since it aims to have a bigger feature set
% than \pkg{xkeyval}, it most definitely also has a bigger feature set than
% \expkv. Also, it can't parse |\long| input, so as soon as your values contain
% a |\par|, it'll throw errors. Furthermore, \pkg{ltxkeys} doesn't strip outer
% braces at all by design, which, imho, is a weird design choice. In addition
% \pkg{ltxkeys} loads \pkg{catoptions} which is known to introduce bugs (e.g.,
% see \url{https://tex.stackexchange.com/questions/461783}).
%
% \paragraph{\pkg{l3keys}} is almost five times slower, but has an, imho, great
% interface to define keys. It strips \emph{all} outer spaces, even if somehow
% multiple spaces ended up on either end. It offers more features, but is pretty
% much bound to \pkg{expl3} code. Whether that's a drawback is up to you.
%
% \paragraph{\pkg{pgfkeys}} is a bit more than two times slower for one key,
% but has an \emph{enormous} feature set. However, since adding additional keys
% doesn't add as much needed time for \pkg{pgfkeys} compared to \expkv, it gets
% faster than \expkv\ at around eight \kv\ pairs.
% It has the same or a very similar bug \pkg{keyval} has. The brace bug (and
% also the category fragility) can be fixed by \pkg{pgfkeyx}, but this package
% was last updated in 2012 and it slows down |\pgfkeys| by factor~$8$. Also I
% don't know whether this might introduce new bugs.
%
% \paragraph{\pkg{kvsetkeys} with \pkg{kvdefinekeys}} is about three times
% slower, but it works even if commas and equals have category codes different
% from 12 (just as some other packages in this list). Else the features of the
% keys are equal to those of \pkg{keyval}, the parser has more features, though.
%
% \paragraph{\pkg{options}} is a bit slower for only a single value, but gets a
% tad faster than \expkv\ at around 10 \kv\ pairs. It has a much bigger feature
% set. Unfortunately it also suffers from the premature unbracing bug
% \pkg{keyval} has.
%
% \paragraph{\pkg{simplekv}} is hard to compare because I don't speak French (so
% I don't understand the documentation) and from what I can see, there is no
% direct way to define the equivalent test key. Nevertheless, I tested the
% closest possible equivalent of my test key while siding for \pkg{simplekv}'s
% design not forcing something into it it doesn't seem to be designed for. It is
% more than five times slower and has hard to predict behaviour regarding brace
% and space stripping, similar to \pkg{keyval}. The tested definition was:
% \begin{lstlisting}[belowskip=0pt]
% \usepackage{simplekv}
% \setKVdefault[simplekv]{height={ abc}}                     % key setup
% \setKV[simplekv]{ height = 6 }                             % benchmarked
% \end{lstlisting}
%
% \paragraph{\pkg{yax}} is over eighteen times slower. It has a pretty strange
% syntax, imho, and again a direct equivalent is hard to define. It has the
% premature unbracing bug, too. Also somehow loading \pkg{yax} broke
% \pkg{options} for me. The tested definition was:
% \begin{lstlisting}[belowskip=0pt]
% \usepackage{yax}
% \defactiveparameter yax {\storevalue\myheight yax:height } % key setup
% \setparameterlist{yax}{ height = 6 }                       % benchmarked
% \end{lstlisting}
%
% \begin{table}
%   \def\fnsym{\textcolor{red!80!black}{*}}%
%   \sisetup{round-precision=1, round-mode=places}%
%   \begingroup
%   \centering
%   \newcommand*\yes{\textcolor{red!80!black}  {yes}}^^A
%   \newcommand*\no {\textcolor{green!80!black}{no}}^^A
%   \caption[Comparison of \kv\ packages]
%     {^^A
%       Comparison of \kv\ packages. The packages are ordered from
%       fastest to slowest for one \kv\ pair. Benchmarking was done using
%       \pkg{l3benchmark} and the scripts in the \file{Benchmarks} folder
%       of the \href{https://github.com/Skillmon/tex_expkv}{git repository}.
%       The columns $p_i$ are the polynomial coefficients of a linear fit to the
%       run-time, $p_0$ can be interpreted as the overhead for initialisation
%       and $p_1$ the cost per key. The $T_0$ column is the actual mean ops
%       needed for an empty list argument, as the linear fit doesn't match that
%       point well in general. The column ``BB'' lists whether the
%       parsing is affected by some sort of brace bug, ``CF'' stands for
%       category code fragile and lists whether the parsing breaks with active
%       commas or equal signs.^^A
%       \label{tab:comp}^^A
%     }
%   \begin{tabular}
%       {>{\collectcell\pkg}l<{\endcollectcell}*3{S[table-format=4.1]}ccc}
%     \toprule
%     \rmfamily Package & {$p_1$} & {$p_0$} & {$T_0$}& BB & CF & Date \\
%     \midrule
%     keyval    &   13.539 &    1.844 &    6.969 & \yes & \yes & 2014-10-28 \\
%     \expkv    &   26.411 &    2.391 &   12.610 & \no  & \no  & 2020-02-22 \\
%     options   &   25.532 &   10.518 &   21.393 & \yes & \yes & 2015-03-01 \\
%     pgfkeys   &   26.444 &   40.939 &   55.790 & \yes & \yes & 2020-01-08 \\
%     kvsetkeys & {\fnsym} & {\fnsym} &   41.620 & \no  & \no  & 2019-12-15 \\
%     l3keys    &  114.456 &   35.357 &   52.667 & \no  & \no  & 2020-02-14 \\
%     simplekv  &  160.523 &   10.809 &    8.698 & \yes & \yes & 2017-08-08 \\
%     xkeyval   &  260.872 &  180.435 &  164.167 & \yes & \yes & 2014-12-03 \\
%     yax       &  507.643 &   62.863 &  123.533 & \yes & \yes & 2010-01-22 \\
%     ltxkeys   & 3932.759 & 4737.818 & 5883.000 & \no  & \no  & 2012-11-17 \\
%     \bottomrule
%   \end{tabular}
%   \par
%   \endgroup
%   \medskip
%   \fnsym For \pkg{kvsetkeys} the linear model used for the other
%   packages is a poor fit, \pkg{kvsetkeys} seems to have approximately
%   quadratic run-time, the coefficients of the second degree polynomial fit are
%   $p_2=\num{9.911}$, $p_1=\num{40.523}$, and $p_0=\num{61.500}$. Of course the
%   other packages might not really have linear run-time, but at least from 1~to
%   20~keys the fits don't seem too bad (the maximum ratio $p_2/p_1$
%   for the other packages is \num{9.697e-3}). If one extrapolates the fits
%   for 100 \kv\ pairs one finds that most of them match pretty well, the
%   exception being \pkg{ltxkeys}, which behaves quadratic as well with
%   $p_2=\num{31.685}$, $p_1=\num{3267.368}$, and $p_0=\num{7177.582}$.
% \end{table}
%
% \subsection{Examples}
%
% \subsubsection{Standard Use-Case}
%
% Say we have a macro for which we want to create a \kv\ interface. The macro
% has a parameter, which is stored in the dimension |\ourdim| having a default
% value from its initialization. Now we want to be able to change that dimension
% with the |width| key to some specified value. For that we'd do
% \begin{lstlisting}
% \newdimen\ourdim
% \ourdim=150pt
% \protected\ekvdef{our}{width}{\ourdim=#1\relax}
% \end{lstlisting}
% as you can see, we use the set |our| here. We want the key to behave different
% if no value is specified. In that case the key should not use its initial
% value, but be smart and determine the available space from
% \cs[no-index]{hsize}, so we also define
% \begin{lstlisting}
% \protected\ekvdefNoVal{our}{width}{\ourdim=.9\hsize}
% \end{lstlisting}
% Now we set up our macro to use this \kv\ interface
% \begin{lstlisting}
% \protected\def\ourmacro#1{\begingroup\ekvset{our}{#1}\the\ourdim\endgroup}
% \end{lstlisting}
% Finally we can use our macro like in the following\\
% \begin{minipage}{6cm}
% \begin{lstlisting}
% \ourmacro{}\par
% \ourmacro{width}\par
% \ourmacro{width=5pt}\par
% \end{lstlisting}
% \end{minipage}^^A
% \begin{minipage}{\dimexpr\linewidth-6cm\relax}
%   \newdimen\ourdim\ourdim=150pt
%   \protected\ekvdef{our}{width}{\ourdim=#1\relax}^^A
%   \protected\ekvdefNoVal{our}{width}{\ourdim=.9\hsize}^^A
%   \protected\def\ourmacro#1^^A
%     {\begingroup\ekvset{our}{#1}\the\ourdim\endgroup}^^A
%   \ourmacro{}\par
%   \ourmacro{width}\par
%   \ourmacro{width=5pt}\par
% \end{minipage}
%
% \paragraph{The same key using \protect\expkvd}
% Using \expkvd\ we can set up the equivalent key using a \kv\ interface, after
% the following we could use |\ourmacro| in the same way as above. \expkvd\ will
% allocate and initialise |\ourdim| and define the |width| key |\protected| for
% us, so the result will be exactly the same -- with the exception that the
% default will use |\ourdim=.9\hsize\relax| instead.
% \begin{lstlisting}
% \input expkv-def                 % or \usepackage{expkv-def}
% \ekvdefinekeys{our}
%   {
%     dimen    width = \ourdim,
%     qdefault width = .9\hsize,
%     initial  width = 150pt
%   }
% \end{lstlisting}
%
% \subsubsection{An Expandable \kv\ Macro Using \cs[no-index]{ekvsneak}}
% \label{sec:sneakex}
%
% Let's set up an expandable macro, that uses a \kv\ interface. The problems
% we'll face for this are:
% \begin{enumerate}
%   \item ignoring duplicate keys
%   \item default values for keys which weren't used
%   \item providing the values as the correct argument to a macro (ordered)
% \end{enumerate}
%
% First we need to decide which \kv\ parsing macro we want to do this with,
% |\ekvset| or |\ekvparse|. For this example we also want to show the
% usage of |\ekvsneak|, hence we'll choose |\ekvset|. And we'll have to use
% |\ekvset| such that it builds a parsable list for our macro internals. To
% gain back control after |\ekvset| is done we have to put an internal of our
% macro at the start of that list, so we use an internal key that uses
% |\ekvsneakPre| after any user input.
%
% To ignore duplicates will be easy if the value of the key used last will be
% put first in the list, so the following will use |\ekvsneakPre| for the
% user-level keys. If we wanted some key for which the first usage should be the
% the binding one we would use |\ekvsneak| instead for that key.
%
% Providing default values can be done in different ways, we'll use a simple
% approach in which we'll just put the outcome of our keys if they were used
% with default values before the parsing list terminator.
%
% Ordering the keys can be done simply by searching for a specific token for
% each argument which acts like a flag, so our sneaked out values will include
% specific tokens acting as markers.
%
% Now that we have answers for our technical problems, we have to decide what
% our example macro should do. How about we define a macro that calculates the
% sine of a number and rounds that to a specified precision? As a small extra
% this macro should understand input in radian and degree and the used
% trigonometric function should be selectable as well. For the hard part of this
% task (expandably evaluating trigonometric functions) we'll use the \pkg{xfp}
% package.
%
% First we set up our keys according to our earlier considerations and set up
% the user facing macro |\sine|. The end marker of the parsing list will be a
% |\sine@stop| token, which we don't need to define and we put our defaults
% right before it.
%
% \begin{lstlisting}
% \RequirePackage{xfp}
% \makeatletter
% \ekvdef{expex}{f}{\ekvsneakPre{\f{#1}}}
% \ekvdef{expex}{round}{\ekvsneakPre{\rnd{#1}}}
% \ekvdefNoVal{expex}{degree}{\ekvsneakPre{\deg{d}}}
% \ekvdefNoVal{expex}{radian}{\ekvsneakPre{\deg{}}}
% \ekvdefNoVal{expex}{internal}{\ekvsneakPre{\sine@rnd}}
% \newcommand*\sine[2]
%   {\ekvset{expex}{#1,internal}\rnd{3}\deg{d}\f{sin}\sine@stop{#2}}
% \end{lstlisting}
% For the sake of simplicity we defined the macro |\sine| with two mandatory
% arguments, the first being the \kv\ list, the second the argument to the
% trigonometric function. We could've used \pkg{xparse}'s facilities here to
% define an expandable macro which takes an optional argument instead.
%
% Now we need to define some internal macros to extract the value of each key's
% last usage (remember that this will be the group after the first special
% flag-token). For that we use one delimited macro per key.
% \begin{lstlisting}
% \def\sine@rnd#1\rnd#2#3\sine@stop{\sine@deg#1#3\sine@stop{#2}}
% \def\sine@deg#1\deg#2#3\sine@stop{\sine@f#1#3\sine@stop{#2}}
% \def\sine@f#1\f#2#3\sine@stop{\sine@final{#2}}
% \end{lstlisting}
% After the macros |\sine@rnd|, |\sine@deg|, and |\sine@f| the macro
% |\sine@final| will see
% \texttt
%   {^^A
%     \cs[no-index]{sine@final}^^A
%     \marg{f}\marg{degree/radian}\marg{round}\marg{num}^^A
%   }.
% Now |\sine@final| has to expandably deal with those arguments such that the
% |\fpeval| macro of \pkg{xfp} gets the correct input. Luckily this is pretty
% straight forward in this example. In |\fpeval| the trigonometric functions
% have names such as |sin| or |cos| and the degree taking variants |sind| or
% |cosd|. And since the |degree| key puts a |d| in |#2| and the |radian| key
% leaves |#2| empty all we have to do to get the correct function name is stick
% the two together.
% \begin{lstlisting}
% \newcommand*\sine@final[4]{\fpeval{round(#1#2(#4),#3)}}
% \makeatother
% \end{lstlisting}
% Let's test our macro:\\
% \begin{minipage}{.7\linewidth}
%   \begin{lstlisting}[gobble=4]
%   \sine{}{60}\par
%   \sine{round=10}{60}\par
%   \sine{f=cos,radian}{pi}\par
%   \edef\myval{\sine{f=tan}{1}}\texttt{\meaning\myval}
%   \end{lstlisting}
% \end{minipage}\hfill
% \begin{minipage}{.2\linewidth}
%   \makeatletter
%   \ekvdef{expex}{f}{\ekvsneakPre{\f{#1}}}^^A
%   \ekvdef{expex}{round}{\ekvsneakPre{\rnd{#1}}}^^A
%   \ekvdefNoVal{expex}{degree}{\ekvsneakPre{\deg{d}}}^^A
%   \ekvdefNoVal{expex}{radian}{\ekvsneakPre{\deg{}}}^^A
%   \ekvdefNoVal{expex}{internal}{\ekvsneakPre{\sine@rnd}}^^A
%   \newcommand*\sine[2]
%     {\ekvset{expex}{#1,internal}\rnd{3}\deg{d}\f{sin}\sine@stop{#2}}^^A
%   \def\sine@rnd#1\rnd#2#3\sine@stop{\sine@deg#1#3\sine@stop{#2}}^^A
%   \def\sine@deg#1\deg#2#3\sine@stop{\sine@f#1#3\sine@stop{#2}}^^A
%   \def\sine@f#1\f#2#3\sine@stop{\sine@final{#2}}^^A
%   \newcommand*\sine@final[4]{\fpeval{round(#1#2(#4),#3)}}^^A
%   \makeatother
%   \sine{}{60}\par
%   \sine{round=10}{60}\par
%   \sine{f=cos,radian}{pi}\par
%   \edef\myval{\sine{f=tan}{1}}\texttt{\meaning\myval}
% \end{minipage}
%
%
% \subsection{Error Messages}
%
% \expkv\ should only send messages in case of errors, there are no warnings and
% no info messages. In this subsection those errors are listed.
%
% \subsubsection{Load Time}
%
% \file{expkv.tex} checks whether \eTeX\ is available. If it isn't, an error
% will be thrown using |\errmessage|:
% \begin{lstlisting}
% ! expkv Error: e-TeX required.
% \end{lstlisting}
%
% \subsubsection{Defining Keys}
%
% If you get any error from \expkv\ while you're trying to define a key, the
% definition will be aborted and gobbled.
%
% If you try to define a key with an empty set name you'll get:
% \begin{lstlisting}
% ! expkv Error: empty set name not allowed.
% \end{lstlisting}
%
% Similarly, if you try to define a key with an empty key name:
% \begin{lstlisting}
% ! expkv Error: empty key name not allowed.
% \end{lstlisting}
%
% Both of these messages are done in a way that doesn't throw additional errors
% due to |\global|, |\long|, etc., not being used correctly if you prefixed one
% of the defining macros.
%
% \subsubsection{Using Keys}
%
% This subsubsection contains the errors thrown during |\ekvset|. The errors are
% thrown in an expandable manner by providing an undefined macro. In the
% following messages \lstinline|<key>| gets replaced with the problematic key's
% name, and \lstinline|<set>| with the corresponding set. If any errors during
% \kv\ handling are encountered, the entry in the comma separated list will be
% omitted after the error is thrown and the next \kv\ pair will be parsed.
%
% If you're using an undefined key you'll get:
% \begin{lstlisting}
% ! Undefined control sequence.
% <argument> \! expkv Error:
%                            unknown key (`<key>', set `<set>').
% \end{lstlisting}
%
% If you're using a key for which only a normal version and no |NoVal| version
% is defined, but don't provide a value, you'll get:
% \begin{lstlisting}
% ! Undefined control sequence.
% <argument> \! expkv Error:
%                            value required (`<key>', set `<set>').
% \end{lstlisting}
%
% If you're using a key for which only a |NoVal| version and no normal version
% is defined, but provide a value, you'll get:
% \begin{lstlisting}
% ! Undefined control sequence.
% <argument> \! expkv Error:
%                            value forbidden (`<key>', set `<set>').
% \end{lstlisting}
%
% If you're using a set for which you never executed one of the defining macros
% from \autoref{sec:define} you'll get a low level \TeX\ error, as that isn't
% actively tested by the parser (and hence will lead to undefined behaviour and
% not be gracefully ignored). The error will look like
% \begin{lstlisting}
% ! Missing \endcsname inserted.
% <to be read again>
%                    \! expkv Error: Set `<set>' undefined.
% \end{lstlisting}
%
% \subsection{License}
%
% Copyright \textcopyright\ 2020\unless\ifnum\year=2020--\the\year\fi\
% Jonathan P. Spratte
%
% \medskip\noindent
% This work may be distributed and/or modified under the conditions of the
% \LaTeX\ Project Public License (LPPL), either version 1.3c of this license or
% (at your option) any later version. The latest version of this license is in
% the file:
%
% \url{http://www.latex-project.org/lppl.txt}
%
% \noindent
% This work is ``maintained'' (as per LPPL maintenance status) by
%
% \mbox{Jonathan P. Spratte}.
%
% \end{documentation}^^A=<<
%
% \begin{implementation}^^A>>=
%
% \clearpage
%
% \section{Implementation}^^A>>=
%^^A the LaTeX package >>=
% \subsection{The \LaTeX\ Package}
% First we set up the \LaTeX\ package. That one doesn't really do much except
% |\input|ting the generic code and identifying itself as a package.
% \gobbledocstriptag
%<*pkg>
%    \begin{macrocode}
\def\ekv@tmp
  {%
    \ProvidesFile{expkv.tex}%
      [\ekvDate\space v\ekvVersion\space an expandable key=val implementation]%
  }
\input{expkv.tex}
\ProvidesPackage{expkv}%
  [\ekvDate\space v\ekvVersion\space an expandable key=val implementation]
%    \end{macrocode}
% \gobbledocstriptag
%</pkg>
%^^A=<<
%^^A main file >>=
% \subsection{The Generic Code}
% The rest of this implementation will be the generic code.
% \gobbledocstriptag
%<*tex>
%
% Check whether \eTeX\ is available -- \expkv\ requires \eTeX.
%    \begin{macrocode}
\begingroup\expandafter\expandafter\expandafter\endgroup
\expandafter\ifx\csname numexpr\endcsname\relax
  \errmessage{expkv requires e-TeX}
  \expandafter\endinput
\fi
%    \end{macrocode}
%
% We make sure that it's only input once:
%    \begin{macrocode}
\expandafter\ifx\csname ekvVersion\endcsname\relax
\else
  \expandafter\endinput
\fi
%    \end{macrocode}
% \begin{macro}{\ekvVersion,\ekvDate}
% We're on our first input, so lets store the version and date in a macro.
%    \begin{macrocode}
\def\ekvVersion{0.5a}
\def\ekvDate{2020-02-27}
%    \end{macrocode}
% \end{macro}
%
% If the \LaTeX\ format is loaded we want to be a good file and report back who
% we are, for this the package will have defined |\ekv@tmp| to use
% |\ProvidesFile|, else this will expand to a |\relax| and do no harm.
%    \begin{macrocode}
\csname ekv@tmp\endcsname
%    \end{macrocode}
%
% Store the category code of |@| to later be able to reset it and change it to
% 11 for now.
%    \begin{macrocode}
\expandafter\chardef\csname ekv@tmp\endcsname=\catcode`\@
\catcode`\@=11
%    \end{macrocode}
% |\ekv@tmp| might later be reused to gobble any prefixes which might be
% provided to |\ekvdef| and similar in case the names are invalid, we just
% temporarily use it here as means to store the current category code of |@| to
% restore it at the end of the file, we never care for the actual definition of
% it.
%
% \begin{macro}[internal]
%   {
%     \@gobble,\@firstofone,\@firstoftwo,\@secondoftwo,
%     \ekv@gobbleto@stop,\ekv@fi@secondoftwo,\ekv@gobble@mark
%   }
% Since branching tests are often more versatile than |\if...\else...\fi|
% constructs, we define helpers that are branching pretty fast. Also here are
% some other utility functions that just grab some tokens. The ones that are
% also contained in \LaTeX\ don't use the |ekv| prefix.
%    \begin{macrocode}
\long\def\@gobble#1{}
\long\def\@firstofone#1{#1}
\long\def\@firstoftwo#1#2{#1}
\long\def\@secondoftwo#1#2{#2}
\long\def\ekv@fi@secondoftwo\fi\@firstoftwo#1#2{\fi#2}
\long\def\ekv@gobbleto@stop#1\ekv@stop{}
\def\ekv@gobble@mark\ekv@mark{}
%    \end{macrocode}
% \end{macro}
% As you can see |\ekv@gobbleto@stop| uses a special marker |\ekv@stop|. The
% package will use three such markers, the one you've seen already, |\ekv@mark|
% and |\ekv@nil|. Contrarily to how for instance \pkg{expl3} does things, we
% don't define them, as we don't need them to have an actual meaning. This has
% the advantage that if they somehow get expanded -- which should never happen
% if things work out -- they'll throw an error directly.
%
% \begin{macro}[internal]
%   {
%     \ekv@ifempty,\ekv@ifempty@,\ekv@ifempty@true,\ekv@ifempty@false,
%     \ekv@ifempty@true@F
%   }
% We can test for a lot of things building on an if-empty test, so lets define a
% really fast one. Since some tests might have reversed logic (true if something
% is not empty) we also set up macros for the reversed branches.
%    \begin{macrocode}
\long\def\ekv@ifempty#1%
  {%
    \ekv@ifempty@\ekv@ifempty@A#1\ekv@ifempty@B\ekv@ifempty@true
      \ekv@ifempty@A\ekv@ifempty@B\@secondoftwo
  }
\long\def\ekv@ifempty@#1\ekv@ifempty@A\ekv@ifempty@B{}
\long\def\ekv@ifempty@true\ekv@ifempty@A\ekv@ifempty@B\@secondoftwo#1#2{#1}
\long\def\ekv@ifempty@false\ekv@ifempty@A\ekv@ifempty@B\@firstoftwo#1#2{#2}
\long\def\ekv@ifempty@true@F\ekv@ifempty@A\ekv@ifempty@B\@firstofone#1{}
%    \end{macrocode}
% \end{macro}
%
% \begin{macro}[internal]{\ekv@ifblank@}
% The obvious test that can be based on an if-empty is if-blank, meaning a test
% checking whether the argument is empty or consists only of spaces. Our version
% here will be tweaked a bit, as we want to check this, but with one leading
% |\ekv@mark| token that is to be ignored.
%    \begin{macrocode}
\long\def\ekv@ifblank@\ekv@mark#1{\ekv@ifempty@\ekv@ifempty@A}
%    \end{macrocode}
% \end{macro}
%
% \begin{macro}[internal]{\ekv@ifdefined}
% We'll need to check whether something is defined quite frequently, so why not
% define a macro that does this. The following test is expandable, slower than
% the typical expandable test for undefined control sequences, but faster for
% defined ones. Since we want to be as fast as possible for correct input, this
% is to be preferred.
%    \begin{macrocode}
\def\ekv@ifdefined#1%
  {%
    \expandafter
    \ifx\csname\ifcsname #1\endcsname #1\else relax\fi\endcsname\relax
      \ekv@fi@secondoftwo
    \fi
    \@firstoftwo
  }
%    \end{macrocode}
% \end{macro}
  % \begin{macro}[internal]
%   {
%     \ekv@ifdefined@pair,\ekv@ifdefined@pair@,\ekv@ifdefined@key,
%     \ekv@ifdefined@key@
%   }
% Since we can save some time if we only have to create the control sequence
% once when we know beforehand how we want to use it, we build some other macros
% for those cases (which we'll have quite often, once per key usage).
%    \begin{macrocode}
\def\ekv@ifdefined@pair#1#2%
  {%
    \expandafter\ekv@ifdefined@pair@
      \csname
        \ifcsname #1{#2}\endcsname
          #1{#2}%
        \else
          relax%
        \fi
      \endcsname
  }
\def\ekv@ifdefined@pair@#1%
  {%
    \ifx#1\relax
      \ekv@fi@secondoftwo
    \fi
    \@firstoftwo
    {\ekv@set@pair@#1\ekv@mark}%
  }
\def\ekv@ifdefined@key#1#2%
  {%
    \expandafter\ekv@ifdefined@key@
      \csname
        \ifcsname #1{#2}N\endcsname
          #1{#2}N%
        \else
          relax%
        \fi
      \endcsname
  }
\def\ekv@ifdefined@key@#1%
  {%
    \ifx#1\relax
      \ekv@fi@secondoftwo
    \fi
    \@firstoftwo#1%
  }
%    \end{macrocode}
% \end{macro}
%
% \begin{macro}{\ekv@name,\ekv@name@set,\ekv@name@key}
% The keys will all follow the same naming scheme, so we define it here.
%    \begin{macrocode}
\def\ekv@name#1#2{\ekv@name@set{#1}\ekv@name@key{#2}}
\def\ekv@name@set#1{ekv#1(}
\def\ekv@name@key#1{#1)}
%    \end{macrocode}
% \end{macro}
% \begin{macro}[internal]{\ekv@undefined@set}
% We can misuse the macro name we use to expandably store the set-name in a
% single token -- since this increases performance drastically, especially for
% long set-names -- to throw a more meaningful error message in case a set isn't
% defined. The name of |\ekv@undefined@set| is a little bit misleading, as it is
% called in either case inside of |\csname|, but the result will be a control
% sequence with meaning |\relax| if the set is undefined, hence will break the
% |\csname| building the key-macro which will throw the error message.
%    \begin{macrocode}
\def\ekv@undefined@set#1{! expkv Error: Set `#1' undefined.}
%    \end{macrocode}
% \end{macro}
% \begin{macro}[internal]{\ekv@checkvalid}
% We place some restrictions on the allowed names, though, namely sets and
% keys are not allowed to be empty -- blanks are fine (meaning \mbox{set-
% or} key-names consisting of spaces).
%    \begin{macrocode}
\protected\def\ekv@checkvalid#1#2%
  {%
    \ekv@ifempty{#1}%
      {%
        \def\ekv@tmp{}%
        \errmessage{expkv Error: empty set name not allowed}%
      }%
      {%
        \ekv@ifempty{#2}%
          {%
            \def\ekv@tmp{}%
            \errmessage{expkv Error: empty key name not allowed}%
          }%
          \@secondoftwo
      }%
    \@gobble
  }
%    \end{macrocode}
% \end{macro}
% \begin{macro}{\ekvifdefined,\ekvifdefinedNoVal}
% And provide user-level macros to test whether a key is defined.
%    \begin{macrocode}
\def\ekvifdefined#1#2{\ekv@ifdefined{\ekv@name{#1}{#2}}}
\def\ekvifdefinedNoVal#1#2{\ekv@ifdefined{\ekv@name{#1}{#2}N}}
%    \end{macrocode}
% \end{macro}
%
% \begin{macro}
%   {\ekvdef,\ekvdefNoVal,\ekvlet,\ekvletNoVal,\ekvletkv,\ekvletkvNoVal}
% Set up the key defining macros |\ekvdef| etc.
%    \begin{macrocode}
\protected\long\def\ekvdef#1#2#3%
  {%
    \ekv@checkvalid{#1}{#2}%
      {%
        \expandafter\def\csname\ekv@name{#1}{#2}\endcsname##1{#3}%
        \ekv@defset{#1}%
      }%
  }
\protected\long\def\ekvdefNoVal#1#2#3%
  {%
    \ekv@checkvalid{#1}{#2}%
      {%
        \expandafter\def\csname\ekv@name{#1}{#2}N\endcsname{#3}%
        \ekv@defset{#1}%
      }%
  }
\protected\def\ekvlet#1#2#3%
  {%
    \ekv@checkvalid{#1}{#2}%
      {%
        \expandafter\let\csname\ekv@name{#1}{#2}\endcsname#3%
        \ekv@defset{#1}%
      }%
  }
\protected\def\ekvletNoVal#1#2#3%
  {%
    \ekv@checkvalid{#1}{#2}%
      {%
        \expandafter\let\csname\ekv@name{#1}{#2}N\endcsname#3%
        \ekv@defset{#1}%
      }%
  }
\protected\def\ekvletkv#1#2#3#4%
  {%
    \ekv@checkvalid{#1}{#2}%
      {%
        \expandafter\let\csname\ekv@name{#1}{#2}\expandafter\endcsname
        \csname\ekv@name{#3}{#4}\endcsname
        \ekv@defset{#1}%
      }%
  }
\protected\def\ekvletkvNoVal#1#2#3#4%
  {%
    \ekv@checkvalid{#1}{#2}%
      {%
        \expandafter\let\csname\ekv@name{#1}{#2}N\expandafter\endcsname
        \csname\ekv@name{#3}{#4}N\endcsname
        \ekv@defset{#1}%
      }%
  }
%    \end{macrocode}
% \end{macro}
% \begin{macro}[internal]{\ekv@defset}
%   In order to enhance the speed the set name given to |\ekvset| will be turned
%   into a control sequence pretty early, so we have to define that control
%   sequence.
%    \begin{macrocode}
\protected\def\ekv@defset#1%
  {%
    \expandafter\edef\csname\ekv@undefined@set{#1}\endcsname##1%
      {\ekv@name@set{#1}\ekv@name@key{##1}}%
  }
%    \end{macrocode}
% \end{macro}
%
% \begin{macro}{\ekvset}
% Set up |\ekvset|, which should not be affected by active commas and equal
% signs. The equal signs are a bit harder to cope with and we'll do that later,
% but replacing the active commas with commas of category other can be done
% beforehand. That's why we define |\ekvset| here with a temporary meaning just
% to set up the things with two different category codes. |#1| will be a
% \texttt{,\textsubscript{13}} and |#2| will be a \texttt{=\textsubscript{13}}.
%    \begin{macrocode}
\def\ekvset#1#2{%
\endgroup
\long\def\ekvset##1##2%
  {%
    \expandafter\ekv@set\csname\ekv@undefined@set{##1}\endcsname
      \ekv@mark##2#1\ekv@stop#1{}%
  }
%    \end{macrocode}
% \end{macro}
% \begin{macro}[internal]{\ekv@set}
% |\ekv@set| will split the \kv\ list at active commas. Then it has to check
% whether there were unprotected other commas and resplit there.
%    \begin{macrocode}
\long\def\ekv@set##1##2#1%
  {%
%    \end{macrocode}
% Test whether we're at the end, if so invoke |\ekv@endset|,
%    \begin{macrocode}
    \ekv@gobbleto@markstop##2\ekv@endset\ekv@mark\ekv@stop
%    \end{macrocode}
% else go on with other commas,
%    \begin{macrocode}
    \ekv@set@other##1##2,\ekv@stop,%
%    \end{macrocode}
% and get the next active comma delimited \kv\ pair.
%    \begin{macrocode}
    \ekv@set##1\ekv@mark
  }
%    \end{macrocode}
% \end{macro}
% \begin{macro}[internal]{\ekv@endset}
% |\ekv@endset| is a hungry little macro. It will eat everything that remains
% of |\ekv@set| and unbrace the sneaked stuff.
%    \begin{macrocode}
\long\def\ekv@endset
    \ekv@mark\ekv@stop\ekv@set@other##1,\ekv@stop,\ekv@set##2\ekv@mark
    ##3%
  {##3}
%    \end{macrocode}
% \end{macro}
% \begin{macro}[internal]{\ekv@set@other}
% The macro |\ekv@set@other| is guaranteed to get only single \kv\ pairs.
% First we test whether we're done, if not split at equal signs. It is faster to
% first split at category 12 equal signs and only after that on actives.
% If there is no equal sign, we need to test whether we got a blank argument and
% if not this is a |NoVal| key.
%    \begin{macrocode}
\long\def\ekv@set@other##1##2,%
  {%
    \ekv@gobbleto@markstop##2\ekv@endset@other\ekv@mark\ekv@stop
    \ekv@ifhas@eq@other##2=\ekv@ifempty@B\ekv@ifempty@false
      \ekv@ifempty@A\ekv@ifempty@B\@firstoftwo
      {\ekv@set@eq@other##1##2\ekv@stop}%
      {%
        \ekv@ifhas@eq@active##2#2\ekv@ifempty@B\ekv@ifempty@false
          \ekv@ifempty@A\ekv@ifempty@B\@firstoftwo
          {\ekv@set@eq@active##1##2\ekv@stop}%
          {%
            \ekv@ifblank@##2\ekv@nil\ekv@ifempty@B\ekv@ifempty@true@F
              \ekv@ifempty@A\ekv@ifempty@B\@firstofone
              {\ekv@strip{##2}\ekv@set@key##1}%
          }%
      }%
    \ekv@set@other##1\ekv@mark%
  }
%    \end{macrocode}
% \end{macro}
% \begin{macro}[internal]{\ekv@set@eq@other}
% |\ekv@set@eq@other| might not be the correct break point, there might be an
% active equal sign in the currently parsed key-name. If so, we have to resplit.
% If the split is correct strip the key-name of outer spaces and braces and feed
% it to |\ekv@set@pair|.
%    \begin{macrocode}
\long\def\ekv@set@eq@other##1##2=%
  {%
    \ekv@ifhas@eq@active##2#2\ekv@ifempty@B\ekv@ifempty@false
      \ekv@ifempty@A\ekv@ifempty@B\@firstoftwo
      {\ekv@set@eq@active##1##2=}%
      {\ekv@strip{##2}\ekv@set@pair##1}%
  }
%    \end{macrocode}
% \end{macro}
% \begin{macro}[internal]{\ekv@set@eq@active}
% We need to handle the active equal signs.
%    \begin{macrocode}
\long\def\ekv@set@eq@active##1##2#2%
  {%
    \ekv@strip{##2}\ekv@set@pair##1%
  }
%    \end{macrocode}
% \end{macro}
% \begin{macro}[internal]
%   {\ekv@ifhas@eq@other,\ekv@ifhas@eq@active,\ekv@endset@other}
% And we have to set up the testing macros for our equal signs and
% |\ekv@endset@other|.
%    \begin{macrocode}
\long\def\ekv@ifhas@eq@other\ekv@mark##1={\ekv@ifempty@\ekv@ifempty@A}
\long\def\ekv@ifhas@eq@active\ekv@mark##1#2{\ekv@ifempty@\ekv@ifempty@A}
\long\def\ekv@endset@other
    \ekv@mark\ekv@stop
    \ekv@ifhas@eq@other##1=\ekv@ifempty@B\ekv@ifempty@false
    \ekv@ifempty@A\ekv@ifempty@B\@firstoftwo
    ##2%
    \ekv@set@other##3\ekv@mark
  {}
%    \end{macrocode}
% \end{macro}
% \begin{macro}{\ekvbreak,\ekvbreakPreSneak,\ekvbreakPostSneak}
% Provide macros that can completely stop the parsing of |\ekvset|, who knows
% what it'll be useful for.
%    \begin{macrocode}
\long\def\ekvbreak##1##2\ekv@stop#1##3{##1}
\long\def\ekvbreakPreSneak ##1##2\ekv@stop#1##3{##1##3}
\long\def\ekvbreakPostSneak##1##2\ekv@stop#1##3{##3##1}
%    \end{macrocode}
% \end{macro}
% \begin{macro}{\ekvsneak,\ekvsneakPre}
% One last thing we want to do for |\ekvset| is to provide macros that just
% smuggle stuff after |\ekvset|'s effects.
%    \begin{macrocode}
\long\def\ekvsneak##1##2\ekv@stop#1##3%
  {%
    ##2\ekv@stop#1{##3##1}%
  }
\long\def\ekvsneakPre##1##2\ekv@stop#1##3%
  {%
    ##2\ekv@stop#1{##1##3}%
  }
%    \end{macrocode}
% \end{macro}
%
% \begin{macro}{\ekvparse}
% Additionally to the |\ekvset| macro we also want to provide an |\ekvparse|
% macro, that has the same scope as |\keyval_parse:NNn| from \pkg{expl3}.
% This is pretty analogue to the |\ekvset| implementation, we just put an
% |\unexpanded| here and there instead of other macros to stop the |\expanded|
% on our output.
%    \begin{macrocode}
\long\def\ekvparse##1##2##3%
  {%
    \ekv@parse##1##2\ekv@mark##3#1\ekv@stop#1%
  }
%    \end{macrocode}
% \end{macro}
% \begin{macro}[internal]{\ekv@parse}
%    \begin{macrocode}
\long\def\ekv@parse##1##2##3#1%
  {%
    \ekv@gobbleto@markstop##3\ekv@endparse\ekv@mark\ekv@stop
    \ekv@parse@other##1##2##3,\ekv@stop,%
    \ekv@parse##1##2\ekv@mark
  }
%    \end{macrocode}
% \end{macro}
% \begin{macro}[internal]{\ekv@endparse}
%    \begin{macrocode}
\long\def\ekv@endparse
    \ekv@mark\ekv@stop\ekv@parse@other##1,\ekv@stop,\ekv@parse##2\ekv@mark
  {}
%    \end{macrocode}
% \end{macro}
% \begin{macro}[internal]{\ekv@parse@other}
%    \begin{macrocode}
\long\def\ekv@parse@other##1##2##3,%
  {%
    \ekv@gobbleto@markstop##3\ekv@endparse@other\ekv@mark\ekv@stop
    \ekv@ifhas@eq@other##3=\ekv@ifempty@B\ekv@ifempty@false
      \ekv@ifempty@A\ekv@ifempty@B\@firstoftwo
      {\ekv@parse@eq@other##3\ekv@stop##2}%
      {%
        \ekv@ifhas@eq@active##3#2\ekv@ifempty@B\ekv@ifempty@false
          \ekv@ifempty@A\ekv@ifempty@B\@firstoftwo
          {\ekv@parse@eq@active##3\ekv@stop##2}%
          {%
            \ekv@ifblank@##3\ekv@nil\ekv@ifempty@B\ekv@ifempty@true@F
              \ekv@ifempty@A\ekv@ifempty@B\@firstofone
              {\ekv@strip{##3}\ekv@parse@key##1}%
          }%
      }%
    \ekv@parse@other##1##2\ekv@mark
  }
%    \end{macrocode}
% \end{macro}
% \begin{macro}[internal]{\ekv@parse@eq@other}
%    \begin{macrocode}
\long\def\ekv@parse@eq@other##1=%
  {%
    \ekv@ifhas@eq@active##1#2\ekv@ifempty@B\ekv@ifempty@false
      \ekv@ifempty@A\ekv@ifempty@B\@firstoftwo
      {\ekv@parse@eq@active##1=}%
      {\ekv@strip{##1}\ekv@parse@pair\ekv@mark}%
  }
%    \end{macrocode}
% \end{macro}
% \begin{macro}[internal]{\ekv@parse@eq@active}
%    \begin{macrocode}
\long\def\ekv@parse@eq@active##1#2%
  {%
    \ekv@strip{##1}\ekv@parse@pair\ekv@mark
  }
%    \end{macrocode}
% \end{macro}
% \begin{macro}[internal]{\ekv@endparse@other}
%    \begin{macrocode}
\long\def\ekv@endparse@other
    \ekv@mark\ekv@stop
    \ekv@ifhas@eq@other##1=\ekv@ifempty@B\ekv@ifempty@false
    \ekv@ifempty@A\ekv@ifempty@B\@firstoftwo
    ##2%
    \ekv@parse@other##3\ekv@mark
  {}
%    \end{macrocode}
% \end{macro}
% \begin{macro}[internal]{\ekv@parse@pair,\ekv@parse@pair@}
%    \begin{macrocode}
\long\def\ekv@parse@pair##1##2\ekv@stop
  {%
    \ekv@strip{##2}\ekv@parse@pair@{##1}%
  }
\long\def\ekv@parse@pair@##1##2##3%
  {%
    \unexpanded{##3{##2}{##1}}%
  }
%    \end{macrocode}
% \end{macro}
% \begin{macro}[internal]{\ekv@parse@key}
%    \begin{macrocode}
\long\def\ekv@parse@key##1##2%
  {%
    \unexpanded{##2{##1}}%
  }
%    \end{macrocode}
% \end{macro}
%
% Finally really setting things up with |\ekvset|'s temporary meaning:
%    \begin{macrocode}
}
\begingroup
\catcode`\,=13
\catcode`\==13
\ekvset,=
%    \end{macrocode}
%
% \begin{macro}{\ekvchangeset}
% Provide a macro that is able to switch out the current \set\ in |\ekvset|.
% This operation is slow (by comparison, it should be slightly faster than
% |\ekvsneak|), but allows for something similar to \pkg{pgfkeys}'s
% \texttt{\meta{key}/.cd} mechanism. However this operation is more expensive
% than |/.cd| as we can't just redefine some token to reflect this, but have to
% switch out the set expandably, so this works similar to the |\ekvsneak| macros
% reading and reinserting the remainder of the \kv\ list.
%    \begin{macrocode}
\def\ekvchangeset#1%
  {%
    \expandafter\ekv@changeset\csname\ekv@undefined@set{#1}\endcsname\ekv@mark
  }
%    \end{macrocode}
% \end{macro}
% \begin{macro}[internal]{\ekv@changeset}
% This macro does the real change-out of |\ekvchangeset|. We introduced an
% |\ekv@mark| to not accidentally remove some braces which we have to remove
% again.
%    \begin{macrocode}
\long\def\ekv@changeset#1#2\ekv@set@other#3#4\ekv@set#5%
  {%
    \ekv@gobble@mark#2\ekv@set@other#1#4\ekv@set#1%
  }
%    \end{macrocode}
% \end{macro}
%
%
% \begin{macro}[internal]{\ekv@gobbleto@markstop}
% The |\ekv@gobbleto@markstop| can be used to test for |\ekv@stop| similar to
% our if-empty test, but instead of using tokens which are used nowhere else
% (|\ekv@ifempty@A| and |\ekv@ifempty@B|) it uses |\ekv@mark| and |\ekv@stop|.
%    \begin{macrocode}
\long\def\ekv@gobbleto@markstop#1\ekv@mark\ekv@stop{}
%    \end{macrocode}
% \end{macro}
%
% \begin{macro}[internal]{\ekv@set@pair,\ekv@set@pair@}
% |\ekv@set@pair| gets invoked with the space and brace stripped key-name as its
% first argument, the set-macro as the second argument, and following that is
% the key-value right delimited by an |\ekv@stop|.
%    \begin{macrocode}
\long\def\ekv@set@pair#1#2%
  {%
    \ekv@ifdefined@pair#2{#1}%
      {%
%    \end{macrocode}
% This branch will be executed if the key is not defined as an argument grabbing
% one. If so test whether there is a |NoVal| key of the same name or whether the
% key is unknown. Throw a meaningful error message and gobble the value.
%    \begin{macrocode}
        \ekv@ifdefined{#2{#1}N}%
          \ekv@err@noarg
          \ekv@err@unknown
          #2{#1}%
        \ekv@gobbleto@stop
      }%
  }
%    \end{macrocode}
% |\ekv@ifdefined@pair| will call |\ekv@set@pair@| if the key is correctly
% defined. This will then grab the value, strip outer spaces and braces from it
% and feed it to the key-macro. Afterwards |\ekv@set@other| will take control
% again.
%    \begin{macrocode}
\long\def\ekv@set@pair@#1#2\ekv@stop
  {%
    \ekv@strip{#2}#1%
  }
%    \end{macrocode}
% \end{macro}
%
% \begin{macro}[internal]{\ekv@set@key}
% Analogous to |\ekv@set@pair|, |\ekv@set@key| lets |\ekv@ifdefined@key| test
% whether a |NoVal| key is defined, else it'll throw a meaningful error message.
% Since we don't have to grab any value |\ekv@ifdefined@key| will invoke the
% key-macro and we're done here, |\ekv@set@other| will take over again.
%    \begin{macrocode}
\long\def\ekv@set@key#1#2%
  {%
    \ekv@ifdefined@key#2{#1}%
      {%
        \ekv@ifdefined{#2{#1}}%
          \ekv@err@reqval
          \ekv@err@unknown
          #2{#1}%
      }%
  }
%    \end{macrocode}
% \end{macro}
%
% \begin{macro}[internal]{\ekv@err,\ekv@err@}
% Since |\ekvset| is fully expandable as long as the code of the keys is (which
% is unlikely) we want to somehow throw expandable errors, in our case via
% undefined control sequences.
%    \begin{macrocode}
\begingroup
\edef\ekv@err
  {%
    \endgroup
    \unexpanded{\long\def\ekv@err}##1%
      {%
        \unexpanded{\expandafter\ekv@err@\@firstofone}%
        {\expandafter\noexpand\csname ! expkv Error:\endcsname ##1.}%
        \unexpanded{\ekv@stop}%
      }%
  }
\ekv@err
\def\ekv@err@{\expandafter\ekv@gobbleto@stop}
%    \end{macrocode}
% \end{macro}
% \begin{macro}[internal]
%   {
%     \ekv@err@common,\ekv@err@common@,
%     \ekv@err@unknown,\ekv@err@noarg,\ekv@err@reqval
%   }
% Now we can use |\ekv@err| to set up some error messages so that we can later
% use those instead of the full strings.
%    \begin{macrocode}
\long\def\ekv@err@common #1#2{\expandafter\ekv@err@common@\string#2{#1}}
\long\def\ekv@err@common@#1`#2' #3.#4#5{\ekv@err{#4 (`#5', set `#2')}}
\long\def\ekv@err@unknown#1#2{\ekv@err@common{unknown key}#1{#2}}
\long\def\ekv@err@noarg  #1#2{\ekv@err@common{value forbidden}#1{#2}}
\long\def\ekv@err@reqval #1#2{\ekv@err@common{value required}#1{#2}}
%    \end{macrocode}
% \end{macro}
%
% \begin{macro}[internal]{\ekv@strip,\ekv@strip@a,\ekv@strip@b,\ekv@strip@c}
% Finally we borrow some ideas of \pkg{expl3}'s \pkg{l3tl} to strip spaces
% from keys and values. This |\ekv@strip| also strips one level of outer braces
% \emph{after} stripping spaces, so an input of | {abc} | becomes |abc| after
% stripping. It should be used with |#1| prefixed by |\ekv@mark|. Also this
% implementation at most strips \emph{one} space from both sides.
%    \begin{macrocode}
\def\ekv@strip#1%
  {%
    \long\def\ekv@strip##1%
      {%
        \ekv@strip@a
          ##1%
          \ekv@nil
          \ekv@mark#1%
          #1\ekv@nil{}%
        \ekv@stop
      }%
    \long\def\ekv@strip@a##1\ekv@mark#1##2\ekv@nil##3%
      {%
        \ekv@strip@b##3##1##2\ekv@nil
      }%
    \long\def\ekv@strip@b##1#1\ekv@nil
      {%
        \ekv@strip@c##1\ekv@nil
      }%
    \long\def\ekv@strip@c\ekv@mark##1\ekv@nil##2\ekv@stop##3%
      {%
        ##3{##1}%
      }%
  }
\ekv@strip{ }
%    \end{macrocode}
% \end{macro}
%
% Now everything that's left is to reset the category code of |@|.
%    \begin{macrocode}
\catcode`\@=\ekv@tmp
%    \end{macrocode}
%
% \gobbledocstriptag
%</tex>
%^^A=<<
%
%^^A=<<
%
% \end{implementation}^^A=<<
%
% \clearpage
% \PrintIndex
%
\endinput
%
^^A vim: ft=tex fdm=marker fmr=>>=,=<<
}{}
\ProvidesFile{expkv.dtx}
  [\csname ekvDate\endcsname\ an expandable key=val implementation]
\PassOptionsToPackage{full}{textcomp}
\documentclass{l3doc}
\RequirePackage[oldstylenums,nott]{kpfonts}
\input{glyphtounicode}
\pdfgentounicode=1
\RequirePackage{xfp} ^^A required for an example
\RequirePackage{listings}
\RequirePackage{booktabs}
\RequirePackage{array}
\RequirePackage{collcell}
\RequirePackage{siunitx}
\RequirePackage{xcolor}
\RequirePackage{caption}
\RequirePackage{microtype}
\RequirePackage{accsupp}
\lstset
  {
    ,flexiblecolumns=false
    ,basewidth=.53em
    ,gobble=2
    ,basicstyle=\fontfamily{jkp}\itshape
    ,morekeywords=^^A
      {^^A
        \ekvdef,\ekvdefNoVal,\ekvlet,\ekvletNoVal,\ekvletkv,\ekvletkvNoVal,^^A
        \ekvset,\ekvparse,\ekvVersion,\ekvDate,\ekvifdefined,^^A
        \ekvifdefinedNoVal,\ekvbreak,\ekvbreakPreSneak,\ekvbreakPostSneak,^^A
        \ekvsneak,\ekvsneakPre,\ekvchangeset
      }
    ,morecomment=[l]\%
    ,commentstyle=\color[gray]{0.4}
    ,literate={\{}{{\CodeSymbol\{}}{1}
              {\}}{{\CodeSymbol\}}}{1}
    ^^A,literate=*{<key>}{\key}{4}{<set>}{\set}{4}
  }
\newcommand*\CodeSymbol[1]{\textbf{#1}}
\RequirePackage{randtext}
\let\metaORIG\meta
\protected\def\meta #1{\texttt{\metaORIG{#1}}}
\renewcommand*\thefootnote{\fnsymbol{footnote}}
\definecolor{expkvred}{HTML}{9F393D}
\colorlet{expkvgrey}{black!75}
\makeatletter
\newcommand*\expkv
  {^^A
    \texorpdfstring
      {^^A
        \mbox
          {^^A
            \BeginAccSupp{ActualText=expkv}^^A
            \href{https://github.com/Skillmon/tex_expkv}
              {^^A
                \rmfamily
                \bfseries
                {\color{expkvgrey}e\kern-.05em x\kern-.05em}^^A
                \lower.493ex
                  \hbox{{\color{expkvgrey}P}\kern-.1em{\color{expkvred}k}}^^A
                \kern-.18em{\color{expkvred}v}^^A
              }^^A
            \EndAccSupp{}^^A
          }^^A
      }
      {expkv}^^A
  }
\newcommand*\expkvd
  {^^A
    \texorpdfstring
      {^^A
        \mbox
          {^^A
            \BeginAccSupp{ActualText=expkv-def}^^A
            \rmfamily
            \bfseries
            {\color{expkvgrey}e\kern-.05em x\kern-.05em}^^A
            \lower.493ex
              \hbox{{\color{expkvgrey}P}\kern-.1em{\color{expkvred}k}}^^A
            \kern-.18em{\color{expkvred}v}^^A
            {\color{expkvgrey}^^A
              \kern.05em\rule[-.1ex]{.08em}{1.2ex}\kern.05em\textsc{def}^^A
            }^^A
            \EndAccSupp{}^^A
          }^^A
      }
      {expkv-def}^^A
  }
\newcommand\kv{\meta{key}=\meta{value}}
\newcommand\key{\meta{key}}
\newcommand\val{\meta{value}}
\newcommand\set{\meta{set}}
\hypersetup{linkcolor=red!80!black,urlcolor=purple!80!black}
\DoNotIndex{\def,\edef,\,,\=,\begingroup,\catcode,\chardef,\csname,\endcsname}
\DoNotIndex{\endgroup,\endinput,\errmessage,\expandafter,\input,\let,\long}
\DoNotIndex{\noexpand,\protected,\ProvidesFile,\ProvidesPackage,\relax,\space}
\DoNotIndex{\@,\unexpanded,\string,\expanded}
\DoNotIndex{\ifcsname}
\DoNotIndex{\ifx}
\DoNotIndex{\else}
\DoNotIndex{\fi}
\@gobble\fi ^^A ignoring \ifx and \ifcsname, but only one \fi
\@ifdefinable\gobbledocstriptag{\def\gobbledocstriptag#1>{}}
\newcommand*\pmso[1] ^^A poor man's strike out
  {^^A
    \leavevmode
    \begingroup
      \sbox0{#1}^^A
      \rlap{\vrule height .6ex depth -.5ex width \wd0\relax}^^A
      \usebox0\relax
    \endgroup
  }
\makeatother
\begin{document}
  \title
    {^^A
      \texorpdfstring
        {^^A
          \huge\expkv
          \\[\medskipamount]
          \Large an expandable \kv\ implementation^^A
        }{expkv - an expandable <key>=<value> implementation}^^A
    }
  \date{\ekvDate\space v\ekvVersion}
  \author{Jonathan P. Spratte\thanks{\protect\randomize{jspratte@yahoo.de}}}
  \DocInput{expkv.dtx}
\end{document}
%</driver>^^A=<<
% \fi
%
% \maketitle
% \renewcommand*\thefootnote{\arabic{footnote}}
%
% \begin{abstract}
% \noindent\parfillskip=0pt
% \expkv\ provides a small interface for \kv\ parsing. The parsing macro is
% fully expandable, the \meta{code} of your keys might be not. \expkv\ is pretty
% fast, but not the fastest available \kv\ solution (\pkg{keyval} for instance
% is thrice as fast, but not expandable and it might strip braces it shouldn't
% have stripped).
% \end{abstract}
%
% \tableofcontents
%
% \begin{documentation}^^A>>=
%
% \section{Documentation}
%
% \expkv\ provides an expandable \kv\ parser. The \kv\ pairs should be given as
% a comma separated list and the separator between a \key\ and the associated
% \val\ should be an equal sign. Both, the commas and the equal signs, might be
% of category 12 (other) or 13 (active). To support this is necessary as for
% example \pkg{babel} turns characters active for some languages, for instance
% the equal sign is turned active for Turkish.
%
% \expkv\ is usable as generic code or as a \LaTeX\ package. To use it, just
% use one of:
% \begin{lstlisting}
% \usepackage{expkv} % LaTeX
% \input expkv       % plainTeX
% \end{lstlisting}
% The \LaTeX\ package doesn't do more than \file{expkv.tex}, except calling
% |\ProvidesPackage| and setting things up such that \file{expkv.tex} will use
% |\ProvidesFile|.
%
% \subsection{Setting up Keys}\label{sec:define}
% \expkv\ provides a rather simple approach to setting up keys, similar to
% \pkg{keyval}. However there is an auxiliary package named
% \href{https://github.com/Skillmon/tex_expkv-def}{\expkvd} (the files use
% \file{expkv-def} instead because \string| isn't allowed in package names by
% CTAN) which provides a more sophisticated interface, similar to well
% established packages like \pkg{pgfkeys} or \pkg{l3keys}.
%
% Keys in \expkv\ (as in almost all other \kv\ implementations) belong to a
% \emph{set} such that different sets can contain keys of the same name. Unlike
% many other implementations \expkv\ doesn't provide means to set a default
% value, instead we have keys that take values and keys that don't (the latter
% are called |NoVal| keys by \expkv), but both can have the same name (on the
% user level).
%
% The following macros are available to define new keys. Those macros containing
% ``|def|'' in their name can be prefixed by anything allowed to prefix |\def|,
% prefixes allowed for |\let| can prefix those with ``|let|'' in their name,
% accordingly. Neither \set\ nor \key\ are allowed to be empty for new keys and
% must not contain a |\par| or tokens that expand to it -- they must be legal
% inside of |\csname ...\endcsname|.
%
% \begin{function}{\ekvdef}
%   \begin{syntax}
%     \cs{ekvdef}\marg{set}\marg{key}\marg{code}
%   \end{syntax}
%   Defines a \key\ taking a value in a \set\ to expand to \meta{code}. In
%   \meta{code} you can use |#1| to refer to the given value.
% \end{function}
%
% \begin{function}{\ekvdefNoVal}
%   \begin{syntax}
%     \cs{ekvdefNoVal}\marg{set}\marg{key}\marg{code}
%   \end{syntax}
%   Defines a no value taking \key\ in a \set\ to expand to \meta{code}.
% \end{function}
%
% \begin{function}{\ekvlet}
%   \begin{syntax}
%     \cs{ekvlet}\marg{set}\marg{key}\meta{cs}
%   \end{syntax}
%   Let the value taking \key\ in \set\ to \meta{cs}, there are no checks on
%   \meta{cs} enforced.
% \end{function}
%
% \begin{function}{\ekvletNoVal}
%   \begin{syntax}
%     \cs{ekvletNoVal}\marg{set}\marg{key}\meta{cs}
%   \end{syntax}
%   Let the no value taking \key\ in \set\ to \meta{cs}, it is not checked
%   whether \meta{cs} exists or that it takes no parameter.
% \end{function}
%
% \begin{function}{\ekvletkv}
%   \begin{syntax}
%     \cs{ekvletkv}\marg{set}\marg{key}\marg{set2}\marg{key2}
%   \end{syntax}
%   Let the \key\ in \set\ to \meta{key2} in \meta{set2}, it is not checked
%   whether that second key exists.
% \end{function}
%
% \begin{function}{\ekvletkvNoVal}
%   \begin{syntax}
%     \cs{ekvletkvNoVal}\marg{set}\marg{key}\marg{set2}\marg{key2}
%   \end{syntax}
%   Let the \key\ in \set\ to \meta{key2} in \meta{set2}, it is not checked
%   whether that second key exists.
% \end{function}
%
% \subsection{Parsing Keys}
%
% \begin{function}{\ekvset}
%   \begin{syntax}
%     \cs{ekvset}\marg{set}\{\kv,\ldots\}
%   \end{syntax}
%   Splits \kv\ pairs on commas. From both \key\ and \val\ up to one space is
%   stripped from both ends, if then only a braced group remains the braces are
%   stripped as well. So |\ekvset{foo}{bar=baz}| and
%   |\ekvset{foo}{ {bar}= {baz} }| will both do
%   \texttt{\cs[no-index]{\meta{foobarcode}}\{baz\}}, so you can hide commas,
%   equal signs and spaces at the ends of either \key\ or \val\ by putting
%   braces around them. If you omit the equal sign the code of the key created
%   with the |NoVal| variants described in \autoref{sec:define} will be
%   executed. If \kv\ contains more than a single unhidden equal sign, it will
%   be split at the first one and the others are considered part of the value.
%   |\ekvset| should be nestable.
% \end{function}
%
% \begin{function}{\ekvparse}
%   \begin{syntax}
%     \cs{ekvparse}\meta{cs1}\meta{cs2}\{\kv,\ldots\}
%   \end{syntax}
%   This macro parses the \kv\ pairs and provides those list elements which are
%   only keys as the argument to \meta{cs1}, and those which are a \kv\ pair to
%   \meta{cs2} as two arguments. It is fully expandable as well and returns the
%   parsed list in |\unexpanded|, which has no effect outside of an |\expanded|
%   or |\edef| context\footnotemark.
%   If you need control over the necessary steps of expansion you
%   can use |\expanded| around it.
%
%   |\ekvbreak|, |\ekvsneak|, and |\ekvchangeset| and their relatives don't work
%   in |\ekvparse|.  It is analogue to \pkg{expl3}'s |\keyval_parse:NNn|, but
%   not with the same parsing rules -- |\keyval_parse:NNn| throws an error on
%   multiple equal signs per \kv\ pair and on empty \key\ names in a \kv\ pair,
%   both of which |\ekvparse| doesn't deal with.
%
%   As a small example:
% \begin{lstlisting}
% \ekvparse\handlekey\handlekeyval{foo = bar, key, baz={zzz}}
% \end{lstlisting}
%   would expand to
% \begin{lstlisting}
% \handlekeyval{foo}{bar}\handlekey{key}\handlekeyval{baz}{zzz}
% \end{lstlisting}
%   and afterwards |\handlekey| and |\handlekeyval| would have to further handle
%   the \key. There are no macros like these two contained in \expkv, you have
%   to set them up yourself if you want to use |\ekvparse| (of course the names
%   might differ). If you need the results of |\ekvparse| as the argument for
%   another macro, you should use |\expanded| as only then the input stream will
%   contain the output above:
% \begin{lstlisting}
% \expandafter\handle\expanded{\ekvparse\k\kv{foo = bar, key, baz={zzz}}}
% \end{lstlisting}
%   would expand to
% \begin{lstlisting}
% \handle\kv{foo}{bar}\k{key}\kv{baz}{zzz}
% \end{lstlisting}
% \end{function}
% \footnotetext 
%   {^^A
%     This is a change in behaviour, previously (v0.3 and before) \cs{ekvparse}
%     would expand in exactly two steps. This isn't always necessary, but makes
%     the parsing considerably slower. If this is necessary for your application
%     you can put an \cs[no-index]{expanded} around it and will still be faster
%     since you need only a single \cs[no-index]{expandafter} this way.^^A
%   }.
%
% \subsection{Miscellaneous}
%
% \subsubsection{Other Macros}
%
% \expkv\ provides some other macros which might be of interest.
%
% \begin{function}{\ekvVersion,\ekvDate}
%   These two macros store the version and date of the package.
% \end{function}
% \begin{function}{\ekvifdefined,\ekvifdefinedNoVal}
%   \begin{syntax}
%     \cs{ekvifdefined}\marg{set}\marg{key}\marg{true}\marg{false}
%     \cs{ekvifdefinedNoVal}\marg{set}\marg{key}\marg{true}\marg{false}
%   \end{syntax}
%   These two macros test whether there is a \key\ in \set. It is false if
%   either a hash table entry doesn't exist for that key or its meaning is
%   |\relax|.
% \end{function}
%
% \begin{function}{\ekvbreak,\ekvbreakPreSneak,\ekvbreakPostSneak}
%   \begin{syntax}
%     \cs{ekvbreak}\marg{after}
%   \end{syntax}
%   Gobbles the remainder of the current |\ekvset| macro and its argument list
%   and reinserts \meta{after}. So this can be used to break out of |\ekvset|.
%   The first variant will also gobble anything that has been sneaked out using
%   |\ekvsneak| or |\ekvsneakPre|, while |\ekvbreakPreSneak| will put
%   \meta{after} before anything that has been smuggled and |\ekvbreakPostSneak|
%   will put \meta{after} after the stuff that has been sneaked out.
% \end{function}
%
% \begin{function}{\ekvsneak,\ekvsneakPre}
%   \begin{syntax}
%     \cs{ekvsneak}\marg{after}
%   \end{syntax}
%   Puts \meta{after} after the effects of |\ekvset|. The first variant will put
%   \meta{after} after any other tokens which might have been sneaked before,
%   while |\ekvsneakPre| will put \meta{after} before other smuggled stuff.
%   This reads and reinserts the remainder of the current |\ekvset| macro and
%   its argument list to do its job. A small usage example is shown in
%   \autoref{sec:sneakex}.
% \end{function}
%
% \begin{function}{\ekvchangeset}
%   \begin{syntax}
%     \cs{ekvchangeset}\marg{new-set}
%   \end{syntax}
%   Replaces the current set with \meta{new-set}, so for the rest of the current
%   |\ekvset| call, that call behaves as if it was called with
%   \texttt{\cs[no-index]{ekvset}\marg{new-set}}. Just like |\ekvsneak| this
%   reads and reinserts the remainder of the current |\ekvset| macro to do its
%   job. It is comparable to using \texttt{\meta{key}/.cd} in \pkg{pgfkeys}.
% \end{function}
%
% \bigskip
%
% \begin{function}{\ekv@name,\ekv@name@set,\ekv@name@key}
%   \begin{syntax}
%     \cs{ekv@name}\marg{set}\marg{key}\\
%     \cs{ekv@name@set}\marg{set}\\
%     \cs{ekv@name@key}\marg{key}
%   \end{syntax}
%   The names of the macros that correspond to a key in a set are build with
%   these macros. The default definition of |\ekv@name@set| is
%   ``\texttt{\csname ekv@name@set\endcsname{\set}}'' and the default of
%   |\ekv@name@key| is ``\texttt{\csname ekv@name@key\endcsname{\key}}''.
%   The complete name is build using |\ekv@name| which is equivalent to
%   \texttt{\cs[no-index]{ekv@name@set}\marg{set}\cs{ekv@name@key}\marg{key}}.
%   For |NoVal| keys an additional |N| gets appended irrespective of these
%   macros' definition, so their name is
%   \cs[no-index]{\csname ekv@name\endcsname{\set}{\key}N}. You might redefine
%   |\ekv@name@set| and |\ekv@name@key| locally but \emph{don't redefine}
%   |\ekv@name|!
% \end{function}
%
% \subsubsection{Bugs}
%
% Just like \pkg{keyval}, \expkv\ is bug free. But if you find \pmso{bugs}hidden
% features\footnote{Thanks, David!} you can tell me about them either via mail
% (see the first page) or directly on GitHub if you have an account there:
% \url{https://github.com/Skillmon/tex_expkv}
%
% \subsubsection{Comparisons}
%
% Comparisons of speed are done with a very simple test key and the help of the
% \pkg{l3benchmark} package. The key and its usage should be equivalent to
% \begin{lstlisting}
% \protected\ekvdef{test}{height}{\def\myheight{#1}}
% \ekvset{test}{ height = 6 }
% \end{lstlisting}
% and only the usage of the key, not its definition, is benchmarked. For the
% impatient, the essence of these comparisons regarding speed and buggy
% behaviour is contained in \autoref{tab:comp}.
%
% As far as I know \expkv\ is the only fully expandable \kv\ parser. I tried to
% compare \expkv\ to every \kv\ package listed on
% \href{https://ctan.org/topic/keyval}{CTAN}, however, one might notice that
% some of those are missing from this list. That's because I didn't get the
% others to work due to bugs, or because they just provide wrappers around other
% packages in this list.
%
% In this subsubsection is no benchmark of |\ekvparse| and |\keyval_parse:NNn|
% contained, as most other packages don't provide equivalent features to my
% knowledge. |\ekvparse| is slightly faster than |\ekvset|, but keep in mind
% that it does less. The same is true for |\keyval_parse:NNn| compared to
% |\keys_set:nn| of \pkg{expl3} (where the difference is much bigger).
%
% \paragraph{\pkg{keyval}} is about two times faster and has a comparable
% feature set just a slightly different way how it handles keys without values.
% That might be considered a drawback, as it limits the versatility, but also as
% an advantage, as it might reduce doubled code. Keep in mind that as soon as
% someone loads \pkg{xkeyval} the performance of \pkg{keyval} gets replaced by
% \pkg{xkeyval}'s.
%
% Also \pkg{keyval} has a bug, which unfortunately can't really be resolved
% without breaking backwards compatibility for \emph{many} documents, namely it
% strips braces from the argument before stripping spaces if the argument isn't
% surrounded by spaces, also it might strip more than one set of braces. Hence
% all of the following are equivalent in their outcome, though the last two
% lines should result in something different than the first two:
% \begin{lstlisting}[belowskip=0pt]
% \setkeys{foo}{bar=baz}
% \setkeys{foo}{bar= {baz}}
% \setkeys{foo}{bar={ baz}}
% \setkeys{foo}{bar={{baz}}}
% \end{lstlisting}
%
% \paragraph{\pkg{xkeyval}} is roughly fourteen times slower, but it provides
% more functionality, e.g., it has choice keys, boolean keys, and so on. It
% contains the same bug as \pkg{keyval} as it has to be compatible with it by
% design (it replaces \pkg{keyval}'s frontend), but also adds even more cases in
% which braces are stripped that shouldn't be stripped, worsening the situation.
%
% \paragraph{\pkg{ltxkeys}} is over 300 times slower -- which is funny, because
% it aims to be ``[\ldots] faster than these earlier packages [referring to
% \pkg{keyval} and \pkg{xkeyval}].'' Since it aims to have a bigger feature set
% than \pkg{xkeyval}, it most definitely also has a bigger feature set than
% \expkv. Also, it can't parse |\long| input, so as soon as your values contain
% a |\par|, it'll throw errors. Furthermore, \pkg{ltxkeys} doesn't strip outer
% braces at all by design, which, imho, is a weird design choice. In addition
% \pkg{ltxkeys} loads \pkg{catoptions} which is known to introduce bugs (e.g.,
% see \url{https://tex.stackexchange.com/questions/461783}).
%
% \paragraph{\pkg{l3keys}} is almost five times slower, but has an, imho, great
% interface to define keys. It strips \emph{all} outer spaces, even if somehow
% multiple spaces ended up on either end. It offers more features, but is pretty
% much bound to \pkg{expl3} code. Whether that's a drawback is up to you.
%
% \paragraph{\pkg{pgfkeys}} is a bit more than two times slower for one key,
% but has an \emph{enormous} feature set. However, since adding additional keys
% doesn't add as much needed time for \pkg{pgfkeys} compared to \expkv, it gets
% faster than \expkv\ at around eight \kv\ pairs.
% It has the same or a very similar bug \pkg{keyval} has. The brace bug (and
% also the category fragility) can be fixed by \pkg{pgfkeyx}, but this package
% was last updated in 2012 and it slows down |\pgfkeys| by factor~$8$. Also I
% don't know whether this might introduce new bugs.
%
% \paragraph{\pkg{kvsetkeys} with \pkg{kvdefinekeys}} is about three times
% slower, but it works even if commas and equals have category codes different
% from 12 (just as some other packages in this list). Else the features of the
% keys are equal to those of \pkg{keyval}, the parser has more features, though.
%
% \paragraph{\pkg{options}} is a bit slower for only a single value, but gets a
% tad faster than \expkv\ at around 10 \kv\ pairs. It has a much bigger feature
% set. Unfortunately it also suffers from the premature unbracing bug
% \pkg{keyval} has.
%
% \paragraph{\pkg{simplekv}} is hard to compare because I don't speak French (so
% I don't understand the documentation) and from what I can see, there is no
% direct way to define the equivalent test key. Nevertheless, I tested the
% closest possible equivalent of my test key while siding for \pkg{simplekv}'s
% design not forcing something into it it doesn't seem to be designed for. It is
% more than five times slower and has hard to predict behaviour regarding brace
% and space stripping, similar to \pkg{keyval}. The tested definition was:
% \begin{lstlisting}[belowskip=0pt]
% \usepackage{simplekv}
% \setKVdefault[simplekv]{height={ abc}}                     % key setup
% \setKV[simplekv]{ height = 6 }                             % benchmarked
% \end{lstlisting}
%
% \paragraph{\pkg{yax}} is over eighteen times slower. It has a pretty strange
% syntax, imho, and again a direct equivalent is hard to define. It has the
% premature unbracing bug, too. Also somehow loading \pkg{yax} broke
% \pkg{options} for me. The tested definition was:
% \begin{lstlisting}[belowskip=0pt]
% \usepackage{yax}
% \defactiveparameter yax {\storevalue\myheight yax:height } % key setup
% \setparameterlist{yax}{ height = 6 }                       % benchmarked
% \end{lstlisting}
%
% \begin{table}
%   \def\fnsym{\textcolor{red!80!black}{*}}%
%   \sisetup{round-precision=1, round-mode=places}%
%   \begingroup
%   \centering
%   \newcommand*\yes{\textcolor{red!80!black}  {yes}}^^A
%   \newcommand*\no {\textcolor{green!80!black}{no}}^^A
%   \caption[Comparison of \kv\ packages]
%     {^^A
%       Comparison of \kv\ packages. The packages are ordered from
%       fastest to slowest for one \kv\ pair. Benchmarking was done using
%       \pkg{l3benchmark} and the scripts in the \file{Benchmarks} folder
%       of the \href{https://github.com/Skillmon/tex_expkv}{git repository}.
%       The columns $p_i$ are the polynomial coefficients of a linear fit to the
%       run-time, $p_0$ can be interpreted as the overhead for initialisation
%       and $p_1$ the cost per key. The $T_0$ column is the actual mean ops
%       needed for an empty list argument, as the linear fit doesn't match that
%       point well in general. The column ``BB'' lists whether the
%       parsing is affected by some sort of brace bug, ``CF'' stands for
%       category code fragile and lists whether the parsing breaks with active
%       commas or equal signs.^^A
%       \label{tab:comp}^^A
%     }
%   \begin{tabular}
%       {>{\collectcell\pkg}l<{\endcollectcell}*3{S[table-format=4.1]}ccc}
%     \toprule
%     \rmfamily Package & {$p_1$} & {$p_0$} & {$T_0$}& BB & CF & Date \\
%     \midrule
%     keyval    &   13.539 &    1.844 &    6.969 & \yes & \yes & 2014-10-28 \\
%     \expkv    &   26.411 &    2.391 &   12.610 & \no  & \no  & 2020-02-22 \\
%     options   &   25.532 &   10.518 &   21.393 & \yes & \yes & 2015-03-01 \\
%     pgfkeys   &   26.444 &   40.939 &   55.790 & \yes & \yes & 2020-01-08 \\
%     kvsetkeys & {\fnsym} & {\fnsym} &   41.620 & \no  & \no  & 2019-12-15 \\
%     l3keys    &  114.456 &   35.357 &   52.667 & \no  & \no  & 2020-02-14 \\
%     simplekv  &  160.523 &   10.809 &    8.698 & \yes & \yes & 2017-08-08 \\
%     xkeyval   &  260.872 &  180.435 &  164.167 & \yes & \yes & 2014-12-03 \\
%     yax       &  507.643 &   62.863 &  123.533 & \yes & \yes & 2010-01-22 \\
%     ltxkeys   & 3932.759 & 4737.818 & 5883.000 & \no  & \no  & 2012-11-17 \\
%     \bottomrule
%   \end{tabular}
%   \par
%   \endgroup
%   \medskip
%   \fnsym For \pkg{kvsetkeys} the linear model used for the other
%   packages is a poor fit, \pkg{kvsetkeys} seems to have approximately
%   quadratic run-time, the coefficients of the second degree polynomial fit are
%   $p_2=\num{9.911}$, $p_1=\num{40.523}$, and $p_0=\num{61.500}$. Of course the
%   other packages might not really have linear run-time, but at least from 1~to
%   20~keys the fits don't seem too bad (the maximum ratio $p_2/p_1$
%   for the other packages is \num{9.697e-3}). If one extrapolates the fits
%   for 100 \kv\ pairs one finds that most of them match pretty well, the
%   exception being \pkg{ltxkeys}, which behaves quadratic as well with
%   $p_2=\num{31.685}$, $p_1=\num{3267.368}$, and $p_0=\num{7177.582}$.
% \end{table}
%
% \subsection{Examples}
%
% \subsubsection{Standard Use-Case}
%
% Say we have a macro for which we want to create a \kv\ interface. The macro
% has a parameter, which is stored in the dimension |\ourdim| having a default
% value from its initialization. Now we want to be able to change that dimension
% with the |width| key to some specified value. For that we'd do
% \begin{lstlisting}
% \newdimen\ourdim
% \ourdim=150pt
% \protected\ekvdef{our}{width}{\ourdim=#1\relax}
% \end{lstlisting}
% as you can see, we use the set |our| here. We want the key to behave different
% if no value is specified. In that case the key should not use its initial
% value, but be smart and determine the available space from
% \cs[no-index]{hsize}, so we also define
% \begin{lstlisting}
% \protected\ekvdefNoVal{our}{width}{\ourdim=.9\hsize}
% \end{lstlisting}
% Now we set up our macro to use this \kv\ interface
% \begin{lstlisting}
% \protected\def\ourmacro#1{\begingroup\ekvset{our}{#1}\the\ourdim\endgroup}
% \end{lstlisting}
% Finally we can use our macro like in the following\\
% \begin{minipage}{6cm}
% \begin{lstlisting}
% \ourmacro{}\par
% \ourmacro{width}\par
% \ourmacro{width=5pt}\par
% \end{lstlisting}
% \end{minipage}^^A
% \begin{minipage}{\dimexpr\linewidth-6cm\relax}
%   \newdimen\ourdim\ourdim=150pt
%   \protected\ekvdef{our}{width}{\ourdim=#1\relax}^^A
%   \protected\ekvdefNoVal{our}{width}{\ourdim=.9\hsize}^^A
%   \protected\def\ourmacro#1^^A
%     {\begingroup\ekvset{our}{#1}\the\ourdim\endgroup}^^A
%   \ourmacro{}\par
%   \ourmacro{width}\par
%   \ourmacro{width=5pt}\par
% \end{minipage}
%
% \paragraph{The same key using \protect\expkvd}
% Using \expkvd\ we can set up the equivalent key using a \kv\ interface, after
% the following we could use |\ourmacro| in the same way as above. \expkvd\ will
% allocate and initialise |\ourdim| and define the |width| key |\protected| for
% us, so the result will be exactly the same -- with the exception that the
% default will use |\ourdim=.9\hsize\relax| instead.
% \begin{lstlisting}
% \input expkv-def                 % or \usepackage{expkv-def}
% \ekvdefinekeys{our}
%   {
%     dimen    width = \ourdim,
%     qdefault width = .9\hsize,
%     initial  width = 150pt
%   }
% \end{lstlisting}
%
% \subsubsection{An Expandable \kv\ Macro Using \cs[no-index]{ekvsneak}}
% \label{sec:sneakex}
%
% Let's set up an expandable macro, that uses a \kv\ interface. The problems
% we'll face for this are:
% \begin{enumerate}
%   \item ignoring duplicate keys
%   \item default values for keys which weren't used
%   \item providing the values as the correct argument to a macro (ordered)
% \end{enumerate}
%
% First we need to decide which \kv\ parsing macro we want to do this with,
% |\ekvset| or |\ekvparse|. For this example we also want to show the
% usage of |\ekvsneak|, hence we'll choose |\ekvset|. And we'll have to use
% |\ekvset| such that it builds a parsable list for our macro internals. To
% gain back control after |\ekvset| is done we have to put an internal of our
% macro at the start of that list, so we use an internal key that uses
% |\ekvsneakPre| after any user input.
%
% To ignore duplicates will be easy if the value of the key used last will be
% put first in the list, so the following will use |\ekvsneakPre| for the
% user-level keys. If we wanted some key for which the first usage should be the
% the binding one we would use |\ekvsneak| instead for that key.
%
% Providing default values can be done in different ways, we'll use a simple
% approach in which we'll just put the outcome of our keys if they were used
% with default values before the parsing list terminator.
%
% Ordering the keys can be done simply by searching for a specific token for
% each argument which acts like a flag, so our sneaked out values will include
% specific tokens acting as markers.
%
% Now that we have answers for our technical problems, we have to decide what
% our example macro should do. How about we define a macro that calculates the
% sine of a number and rounds that to a specified precision? As a small extra
% this macro should understand input in radian and degree and the used
% trigonometric function should be selectable as well. For the hard part of this
% task (expandably evaluating trigonometric functions) we'll use the \pkg{xfp}
% package.
%
% First we set up our keys according to our earlier considerations and set up
% the user facing macro |\sine|. The end marker of the parsing list will be a
% |\sine@stop| token, which we don't need to define and we put our defaults
% right before it.
%
% \begin{lstlisting}
% \RequirePackage{xfp}
% \makeatletter
% \ekvdef{expex}{f}{\ekvsneakPre{\f{#1}}}
% \ekvdef{expex}{round}{\ekvsneakPre{\rnd{#1}}}
% \ekvdefNoVal{expex}{degree}{\ekvsneakPre{\deg{d}}}
% \ekvdefNoVal{expex}{radian}{\ekvsneakPre{\deg{}}}
% \ekvdefNoVal{expex}{internal}{\ekvsneakPre{\sine@rnd}}
% \newcommand*\sine[2]
%   {\ekvset{expex}{#1,internal}\rnd{3}\deg{d}\f{sin}\sine@stop{#2}}
% \end{lstlisting}
% For the sake of simplicity we defined the macro |\sine| with two mandatory
% arguments, the first being the \kv\ list, the second the argument to the
% trigonometric function. We could've used \pkg{xparse}'s facilities here to
% define an expandable macro which takes an optional argument instead.
%
% Now we need to define some internal macros to extract the value of each key's
% last usage (remember that this will be the group after the first special
% flag-token). For that we use one delimited macro per key.
% \begin{lstlisting}
% \def\sine@rnd#1\rnd#2#3\sine@stop{\sine@deg#1#3\sine@stop{#2}}
% \def\sine@deg#1\deg#2#3\sine@stop{\sine@f#1#3\sine@stop{#2}}
% \def\sine@f#1\f#2#3\sine@stop{\sine@final{#2}}
% \end{lstlisting}
% After the macros |\sine@rnd|, |\sine@deg|, and |\sine@f| the macro
% |\sine@final| will see
% \texttt
%   {^^A
%     \cs[no-index]{sine@final}^^A
%     \marg{f}\marg{degree/radian}\marg{round}\marg{num}^^A
%   }.
% Now |\sine@final| has to expandably deal with those arguments such that the
% |\fpeval| macro of \pkg{xfp} gets the correct input. Luckily this is pretty
% straight forward in this example. In |\fpeval| the trigonometric functions
% have names such as |sin| or |cos| and the degree taking variants |sind| or
% |cosd|. And since the |degree| key puts a |d| in |#2| and the |radian| key
% leaves |#2| empty all we have to do to get the correct function name is stick
% the two together.
% \begin{lstlisting}
% \newcommand*\sine@final[4]{\fpeval{round(#1#2(#4),#3)}}
% \makeatother
% \end{lstlisting}
% Let's test our macro:\\
% \begin{minipage}{.7\linewidth}
%   \begin{lstlisting}[gobble=4]
%   \sine{}{60}\par
%   \sine{round=10}{60}\par
%   \sine{f=cos,radian}{pi}\par
%   \edef\myval{\sine{f=tan}{1}}\texttt{\meaning\myval}
%   \end{lstlisting}
% \end{minipage}\hfill
% \begin{minipage}{.2\linewidth}
%   \makeatletter
%   \ekvdef{expex}{f}{\ekvsneakPre{\f{#1}}}^^A
%   \ekvdef{expex}{round}{\ekvsneakPre{\rnd{#1}}}^^A
%   \ekvdefNoVal{expex}{degree}{\ekvsneakPre{\deg{d}}}^^A
%   \ekvdefNoVal{expex}{radian}{\ekvsneakPre{\deg{}}}^^A
%   \ekvdefNoVal{expex}{internal}{\ekvsneakPre{\sine@rnd}}^^A
%   \newcommand*\sine[2]
%     {\ekvset{expex}{#1,internal}\rnd{3}\deg{d}\f{sin}\sine@stop{#2}}^^A
%   \def\sine@rnd#1\rnd#2#3\sine@stop{\sine@deg#1#3\sine@stop{#2}}^^A
%   \def\sine@deg#1\deg#2#3\sine@stop{\sine@f#1#3\sine@stop{#2}}^^A
%   \def\sine@f#1\f#2#3\sine@stop{\sine@final{#2}}^^A
%   \newcommand*\sine@final[4]{\fpeval{round(#1#2(#4),#3)}}^^A
%   \makeatother
%   \sine{}{60}\par
%   \sine{round=10}{60}\par
%   \sine{f=cos,radian}{pi}\par
%   \edef\myval{\sine{f=tan}{1}}\texttt{\meaning\myval}
% \end{minipage}
%
%
% \subsection{Error Messages}
%
% \expkv\ should only send messages in case of errors, there are no warnings and
% no info messages. In this subsection those errors are listed.
%
% \subsubsection{Load Time}
%
% \file{expkv.tex} checks whether \eTeX\ is available. If it isn't, an error
% will be thrown using |\errmessage|:
% \begin{lstlisting}
% ! expkv Error: e-TeX required.
% \end{lstlisting}
%
% \subsubsection{Defining Keys}
%
% If you get any error from \expkv\ while you're trying to define a key, the
% definition will be aborted and gobbled.
%
% If you try to define a key with an empty set name you'll get:
% \begin{lstlisting}
% ! expkv Error: empty set name not allowed.
% \end{lstlisting}
%
% Similarly, if you try to define a key with an empty key name:
% \begin{lstlisting}
% ! expkv Error: empty key name not allowed.
% \end{lstlisting}
%
% Both of these messages are done in a way that doesn't throw additional errors
% due to |\global|, |\long|, etc., not being used correctly if you prefixed one
% of the defining macros.
%
% \subsubsection{Using Keys}
%
% This subsubsection contains the errors thrown during |\ekvset|. The errors are
% thrown in an expandable manner by providing an undefined macro. In the
% following messages \lstinline|<key>| gets replaced with the problematic key's
% name, and \lstinline|<set>| with the corresponding set. If any errors during
% \kv\ handling are encountered, the entry in the comma separated list will be
% omitted after the error is thrown and the next \kv\ pair will be parsed.
%
% If you're using an undefined key you'll get:
% \begin{lstlisting}
% ! Undefined control sequence.
% <argument> \! expkv Error:
%                            unknown key (`<key>', set `<set>').
% \end{lstlisting}
%
% If you're using a key for which only a normal version and no |NoVal| version
% is defined, but don't provide a value, you'll get:
% \begin{lstlisting}
% ! Undefined control sequence.
% <argument> \! expkv Error:
%                            value required (`<key>', set `<set>').
% \end{lstlisting}
%
% If you're using a key for which only a |NoVal| version and no normal version
% is defined, but provide a value, you'll get:
% \begin{lstlisting}
% ! Undefined control sequence.
% <argument> \! expkv Error:
%                            value forbidden (`<key>', set `<set>').
% \end{lstlisting}
%
% If you're using a set for which you never executed one of the defining macros
% from \autoref{sec:define} you'll get a low level \TeX\ error, as that isn't
% actively tested by the parser (and hence will lead to undefined behaviour and
% not be gracefully ignored). The error will look like
% \begin{lstlisting}
% ! Missing \endcsname inserted.
% <to be read again>
%                    \! expkv Error: Set `<set>' undefined.
% \end{lstlisting}
%
% \subsection{License}
%
% Copyright \textcopyright\ 2020\unless\ifnum\year=2020--\the\year\fi\
% Jonathan P. Spratte
%
% \medskip\noindent
% This work may be distributed and/or modified under the conditions of the
% \LaTeX\ Project Public License (LPPL), either version 1.3c of this license or
% (at your option) any later version. The latest version of this license is in
% the file:
%
% \url{http://www.latex-project.org/lppl.txt}
%
% \noindent
% This work is ``maintained'' (as per LPPL maintenance status) by
%
% \mbox{Jonathan P. Spratte}.
%
% \end{documentation}^^A=<<
%
% \begin{implementation}^^A>>=
%
% \clearpage
%
% \section{Implementation}^^A>>=
%^^A the LaTeX package >>=
% \subsection{The \LaTeX\ Package}
% First we set up the \LaTeX\ package. That one doesn't really do much except
% |\input|ting the generic code and identifying itself as a package.
% \gobbledocstriptag
%<*pkg>
%    \begin{macrocode}
\def\ekv@tmp
  {%
    \ProvidesFile{expkv.tex}%
      [\ekvDate\space v\ekvVersion\space an expandable key=val implementation]%
  }
% \iffalse meta-comment
%
% File: expkv.dtx Copyright (C) 2020 Jonathan P. Spratte
%
% This work  may be  distributed and/or  modified under  the conditions  of the
% LaTeX Project Public License (LPPL),  either version 1.3c  of this license or
% (at your option) any later version.  The latest version of this license is in
% the file:
%
%   http://www.latex-project.org/lppl.txt
%
% ------------------------------------------------------------------------------
%
%<*driver>^^A>>=
\def\nameofplainTeX{plain}
\ifx\fmtname\nameofplainTeX\else
  \expandafter\begingroup
\fi
\input l3docstrip.tex
\askforoverwritefalse
\preamble

--------------------------------------------------------------
expkv -- an expandable key=val implementation
E-mail: jspratte@yahoo.de
Released under the LaTeX Project Public License v1.3c or later
See http://www.latex-project.org/lppl.txt
--------------------------------------------------------------

Copyright (C) 2020 Jonathan P. Spratte

This  work may be  distributed and/or  modified under  the conditions  of the
LaTeX Project Public License (LPPL),  either version 1.3c  of this license or
(at your option) any later version.  The latest version of this license is in
the file:

  http://www.latex-project.org/lppl.txt

This work is "maintained" (as per LPPL maintenance status) by
  Jonathan P. Spratte.

This work consists of the file  expkv.dtx
and the derived files           expkv.pdf
                                expkv.sty
                                expkv.tex

\endpreamble
% stop docstrip adding \endinput
\postamble
\endpostamble
\generate{\file{expkv.sty}{\from{expkv.dtx}{pkg}}}
\generate{\file{expkv.tex}{\from{expkv.dtx}{tex}}}
\ifx\fmtname\nameofplainTeX
  \expandafter\endbatchfile
\else
  \expandafter\endgroup
\fi
%
\IfFileExists{expkv.tex}{\input{expkv.tex}}{}
\ProvidesFile{expkv.dtx}
  [\csname ekvDate\endcsname\ an expandable key=val implementation]
\PassOptionsToPackage{full}{textcomp}
\documentclass{l3doc}
\RequirePackage[oldstylenums,nott]{kpfonts}
\input{glyphtounicode}
\pdfgentounicode=1
\RequirePackage{xfp} ^^A required for an example
\RequirePackage{listings}
\RequirePackage{booktabs}
\RequirePackage{array}
\RequirePackage{collcell}
\RequirePackage{siunitx}
\RequirePackage{xcolor}
\RequirePackage{caption}
\RequirePackage{microtype}
\RequirePackage{accsupp}
\lstset
  {
    ,flexiblecolumns=false
    ,basewidth=.53em
    ,gobble=2
    ,basicstyle=\fontfamily{jkp}\itshape
    ,morekeywords=^^A
      {^^A
        \ekvdef,\ekvdefNoVal,\ekvlet,\ekvletNoVal,\ekvletkv,\ekvletkvNoVal,^^A
        \ekvset,\ekvparse,\ekvVersion,\ekvDate,\ekvifdefined,^^A
        \ekvifdefinedNoVal,\ekvbreak,\ekvbreakPreSneak,\ekvbreakPostSneak,^^A
        \ekvsneak,\ekvsneakPre,\ekvchangeset
      }
    ,morecomment=[l]\%
    ,commentstyle=\color[gray]{0.4}
    ,literate={\{}{{\CodeSymbol\{}}{1}
              {\}}{{\CodeSymbol\}}}{1}
    ^^A,literate=*{<key>}{\key}{4}{<set>}{\set}{4}
  }
\newcommand*\CodeSymbol[1]{\textbf{#1}}
\RequirePackage{randtext}
\let\metaORIG\meta
\protected\def\meta #1{\texttt{\metaORIG{#1}}}
\renewcommand*\thefootnote{\fnsymbol{footnote}}
\definecolor{expkvred}{HTML}{9F393D}
\colorlet{expkvgrey}{black!75}
\makeatletter
\newcommand*\expkv
  {^^A
    \texorpdfstring
      {^^A
        \mbox
          {^^A
            \BeginAccSupp{ActualText=expkv}^^A
            \href{https://github.com/Skillmon/tex_expkv}
              {^^A
                \rmfamily
                \bfseries
                {\color{expkvgrey}e\kern-.05em x\kern-.05em}^^A
                \lower.493ex
                  \hbox{{\color{expkvgrey}P}\kern-.1em{\color{expkvred}k}}^^A
                \kern-.18em{\color{expkvred}v}^^A
              }^^A
            \EndAccSupp{}^^A
          }^^A
      }
      {expkv}^^A
  }
\newcommand*\expkvd
  {^^A
    \texorpdfstring
      {^^A
        \mbox
          {^^A
            \BeginAccSupp{ActualText=expkv-def}^^A
            \rmfamily
            \bfseries
            {\color{expkvgrey}e\kern-.05em x\kern-.05em}^^A
            \lower.493ex
              \hbox{{\color{expkvgrey}P}\kern-.1em{\color{expkvred}k}}^^A
            \kern-.18em{\color{expkvred}v}^^A
            {\color{expkvgrey}^^A
              \kern.05em\rule[-.1ex]{.08em}{1.2ex}\kern.05em\textsc{def}^^A
            }^^A
            \EndAccSupp{}^^A
          }^^A
      }
      {expkv-def}^^A
  }
\newcommand\kv{\meta{key}=\meta{value}}
\newcommand\key{\meta{key}}
\newcommand\val{\meta{value}}
\newcommand\set{\meta{set}}
\hypersetup{linkcolor=red!80!black,urlcolor=purple!80!black}
\DoNotIndex{\def,\edef,\,,\=,\begingroup,\catcode,\chardef,\csname,\endcsname}
\DoNotIndex{\endgroup,\endinput,\errmessage,\expandafter,\input,\let,\long}
\DoNotIndex{\noexpand,\protected,\ProvidesFile,\ProvidesPackage,\relax,\space}
\DoNotIndex{\@,\unexpanded,\string,\expanded}
\DoNotIndex{\ifcsname}
\DoNotIndex{\ifx}
\DoNotIndex{\else}
\DoNotIndex{\fi}
\@gobble\fi ^^A ignoring \ifx and \ifcsname, but only one \fi
\@ifdefinable\gobbledocstriptag{\def\gobbledocstriptag#1>{}}
\newcommand*\pmso[1] ^^A poor man's strike out
  {^^A
    \leavevmode
    \begingroup
      \sbox0{#1}^^A
      \rlap{\vrule height .6ex depth -.5ex width \wd0\relax}^^A
      \usebox0\relax
    \endgroup
  }
\makeatother
\begin{document}
  \title
    {^^A
      \texorpdfstring
        {^^A
          \huge\expkv
          \\[\medskipamount]
          \Large an expandable \kv\ implementation^^A
        }{expkv - an expandable <key>=<value> implementation}^^A
    }
  \date{\ekvDate\space v\ekvVersion}
  \author{Jonathan P. Spratte\thanks{\protect\randomize{jspratte@yahoo.de}}}
  \DocInput{expkv.dtx}
\end{document}
%</driver>^^A=<<
% \fi
%
% \maketitle
% \renewcommand*\thefootnote{\arabic{footnote}}
%
% \begin{abstract}
% \noindent\parfillskip=0pt
% \expkv\ provides a small interface for \kv\ parsing. The parsing macro is
% fully expandable, the \meta{code} of your keys might be not. \expkv\ is pretty
% fast, but not the fastest available \kv\ solution (\pkg{keyval} for instance
% is thrice as fast, but not expandable and it might strip braces it shouldn't
% have stripped).
% \end{abstract}
%
% \tableofcontents
%
% \begin{documentation}^^A>>=
%
% \section{Documentation}
%
% \expkv\ provides an expandable \kv\ parser. The \kv\ pairs should be given as
% a comma separated list and the separator between a \key\ and the associated
% \val\ should be an equal sign. Both, the commas and the equal signs, might be
% of category 12 (other) or 13 (active). To support this is necessary as for
% example \pkg{babel} turns characters active for some languages, for instance
% the equal sign is turned active for Turkish.
%
% \expkv\ is usable as generic code or as a \LaTeX\ package. To use it, just
% use one of:
% \begin{lstlisting}
% \usepackage{expkv} % LaTeX
% \input expkv       % plainTeX
% \end{lstlisting}
% The \LaTeX\ package doesn't do more than \file{expkv.tex}, except calling
% |\ProvidesPackage| and setting things up such that \file{expkv.tex} will use
% |\ProvidesFile|.
%
% \subsection{Setting up Keys}\label{sec:define}
% \expkv\ provides a rather simple approach to setting up keys, similar to
% \pkg{keyval}. However there is an auxiliary package named
% \href{https://github.com/Skillmon/tex_expkv-def}{\expkvd} (the files use
% \file{expkv-def} instead because \string| isn't allowed in package names by
% CTAN) which provides a more sophisticated interface, similar to well
% established packages like \pkg{pgfkeys} or \pkg{l3keys}.
%
% Keys in \expkv\ (as in almost all other \kv\ implementations) belong to a
% \emph{set} such that different sets can contain keys of the same name. Unlike
% many other implementations \expkv\ doesn't provide means to set a default
% value, instead we have keys that take values and keys that don't (the latter
% are called |NoVal| keys by \expkv), but both can have the same name (on the
% user level).
%
% The following macros are available to define new keys. Those macros containing
% ``|def|'' in their name can be prefixed by anything allowed to prefix |\def|,
% prefixes allowed for |\let| can prefix those with ``|let|'' in their name,
% accordingly. Neither \set\ nor \key\ are allowed to be empty for new keys and
% must not contain a |\par| or tokens that expand to it -- they must be legal
% inside of |\csname ...\endcsname|.
%
% \begin{function}{\ekvdef}
%   \begin{syntax}
%     \cs{ekvdef}\marg{set}\marg{key}\marg{code}
%   \end{syntax}
%   Defines a \key\ taking a value in a \set\ to expand to \meta{code}. In
%   \meta{code} you can use |#1| to refer to the given value.
% \end{function}
%
% \begin{function}{\ekvdefNoVal}
%   \begin{syntax}
%     \cs{ekvdefNoVal}\marg{set}\marg{key}\marg{code}
%   \end{syntax}
%   Defines a no value taking \key\ in a \set\ to expand to \meta{code}.
% \end{function}
%
% \begin{function}{\ekvlet}
%   \begin{syntax}
%     \cs{ekvlet}\marg{set}\marg{key}\meta{cs}
%   \end{syntax}
%   Let the value taking \key\ in \set\ to \meta{cs}, there are no checks on
%   \meta{cs} enforced.
% \end{function}
%
% \begin{function}{\ekvletNoVal}
%   \begin{syntax}
%     \cs{ekvletNoVal}\marg{set}\marg{key}\meta{cs}
%   \end{syntax}
%   Let the no value taking \key\ in \set\ to \meta{cs}, it is not checked
%   whether \meta{cs} exists or that it takes no parameter.
% \end{function}
%
% \begin{function}{\ekvletkv}
%   \begin{syntax}
%     \cs{ekvletkv}\marg{set}\marg{key}\marg{set2}\marg{key2}
%   \end{syntax}
%   Let the \key\ in \set\ to \meta{key2} in \meta{set2}, it is not checked
%   whether that second key exists.
% \end{function}
%
% \begin{function}{\ekvletkvNoVal}
%   \begin{syntax}
%     \cs{ekvletkvNoVal}\marg{set}\marg{key}\marg{set2}\marg{key2}
%   \end{syntax}
%   Let the \key\ in \set\ to \meta{key2} in \meta{set2}, it is not checked
%   whether that second key exists.
% \end{function}
%
% \subsection{Parsing Keys}
%
% \begin{function}{\ekvset}
%   \begin{syntax}
%     \cs{ekvset}\marg{set}\{\kv,\ldots\}
%   \end{syntax}
%   Splits \kv\ pairs on commas. From both \key\ and \val\ up to one space is
%   stripped from both ends, if then only a braced group remains the braces are
%   stripped as well. So |\ekvset{foo}{bar=baz}| and
%   |\ekvset{foo}{ {bar}= {baz} }| will both do
%   \texttt{\cs[no-index]{\meta{foobarcode}}\{baz\}}, so you can hide commas,
%   equal signs and spaces at the ends of either \key\ or \val\ by putting
%   braces around them. If you omit the equal sign the code of the key created
%   with the |NoVal| variants described in \autoref{sec:define} will be
%   executed. If \kv\ contains more than a single unhidden equal sign, it will
%   be split at the first one and the others are considered part of the value.
%   |\ekvset| should be nestable.
% \end{function}
%
% \begin{function}{\ekvparse}
%   \begin{syntax}
%     \cs{ekvparse}\meta{cs1}\meta{cs2}\{\kv,\ldots\}
%   \end{syntax}
%   This macro parses the \kv\ pairs and provides those list elements which are
%   only keys as the argument to \meta{cs1}, and those which are a \kv\ pair to
%   \meta{cs2} as two arguments. It is fully expandable as well and returns the
%   parsed list in |\unexpanded|, which has no effect outside of an |\expanded|
%   or |\edef| context\footnotemark.
%   If you need control over the necessary steps of expansion you
%   can use |\expanded| around it.
%
%   |\ekvbreak|, |\ekvsneak|, and |\ekvchangeset| and their relatives don't work
%   in |\ekvparse|.  It is analogue to \pkg{expl3}'s |\keyval_parse:NNn|, but
%   not with the same parsing rules -- |\keyval_parse:NNn| throws an error on
%   multiple equal signs per \kv\ pair and on empty \key\ names in a \kv\ pair,
%   both of which |\ekvparse| doesn't deal with.
%
%   As a small example:
% \begin{lstlisting}
% \ekvparse\handlekey\handlekeyval{foo = bar, key, baz={zzz}}
% \end{lstlisting}
%   would expand to
% \begin{lstlisting}
% \handlekeyval{foo}{bar}\handlekey{key}\handlekeyval{baz}{zzz}
% \end{lstlisting}
%   and afterwards |\handlekey| and |\handlekeyval| would have to further handle
%   the \key. There are no macros like these two contained in \expkv, you have
%   to set them up yourself if you want to use |\ekvparse| (of course the names
%   might differ). If you need the results of |\ekvparse| as the argument for
%   another macro, you should use |\expanded| as only then the input stream will
%   contain the output above:
% \begin{lstlisting}
% \expandafter\handle\expanded{\ekvparse\k\kv{foo = bar, key, baz={zzz}}}
% \end{lstlisting}
%   would expand to
% \begin{lstlisting}
% \handle\kv{foo}{bar}\k{key}\kv{baz}{zzz}
% \end{lstlisting}
% \end{function}
% \footnotetext 
%   {^^A
%     This is a change in behaviour, previously (v0.3 and before) \cs{ekvparse}
%     would expand in exactly two steps. This isn't always necessary, but makes
%     the parsing considerably slower. If this is necessary for your application
%     you can put an \cs[no-index]{expanded} around it and will still be faster
%     since you need only a single \cs[no-index]{expandafter} this way.^^A
%   }.
%
% \subsection{Miscellaneous}
%
% \subsubsection{Other Macros}
%
% \expkv\ provides some other macros which might be of interest.
%
% \begin{function}{\ekvVersion,\ekvDate}
%   These two macros store the version and date of the package.
% \end{function}
% \begin{function}{\ekvifdefined,\ekvifdefinedNoVal}
%   \begin{syntax}
%     \cs{ekvifdefined}\marg{set}\marg{key}\marg{true}\marg{false}
%     \cs{ekvifdefinedNoVal}\marg{set}\marg{key}\marg{true}\marg{false}
%   \end{syntax}
%   These two macros test whether there is a \key\ in \set. It is false if
%   either a hash table entry doesn't exist for that key or its meaning is
%   |\relax|.
% \end{function}
%
% \begin{function}{\ekvbreak,\ekvbreakPreSneak,\ekvbreakPostSneak}
%   \begin{syntax}
%     \cs{ekvbreak}\marg{after}
%   \end{syntax}
%   Gobbles the remainder of the current |\ekvset| macro and its argument list
%   and reinserts \meta{after}. So this can be used to break out of |\ekvset|.
%   The first variant will also gobble anything that has been sneaked out using
%   |\ekvsneak| or |\ekvsneakPre|, while |\ekvbreakPreSneak| will put
%   \meta{after} before anything that has been smuggled and |\ekvbreakPostSneak|
%   will put \meta{after} after the stuff that has been sneaked out.
% \end{function}
%
% \begin{function}{\ekvsneak,\ekvsneakPre}
%   \begin{syntax}
%     \cs{ekvsneak}\marg{after}
%   \end{syntax}
%   Puts \meta{after} after the effects of |\ekvset|. The first variant will put
%   \meta{after} after any other tokens which might have been sneaked before,
%   while |\ekvsneakPre| will put \meta{after} before other smuggled stuff.
%   This reads and reinserts the remainder of the current |\ekvset| macro and
%   its argument list to do its job. A small usage example is shown in
%   \autoref{sec:sneakex}.
% \end{function}
%
% \begin{function}{\ekvchangeset}
%   \begin{syntax}
%     \cs{ekvchangeset}\marg{new-set}
%   \end{syntax}
%   Replaces the current set with \meta{new-set}, so for the rest of the current
%   |\ekvset| call, that call behaves as if it was called with
%   \texttt{\cs[no-index]{ekvset}\marg{new-set}}. Just like |\ekvsneak| this
%   reads and reinserts the remainder of the current |\ekvset| macro to do its
%   job. It is comparable to using \texttt{\meta{key}/.cd} in \pkg{pgfkeys}.
% \end{function}
%
% \bigskip
%
% \begin{function}{\ekv@name,\ekv@name@set,\ekv@name@key}
%   \begin{syntax}
%     \cs{ekv@name}\marg{set}\marg{key}\\
%     \cs{ekv@name@set}\marg{set}\\
%     \cs{ekv@name@key}\marg{key}
%   \end{syntax}
%   The names of the macros that correspond to a key in a set are build with
%   these macros. The default definition of |\ekv@name@set| is
%   ``\texttt{\csname ekv@name@set\endcsname{\set}}'' and the default of
%   |\ekv@name@key| is ``\texttt{\csname ekv@name@key\endcsname{\key}}''.
%   The complete name is build using |\ekv@name| which is equivalent to
%   \texttt{\cs[no-index]{ekv@name@set}\marg{set}\cs{ekv@name@key}\marg{key}}.
%   For |NoVal| keys an additional |N| gets appended irrespective of these
%   macros' definition, so their name is
%   \cs[no-index]{\csname ekv@name\endcsname{\set}{\key}N}. You might redefine
%   |\ekv@name@set| and |\ekv@name@key| locally but \emph{don't redefine}
%   |\ekv@name|!
% \end{function}
%
% \subsubsection{Bugs}
%
% Just like \pkg{keyval}, \expkv\ is bug free. But if you find \pmso{bugs}hidden
% features\footnote{Thanks, David!} you can tell me about them either via mail
% (see the first page) or directly on GitHub if you have an account there:
% \url{https://github.com/Skillmon/tex_expkv}
%
% \subsubsection{Comparisons}
%
% Comparisons of speed are done with a very simple test key and the help of the
% \pkg{l3benchmark} package. The key and its usage should be equivalent to
% \begin{lstlisting}
% \protected\ekvdef{test}{height}{\def\myheight{#1}}
% \ekvset{test}{ height = 6 }
% \end{lstlisting}
% and only the usage of the key, not its definition, is benchmarked. For the
% impatient, the essence of these comparisons regarding speed and buggy
% behaviour is contained in \autoref{tab:comp}.
%
% As far as I know \expkv\ is the only fully expandable \kv\ parser. I tried to
% compare \expkv\ to every \kv\ package listed on
% \href{https://ctan.org/topic/keyval}{CTAN}, however, one might notice that
% some of those are missing from this list. That's because I didn't get the
% others to work due to bugs, or because they just provide wrappers around other
% packages in this list.
%
% In this subsubsection is no benchmark of |\ekvparse| and |\keyval_parse:NNn|
% contained, as most other packages don't provide equivalent features to my
% knowledge. |\ekvparse| is slightly faster than |\ekvset|, but keep in mind
% that it does less. The same is true for |\keyval_parse:NNn| compared to
% |\keys_set:nn| of \pkg{expl3} (where the difference is much bigger).
%
% \paragraph{\pkg{keyval}} is about two times faster and has a comparable
% feature set just a slightly different way how it handles keys without values.
% That might be considered a drawback, as it limits the versatility, but also as
% an advantage, as it might reduce doubled code. Keep in mind that as soon as
% someone loads \pkg{xkeyval} the performance of \pkg{keyval} gets replaced by
% \pkg{xkeyval}'s.
%
% Also \pkg{keyval} has a bug, which unfortunately can't really be resolved
% without breaking backwards compatibility for \emph{many} documents, namely it
% strips braces from the argument before stripping spaces if the argument isn't
% surrounded by spaces, also it might strip more than one set of braces. Hence
% all of the following are equivalent in their outcome, though the last two
% lines should result in something different than the first two:
% \begin{lstlisting}[belowskip=0pt]
% \setkeys{foo}{bar=baz}
% \setkeys{foo}{bar= {baz}}
% \setkeys{foo}{bar={ baz}}
% \setkeys{foo}{bar={{baz}}}
% \end{lstlisting}
%
% \paragraph{\pkg{xkeyval}} is roughly fourteen times slower, but it provides
% more functionality, e.g., it has choice keys, boolean keys, and so on. It
% contains the same bug as \pkg{keyval} as it has to be compatible with it by
% design (it replaces \pkg{keyval}'s frontend), but also adds even more cases in
% which braces are stripped that shouldn't be stripped, worsening the situation.
%
% \paragraph{\pkg{ltxkeys}} is over 300 times slower -- which is funny, because
% it aims to be ``[\ldots] faster than these earlier packages [referring to
% \pkg{keyval} and \pkg{xkeyval}].'' Since it aims to have a bigger feature set
% than \pkg{xkeyval}, it most definitely also has a bigger feature set than
% \expkv. Also, it can't parse |\long| input, so as soon as your values contain
% a |\par|, it'll throw errors. Furthermore, \pkg{ltxkeys} doesn't strip outer
% braces at all by design, which, imho, is a weird design choice. In addition
% \pkg{ltxkeys} loads \pkg{catoptions} which is known to introduce bugs (e.g.,
% see \url{https://tex.stackexchange.com/questions/461783}).
%
% \paragraph{\pkg{l3keys}} is almost five times slower, but has an, imho, great
% interface to define keys. It strips \emph{all} outer spaces, even if somehow
% multiple spaces ended up on either end. It offers more features, but is pretty
% much bound to \pkg{expl3} code. Whether that's a drawback is up to you.
%
% \paragraph{\pkg{pgfkeys}} is a bit more than two times slower for one key,
% but has an \emph{enormous} feature set. However, since adding additional keys
% doesn't add as much needed time for \pkg{pgfkeys} compared to \expkv, it gets
% faster than \expkv\ at around eight \kv\ pairs.
% It has the same or a very similar bug \pkg{keyval} has. The brace bug (and
% also the category fragility) can be fixed by \pkg{pgfkeyx}, but this package
% was last updated in 2012 and it slows down |\pgfkeys| by factor~$8$. Also I
% don't know whether this might introduce new bugs.
%
% \paragraph{\pkg{kvsetkeys} with \pkg{kvdefinekeys}} is about three times
% slower, but it works even if commas and equals have category codes different
% from 12 (just as some other packages in this list). Else the features of the
% keys are equal to those of \pkg{keyval}, the parser has more features, though.
%
% \paragraph{\pkg{options}} is a bit slower for only a single value, but gets a
% tad faster than \expkv\ at around 10 \kv\ pairs. It has a much bigger feature
% set. Unfortunately it also suffers from the premature unbracing bug
% \pkg{keyval} has.
%
% \paragraph{\pkg{simplekv}} is hard to compare because I don't speak French (so
% I don't understand the documentation) and from what I can see, there is no
% direct way to define the equivalent test key. Nevertheless, I tested the
% closest possible equivalent of my test key while siding for \pkg{simplekv}'s
% design not forcing something into it it doesn't seem to be designed for. It is
% more than five times slower and has hard to predict behaviour regarding brace
% and space stripping, similar to \pkg{keyval}. The tested definition was:
% \begin{lstlisting}[belowskip=0pt]
% \usepackage{simplekv}
% \setKVdefault[simplekv]{height={ abc}}                     % key setup
% \setKV[simplekv]{ height = 6 }                             % benchmarked
% \end{lstlisting}
%
% \paragraph{\pkg{yax}} is over eighteen times slower. It has a pretty strange
% syntax, imho, and again a direct equivalent is hard to define. It has the
% premature unbracing bug, too. Also somehow loading \pkg{yax} broke
% \pkg{options} for me. The tested definition was:
% \begin{lstlisting}[belowskip=0pt]
% \usepackage{yax}
% \defactiveparameter yax {\storevalue\myheight yax:height } % key setup
% \setparameterlist{yax}{ height = 6 }                       % benchmarked
% \end{lstlisting}
%
% \begin{table}
%   \def\fnsym{\textcolor{red!80!black}{*}}%
%   \sisetup{round-precision=1, round-mode=places}%
%   \begingroup
%   \centering
%   \newcommand*\yes{\textcolor{red!80!black}  {yes}}^^A
%   \newcommand*\no {\textcolor{green!80!black}{no}}^^A
%   \caption[Comparison of \kv\ packages]
%     {^^A
%       Comparison of \kv\ packages. The packages are ordered from
%       fastest to slowest for one \kv\ pair. Benchmarking was done using
%       \pkg{l3benchmark} and the scripts in the \file{Benchmarks} folder
%       of the \href{https://github.com/Skillmon/tex_expkv}{git repository}.
%       The columns $p_i$ are the polynomial coefficients of a linear fit to the
%       run-time, $p_0$ can be interpreted as the overhead for initialisation
%       and $p_1$ the cost per key. The $T_0$ column is the actual mean ops
%       needed for an empty list argument, as the linear fit doesn't match that
%       point well in general. The column ``BB'' lists whether the
%       parsing is affected by some sort of brace bug, ``CF'' stands for
%       category code fragile and lists whether the parsing breaks with active
%       commas or equal signs.^^A
%       \label{tab:comp}^^A
%     }
%   \begin{tabular}
%       {>{\collectcell\pkg}l<{\endcollectcell}*3{S[table-format=4.1]}ccc}
%     \toprule
%     \rmfamily Package & {$p_1$} & {$p_0$} & {$T_0$}& BB & CF & Date \\
%     \midrule
%     keyval    &   13.539 &    1.844 &    6.969 & \yes & \yes & 2014-10-28 \\
%     \expkv    &   26.411 &    2.391 &   12.610 & \no  & \no  & 2020-02-22 \\
%     options   &   25.532 &   10.518 &   21.393 & \yes & \yes & 2015-03-01 \\
%     pgfkeys   &   26.444 &   40.939 &   55.790 & \yes & \yes & 2020-01-08 \\
%     kvsetkeys & {\fnsym} & {\fnsym} &   41.620 & \no  & \no  & 2019-12-15 \\
%     l3keys    &  114.456 &   35.357 &   52.667 & \no  & \no  & 2020-02-14 \\
%     simplekv  &  160.523 &   10.809 &    8.698 & \yes & \yes & 2017-08-08 \\
%     xkeyval   &  260.872 &  180.435 &  164.167 & \yes & \yes & 2014-12-03 \\
%     yax       &  507.643 &   62.863 &  123.533 & \yes & \yes & 2010-01-22 \\
%     ltxkeys   & 3932.759 & 4737.818 & 5883.000 & \no  & \no  & 2012-11-17 \\
%     \bottomrule
%   \end{tabular}
%   \par
%   \endgroup
%   \medskip
%   \fnsym For \pkg{kvsetkeys} the linear model used for the other
%   packages is a poor fit, \pkg{kvsetkeys} seems to have approximately
%   quadratic run-time, the coefficients of the second degree polynomial fit are
%   $p_2=\num{9.911}$, $p_1=\num{40.523}$, and $p_0=\num{61.500}$. Of course the
%   other packages might not really have linear run-time, but at least from 1~to
%   20~keys the fits don't seem too bad (the maximum ratio $p_2/p_1$
%   for the other packages is \num{9.697e-3}). If one extrapolates the fits
%   for 100 \kv\ pairs one finds that most of them match pretty well, the
%   exception being \pkg{ltxkeys}, which behaves quadratic as well with
%   $p_2=\num{31.685}$, $p_1=\num{3267.368}$, and $p_0=\num{7177.582}$.
% \end{table}
%
% \subsection{Examples}
%
% \subsubsection{Standard Use-Case}
%
% Say we have a macro for which we want to create a \kv\ interface. The macro
% has a parameter, which is stored in the dimension |\ourdim| having a default
% value from its initialization. Now we want to be able to change that dimension
% with the |width| key to some specified value. For that we'd do
% \begin{lstlisting}
% \newdimen\ourdim
% \ourdim=150pt
% \protected\ekvdef{our}{width}{\ourdim=#1\relax}
% \end{lstlisting}
% as you can see, we use the set |our| here. We want the key to behave different
% if no value is specified. In that case the key should not use its initial
% value, but be smart and determine the available space from
% \cs[no-index]{hsize}, so we also define
% \begin{lstlisting}
% \protected\ekvdefNoVal{our}{width}{\ourdim=.9\hsize}
% \end{lstlisting}
% Now we set up our macro to use this \kv\ interface
% \begin{lstlisting}
% \protected\def\ourmacro#1{\begingroup\ekvset{our}{#1}\the\ourdim\endgroup}
% \end{lstlisting}
% Finally we can use our macro like in the following\\
% \begin{minipage}{6cm}
% \begin{lstlisting}
% \ourmacro{}\par
% \ourmacro{width}\par
% \ourmacro{width=5pt}\par
% \end{lstlisting}
% \end{minipage}^^A
% \begin{minipage}{\dimexpr\linewidth-6cm\relax}
%   \newdimen\ourdim\ourdim=150pt
%   \protected\ekvdef{our}{width}{\ourdim=#1\relax}^^A
%   \protected\ekvdefNoVal{our}{width}{\ourdim=.9\hsize}^^A
%   \protected\def\ourmacro#1^^A
%     {\begingroup\ekvset{our}{#1}\the\ourdim\endgroup}^^A
%   \ourmacro{}\par
%   \ourmacro{width}\par
%   \ourmacro{width=5pt}\par
% \end{minipage}
%
% \paragraph{The same key using \protect\expkvd}
% Using \expkvd\ we can set up the equivalent key using a \kv\ interface, after
% the following we could use |\ourmacro| in the same way as above. \expkvd\ will
% allocate and initialise |\ourdim| and define the |width| key |\protected| for
% us, so the result will be exactly the same -- with the exception that the
% default will use |\ourdim=.9\hsize\relax| instead.
% \begin{lstlisting}
% \input expkv-def                 % or \usepackage{expkv-def}
% \ekvdefinekeys{our}
%   {
%     dimen    width = \ourdim,
%     qdefault width = .9\hsize,
%     initial  width = 150pt
%   }
% \end{lstlisting}
%
% \subsubsection{An Expandable \kv\ Macro Using \cs[no-index]{ekvsneak}}
% \label{sec:sneakex}
%
% Let's set up an expandable macro, that uses a \kv\ interface. The problems
% we'll face for this are:
% \begin{enumerate}
%   \item ignoring duplicate keys
%   \item default values for keys which weren't used
%   \item providing the values as the correct argument to a macro (ordered)
% \end{enumerate}
%
% First we need to decide which \kv\ parsing macro we want to do this with,
% |\ekvset| or |\ekvparse|. For this example we also want to show the
% usage of |\ekvsneak|, hence we'll choose |\ekvset|. And we'll have to use
% |\ekvset| such that it builds a parsable list for our macro internals. To
% gain back control after |\ekvset| is done we have to put an internal of our
% macro at the start of that list, so we use an internal key that uses
% |\ekvsneakPre| after any user input.
%
% To ignore duplicates will be easy if the value of the key used last will be
% put first in the list, so the following will use |\ekvsneakPre| for the
% user-level keys. If we wanted some key for which the first usage should be the
% the binding one we would use |\ekvsneak| instead for that key.
%
% Providing default values can be done in different ways, we'll use a simple
% approach in which we'll just put the outcome of our keys if they were used
% with default values before the parsing list terminator.
%
% Ordering the keys can be done simply by searching for a specific token for
% each argument which acts like a flag, so our sneaked out values will include
% specific tokens acting as markers.
%
% Now that we have answers for our technical problems, we have to decide what
% our example macro should do. How about we define a macro that calculates the
% sine of a number and rounds that to a specified precision? As a small extra
% this macro should understand input in radian and degree and the used
% trigonometric function should be selectable as well. For the hard part of this
% task (expandably evaluating trigonometric functions) we'll use the \pkg{xfp}
% package.
%
% First we set up our keys according to our earlier considerations and set up
% the user facing macro |\sine|. The end marker of the parsing list will be a
% |\sine@stop| token, which we don't need to define and we put our defaults
% right before it.
%
% \begin{lstlisting}
% \RequirePackage{xfp}
% \makeatletter
% \ekvdef{expex}{f}{\ekvsneakPre{\f{#1}}}
% \ekvdef{expex}{round}{\ekvsneakPre{\rnd{#1}}}
% \ekvdefNoVal{expex}{degree}{\ekvsneakPre{\deg{d}}}
% \ekvdefNoVal{expex}{radian}{\ekvsneakPre{\deg{}}}
% \ekvdefNoVal{expex}{internal}{\ekvsneakPre{\sine@rnd}}
% \newcommand*\sine[2]
%   {\ekvset{expex}{#1,internal}\rnd{3}\deg{d}\f{sin}\sine@stop{#2}}
% \end{lstlisting}
% For the sake of simplicity we defined the macro |\sine| with two mandatory
% arguments, the first being the \kv\ list, the second the argument to the
% trigonometric function. We could've used \pkg{xparse}'s facilities here to
% define an expandable macro which takes an optional argument instead.
%
% Now we need to define some internal macros to extract the value of each key's
% last usage (remember that this will be the group after the first special
% flag-token). For that we use one delimited macro per key.
% \begin{lstlisting}
% \def\sine@rnd#1\rnd#2#3\sine@stop{\sine@deg#1#3\sine@stop{#2}}
% \def\sine@deg#1\deg#2#3\sine@stop{\sine@f#1#3\sine@stop{#2}}
% \def\sine@f#1\f#2#3\sine@stop{\sine@final{#2}}
% \end{lstlisting}
% After the macros |\sine@rnd|, |\sine@deg|, and |\sine@f| the macro
% |\sine@final| will see
% \texttt
%   {^^A
%     \cs[no-index]{sine@final}^^A
%     \marg{f}\marg{degree/radian}\marg{round}\marg{num}^^A
%   }.
% Now |\sine@final| has to expandably deal with those arguments such that the
% |\fpeval| macro of \pkg{xfp} gets the correct input. Luckily this is pretty
% straight forward in this example. In |\fpeval| the trigonometric functions
% have names such as |sin| or |cos| and the degree taking variants |sind| or
% |cosd|. And since the |degree| key puts a |d| in |#2| and the |radian| key
% leaves |#2| empty all we have to do to get the correct function name is stick
% the two together.
% \begin{lstlisting}
% \newcommand*\sine@final[4]{\fpeval{round(#1#2(#4),#3)}}
% \makeatother
% \end{lstlisting}
% Let's test our macro:\\
% \begin{minipage}{.7\linewidth}
%   \begin{lstlisting}[gobble=4]
%   \sine{}{60}\par
%   \sine{round=10}{60}\par
%   \sine{f=cos,radian}{pi}\par
%   \edef\myval{\sine{f=tan}{1}}\texttt{\meaning\myval}
%   \end{lstlisting}
% \end{minipage}\hfill
% \begin{minipage}{.2\linewidth}
%   \makeatletter
%   \ekvdef{expex}{f}{\ekvsneakPre{\f{#1}}}^^A
%   \ekvdef{expex}{round}{\ekvsneakPre{\rnd{#1}}}^^A
%   \ekvdefNoVal{expex}{degree}{\ekvsneakPre{\deg{d}}}^^A
%   \ekvdefNoVal{expex}{radian}{\ekvsneakPre{\deg{}}}^^A
%   \ekvdefNoVal{expex}{internal}{\ekvsneakPre{\sine@rnd}}^^A
%   \newcommand*\sine[2]
%     {\ekvset{expex}{#1,internal}\rnd{3}\deg{d}\f{sin}\sine@stop{#2}}^^A
%   \def\sine@rnd#1\rnd#2#3\sine@stop{\sine@deg#1#3\sine@stop{#2}}^^A
%   \def\sine@deg#1\deg#2#3\sine@stop{\sine@f#1#3\sine@stop{#2}}^^A
%   \def\sine@f#1\f#2#3\sine@stop{\sine@final{#2}}^^A
%   \newcommand*\sine@final[4]{\fpeval{round(#1#2(#4),#3)}}^^A
%   \makeatother
%   \sine{}{60}\par
%   \sine{round=10}{60}\par
%   \sine{f=cos,radian}{pi}\par
%   \edef\myval{\sine{f=tan}{1}}\texttt{\meaning\myval}
% \end{minipage}
%
%
% \subsection{Error Messages}
%
% \expkv\ should only send messages in case of errors, there are no warnings and
% no info messages. In this subsection those errors are listed.
%
% \subsubsection{Load Time}
%
% \file{expkv.tex} checks whether \eTeX\ is available. If it isn't, an error
% will be thrown using |\errmessage|:
% \begin{lstlisting}
% ! expkv Error: e-TeX required.
% \end{lstlisting}
%
% \subsubsection{Defining Keys}
%
% If you get any error from \expkv\ while you're trying to define a key, the
% definition will be aborted and gobbled.
%
% If you try to define a key with an empty set name you'll get:
% \begin{lstlisting}
% ! expkv Error: empty set name not allowed.
% \end{lstlisting}
%
% Similarly, if you try to define a key with an empty key name:
% \begin{lstlisting}
% ! expkv Error: empty key name not allowed.
% \end{lstlisting}
%
% Both of these messages are done in a way that doesn't throw additional errors
% due to |\global|, |\long|, etc., not being used correctly if you prefixed one
% of the defining macros.
%
% \subsubsection{Using Keys}
%
% This subsubsection contains the errors thrown during |\ekvset|. The errors are
% thrown in an expandable manner by providing an undefined macro. In the
% following messages \lstinline|<key>| gets replaced with the problematic key's
% name, and \lstinline|<set>| with the corresponding set. If any errors during
% \kv\ handling are encountered, the entry in the comma separated list will be
% omitted after the error is thrown and the next \kv\ pair will be parsed.
%
% If you're using an undefined key you'll get:
% \begin{lstlisting}
% ! Undefined control sequence.
% <argument> \! expkv Error:
%                            unknown key (`<key>', set `<set>').
% \end{lstlisting}
%
% If you're using a key for which only a normal version and no |NoVal| version
% is defined, but don't provide a value, you'll get:
% \begin{lstlisting}
% ! Undefined control sequence.
% <argument> \! expkv Error:
%                            value required (`<key>', set `<set>').
% \end{lstlisting}
%
% If you're using a key for which only a |NoVal| version and no normal version
% is defined, but provide a value, you'll get:
% \begin{lstlisting}
% ! Undefined control sequence.
% <argument> \! expkv Error:
%                            value forbidden (`<key>', set `<set>').
% \end{lstlisting}
%
% If you're using a set for which you never executed one of the defining macros
% from \autoref{sec:define} you'll get a low level \TeX\ error, as that isn't
% actively tested by the parser (and hence will lead to undefined behaviour and
% not be gracefully ignored). The error will look like
% \begin{lstlisting}
% ! Missing \endcsname inserted.
% <to be read again>
%                    \! expkv Error: Set `<set>' undefined.
% \end{lstlisting}
%
% \subsection{License}
%
% Copyright \textcopyright\ 2020\unless\ifnum\year=2020--\the\year\fi\
% Jonathan P. Spratte
%
% \medskip\noindent
% This work may be distributed and/or modified under the conditions of the
% \LaTeX\ Project Public License (LPPL), either version 1.3c of this license or
% (at your option) any later version. The latest version of this license is in
% the file:
%
% \url{http://www.latex-project.org/lppl.txt}
%
% \noindent
% This work is ``maintained'' (as per LPPL maintenance status) by
%
% \mbox{Jonathan P. Spratte}.
%
% \end{documentation}^^A=<<
%
% \begin{implementation}^^A>>=
%
% \clearpage
%
% \section{Implementation}^^A>>=
%^^A the LaTeX package >>=
% \subsection{The \LaTeX\ Package}
% First we set up the \LaTeX\ package. That one doesn't really do much except
% |\input|ting the generic code and identifying itself as a package.
% \gobbledocstriptag
%<*pkg>
%    \begin{macrocode}
\def\ekv@tmp
  {%
    \ProvidesFile{expkv.tex}%
      [\ekvDate\space v\ekvVersion\space an expandable key=val implementation]%
  }
\input{expkv.tex}
\ProvidesPackage{expkv}%
  [\ekvDate\space v\ekvVersion\space an expandable key=val implementation]
%    \end{macrocode}
% \gobbledocstriptag
%</pkg>
%^^A=<<
%^^A main file >>=
% \subsection{The Generic Code}
% The rest of this implementation will be the generic code.
% \gobbledocstriptag
%<*tex>
%
% Check whether \eTeX\ is available -- \expkv\ requires \eTeX.
%    \begin{macrocode}
\begingroup\expandafter\expandafter\expandafter\endgroup
\expandafter\ifx\csname numexpr\endcsname\relax
  \errmessage{expkv requires e-TeX}
  \expandafter\endinput
\fi
%    \end{macrocode}
%
% We make sure that it's only input once:
%    \begin{macrocode}
\expandafter\ifx\csname ekvVersion\endcsname\relax
\else
  \expandafter\endinput
\fi
%    \end{macrocode}
% \begin{macro}{\ekvVersion,\ekvDate}
% We're on our first input, so lets store the version and date in a macro.
%    \begin{macrocode}
\def\ekvVersion{0.5a}
\def\ekvDate{2020-02-27}
%    \end{macrocode}
% \end{macro}
%
% If the \LaTeX\ format is loaded we want to be a good file and report back who
% we are, for this the package will have defined |\ekv@tmp| to use
% |\ProvidesFile|, else this will expand to a |\relax| and do no harm.
%    \begin{macrocode}
\csname ekv@tmp\endcsname
%    \end{macrocode}
%
% Store the category code of |@| to later be able to reset it and change it to
% 11 for now.
%    \begin{macrocode}
\expandafter\chardef\csname ekv@tmp\endcsname=\catcode`\@
\catcode`\@=11
%    \end{macrocode}
% |\ekv@tmp| might later be reused to gobble any prefixes which might be
% provided to |\ekvdef| and similar in case the names are invalid, we just
% temporarily use it here as means to store the current category code of |@| to
% restore it at the end of the file, we never care for the actual definition of
% it.
%
% \begin{macro}[internal]
%   {
%     \@gobble,\@firstofone,\@firstoftwo,\@secondoftwo,
%     \ekv@gobbleto@stop,\ekv@fi@secondoftwo,\ekv@gobble@mark
%   }
% Since branching tests are often more versatile than |\if...\else...\fi|
% constructs, we define helpers that are branching pretty fast. Also here are
% some other utility functions that just grab some tokens. The ones that are
% also contained in \LaTeX\ don't use the |ekv| prefix.
%    \begin{macrocode}
\long\def\@gobble#1{}
\long\def\@firstofone#1{#1}
\long\def\@firstoftwo#1#2{#1}
\long\def\@secondoftwo#1#2{#2}
\long\def\ekv@fi@secondoftwo\fi\@firstoftwo#1#2{\fi#2}
\long\def\ekv@gobbleto@stop#1\ekv@stop{}
\def\ekv@gobble@mark\ekv@mark{}
%    \end{macrocode}
% \end{macro}
% As you can see |\ekv@gobbleto@stop| uses a special marker |\ekv@stop|. The
% package will use three such markers, the one you've seen already, |\ekv@mark|
% and |\ekv@nil|. Contrarily to how for instance \pkg{expl3} does things, we
% don't define them, as we don't need them to have an actual meaning. This has
% the advantage that if they somehow get expanded -- which should never happen
% if things work out -- they'll throw an error directly.
%
% \begin{macro}[internal]
%   {
%     \ekv@ifempty,\ekv@ifempty@,\ekv@ifempty@true,\ekv@ifempty@false,
%     \ekv@ifempty@true@F
%   }
% We can test for a lot of things building on an if-empty test, so lets define a
% really fast one. Since some tests might have reversed logic (true if something
% is not empty) we also set up macros for the reversed branches.
%    \begin{macrocode}
\long\def\ekv@ifempty#1%
  {%
    \ekv@ifempty@\ekv@ifempty@A#1\ekv@ifempty@B\ekv@ifempty@true
      \ekv@ifempty@A\ekv@ifempty@B\@secondoftwo
  }
\long\def\ekv@ifempty@#1\ekv@ifempty@A\ekv@ifempty@B{}
\long\def\ekv@ifempty@true\ekv@ifempty@A\ekv@ifempty@B\@secondoftwo#1#2{#1}
\long\def\ekv@ifempty@false\ekv@ifempty@A\ekv@ifempty@B\@firstoftwo#1#2{#2}
\long\def\ekv@ifempty@true@F\ekv@ifempty@A\ekv@ifempty@B\@firstofone#1{}
%    \end{macrocode}
% \end{macro}
%
% \begin{macro}[internal]{\ekv@ifblank@}
% The obvious test that can be based on an if-empty is if-blank, meaning a test
% checking whether the argument is empty or consists only of spaces. Our version
% here will be tweaked a bit, as we want to check this, but with one leading
% |\ekv@mark| token that is to be ignored.
%    \begin{macrocode}
\long\def\ekv@ifblank@\ekv@mark#1{\ekv@ifempty@\ekv@ifempty@A}
%    \end{macrocode}
% \end{macro}
%
% \begin{macro}[internal]{\ekv@ifdefined}
% We'll need to check whether something is defined quite frequently, so why not
% define a macro that does this. The following test is expandable, slower than
% the typical expandable test for undefined control sequences, but faster for
% defined ones. Since we want to be as fast as possible for correct input, this
% is to be preferred.
%    \begin{macrocode}
\def\ekv@ifdefined#1%
  {%
    \expandafter
    \ifx\csname\ifcsname #1\endcsname #1\else relax\fi\endcsname\relax
      \ekv@fi@secondoftwo
    \fi
    \@firstoftwo
  }
%    \end{macrocode}
% \end{macro}
  % \begin{macro}[internal]
%   {
%     \ekv@ifdefined@pair,\ekv@ifdefined@pair@,\ekv@ifdefined@key,
%     \ekv@ifdefined@key@
%   }
% Since we can save some time if we only have to create the control sequence
% once when we know beforehand how we want to use it, we build some other macros
% for those cases (which we'll have quite often, once per key usage).
%    \begin{macrocode}
\def\ekv@ifdefined@pair#1#2%
  {%
    \expandafter\ekv@ifdefined@pair@
      \csname
        \ifcsname #1{#2}\endcsname
          #1{#2}%
        \else
          relax%
        \fi
      \endcsname
  }
\def\ekv@ifdefined@pair@#1%
  {%
    \ifx#1\relax
      \ekv@fi@secondoftwo
    \fi
    \@firstoftwo
    {\ekv@set@pair@#1\ekv@mark}%
  }
\def\ekv@ifdefined@key#1#2%
  {%
    \expandafter\ekv@ifdefined@key@
      \csname
        \ifcsname #1{#2}N\endcsname
          #1{#2}N%
        \else
          relax%
        \fi
      \endcsname
  }
\def\ekv@ifdefined@key@#1%
  {%
    \ifx#1\relax
      \ekv@fi@secondoftwo
    \fi
    \@firstoftwo#1%
  }
%    \end{macrocode}
% \end{macro}
%
% \begin{macro}{\ekv@name,\ekv@name@set,\ekv@name@key}
% The keys will all follow the same naming scheme, so we define it here.
%    \begin{macrocode}
\def\ekv@name#1#2{\ekv@name@set{#1}\ekv@name@key{#2}}
\def\ekv@name@set#1{ekv#1(}
\def\ekv@name@key#1{#1)}
%    \end{macrocode}
% \end{macro}
% \begin{macro}[internal]{\ekv@undefined@set}
% We can misuse the macro name we use to expandably store the set-name in a
% single token -- since this increases performance drastically, especially for
% long set-names -- to throw a more meaningful error message in case a set isn't
% defined. The name of |\ekv@undefined@set| is a little bit misleading, as it is
% called in either case inside of |\csname|, but the result will be a control
% sequence with meaning |\relax| if the set is undefined, hence will break the
% |\csname| building the key-macro which will throw the error message.
%    \begin{macrocode}
\def\ekv@undefined@set#1{! expkv Error: Set `#1' undefined.}
%    \end{macrocode}
% \end{macro}
% \begin{macro}[internal]{\ekv@checkvalid}
% We place some restrictions on the allowed names, though, namely sets and
% keys are not allowed to be empty -- blanks are fine (meaning \mbox{set-
% or} key-names consisting of spaces).
%    \begin{macrocode}
\protected\def\ekv@checkvalid#1#2%
  {%
    \ekv@ifempty{#1}%
      {%
        \def\ekv@tmp{}%
        \errmessage{expkv Error: empty set name not allowed}%
      }%
      {%
        \ekv@ifempty{#2}%
          {%
            \def\ekv@tmp{}%
            \errmessage{expkv Error: empty key name not allowed}%
          }%
          \@secondoftwo
      }%
    \@gobble
  }
%    \end{macrocode}
% \end{macro}
% \begin{macro}{\ekvifdefined,\ekvifdefinedNoVal}
% And provide user-level macros to test whether a key is defined.
%    \begin{macrocode}
\def\ekvifdefined#1#2{\ekv@ifdefined{\ekv@name{#1}{#2}}}
\def\ekvifdefinedNoVal#1#2{\ekv@ifdefined{\ekv@name{#1}{#2}N}}
%    \end{macrocode}
% \end{macro}
%
% \begin{macro}
%   {\ekvdef,\ekvdefNoVal,\ekvlet,\ekvletNoVal,\ekvletkv,\ekvletkvNoVal}
% Set up the key defining macros |\ekvdef| etc.
%    \begin{macrocode}
\protected\long\def\ekvdef#1#2#3%
  {%
    \ekv@checkvalid{#1}{#2}%
      {%
        \expandafter\def\csname\ekv@name{#1}{#2}\endcsname##1{#3}%
        \ekv@defset{#1}%
      }%
  }
\protected\long\def\ekvdefNoVal#1#2#3%
  {%
    \ekv@checkvalid{#1}{#2}%
      {%
        \expandafter\def\csname\ekv@name{#1}{#2}N\endcsname{#3}%
        \ekv@defset{#1}%
      }%
  }
\protected\def\ekvlet#1#2#3%
  {%
    \ekv@checkvalid{#1}{#2}%
      {%
        \expandafter\let\csname\ekv@name{#1}{#2}\endcsname#3%
        \ekv@defset{#1}%
      }%
  }
\protected\def\ekvletNoVal#1#2#3%
  {%
    \ekv@checkvalid{#1}{#2}%
      {%
        \expandafter\let\csname\ekv@name{#1}{#2}N\endcsname#3%
        \ekv@defset{#1}%
      }%
  }
\protected\def\ekvletkv#1#2#3#4%
  {%
    \ekv@checkvalid{#1}{#2}%
      {%
        \expandafter\let\csname\ekv@name{#1}{#2}\expandafter\endcsname
        \csname\ekv@name{#3}{#4}\endcsname
        \ekv@defset{#1}%
      }%
  }
\protected\def\ekvletkvNoVal#1#2#3#4%
  {%
    \ekv@checkvalid{#1}{#2}%
      {%
        \expandafter\let\csname\ekv@name{#1}{#2}N\expandafter\endcsname
        \csname\ekv@name{#3}{#4}N\endcsname
        \ekv@defset{#1}%
      }%
  }
%    \end{macrocode}
% \end{macro}
% \begin{macro}[internal]{\ekv@defset}
%   In order to enhance the speed the set name given to |\ekvset| will be turned
%   into a control sequence pretty early, so we have to define that control
%   sequence.
%    \begin{macrocode}
\protected\def\ekv@defset#1%
  {%
    \expandafter\edef\csname\ekv@undefined@set{#1}\endcsname##1%
      {\ekv@name@set{#1}\ekv@name@key{##1}}%
  }
%    \end{macrocode}
% \end{macro}
%
% \begin{macro}{\ekvset}
% Set up |\ekvset|, which should not be affected by active commas and equal
% signs. The equal signs are a bit harder to cope with and we'll do that later,
% but replacing the active commas with commas of category other can be done
% beforehand. That's why we define |\ekvset| here with a temporary meaning just
% to set up the things with two different category codes. |#1| will be a
% \texttt{,\textsubscript{13}} and |#2| will be a \texttt{=\textsubscript{13}}.
%    \begin{macrocode}
\def\ekvset#1#2{%
\endgroup
\long\def\ekvset##1##2%
  {%
    \expandafter\ekv@set\csname\ekv@undefined@set{##1}\endcsname
      \ekv@mark##2#1\ekv@stop#1{}%
  }
%    \end{macrocode}
% \end{macro}
% \begin{macro}[internal]{\ekv@set}
% |\ekv@set| will split the \kv\ list at active commas. Then it has to check
% whether there were unprotected other commas and resplit there.
%    \begin{macrocode}
\long\def\ekv@set##1##2#1%
  {%
%    \end{macrocode}
% Test whether we're at the end, if so invoke |\ekv@endset|,
%    \begin{macrocode}
    \ekv@gobbleto@markstop##2\ekv@endset\ekv@mark\ekv@stop
%    \end{macrocode}
% else go on with other commas,
%    \begin{macrocode}
    \ekv@set@other##1##2,\ekv@stop,%
%    \end{macrocode}
% and get the next active comma delimited \kv\ pair.
%    \begin{macrocode}
    \ekv@set##1\ekv@mark
  }
%    \end{macrocode}
% \end{macro}
% \begin{macro}[internal]{\ekv@endset}
% |\ekv@endset| is a hungry little macro. It will eat everything that remains
% of |\ekv@set| and unbrace the sneaked stuff.
%    \begin{macrocode}
\long\def\ekv@endset
    \ekv@mark\ekv@stop\ekv@set@other##1,\ekv@stop,\ekv@set##2\ekv@mark
    ##3%
  {##3}
%    \end{macrocode}
% \end{macro}
% \begin{macro}[internal]{\ekv@set@other}
% The macro |\ekv@set@other| is guaranteed to get only single \kv\ pairs.
% First we test whether we're done, if not split at equal signs. It is faster to
% first split at category 12 equal signs and only after that on actives.
% If there is no equal sign, we need to test whether we got a blank argument and
% if not this is a |NoVal| key.
%    \begin{macrocode}
\long\def\ekv@set@other##1##2,%
  {%
    \ekv@gobbleto@markstop##2\ekv@endset@other\ekv@mark\ekv@stop
    \ekv@ifhas@eq@other##2=\ekv@ifempty@B\ekv@ifempty@false
      \ekv@ifempty@A\ekv@ifempty@B\@firstoftwo
      {\ekv@set@eq@other##1##2\ekv@stop}%
      {%
        \ekv@ifhas@eq@active##2#2\ekv@ifempty@B\ekv@ifempty@false
          \ekv@ifempty@A\ekv@ifempty@B\@firstoftwo
          {\ekv@set@eq@active##1##2\ekv@stop}%
          {%
            \ekv@ifblank@##2\ekv@nil\ekv@ifempty@B\ekv@ifempty@true@F
              \ekv@ifempty@A\ekv@ifempty@B\@firstofone
              {\ekv@strip{##2}\ekv@set@key##1}%
          }%
      }%
    \ekv@set@other##1\ekv@mark%
  }
%    \end{macrocode}
% \end{macro}
% \begin{macro}[internal]{\ekv@set@eq@other}
% |\ekv@set@eq@other| might not be the correct break point, there might be an
% active equal sign in the currently parsed key-name. If so, we have to resplit.
% If the split is correct strip the key-name of outer spaces and braces and feed
% it to |\ekv@set@pair|.
%    \begin{macrocode}
\long\def\ekv@set@eq@other##1##2=%
  {%
    \ekv@ifhas@eq@active##2#2\ekv@ifempty@B\ekv@ifempty@false
      \ekv@ifempty@A\ekv@ifempty@B\@firstoftwo
      {\ekv@set@eq@active##1##2=}%
      {\ekv@strip{##2}\ekv@set@pair##1}%
  }
%    \end{macrocode}
% \end{macro}
% \begin{macro}[internal]{\ekv@set@eq@active}
% We need to handle the active equal signs.
%    \begin{macrocode}
\long\def\ekv@set@eq@active##1##2#2%
  {%
    \ekv@strip{##2}\ekv@set@pair##1%
  }
%    \end{macrocode}
% \end{macro}
% \begin{macro}[internal]
%   {\ekv@ifhas@eq@other,\ekv@ifhas@eq@active,\ekv@endset@other}
% And we have to set up the testing macros for our equal signs and
% |\ekv@endset@other|.
%    \begin{macrocode}
\long\def\ekv@ifhas@eq@other\ekv@mark##1={\ekv@ifempty@\ekv@ifempty@A}
\long\def\ekv@ifhas@eq@active\ekv@mark##1#2{\ekv@ifempty@\ekv@ifempty@A}
\long\def\ekv@endset@other
    \ekv@mark\ekv@stop
    \ekv@ifhas@eq@other##1=\ekv@ifempty@B\ekv@ifempty@false
    \ekv@ifempty@A\ekv@ifempty@B\@firstoftwo
    ##2%
    \ekv@set@other##3\ekv@mark
  {}
%    \end{macrocode}
% \end{macro}
% \begin{macro}{\ekvbreak,\ekvbreakPreSneak,\ekvbreakPostSneak}
% Provide macros that can completely stop the parsing of |\ekvset|, who knows
% what it'll be useful for.
%    \begin{macrocode}
\long\def\ekvbreak##1##2\ekv@stop#1##3{##1}
\long\def\ekvbreakPreSneak ##1##2\ekv@stop#1##3{##1##3}
\long\def\ekvbreakPostSneak##1##2\ekv@stop#1##3{##3##1}
%    \end{macrocode}
% \end{macro}
% \begin{macro}{\ekvsneak,\ekvsneakPre}
% One last thing we want to do for |\ekvset| is to provide macros that just
% smuggle stuff after |\ekvset|'s effects.
%    \begin{macrocode}
\long\def\ekvsneak##1##2\ekv@stop#1##3%
  {%
    ##2\ekv@stop#1{##3##1}%
  }
\long\def\ekvsneakPre##1##2\ekv@stop#1##3%
  {%
    ##2\ekv@stop#1{##1##3}%
  }
%    \end{macrocode}
% \end{macro}
%
% \begin{macro}{\ekvparse}
% Additionally to the |\ekvset| macro we also want to provide an |\ekvparse|
% macro, that has the same scope as |\keyval_parse:NNn| from \pkg{expl3}.
% This is pretty analogue to the |\ekvset| implementation, we just put an
% |\unexpanded| here and there instead of other macros to stop the |\expanded|
% on our output.
%    \begin{macrocode}
\long\def\ekvparse##1##2##3%
  {%
    \ekv@parse##1##2\ekv@mark##3#1\ekv@stop#1%
  }
%    \end{macrocode}
% \end{macro}
% \begin{macro}[internal]{\ekv@parse}
%    \begin{macrocode}
\long\def\ekv@parse##1##2##3#1%
  {%
    \ekv@gobbleto@markstop##3\ekv@endparse\ekv@mark\ekv@stop
    \ekv@parse@other##1##2##3,\ekv@stop,%
    \ekv@parse##1##2\ekv@mark
  }
%    \end{macrocode}
% \end{macro}
% \begin{macro}[internal]{\ekv@endparse}
%    \begin{macrocode}
\long\def\ekv@endparse
    \ekv@mark\ekv@stop\ekv@parse@other##1,\ekv@stop,\ekv@parse##2\ekv@mark
  {}
%    \end{macrocode}
% \end{macro}
% \begin{macro}[internal]{\ekv@parse@other}
%    \begin{macrocode}
\long\def\ekv@parse@other##1##2##3,%
  {%
    \ekv@gobbleto@markstop##3\ekv@endparse@other\ekv@mark\ekv@stop
    \ekv@ifhas@eq@other##3=\ekv@ifempty@B\ekv@ifempty@false
      \ekv@ifempty@A\ekv@ifempty@B\@firstoftwo
      {\ekv@parse@eq@other##3\ekv@stop##2}%
      {%
        \ekv@ifhas@eq@active##3#2\ekv@ifempty@B\ekv@ifempty@false
          \ekv@ifempty@A\ekv@ifempty@B\@firstoftwo
          {\ekv@parse@eq@active##3\ekv@stop##2}%
          {%
            \ekv@ifblank@##3\ekv@nil\ekv@ifempty@B\ekv@ifempty@true@F
              \ekv@ifempty@A\ekv@ifempty@B\@firstofone
              {\ekv@strip{##3}\ekv@parse@key##1}%
          }%
      }%
    \ekv@parse@other##1##2\ekv@mark
  }
%    \end{macrocode}
% \end{macro}
% \begin{macro}[internal]{\ekv@parse@eq@other}
%    \begin{macrocode}
\long\def\ekv@parse@eq@other##1=%
  {%
    \ekv@ifhas@eq@active##1#2\ekv@ifempty@B\ekv@ifempty@false
      \ekv@ifempty@A\ekv@ifempty@B\@firstoftwo
      {\ekv@parse@eq@active##1=}%
      {\ekv@strip{##1}\ekv@parse@pair\ekv@mark}%
  }
%    \end{macrocode}
% \end{macro}
% \begin{macro}[internal]{\ekv@parse@eq@active}
%    \begin{macrocode}
\long\def\ekv@parse@eq@active##1#2%
  {%
    \ekv@strip{##1}\ekv@parse@pair\ekv@mark
  }
%    \end{macrocode}
% \end{macro}
% \begin{macro}[internal]{\ekv@endparse@other}
%    \begin{macrocode}
\long\def\ekv@endparse@other
    \ekv@mark\ekv@stop
    \ekv@ifhas@eq@other##1=\ekv@ifempty@B\ekv@ifempty@false
    \ekv@ifempty@A\ekv@ifempty@B\@firstoftwo
    ##2%
    \ekv@parse@other##3\ekv@mark
  {}
%    \end{macrocode}
% \end{macro}
% \begin{macro}[internal]{\ekv@parse@pair,\ekv@parse@pair@}
%    \begin{macrocode}
\long\def\ekv@parse@pair##1##2\ekv@stop
  {%
    \ekv@strip{##2}\ekv@parse@pair@{##1}%
  }
\long\def\ekv@parse@pair@##1##2##3%
  {%
    \unexpanded{##3{##2}{##1}}%
  }
%    \end{macrocode}
% \end{macro}
% \begin{macro}[internal]{\ekv@parse@key}
%    \begin{macrocode}
\long\def\ekv@parse@key##1##2%
  {%
    \unexpanded{##2{##1}}%
  }
%    \end{macrocode}
% \end{macro}
%
% Finally really setting things up with |\ekvset|'s temporary meaning:
%    \begin{macrocode}
}
\begingroup
\catcode`\,=13
\catcode`\==13
\ekvset,=
%    \end{macrocode}
%
% \begin{macro}{\ekvchangeset}
% Provide a macro that is able to switch out the current \set\ in |\ekvset|.
% This operation is slow (by comparison, it should be slightly faster than
% |\ekvsneak|), but allows for something similar to \pkg{pgfkeys}'s
% \texttt{\meta{key}/.cd} mechanism. However this operation is more expensive
% than |/.cd| as we can't just redefine some token to reflect this, but have to
% switch out the set expandably, so this works similar to the |\ekvsneak| macros
% reading and reinserting the remainder of the \kv\ list.
%    \begin{macrocode}
\def\ekvchangeset#1%
  {%
    \expandafter\ekv@changeset\csname\ekv@undefined@set{#1}\endcsname\ekv@mark
  }
%    \end{macrocode}
% \end{macro}
% \begin{macro}[internal]{\ekv@changeset}
% This macro does the real change-out of |\ekvchangeset|. We introduced an
% |\ekv@mark| to not accidentally remove some braces which we have to remove
% again.
%    \begin{macrocode}
\long\def\ekv@changeset#1#2\ekv@set@other#3#4\ekv@set#5%
  {%
    \ekv@gobble@mark#2\ekv@set@other#1#4\ekv@set#1%
  }
%    \end{macrocode}
% \end{macro}
%
%
% \begin{macro}[internal]{\ekv@gobbleto@markstop}
% The |\ekv@gobbleto@markstop| can be used to test for |\ekv@stop| similar to
% our if-empty test, but instead of using tokens which are used nowhere else
% (|\ekv@ifempty@A| and |\ekv@ifempty@B|) it uses |\ekv@mark| and |\ekv@stop|.
%    \begin{macrocode}
\long\def\ekv@gobbleto@markstop#1\ekv@mark\ekv@stop{}
%    \end{macrocode}
% \end{macro}
%
% \begin{macro}[internal]{\ekv@set@pair,\ekv@set@pair@}
% |\ekv@set@pair| gets invoked with the space and brace stripped key-name as its
% first argument, the set-macro as the second argument, and following that is
% the key-value right delimited by an |\ekv@stop|.
%    \begin{macrocode}
\long\def\ekv@set@pair#1#2%
  {%
    \ekv@ifdefined@pair#2{#1}%
      {%
%    \end{macrocode}
% This branch will be executed if the key is not defined as an argument grabbing
% one. If so test whether there is a |NoVal| key of the same name or whether the
% key is unknown. Throw a meaningful error message and gobble the value.
%    \begin{macrocode}
        \ekv@ifdefined{#2{#1}N}%
          \ekv@err@noarg
          \ekv@err@unknown
          #2{#1}%
        \ekv@gobbleto@stop
      }%
  }
%    \end{macrocode}
% |\ekv@ifdefined@pair| will call |\ekv@set@pair@| if the key is correctly
% defined. This will then grab the value, strip outer spaces and braces from it
% and feed it to the key-macro. Afterwards |\ekv@set@other| will take control
% again.
%    \begin{macrocode}
\long\def\ekv@set@pair@#1#2\ekv@stop
  {%
    \ekv@strip{#2}#1%
  }
%    \end{macrocode}
% \end{macro}
%
% \begin{macro}[internal]{\ekv@set@key}
% Analogous to |\ekv@set@pair|, |\ekv@set@key| lets |\ekv@ifdefined@key| test
% whether a |NoVal| key is defined, else it'll throw a meaningful error message.
% Since we don't have to grab any value |\ekv@ifdefined@key| will invoke the
% key-macro and we're done here, |\ekv@set@other| will take over again.
%    \begin{macrocode}
\long\def\ekv@set@key#1#2%
  {%
    \ekv@ifdefined@key#2{#1}%
      {%
        \ekv@ifdefined{#2{#1}}%
          \ekv@err@reqval
          \ekv@err@unknown
          #2{#1}%
      }%
  }
%    \end{macrocode}
% \end{macro}
%
% \begin{macro}[internal]{\ekv@err,\ekv@err@}
% Since |\ekvset| is fully expandable as long as the code of the keys is (which
% is unlikely) we want to somehow throw expandable errors, in our case via
% undefined control sequences.
%    \begin{macrocode}
\begingroup
\edef\ekv@err
  {%
    \endgroup
    \unexpanded{\long\def\ekv@err}##1%
      {%
        \unexpanded{\expandafter\ekv@err@\@firstofone}%
        {\expandafter\noexpand\csname ! expkv Error:\endcsname ##1.}%
        \unexpanded{\ekv@stop}%
      }%
  }
\ekv@err
\def\ekv@err@{\expandafter\ekv@gobbleto@stop}
%    \end{macrocode}
% \end{macro}
% \begin{macro}[internal]
%   {
%     \ekv@err@common,\ekv@err@common@,
%     \ekv@err@unknown,\ekv@err@noarg,\ekv@err@reqval
%   }
% Now we can use |\ekv@err| to set up some error messages so that we can later
% use those instead of the full strings.
%    \begin{macrocode}
\long\def\ekv@err@common #1#2{\expandafter\ekv@err@common@\string#2{#1}}
\long\def\ekv@err@common@#1`#2' #3.#4#5{\ekv@err{#4 (`#5', set `#2')}}
\long\def\ekv@err@unknown#1#2{\ekv@err@common{unknown key}#1{#2}}
\long\def\ekv@err@noarg  #1#2{\ekv@err@common{value forbidden}#1{#2}}
\long\def\ekv@err@reqval #1#2{\ekv@err@common{value required}#1{#2}}
%    \end{macrocode}
% \end{macro}
%
% \begin{macro}[internal]{\ekv@strip,\ekv@strip@a,\ekv@strip@b,\ekv@strip@c}
% Finally we borrow some ideas of \pkg{expl3}'s \pkg{l3tl} to strip spaces
% from keys and values. This |\ekv@strip| also strips one level of outer braces
% \emph{after} stripping spaces, so an input of | {abc} | becomes |abc| after
% stripping. It should be used with |#1| prefixed by |\ekv@mark|. Also this
% implementation at most strips \emph{one} space from both sides.
%    \begin{macrocode}
\def\ekv@strip#1%
  {%
    \long\def\ekv@strip##1%
      {%
        \ekv@strip@a
          ##1%
          \ekv@nil
          \ekv@mark#1%
          #1\ekv@nil{}%
        \ekv@stop
      }%
    \long\def\ekv@strip@a##1\ekv@mark#1##2\ekv@nil##3%
      {%
        \ekv@strip@b##3##1##2\ekv@nil
      }%
    \long\def\ekv@strip@b##1#1\ekv@nil
      {%
        \ekv@strip@c##1\ekv@nil
      }%
    \long\def\ekv@strip@c\ekv@mark##1\ekv@nil##2\ekv@stop##3%
      {%
        ##3{##1}%
      }%
  }
\ekv@strip{ }
%    \end{macrocode}
% \end{macro}
%
% Now everything that's left is to reset the category code of |@|.
%    \begin{macrocode}
\catcode`\@=\ekv@tmp
%    \end{macrocode}
%
% \gobbledocstriptag
%</tex>
%^^A=<<
%
%^^A=<<
%
% \end{implementation}^^A=<<
%
% \clearpage
% \PrintIndex
%
\endinput
%
^^A vim: ft=tex fdm=marker fmr=>>=,=<<

\ProvidesPackage{expkv}%
  [\ekvDate\space v\ekvVersion\space an expandable key=val implementation]
%    \end{macrocode}
% \gobbledocstriptag
%</pkg>
%^^A=<<
%^^A main file >>=
% \subsection{The Generic Code}
% The rest of this implementation will be the generic code.
% \gobbledocstriptag
%<*tex>
%
% Check whether \eTeX\ is available -- \expkv\ requires \eTeX.
%    \begin{macrocode}
\begingroup\expandafter\expandafter\expandafter\endgroup
\expandafter\ifx\csname numexpr\endcsname\relax
  \errmessage{expkv requires e-TeX}
  \expandafter\endinput
\fi
%    \end{macrocode}
%
% We make sure that it's only input once:
%    \begin{macrocode}
\expandafter\ifx\csname ekvVersion\endcsname\relax
\else
  \expandafter\endinput
\fi
%    \end{macrocode}
% \begin{macro}{\ekvVersion,\ekvDate}
% We're on our first input, so lets store the version and date in a macro.
%    \begin{macrocode}
\def\ekvVersion{0.5a}
\def\ekvDate{2020-02-27}
%    \end{macrocode}
% \end{macro}
%
% If the \LaTeX\ format is loaded we want to be a good file and report back who
% we are, for this the package will have defined |\ekv@tmp| to use
% |\ProvidesFile|, else this will expand to a |\relax| and do no harm.
%    \begin{macrocode}
\csname ekv@tmp\endcsname
%    \end{macrocode}
%
% Store the category code of |@| to later be able to reset it and change it to
% 11 for now.
%    \begin{macrocode}
\expandafter\chardef\csname ekv@tmp\endcsname=\catcode`\@
\catcode`\@=11
%    \end{macrocode}
% |\ekv@tmp| might later be reused to gobble any prefixes which might be
% provided to |\ekvdef| and similar in case the names are invalid, we just
% temporarily use it here as means to store the current category code of |@| to
% restore it at the end of the file, we never care for the actual definition of
% it.
%
% \begin{macro}[internal]
%   {
%     \@gobble,\@firstofone,\@firstoftwo,\@secondoftwo,
%     \ekv@gobbleto@stop,\ekv@fi@secondoftwo,\ekv@gobble@mark
%   }
% Since branching tests are often more versatile than |\if...\else...\fi|
% constructs, we define helpers that are branching pretty fast. Also here are
% some other utility functions that just grab some tokens. The ones that are
% also contained in \LaTeX\ don't use the |ekv| prefix.
%    \begin{macrocode}
\long\def\@gobble#1{}
\long\def\@firstofone#1{#1}
\long\def\@firstoftwo#1#2{#1}
\long\def\@secondoftwo#1#2{#2}
\long\def\ekv@fi@secondoftwo\fi\@firstoftwo#1#2{\fi#2}
\long\def\ekv@gobbleto@stop#1\ekv@stop{}
\def\ekv@gobble@mark\ekv@mark{}
%    \end{macrocode}
% \end{macro}
% As you can see |\ekv@gobbleto@stop| uses a special marker |\ekv@stop|. The
% package will use three such markers, the one you've seen already, |\ekv@mark|
% and |\ekv@nil|. Contrarily to how for instance \pkg{expl3} does things, we
% don't define them, as we don't need them to have an actual meaning. This has
% the advantage that if they somehow get expanded -- which should never happen
% if things work out -- they'll throw an error directly.
%
% \begin{macro}[internal]
%   {
%     \ekv@ifempty,\ekv@ifempty@,\ekv@ifempty@true,\ekv@ifempty@false,
%     \ekv@ifempty@true@F
%   }
% We can test for a lot of things building on an if-empty test, so lets define a
% really fast one. Since some tests might have reversed logic (true if something
% is not empty) we also set up macros for the reversed branches.
%    \begin{macrocode}
\long\def\ekv@ifempty#1%
  {%
    \ekv@ifempty@\ekv@ifempty@A#1\ekv@ifempty@B\ekv@ifempty@true
      \ekv@ifempty@A\ekv@ifempty@B\@secondoftwo
  }
\long\def\ekv@ifempty@#1\ekv@ifempty@A\ekv@ifempty@B{}
\long\def\ekv@ifempty@true\ekv@ifempty@A\ekv@ifempty@B\@secondoftwo#1#2{#1}
\long\def\ekv@ifempty@false\ekv@ifempty@A\ekv@ifempty@B\@firstoftwo#1#2{#2}
\long\def\ekv@ifempty@true@F\ekv@ifempty@A\ekv@ifempty@B\@firstofone#1{}
%    \end{macrocode}
% \end{macro}
%
% \begin{macro}[internal]{\ekv@ifblank@}
% The obvious test that can be based on an if-empty is if-blank, meaning a test
% checking whether the argument is empty or consists only of spaces. Our version
% here will be tweaked a bit, as we want to check this, but with one leading
% |\ekv@mark| token that is to be ignored.
%    \begin{macrocode}
\long\def\ekv@ifblank@\ekv@mark#1{\ekv@ifempty@\ekv@ifempty@A}
%    \end{macrocode}
% \end{macro}
%
% \begin{macro}[internal]{\ekv@ifdefined}
% We'll need to check whether something is defined quite frequently, so why not
% define a macro that does this. The following test is expandable, slower than
% the typical expandable test for undefined control sequences, but faster for
% defined ones. Since we want to be as fast as possible for correct input, this
% is to be preferred.
%    \begin{macrocode}
\def\ekv@ifdefined#1%
  {%
    \expandafter
    \ifx\csname\ifcsname #1\endcsname #1\else relax\fi\endcsname\relax
      \ekv@fi@secondoftwo
    \fi
    \@firstoftwo
  }
%    \end{macrocode}
% \end{macro}
  % \begin{macro}[internal]
%   {
%     \ekv@ifdefined@pair,\ekv@ifdefined@pair@,\ekv@ifdefined@key,
%     \ekv@ifdefined@key@
%   }
% Since we can save some time if we only have to create the control sequence
% once when we know beforehand how we want to use it, we build some other macros
% for those cases (which we'll have quite often, once per key usage).
%    \begin{macrocode}
\def\ekv@ifdefined@pair#1#2%
  {%
    \expandafter\ekv@ifdefined@pair@
      \csname
        \ifcsname #1{#2}\endcsname
          #1{#2}%
        \else
          relax%
        \fi
      \endcsname
  }
\def\ekv@ifdefined@pair@#1%
  {%
    \ifx#1\relax
      \ekv@fi@secondoftwo
    \fi
    \@firstoftwo
    {\ekv@set@pair@#1\ekv@mark}%
  }
\def\ekv@ifdefined@key#1#2%
  {%
    \expandafter\ekv@ifdefined@key@
      \csname
        \ifcsname #1{#2}N\endcsname
          #1{#2}N%
        \else
          relax%
        \fi
      \endcsname
  }
\def\ekv@ifdefined@key@#1%
  {%
    \ifx#1\relax
      \ekv@fi@secondoftwo
    \fi
    \@firstoftwo#1%
  }
%    \end{macrocode}
% \end{macro}
%
% \begin{macro}{\ekv@name,\ekv@name@set,\ekv@name@key}
% The keys will all follow the same naming scheme, so we define it here.
%    \begin{macrocode}
\def\ekv@name#1#2{\ekv@name@set{#1}\ekv@name@key{#2}}
\def\ekv@name@set#1{ekv#1(}
\def\ekv@name@key#1{#1)}
%    \end{macrocode}
% \end{macro}
% \begin{macro}[internal]{\ekv@undefined@set}
% We can misuse the macro name we use to expandably store the set-name in a
% single token -- since this increases performance drastically, especially for
% long set-names -- to throw a more meaningful error message in case a set isn't
% defined. The name of |\ekv@undefined@set| is a little bit misleading, as it is
% called in either case inside of |\csname|, but the result will be a control
% sequence with meaning |\relax| if the set is undefined, hence will break the
% |\csname| building the key-macro which will throw the error message.
%    \begin{macrocode}
\def\ekv@undefined@set#1{! expkv Error: Set `#1' undefined.}
%    \end{macrocode}
% \end{macro}
% \begin{macro}[internal]{\ekv@checkvalid}
% We place some restrictions on the allowed names, though, namely sets and
% keys are not allowed to be empty -- blanks are fine (meaning \mbox{set-
% or} key-names consisting of spaces).
%    \begin{macrocode}
\protected\def\ekv@checkvalid#1#2%
  {%
    \ekv@ifempty{#1}%
      {%
        \def\ekv@tmp{}%
        \errmessage{expkv Error: empty set name not allowed}%
      }%
      {%
        \ekv@ifempty{#2}%
          {%
            \def\ekv@tmp{}%
            \errmessage{expkv Error: empty key name not allowed}%
          }%
          \@secondoftwo
      }%
    \@gobble
  }
%    \end{macrocode}
% \end{macro}
% \begin{macro}{\ekvifdefined,\ekvifdefinedNoVal}
% And provide user-level macros to test whether a key is defined.
%    \begin{macrocode}
\def\ekvifdefined#1#2{\ekv@ifdefined{\ekv@name{#1}{#2}}}
\def\ekvifdefinedNoVal#1#2{\ekv@ifdefined{\ekv@name{#1}{#2}N}}
%    \end{macrocode}
% \end{macro}
%
% \begin{macro}
%   {\ekvdef,\ekvdefNoVal,\ekvlet,\ekvletNoVal,\ekvletkv,\ekvletkvNoVal}
% Set up the key defining macros |\ekvdef| etc.
%    \begin{macrocode}
\protected\long\def\ekvdef#1#2#3%
  {%
    \ekv@checkvalid{#1}{#2}%
      {%
        \expandafter\def\csname\ekv@name{#1}{#2}\endcsname##1{#3}%
        \ekv@defset{#1}%
      }%
  }
\protected\long\def\ekvdefNoVal#1#2#3%
  {%
    \ekv@checkvalid{#1}{#2}%
      {%
        \expandafter\def\csname\ekv@name{#1}{#2}N\endcsname{#3}%
        \ekv@defset{#1}%
      }%
  }
\protected\def\ekvlet#1#2#3%
  {%
    \ekv@checkvalid{#1}{#2}%
      {%
        \expandafter\let\csname\ekv@name{#1}{#2}\endcsname#3%
        \ekv@defset{#1}%
      }%
  }
\protected\def\ekvletNoVal#1#2#3%
  {%
    \ekv@checkvalid{#1}{#2}%
      {%
        \expandafter\let\csname\ekv@name{#1}{#2}N\endcsname#3%
        \ekv@defset{#1}%
      }%
  }
\protected\def\ekvletkv#1#2#3#4%
  {%
    \ekv@checkvalid{#1}{#2}%
      {%
        \expandafter\let\csname\ekv@name{#1}{#2}\expandafter\endcsname
        \csname\ekv@name{#3}{#4}\endcsname
        \ekv@defset{#1}%
      }%
  }
\protected\def\ekvletkvNoVal#1#2#3#4%
  {%
    \ekv@checkvalid{#1}{#2}%
      {%
        \expandafter\let\csname\ekv@name{#1}{#2}N\expandafter\endcsname
        \csname\ekv@name{#3}{#4}N\endcsname
        \ekv@defset{#1}%
      }%
  }
%    \end{macrocode}
% \end{macro}
% \begin{macro}[internal]{\ekv@defset}
%   In order to enhance the speed the set name given to |\ekvset| will be turned
%   into a control sequence pretty early, so we have to define that control
%   sequence.
%    \begin{macrocode}
\protected\def\ekv@defset#1%
  {%
    \expandafter\edef\csname\ekv@undefined@set{#1}\endcsname##1%
      {\ekv@name@set{#1}\ekv@name@key{##1}}%
  }
%    \end{macrocode}
% \end{macro}
%
% \begin{macro}{\ekvset}
% Set up |\ekvset|, which should not be affected by active commas and equal
% signs. The equal signs are a bit harder to cope with and we'll do that later,
% but replacing the active commas with commas of category other can be done
% beforehand. That's why we define |\ekvset| here with a temporary meaning just
% to set up the things with two different category codes. |#1| will be a
% \texttt{,\textsubscript{13}} and |#2| will be a \texttt{=\textsubscript{13}}.
%    \begin{macrocode}
\def\ekvset#1#2{%
\endgroup
\long\def\ekvset##1##2%
  {%
    \expandafter\ekv@set\csname\ekv@undefined@set{##1}\endcsname
      \ekv@mark##2#1\ekv@stop#1{}%
  }
%    \end{macrocode}
% \end{macro}
% \begin{macro}[internal]{\ekv@set}
% |\ekv@set| will split the \kv\ list at active commas. Then it has to check
% whether there were unprotected other commas and resplit there.
%    \begin{macrocode}
\long\def\ekv@set##1##2#1%
  {%
%    \end{macrocode}
% Test whether we're at the end, if so invoke |\ekv@endset|,
%    \begin{macrocode}
    \ekv@gobbleto@markstop##2\ekv@endset\ekv@mark\ekv@stop
%    \end{macrocode}
% else go on with other commas,
%    \begin{macrocode}
    \ekv@set@other##1##2,\ekv@stop,%
%    \end{macrocode}
% and get the next active comma delimited \kv\ pair.
%    \begin{macrocode}
    \ekv@set##1\ekv@mark
  }
%    \end{macrocode}
% \end{macro}
% \begin{macro}[internal]{\ekv@endset}
% |\ekv@endset| is a hungry little macro. It will eat everything that remains
% of |\ekv@set| and unbrace the sneaked stuff.
%    \begin{macrocode}
\long\def\ekv@endset
    \ekv@mark\ekv@stop\ekv@set@other##1,\ekv@stop,\ekv@set##2\ekv@mark
    ##3%
  {##3}
%    \end{macrocode}
% \end{macro}
% \begin{macro}[internal]{\ekv@set@other}
% The macro |\ekv@set@other| is guaranteed to get only single \kv\ pairs.
% First we test whether we're done, if not split at equal signs. It is faster to
% first split at category 12 equal signs and only after that on actives.
% If there is no equal sign, we need to test whether we got a blank argument and
% if not this is a |NoVal| key.
%    \begin{macrocode}
\long\def\ekv@set@other##1##2,%
  {%
    \ekv@gobbleto@markstop##2\ekv@endset@other\ekv@mark\ekv@stop
    \ekv@ifhas@eq@other##2=\ekv@ifempty@B\ekv@ifempty@false
      \ekv@ifempty@A\ekv@ifempty@B\@firstoftwo
      {\ekv@set@eq@other##1##2\ekv@stop}%
      {%
        \ekv@ifhas@eq@active##2#2\ekv@ifempty@B\ekv@ifempty@false
          \ekv@ifempty@A\ekv@ifempty@B\@firstoftwo
          {\ekv@set@eq@active##1##2\ekv@stop}%
          {%
            \ekv@ifblank@##2\ekv@nil\ekv@ifempty@B\ekv@ifempty@true@F
              \ekv@ifempty@A\ekv@ifempty@B\@firstofone
              {\ekv@strip{##2}\ekv@set@key##1}%
          }%
      }%
    \ekv@set@other##1\ekv@mark%
  }
%    \end{macrocode}
% \end{macro}
% \begin{macro}[internal]{\ekv@set@eq@other}
% |\ekv@set@eq@other| might not be the correct break point, there might be an
% active equal sign in the currently parsed key-name. If so, we have to resplit.
% If the split is correct strip the key-name of outer spaces and braces and feed
% it to |\ekv@set@pair|.
%    \begin{macrocode}
\long\def\ekv@set@eq@other##1##2=%
  {%
    \ekv@ifhas@eq@active##2#2\ekv@ifempty@B\ekv@ifempty@false
      \ekv@ifempty@A\ekv@ifempty@B\@firstoftwo
      {\ekv@set@eq@active##1##2=}%
      {\ekv@strip{##2}\ekv@set@pair##1}%
  }
%    \end{macrocode}
% \end{macro}
% \begin{macro}[internal]{\ekv@set@eq@active}
% We need to handle the active equal signs.
%    \begin{macrocode}
\long\def\ekv@set@eq@active##1##2#2%
  {%
    \ekv@strip{##2}\ekv@set@pair##1%
  }
%    \end{macrocode}
% \end{macro}
% \begin{macro}[internal]
%   {\ekv@ifhas@eq@other,\ekv@ifhas@eq@active,\ekv@endset@other}
% And we have to set up the testing macros for our equal signs and
% |\ekv@endset@other|.
%    \begin{macrocode}
\long\def\ekv@ifhas@eq@other\ekv@mark##1={\ekv@ifempty@\ekv@ifempty@A}
\long\def\ekv@ifhas@eq@active\ekv@mark##1#2{\ekv@ifempty@\ekv@ifempty@A}
\long\def\ekv@endset@other
    \ekv@mark\ekv@stop
    \ekv@ifhas@eq@other##1=\ekv@ifempty@B\ekv@ifempty@false
    \ekv@ifempty@A\ekv@ifempty@B\@firstoftwo
    ##2%
    \ekv@set@other##3\ekv@mark
  {}
%    \end{macrocode}
% \end{macro}
% \begin{macro}{\ekvbreak,\ekvbreakPreSneak,\ekvbreakPostSneak}
% Provide macros that can completely stop the parsing of |\ekvset|, who knows
% what it'll be useful for.
%    \begin{macrocode}
\long\def\ekvbreak##1##2\ekv@stop#1##3{##1}
\long\def\ekvbreakPreSneak ##1##2\ekv@stop#1##3{##1##3}
\long\def\ekvbreakPostSneak##1##2\ekv@stop#1##3{##3##1}
%    \end{macrocode}
% \end{macro}
% \begin{macro}{\ekvsneak,\ekvsneakPre}
% One last thing we want to do for |\ekvset| is to provide macros that just
% smuggle stuff after |\ekvset|'s effects.
%    \begin{macrocode}
\long\def\ekvsneak##1##2\ekv@stop#1##3%
  {%
    ##2\ekv@stop#1{##3##1}%
  }
\long\def\ekvsneakPre##1##2\ekv@stop#1##3%
  {%
    ##2\ekv@stop#1{##1##3}%
  }
%    \end{macrocode}
% \end{macro}
%
% \begin{macro}{\ekvparse}
% Additionally to the |\ekvset| macro we also want to provide an |\ekvparse|
% macro, that has the same scope as |\keyval_parse:NNn| from \pkg{expl3}.
% This is pretty analogue to the |\ekvset| implementation, we just put an
% |\unexpanded| here and there instead of other macros to stop the |\expanded|
% on our output.
%    \begin{macrocode}
\long\def\ekvparse##1##2##3%
  {%
    \ekv@parse##1##2\ekv@mark##3#1\ekv@stop#1%
  }
%    \end{macrocode}
% \end{macro}
% \begin{macro}[internal]{\ekv@parse}
%    \begin{macrocode}
\long\def\ekv@parse##1##2##3#1%
  {%
    \ekv@gobbleto@markstop##3\ekv@endparse\ekv@mark\ekv@stop
    \ekv@parse@other##1##2##3,\ekv@stop,%
    \ekv@parse##1##2\ekv@mark
  }
%    \end{macrocode}
% \end{macro}
% \begin{macro}[internal]{\ekv@endparse}
%    \begin{macrocode}
\long\def\ekv@endparse
    \ekv@mark\ekv@stop\ekv@parse@other##1,\ekv@stop,\ekv@parse##2\ekv@mark
  {}
%    \end{macrocode}
% \end{macro}
% \begin{macro}[internal]{\ekv@parse@other}
%    \begin{macrocode}
\long\def\ekv@parse@other##1##2##3,%
  {%
    \ekv@gobbleto@markstop##3\ekv@endparse@other\ekv@mark\ekv@stop
    \ekv@ifhas@eq@other##3=\ekv@ifempty@B\ekv@ifempty@false
      \ekv@ifempty@A\ekv@ifempty@B\@firstoftwo
      {\ekv@parse@eq@other##3\ekv@stop##2}%
      {%
        \ekv@ifhas@eq@active##3#2\ekv@ifempty@B\ekv@ifempty@false
          \ekv@ifempty@A\ekv@ifempty@B\@firstoftwo
          {\ekv@parse@eq@active##3\ekv@stop##2}%
          {%
            \ekv@ifblank@##3\ekv@nil\ekv@ifempty@B\ekv@ifempty@true@F
              \ekv@ifempty@A\ekv@ifempty@B\@firstofone
              {\ekv@strip{##3}\ekv@parse@key##1}%
          }%
      }%
    \ekv@parse@other##1##2\ekv@mark
  }
%    \end{macrocode}
% \end{macro}
% \begin{macro}[internal]{\ekv@parse@eq@other}
%    \begin{macrocode}
\long\def\ekv@parse@eq@other##1=%
  {%
    \ekv@ifhas@eq@active##1#2\ekv@ifempty@B\ekv@ifempty@false
      \ekv@ifempty@A\ekv@ifempty@B\@firstoftwo
      {\ekv@parse@eq@active##1=}%
      {\ekv@strip{##1}\ekv@parse@pair\ekv@mark}%
  }
%    \end{macrocode}
% \end{macro}
% \begin{macro}[internal]{\ekv@parse@eq@active}
%    \begin{macrocode}
\long\def\ekv@parse@eq@active##1#2%
  {%
    \ekv@strip{##1}\ekv@parse@pair\ekv@mark
  }
%    \end{macrocode}
% \end{macro}
% \begin{macro}[internal]{\ekv@endparse@other}
%    \begin{macrocode}
\long\def\ekv@endparse@other
    \ekv@mark\ekv@stop
    \ekv@ifhas@eq@other##1=\ekv@ifempty@B\ekv@ifempty@false
    \ekv@ifempty@A\ekv@ifempty@B\@firstoftwo
    ##2%
    \ekv@parse@other##3\ekv@mark
  {}
%    \end{macrocode}
% \end{macro}
% \begin{macro}[internal]{\ekv@parse@pair,\ekv@parse@pair@}
%    \begin{macrocode}
\long\def\ekv@parse@pair##1##2\ekv@stop
  {%
    \ekv@strip{##2}\ekv@parse@pair@{##1}%
  }
\long\def\ekv@parse@pair@##1##2##3%
  {%
    \unexpanded{##3{##2}{##1}}%
  }
%    \end{macrocode}
% \end{macro}
% \begin{macro}[internal]{\ekv@parse@key}
%    \begin{macrocode}
\long\def\ekv@parse@key##1##2%
  {%
    \unexpanded{##2{##1}}%
  }
%    \end{macrocode}
% \end{macro}
%
% Finally really setting things up with |\ekvset|'s temporary meaning:
%    \begin{macrocode}
}
\begingroup
\catcode`\,=13
\catcode`\==13
\ekvset,=
%    \end{macrocode}
%
% \begin{macro}{\ekvchangeset}
% Provide a macro that is able to switch out the current \set\ in |\ekvset|.
% This operation is slow (by comparison, it should be slightly faster than
% |\ekvsneak|), but allows for something similar to \pkg{pgfkeys}'s
% \texttt{\meta{key}/.cd} mechanism. However this operation is more expensive
% than |/.cd| as we can't just redefine some token to reflect this, but have to
% switch out the set expandably, so this works similar to the |\ekvsneak| macros
% reading and reinserting the remainder of the \kv\ list.
%    \begin{macrocode}
\def\ekvchangeset#1%
  {%
    \expandafter\ekv@changeset\csname\ekv@undefined@set{#1}\endcsname\ekv@mark
  }
%    \end{macrocode}
% \end{macro}
% \begin{macro}[internal]{\ekv@changeset}
% This macro does the real change-out of |\ekvchangeset|. We introduced an
% |\ekv@mark| to not accidentally remove some braces which we have to remove
% again.
%    \begin{macrocode}
\long\def\ekv@changeset#1#2\ekv@set@other#3#4\ekv@set#5%
  {%
    \ekv@gobble@mark#2\ekv@set@other#1#4\ekv@set#1%
  }
%    \end{macrocode}
% \end{macro}
%
%
% \begin{macro}[internal]{\ekv@gobbleto@markstop}
% The |\ekv@gobbleto@markstop| can be used to test for |\ekv@stop| similar to
% our if-empty test, but instead of using tokens which are used nowhere else
% (|\ekv@ifempty@A| and |\ekv@ifempty@B|) it uses |\ekv@mark| and |\ekv@stop|.
%    \begin{macrocode}
\long\def\ekv@gobbleto@markstop#1\ekv@mark\ekv@stop{}
%    \end{macrocode}
% \end{macro}
%
% \begin{macro}[internal]{\ekv@set@pair,\ekv@set@pair@}
% |\ekv@set@pair| gets invoked with the space and brace stripped key-name as its
% first argument, the set-macro as the second argument, and following that is
% the key-value right delimited by an |\ekv@stop|.
%    \begin{macrocode}
\long\def\ekv@set@pair#1#2%
  {%
    \ekv@ifdefined@pair#2{#1}%
      {%
%    \end{macrocode}
% This branch will be executed if the key is not defined as an argument grabbing
% one. If so test whether there is a |NoVal| key of the same name or whether the
% key is unknown. Throw a meaningful error message and gobble the value.
%    \begin{macrocode}
        \ekv@ifdefined{#2{#1}N}%
          \ekv@err@noarg
          \ekv@err@unknown
          #2{#1}%
        \ekv@gobbleto@stop
      }%
  }
%    \end{macrocode}
% |\ekv@ifdefined@pair| will call |\ekv@set@pair@| if the key is correctly
% defined. This will then grab the value, strip outer spaces and braces from it
% and feed it to the key-macro. Afterwards |\ekv@set@other| will take control
% again.
%    \begin{macrocode}
\long\def\ekv@set@pair@#1#2\ekv@stop
  {%
    \ekv@strip{#2}#1%
  }
%    \end{macrocode}
% \end{macro}
%
% \begin{macro}[internal]{\ekv@set@key}
% Analogous to |\ekv@set@pair|, |\ekv@set@key| lets |\ekv@ifdefined@key| test
% whether a |NoVal| key is defined, else it'll throw a meaningful error message.
% Since we don't have to grab any value |\ekv@ifdefined@key| will invoke the
% key-macro and we're done here, |\ekv@set@other| will take over again.
%    \begin{macrocode}
\long\def\ekv@set@key#1#2%
  {%
    \ekv@ifdefined@key#2{#1}%
      {%
        \ekv@ifdefined{#2{#1}}%
          \ekv@err@reqval
          \ekv@err@unknown
          #2{#1}%
      }%
  }
%    \end{macrocode}
% \end{macro}
%
% \begin{macro}[internal]{\ekv@err,\ekv@err@}
% Since |\ekvset| is fully expandable as long as the code of the keys is (which
% is unlikely) we want to somehow throw expandable errors, in our case via
% undefined control sequences.
%    \begin{macrocode}
\begingroup
\edef\ekv@err
  {%
    \endgroup
    \unexpanded{\long\def\ekv@err}##1%
      {%
        \unexpanded{\expandafter\ekv@err@\@firstofone}%
        {\expandafter\noexpand\csname ! expkv Error:\endcsname ##1.}%
        \unexpanded{\ekv@stop}%
      }%
  }
\ekv@err
\def\ekv@err@{\expandafter\ekv@gobbleto@stop}
%    \end{macrocode}
% \end{macro}
% \begin{macro}[internal]
%   {
%     \ekv@err@common,\ekv@err@common@,
%     \ekv@err@unknown,\ekv@err@noarg,\ekv@err@reqval
%   }
% Now we can use |\ekv@err| to set up some error messages so that we can later
% use those instead of the full strings.
%    \begin{macrocode}
\long\def\ekv@err@common #1#2{\expandafter\ekv@err@common@\string#2{#1}}
\long\def\ekv@err@common@#1`#2' #3.#4#5{\ekv@err{#4 (`#5', set `#2')}}
\long\def\ekv@err@unknown#1#2{\ekv@err@common{unknown key}#1{#2}}
\long\def\ekv@err@noarg  #1#2{\ekv@err@common{value forbidden}#1{#2}}
\long\def\ekv@err@reqval #1#2{\ekv@err@common{value required}#1{#2}}
%    \end{macrocode}
% \end{macro}
%
% \begin{macro}[internal]{\ekv@strip,\ekv@strip@a,\ekv@strip@b,\ekv@strip@c}
% Finally we borrow some ideas of \pkg{expl3}'s \pkg{l3tl} to strip spaces
% from keys and values. This |\ekv@strip| also strips one level of outer braces
% \emph{after} stripping spaces, so an input of | {abc} | becomes |abc| after
% stripping. It should be used with |#1| prefixed by |\ekv@mark|. Also this
% implementation at most strips \emph{one} space from both sides.
%    \begin{macrocode}
\def\ekv@strip#1%
  {%
    \long\def\ekv@strip##1%
      {%
        \ekv@strip@a
          ##1%
          \ekv@nil
          \ekv@mark#1%
          #1\ekv@nil{}%
        \ekv@stop
      }%
    \long\def\ekv@strip@a##1\ekv@mark#1##2\ekv@nil##3%
      {%
        \ekv@strip@b##3##1##2\ekv@nil
      }%
    \long\def\ekv@strip@b##1#1\ekv@nil
      {%
        \ekv@strip@c##1\ekv@nil
      }%
    \long\def\ekv@strip@c\ekv@mark##1\ekv@nil##2\ekv@stop##3%
      {%
        ##3{##1}%
      }%
  }
\ekv@strip{ }
%    \end{macrocode}
% \end{macro}
%
% Now everything that's left is to reset the category code of |@|.
%    \begin{macrocode}
\catcode`\@=\ekv@tmp
%    \end{macrocode}
%
% \gobbledocstriptag
%</tex>
%^^A=<<
%
%^^A=<<
%
% \end{implementation}^^A=<<
%
% \clearpage
% \PrintIndex
%
\endinput
%
^^A vim: ft=tex fdm=marker fmr=>>=,=<<
}{}
\ProvidesFile{expkv.dtx}
  [\csname ekvDate\endcsname\ an expandable key=val implementation]
\PassOptionsToPackage{full}{textcomp}
\documentclass{l3doc}
\RequirePackage[oldstylenums,nott]{kpfonts}
\input{glyphtounicode}
\pdfgentounicode=1
\RequirePackage{xfp} ^^A required for an example
\RequirePackage{listings}
\RequirePackage{booktabs}
\RequirePackage{array}
\RequirePackage{collcell}
\RequirePackage{siunitx}
\RequirePackage{xcolor}
\RequirePackage{caption}
\RequirePackage{microtype}
\RequirePackage{accsupp}
\lstset
  {
    ,flexiblecolumns=false
    ,basewidth=.53em
    ,gobble=2
    ,basicstyle=\fontfamily{jkp}\itshape
    ,morekeywords=^^A
      {^^A
        \ekvdef,\ekvdefNoVal,\ekvlet,\ekvletNoVal,\ekvletkv,\ekvletkvNoVal,^^A
        \ekvset,\ekvparse,\ekvVersion,\ekvDate,\ekvifdefined,^^A
        \ekvifdefinedNoVal,\ekvbreak,\ekvbreakPreSneak,\ekvbreakPostSneak,^^A
        \ekvsneak,\ekvsneakPre,\ekvchangeset
      }
    ,morecomment=[l]\%
    ,commentstyle=\color[gray]{0.4}
    ,literate={\{}{{\CodeSymbol\{}}{1}
              {\}}{{\CodeSymbol\}}}{1}
    ^^A,literate=*{<key>}{\key}{4}{<set>}{\set}{4}
  }
\newcommand*\CodeSymbol[1]{\textbf{#1}}
\RequirePackage{randtext}
\let\metaORIG\meta
\protected\def\meta #1{\texttt{\metaORIG{#1}}}
\renewcommand*\thefootnote{\fnsymbol{footnote}}
\definecolor{expkvred}{HTML}{9F393D}
\colorlet{expkvgrey}{black!75}
\makeatletter
\newcommand*\expkv
  {^^A
    \texorpdfstring
      {^^A
        \mbox
          {^^A
            \BeginAccSupp{ActualText=expkv}^^A
            \href{https://github.com/Skillmon/tex_expkv}
              {^^A
                \rmfamily
                \bfseries
                {\color{expkvgrey}e\kern-.05em x\kern-.05em}^^A
                \lower.493ex
                  \hbox{{\color{expkvgrey}P}\kern-.1em{\color{expkvred}k}}^^A
                \kern-.18em{\color{expkvred}v}^^A
              }^^A
            \EndAccSupp{}^^A
          }^^A
      }
      {expkv}^^A
  }
\newcommand*\expkvd
  {^^A
    \texorpdfstring
      {^^A
        \mbox
          {^^A
            \BeginAccSupp{ActualText=expkv-def}^^A
            \rmfamily
            \bfseries
            {\color{expkvgrey}e\kern-.05em x\kern-.05em}^^A
            \lower.493ex
              \hbox{{\color{expkvgrey}P}\kern-.1em{\color{expkvred}k}}^^A
            \kern-.18em{\color{expkvred}v}^^A
            {\color{expkvgrey}^^A
              \kern.05em\rule[-.1ex]{.08em}{1.2ex}\kern.05em\textsc{def}^^A
            }^^A
            \EndAccSupp{}^^A
          }^^A
      }
      {expkv-def}^^A
  }
\newcommand\kv{\meta{key}=\meta{value}}
\newcommand\key{\meta{key}}
\newcommand\val{\meta{value}}
\newcommand\set{\meta{set}}
\hypersetup{linkcolor=red!80!black,urlcolor=purple!80!black}
\DoNotIndex{\def,\edef,\,,\=,\begingroup,\catcode,\chardef,\csname,\endcsname}
\DoNotIndex{\endgroup,\endinput,\errmessage,\expandafter,\input,\let,\long}
\DoNotIndex{\noexpand,\protected,\ProvidesFile,\ProvidesPackage,\relax,\space}
\DoNotIndex{\@,\unexpanded,\string,\expanded}
\DoNotIndex{\ifcsname}
\DoNotIndex{\ifx}
\DoNotIndex{\else}
\DoNotIndex{\fi}
\@gobble\fi ^^A ignoring \ifx and \ifcsname, but only one \fi
\@ifdefinable\gobbledocstriptag{\def\gobbledocstriptag#1>{}}
\newcommand*\pmso[1] ^^A poor man's strike out
  {^^A
    \leavevmode
    \begingroup
      \sbox0{#1}^^A
      \rlap{\vrule height .6ex depth -.5ex width \wd0\relax}^^A
      \usebox0\relax
    \endgroup
  }
\makeatother
\begin{document}
  \title
    {^^A
      \texorpdfstring
        {^^A
          \huge\expkv
          \\[\medskipamount]
          \Large an expandable \kv\ implementation^^A
        }{expkv - an expandable <key>=<value> implementation}^^A
    }
  \date{\ekvDate\space v\ekvVersion}
  \author{Jonathan P. Spratte\thanks{\protect\randomize{jspratte@yahoo.de}}}
  \DocInput{expkv.dtx}
\end{document}
%</driver>^^A=<<
% \fi
%
% \maketitle
% \renewcommand*\thefootnote{\arabic{footnote}}
%
% \begin{abstract}
% \noindent\parfillskip=0pt
% \expkv\ provides a small interface for \kv\ parsing. The parsing macro is
% fully expandable, the \meta{code} of your keys might be not. \expkv\ is pretty
% fast, but not the fastest available \kv\ solution (\pkg{keyval} for instance
% is thrice as fast, but not expandable and it might strip braces it shouldn't
% have stripped).
% \end{abstract}
%
% \tableofcontents
%
% \begin{documentation}^^A>>=
%
% \section{Documentation}
%
% \expkv\ provides an expandable \kv\ parser. The \kv\ pairs should be given as
% a comma separated list and the separator between a \key\ and the associated
% \val\ should be an equal sign. Both, the commas and the equal signs, might be
% of category 12 (other) or 13 (active). To support this is necessary as for
% example \pkg{babel} turns characters active for some languages, for instance
% the equal sign is turned active for Turkish.
%
% \expkv\ is usable as generic code or as a \LaTeX\ package. To use it, just
% use one of:
% \begin{lstlisting}
% \usepackage{expkv} % LaTeX
% \input expkv       % plainTeX
% \end{lstlisting}
% The \LaTeX\ package doesn't do more than \file{expkv.tex}, except calling
% |\ProvidesPackage| and setting things up such that \file{expkv.tex} will use
% |\ProvidesFile|.
%
% \subsection{Setting up Keys}\label{sec:define}
% \expkv\ provides a rather simple approach to setting up keys, similar to
% \pkg{keyval}. However there is an auxiliary package named
% \href{https://github.com/Skillmon/tex_expkv-def}{\expkvd} (the files use
% \file{expkv-def} instead because \string| isn't allowed in package names by
% CTAN) which provides a more sophisticated interface, similar to well
% established packages like \pkg{pgfkeys} or \pkg{l3keys}.
%
% Keys in \expkv\ (as in almost all other \kv\ implementations) belong to a
% \emph{set} such that different sets can contain keys of the same name. Unlike
% many other implementations \expkv\ doesn't provide means to set a default
% value, instead we have keys that take values and keys that don't (the latter
% are called |NoVal| keys by \expkv), but both can have the same name (on the
% user level).
%
% The following macros are available to define new keys. Those macros containing
% ``|def|'' in their name can be prefixed by anything allowed to prefix |\def|,
% prefixes allowed for |\let| can prefix those with ``|let|'' in their name,
% accordingly. Neither \set\ nor \key\ are allowed to be empty for new keys and
% must not contain a |\par| or tokens that expand to it -- they must be legal
% inside of |\csname ...\endcsname|.
%
% \begin{function}{\ekvdef}
%   \begin{syntax}
%     \cs{ekvdef}\marg{set}\marg{key}\marg{code}
%   \end{syntax}
%   Defines a \key\ taking a value in a \set\ to expand to \meta{code}. In
%   \meta{code} you can use |#1| to refer to the given value.
% \end{function}
%
% \begin{function}{\ekvdefNoVal}
%   \begin{syntax}
%     \cs{ekvdefNoVal}\marg{set}\marg{key}\marg{code}
%   \end{syntax}
%   Defines a no value taking \key\ in a \set\ to expand to \meta{code}.
% \end{function}
%
% \begin{function}{\ekvlet}
%   \begin{syntax}
%     \cs{ekvlet}\marg{set}\marg{key}\meta{cs}
%   \end{syntax}
%   Let the value taking \key\ in \set\ to \meta{cs}, there are no checks on
%   \meta{cs} enforced.
% \end{function}
%
% \begin{function}{\ekvletNoVal}
%   \begin{syntax}
%     \cs{ekvletNoVal}\marg{set}\marg{key}\meta{cs}
%   \end{syntax}
%   Let the no value taking \key\ in \set\ to \meta{cs}, it is not checked
%   whether \meta{cs} exists or that it takes no parameter.
% \end{function}
%
% \begin{function}{\ekvletkv}
%   \begin{syntax}
%     \cs{ekvletkv}\marg{set}\marg{key}\marg{set2}\marg{key2}
%   \end{syntax}
%   Let the \key\ in \set\ to \meta{key2} in \meta{set2}, it is not checked
%   whether that second key exists.
% \end{function}
%
% \begin{function}{\ekvletkvNoVal}
%   \begin{syntax}
%     \cs{ekvletkvNoVal}\marg{set}\marg{key}\marg{set2}\marg{key2}
%   \end{syntax}
%   Let the \key\ in \set\ to \meta{key2} in \meta{set2}, it is not checked
%   whether that second key exists.
% \end{function}
%
% \subsection{Parsing Keys}
%
% \begin{function}{\ekvset}
%   \begin{syntax}
%     \cs{ekvset}\marg{set}\{\kv,\ldots\}
%   \end{syntax}
%   Splits \kv\ pairs on commas. From both \key\ and \val\ up to one space is
%   stripped from both ends, if then only a braced group remains the braces are
%   stripped as well. So |\ekvset{foo}{bar=baz}| and
%   |\ekvset{foo}{ {bar}= {baz} }| will both do
%   \texttt{\cs[no-index]{\meta{foobarcode}}\{baz\}}, so you can hide commas,
%   equal signs and spaces at the ends of either \key\ or \val\ by putting
%   braces around them. If you omit the equal sign the code of the key created
%   with the |NoVal| variants described in \autoref{sec:define} will be
%   executed. If \kv\ contains more than a single unhidden equal sign, it will
%   be split at the first one and the others are considered part of the value.
%   |\ekvset| should be nestable.
% \end{function}
%
% \begin{function}{\ekvparse}
%   \begin{syntax}
%     \cs{ekvparse}\meta{cs1}\meta{cs2}\{\kv,\ldots\}
%   \end{syntax}
%   This macro parses the \kv\ pairs and provides those list elements which are
%   only keys as the argument to \meta{cs1}, and those which are a \kv\ pair to
%   \meta{cs2} as two arguments. It is fully expandable as well and returns the
%   parsed list in |\unexpanded|, which has no effect outside of an |\expanded|
%   or |\edef| context\footnotemark.
%   If you need control over the necessary steps of expansion you
%   can use |\expanded| around it.
%
%   |\ekvbreak|, |\ekvsneak|, and |\ekvchangeset| and their relatives don't work
%   in |\ekvparse|.  It is analogue to \pkg{expl3}'s |\keyval_parse:NNn|, but
%   not with the same parsing rules -- |\keyval_parse:NNn| throws an error on
%   multiple equal signs per \kv\ pair and on empty \key\ names in a \kv\ pair,
%   both of which |\ekvparse| doesn't deal with.
%
%   As a small example:
% \begin{lstlisting}
% \ekvparse\handlekey\handlekeyval{foo = bar, key, baz={zzz}}
% \end{lstlisting}
%   would expand to
% \begin{lstlisting}
% \handlekeyval{foo}{bar}\handlekey{key}\handlekeyval{baz}{zzz}
% \end{lstlisting}
%   and afterwards |\handlekey| and |\handlekeyval| would have to further handle
%   the \key. There are no macros like these two contained in \expkv, you have
%   to set them up yourself if you want to use |\ekvparse| (of course the names
%   might differ). If you need the results of |\ekvparse| as the argument for
%   another macro, you should use |\expanded| as only then the input stream will
%   contain the output above:
% \begin{lstlisting}
% \expandafter\handle\expanded{\ekvparse\k\kv{foo = bar, key, baz={zzz}}}
% \end{lstlisting}
%   would expand to
% \begin{lstlisting}
% \handle\kv{foo}{bar}\k{key}\kv{baz}{zzz}
% \end{lstlisting}
% \end{function}
% \footnotetext 
%   {^^A
%     This is a change in behaviour, previously (v0.3 and before) \cs{ekvparse}
%     would expand in exactly two steps. This isn't always necessary, but makes
%     the parsing considerably slower. If this is necessary for your application
%     you can put an \cs[no-index]{expanded} around it and will still be faster
%     since you need only a single \cs[no-index]{expandafter} this way.^^A
%   }.
%
% \subsection{Miscellaneous}
%
% \subsubsection{Other Macros}
%
% \expkv\ provides some other macros which might be of interest.
%
% \begin{function}{\ekvVersion,\ekvDate}
%   These two macros store the version and date of the package.
% \end{function}
% \begin{function}{\ekvifdefined,\ekvifdefinedNoVal}
%   \begin{syntax}
%     \cs{ekvifdefined}\marg{set}\marg{key}\marg{true}\marg{false}
%     \cs{ekvifdefinedNoVal}\marg{set}\marg{key}\marg{true}\marg{false}
%   \end{syntax}
%   These two macros test whether there is a \key\ in \set. It is false if
%   either a hash table entry doesn't exist for that key or its meaning is
%   |\relax|.
% \end{function}
%
% \begin{function}{\ekvbreak,\ekvbreakPreSneak,\ekvbreakPostSneak}
%   \begin{syntax}
%     \cs{ekvbreak}\marg{after}
%   \end{syntax}
%   Gobbles the remainder of the current |\ekvset| macro and its argument list
%   and reinserts \meta{after}. So this can be used to break out of |\ekvset|.
%   The first variant will also gobble anything that has been sneaked out using
%   |\ekvsneak| or |\ekvsneakPre|, while |\ekvbreakPreSneak| will put
%   \meta{after} before anything that has been smuggled and |\ekvbreakPostSneak|
%   will put \meta{after} after the stuff that has been sneaked out.
% \end{function}
%
% \begin{function}{\ekvsneak,\ekvsneakPre}
%   \begin{syntax}
%     \cs{ekvsneak}\marg{after}
%   \end{syntax}
%   Puts \meta{after} after the effects of |\ekvset|. The first variant will put
%   \meta{after} after any other tokens which might have been sneaked before,
%   while |\ekvsneakPre| will put \meta{after} before other smuggled stuff.
%   This reads and reinserts the remainder of the current |\ekvset| macro and
%   its argument list to do its job. A small usage example is shown in
%   \autoref{sec:sneakex}.
% \end{function}
%
% \begin{function}{\ekvchangeset}
%   \begin{syntax}
%     \cs{ekvchangeset}\marg{new-set}
%   \end{syntax}
%   Replaces the current set with \meta{new-set}, so for the rest of the current
%   |\ekvset| call, that call behaves as if it was called with
%   \texttt{\cs[no-index]{ekvset}\marg{new-set}}. Just like |\ekvsneak| this
%   reads and reinserts the remainder of the current |\ekvset| macro to do its
%   job. It is comparable to using \texttt{\meta{key}/.cd} in \pkg{pgfkeys}.
% \end{function}
%
% \bigskip
%
% \begin{function}{\ekv@name,\ekv@name@set,\ekv@name@key}
%   \begin{syntax}
%     \cs{ekv@name}\marg{set}\marg{key}\\
%     \cs{ekv@name@set}\marg{set}\\
%     \cs{ekv@name@key}\marg{key}
%   \end{syntax}
%   The names of the macros that correspond to a key in a set are build with
%   these macros. The default definition of |\ekv@name@set| is
%   ``\texttt{\csname ekv@name@set\endcsname{\set}}'' and the default of
%   |\ekv@name@key| is ``\texttt{\csname ekv@name@key\endcsname{\key}}''.
%   The complete name is build using |\ekv@name| which is equivalent to
%   \texttt{\cs[no-index]{ekv@name@set}\marg{set}\cs{ekv@name@key}\marg{key}}.
%   For |NoVal| keys an additional |N| gets appended irrespective of these
%   macros' definition, so their name is
%   \cs[no-index]{\csname ekv@name\endcsname{\set}{\key}N}. You might redefine
%   |\ekv@name@set| and |\ekv@name@key| locally but \emph{don't redefine}
%   |\ekv@name|!
% \end{function}
%
% \subsubsection{Bugs}
%
% Just like \pkg{keyval}, \expkv\ is bug free. But if you find \pmso{bugs}hidden
% features\footnote{Thanks, David!} you can tell me about them either via mail
% (see the first page) or directly on GitHub if you have an account there:
% \url{https://github.com/Skillmon/tex_expkv}
%
% \subsubsection{Comparisons}
%
% Comparisons of speed are done with a very simple test key and the help of the
% \pkg{l3benchmark} package. The key and its usage should be equivalent to
% \begin{lstlisting}
% \protected\ekvdef{test}{height}{\def\myheight{#1}}
% \ekvset{test}{ height = 6 }
% \end{lstlisting}
% and only the usage of the key, not its definition, is benchmarked. For the
% impatient, the essence of these comparisons regarding speed and buggy
% behaviour is contained in \autoref{tab:comp}.
%
% As far as I know \expkv\ is the only fully expandable \kv\ parser. I tried to
% compare \expkv\ to every \kv\ package listed on
% \href{https://ctan.org/topic/keyval}{CTAN}, however, one might notice that
% some of those are missing from this list. That's because I didn't get the
% others to work due to bugs, or because they just provide wrappers around other
% packages in this list.
%
% In this subsubsection is no benchmark of |\ekvparse| and |\keyval_parse:NNn|
% contained, as most other packages don't provide equivalent features to my
% knowledge. |\ekvparse| is slightly faster than |\ekvset|, but keep in mind
% that it does less. The same is true for |\keyval_parse:NNn| compared to
% |\keys_set:nn| of \pkg{expl3} (where the difference is much bigger).
%
% \paragraph{\pkg{keyval}} is about two times faster and has a comparable
% feature set just a slightly different way how it handles keys without values.
% That might be considered a drawback, as it limits the versatility, but also as
% an advantage, as it might reduce doubled code. Keep in mind that as soon as
% someone loads \pkg{xkeyval} the performance of \pkg{keyval} gets replaced by
% \pkg{xkeyval}'s.
%
% Also \pkg{keyval} has a bug, which unfortunately can't really be resolved
% without breaking backwards compatibility for \emph{many} documents, namely it
% strips braces from the argument before stripping spaces if the argument isn't
% surrounded by spaces, also it might strip more than one set of braces. Hence
% all of the following are equivalent in their outcome, though the last two
% lines should result in something different than the first two:
% \begin{lstlisting}[belowskip=0pt]
% \setkeys{foo}{bar=baz}
% \setkeys{foo}{bar= {baz}}
% \setkeys{foo}{bar={ baz}}
% \setkeys{foo}{bar={{baz}}}
% \end{lstlisting}
%
% \paragraph{\pkg{xkeyval}} is roughly fourteen times slower, but it provides
% more functionality, e.g., it has choice keys, boolean keys, and so on. It
% contains the same bug as \pkg{keyval} as it has to be compatible with it by
% design (it replaces \pkg{keyval}'s frontend), but also adds even more cases in
% which braces are stripped that shouldn't be stripped, worsening the situation.
%
% \paragraph{\pkg{ltxkeys}} is over 300 times slower -- which is funny, because
% it aims to be ``[\ldots] faster than these earlier packages [referring to
% \pkg{keyval} and \pkg{xkeyval}].'' Since it aims to have a bigger feature set
% than \pkg{xkeyval}, it most definitely also has a bigger feature set than
% \expkv. Also, it can't parse |\long| input, so as soon as your values contain
% a |\par|, it'll throw errors. Furthermore, \pkg{ltxkeys} doesn't strip outer
% braces at all by design, which, imho, is a weird design choice. In addition
% \pkg{ltxkeys} loads \pkg{catoptions} which is known to introduce bugs (e.g.,
% see \url{https://tex.stackexchange.com/questions/461783}).
%
% \paragraph{\pkg{l3keys}} is almost five times slower, but has an, imho, great
% interface to define keys. It strips \emph{all} outer spaces, even if somehow
% multiple spaces ended up on either end. It offers more features, but is pretty
% much bound to \pkg{expl3} code. Whether that's a drawback is up to you.
%
% \paragraph{\pkg{pgfkeys}} is a bit more than two times slower for one key,
% but has an \emph{enormous} feature set. However, since adding additional keys
% doesn't add as much needed time for \pkg{pgfkeys} compared to \expkv, it gets
% faster than \expkv\ at around eight \kv\ pairs.
% It has the same or a very similar bug \pkg{keyval} has. The brace bug (and
% also the category fragility) can be fixed by \pkg{pgfkeyx}, but this package
% was last updated in 2012 and it slows down |\pgfkeys| by factor~$8$. Also I
% don't know whether this might introduce new bugs.
%
% \paragraph{\pkg{kvsetkeys} with \pkg{kvdefinekeys}} is about three times
% slower, but it works even if commas and equals have category codes different
% from 12 (just as some other packages in this list). Else the features of the
% keys are equal to those of \pkg{keyval}, the parser has more features, though.
%
% \paragraph{\pkg{options}} is a bit slower for only a single value, but gets a
% tad faster than \expkv\ at around 10 \kv\ pairs. It has a much bigger feature
% set. Unfortunately it also suffers from the premature unbracing bug
% \pkg{keyval} has.
%
% \paragraph{\pkg{simplekv}} is hard to compare because I don't speak French (so
% I don't understand the documentation) and from what I can see, there is no
% direct way to define the equivalent test key. Nevertheless, I tested the
% closest possible equivalent of my test key while siding for \pkg{simplekv}'s
% design not forcing something into it it doesn't seem to be designed for. It is
% more than five times slower and has hard to predict behaviour regarding brace
% and space stripping, similar to \pkg{keyval}. The tested definition was:
% \begin{lstlisting}[belowskip=0pt]
% \usepackage{simplekv}
% \setKVdefault[simplekv]{height={ abc}}                     % key setup
% \setKV[simplekv]{ height = 6 }                             % benchmarked
% \end{lstlisting}
%
% \paragraph{\pkg{yax}} is over eighteen times slower. It has a pretty strange
% syntax, imho, and again a direct equivalent is hard to define. It has the
% premature unbracing bug, too. Also somehow loading \pkg{yax} broke
% \pkg{options} for me. The tested definition was:
% \begin{lstlisting}[belowskip=0pt]
% \usepackage{yax}
% \defactiveparameter yax {\storevalue\myheight yax:height } % key setup
% \setparameterlist{yax}{ height = 6 }                       % benchmarked
% \end{lstlisting}
%
% \begin{table}
%   \def\fnsym{\textcolor{red!80!black}{*}}%
%   \sisetup{round-precision=1, round-mode=places}%
%   \begingroup
%   \centering
%   \newcommand*\yes{\textcolor{red!80!black}  {yes}}^^A
%   \newcommand*\no {\textcolor{green!80!black}{no}}^^A
%   \caption[Comparison of \kv\ packages]
%     {^^A
%       Comparison of \kv\ packages. The packages are ordered from
%       fastest to slowest for one \kv\ pair. Benchmarking was done using
%       \pkg{l3benchmark} and the scripts in the \file{Benchmarks} folder
%       of the \href{https://github.com/Skillmon/tex_expkv}{git repository}.
%       The columns $p_i$ are the polynomial coefficients of a linear fit to the
%       run-time, $p_0$ can be interpreted as the overhead for initialisation
%       and $p_1$ the cost per key. The $T_0$ column is the actual mean ops
%       needed for an empty list argument, as the linear fit doesn't match that
%       point well in general. The column ``BB'' lists whether the
%       parsing is affected by some sort of brace bug, ``CF'' stands for
%       category code fragile and lists whether the parsing breaks with active
%       commas or equal signs.^^A
%       \label{tab:comp}^^A
%     }
%   \begin{tabular}
%       {>{\collectcell\pkg}l<{\endcollectcell}*3{S[table-format=4.1]}ccc}
%     \toprule
%     \rmfamily Package & {$p_1$} & {$p_0$} & {$T_0$}& BB & CF & Date \\
%     \midrule
%     keyval    &   13.539 &    1.844 &    6.969 & \yes & \yes & 2014-10-28 \\
%     \expkv    &   26.411 &    2.391 &   12.610 & \no  & \no  & 2020-02-22 \\
%     options   &   25.532 &   10.518 &   21.393 & \yes & \yes & 2015-03-01 \\
%     pgfkeys   &   26.444 &   40.939 &   55.790 & \yes & \yes & 2020-01-08 \\
%     kvsetkeys & {\fnsym} & {\fnsym} &   41.620 & \no  & \no  & 2019-12-15 \\
%     l3keys    &  114.456 &   35.357 &   52.667 & \no  & \no  & 2020-02-14 \\
%     simplekv  &  160.523 &   10.809 &    8.698 & \yes & \yes & 2017-08-08 \\
%     xkeyval   &  260.872 &  180.435 &  164.167 & \yes & \yes & 2014-12-03 \\
%     yax       &  507.643 &   62.863 &  123.533 & \yes & \yes & 2010-01-22 \\
%     ltxkeys   & 3932.759 & 4737.818 & 5883.000 & \no  & \no  & 2012-11-17 \\
%     \bottomrule
%   \end{tabular}
%   \par
%   \endgroup
%   \medskip
%   \fnsym For \pkg{kvsetkeys} the linear model used for the other
%   packages is a poor fit, \pkg{kvsetkeys} seems to have approximately
%   quadratic run-time, the coefficients of the second degree polynomial fit are
%   $p_2=\num{9.911}$, $p_1=\num{40.523}$, and $p_0=\num{61.500}$. Of course the
%   other packages might not really have linear run-time, but at least from 1~to
%   20~keys the fits don't seem too bad (the maximum ratio $p_2/p_1$
%   for the other packages is \num{9.697e-3}). If one extrapolates the fits
%   for 100 \kv\ pairs one finds that most of them match pretty well, the
%   exception being \pkg{ltxkeys}, which behaves quadratic as well with
%   $p_2=\num{31.685}$, $p_1=\num{3267.368}$, and $p_0=\num{7177.582}$.
% \end{table}
%
% \subsection{Examples}
%
% \subsubsection{Standard Use-Case}
%
% Say we have a macro for which we want to create a \kv\ interface. The macro
% has a parameter, which is stored in the dimension |\ourdim| having a default
% value from its initialization. Now we want to be able to change that dimension
% with the |width| key to some specified value. For that we'd do
% \begin{lstlisting}
% \newdimen\ourdim
% \ourdim=150pt
% \protected\ekvdef{our}{width}{\ourdim=#1\relax}
% \end{lstlisting}
% as you can see, we use the set |our| here. We want the key to behave different
% if no value is specified. In that case the key should not use its initial
% value, but be smart and determine the available space from
% \cs[no-index]{hsize}, so we also define
% \begin{lstlisting}
% \protected\ekvdefNoVal{our}{width}{\ourdim=.9\hsize}
% \end{lstlisting}
% Now we set up our macro to use this \kv\ interface
% \begin{lstlisting}
% \protected\def\ourmacro#1{\begingroup\ekvset{our}{#1}\the\ourdim\endgroup}
% \end{lstlisting}
% Finally we can use our macro like in the following\\
% \begin{minipage}{6cm}
% \begin{lstlisting}
% \ourmacro{}\par
% \ourmacro{width}\par
% \ourmacro{width=5pt}\par
% \end{lstlisting}
% \end{minipage}^^A
% \begin{minipage}{\dimexpr\linewidth-6cm\relax}
%   \newdimen\ourdim\ourdim=150pt
%   \protected\ekvdef{our}{width}{\ourdim=#1\relax}^^A
%   \protected\ekvdefNoVal{our}{width}{\ourdim=.9\hsize}^^A
%   \protected\def\ourmacro#1^^A
%     {\begingroup\ekvset{our}{#1}\the\ourdim\endgroup}^^A
%   \ourmacro{}\par
%   \ourmacro{width}\par
%   \ourmacro{width=5pt}\par
% \end{minipage}
%
% \paragraph{The same key using \protect\expkvd}
% Using \expkvd\ we can set up the equivalent key using a \kv\ interface, after
% the following we could use |\ourmacro| in the same way as above. \expkvd\ will
% allocate and initialise |\ourdim| and define the |width| key |\protected| for
% us, so the result will be exactly the same -- with the exception that the
% default will use |\ourdim=.9\hsize\relax| instead.
% \begin{lstlisting}
% \input expkv-def                 % or \usepackage{expkv-def}
% \ekvdefinekeys{our}
%   {
%     dimen    width = \ourdim,
%     qdefault width = .9\hsize,
%     initial  width = 150pt
%   }
% \end{lstlisting}
%
% \subsubsection{An Expandable \kv\ Macro Using \cs[no-index]{ekvsneak}}
% \label{sec:sneakex}
%
% Let's set up an expandable macro, that uses a \kv\ interface. The problems
% we'll face for this are:
% \begin{enumerate}
%   \item ignoring duplicate keys
%   \item default values for keys which weren't used
%   \item providing the values as the correct argument to a macro (ordered)
% \end{enumerate}
%
% First we need to decide which \kv\ parsing macro we want to do this with,
% |\ekvset| or |\ekvparse|. For this example we also want to show the
% usage of |\ekvsneak|, hence we'll choose |\ekvset|. And we'll have to use
% |\ekvset| such that it builds a parsable list for our macro internals. To
% gain back control after |\ekvset| is done we have to put an internal of our
% macro at the start of that list, so we use an internal key that uses
% |\ekvsneakPre| after any user input.
%
% To ignore duplicates will be easy if the value of the key used last will be
% put first in the list, so the following will use |\ekvsneakPre| for the
% user-level keys. If we wanted some key for which the first usage should be the
% the binding one we would use |\ekvsneak| instead for that key.
%
% Providing default values can be done in different ways, we'll use a simple
% approach in which we'll just put the outcome of our keys if they were used
% with default values before the parsing list terminator.
%
% Ordering the keys can be done simply by searching for a specific token for
% each argument which acts like a flag, so our sneaked out values will include
% specific tokens acting as markers.
%
% Now that we have answers for our technical problems, we have to decide what
% our example macro should do. How about we define a macro that calculates the
% sine of a number and rounds that to a specified precision? As a small extra
% this macro should understand input in radian and degree and the used
% trigonometric function should be selectable as well. For the hard part of this
% task (expandably evaluating trigonometric functions) we'll use the \pkg{xfp}
% package.
%
% First we set up our keys according to our earlier considerations and set up
% the user facing macro |\sine|. The end marker of the parsing list will be a
% |\sine@stop| token, which we don't need to define and we put our defaults
% right before it.
%
% \begin{lstlisting}
% \RequirePackage{xfp}
% \makeatletter
% \ekvdef{expex}{f}{\ekvsneakPre{\f{#1}}}
% \ekvdef{expex}{round}{\ekvsneakPre{\rnd{#1}}}
% \ekvdefNoVal{expex}{degree}{\ekvsneakPre{\deg{d}}}
% \ekvdefNoVal{expex}{radian}{\ekvsneakPre{\deg{}}}
% \ekvdefNoVal{expex}{internal}{\ekvsneakPre{\sine@rnd}}
% \newcommand*\sine[2]
%   {\ekvset{expex}{#1,internal}\rnd{3}\deg{d}\f{sin}\sine@stop{#2}}
% \end{lstlisting}
% For the sake of simplicity we defined the macro |\sine| with two mandatory
% arguments, the first being the \kv\ list, the second the argument to the
% trigonometric function. We could've used \pkg{xparse}'s facilities here to
% define an expandable macro which takes an optional argument instead.
%
% Now we need to define some internal macros to extract the value of each key's
% last usage (remember that this will be the group after the first special
% flag-token). For that we use one delimited macro per key.
% \begin{lstlisting}
% \def\sine@rnd#1\rnd#2#3\sine@stop{\sine@deg#1#3\sine@stop{#2}}
% \def\sine@deg#1\deg#2#3\sine@stop{\sine@f#1#3\sine@stop{#2}}
% \def\sine@f#1\f#2#3\sine@stop{\sine@final{#2}}
% \end{lstlisting}
% After the macros |\sine@rnd|, |\sine@deg|, and |\sine@f| the macro
% |\sine@final| will see
% \texttt
%   {^^A
%     \cs[no-index]{sine@final}^^A
%     \marg{f}\marg{degree/radian}\marg{round}\marg{num}^^A
%   }.
% Now |\sine@final| has to expandably deal with those arguments such that the
% |\fpeval| macro of \pkg{xfp} gets the correct input. Luckily this is pretty
% straight forward in this example. In |\fpeval| the trigonometric functions
% have names such as |sin| or |cos| and the degree taking variants |sind| or
% |cosd|. And since the |degree| key puts a |d| in |#2| and the |radian| key
% leaves |#2| empty all we have to do to get the correct function name is stick
% the two together.
% \begin{lstlisting}
% \newcommand*\sine@final[4]{\fpeval{round(#1#2(#4),#3)}}
% \makeatother
% \end{lstlisting}
% Let's test our macro:\\
% \begin{minipage}{.7\linewidth}
%   \begin{lstlisting}[gobble=4]
%   \sine{}{60}\par
%   \sine{round=10}{60}\par
%   \sine{f=cos,radian}{pi}\par
%   \edef\myval{\sine{f=tan}{1}}\texttt{\meaning\myval}
%   \end{lstlisting}
% \end{minipage}\hfill
% \begin{minipage}{.2\linewidth}
%   \makeatletter
%   \ekvdef{expex}{f}{\ekvsneakPre{\f{#1}}}^^A
%   \ekvdef{expex}{round}{\ekvsneakPre{\rnd{#1}}}^^A
%   \ekvdefNoVal{expex}{degree}{\ekvsneakPre{\deg{d}}}^^A
%   \ekvdefNoVal{expex}{radian}{\ekvsneakPre{\deg{}}}^^A
%   \ekvdefNoVal{expex}{internal}{\ekvsneakPre{\sine@rnd}}^^A
%   \newcommand*\sine[2]
%     {\ekvset{expex}{#1,internal}\rnd{3}\deg{d}\f{sin}\sine@stop{#2}}^^A
%   \def\sine@rnd#1\rnd#2#3\sine@stop{\sine@deg#1#3\sine@stop{#2}}^^A
%   \def\sine@deg#1\deg#2#3\sine@stop{\sine@f#1#3\sine@stop{#2}}^^A
%   \def\sine@f#1\f#2#3\sine@stop{\sine@final{#2}}^^A
%   \newcommand*\sine@final[4]{\fpeval{round(#1#2(#4),#3)}}^^A
%   \makeatother
%   \sine{}{60}\par
%   \sine{round=10}{60}\par
%   \sine{f=cos,radian}{pi}\par
%   \edef\myval{\sine{f=tan}{1}}\texttt{\meaning\myval}
% \end{minipage}
%
%
% \subsection{Error Messages}
%
% \expkv\ should only send messages in case of errors, there are no warnings and
% no info messages. In this subsection those errors are listed.
%
% \subsubsection{Load Time}
%
% \file{expkv.tex} checks whether \eTeX\ is available. If it isn't, an error
% will be thrown using |\errmessage|:
% \begin{lstlisting}
% ! expkv Error: e-TeX required.
% \end{lstlisting}
%
% \subsubsection{Defining Keys}
%
% If you get any error from \expkv\ while you're trying to define a key, the
% definition will be aborted and gobbled.
%
% If you try to define a key with an empty set name you'll get:
% \begin{lstlisting}
% ! expkv Error: empty set name not allowed.
% \end{lstlisting}
%
% Similarly, if you try to define a key with an empty key name:
% \begin{lstlisting}
% ! expkv Error: empty key name not allowed.
% \end{lstlisting}
%
% Both of these messages are done in a way that doesn't throw additional errors
% due to |\global|, |\long|, etc., not being used correctly if you prefixed one
% of the defining macros.
%
% \subsubsection{Using Keys}
%
% This subsubsection contains the errors thrown during |\ekvset|. The errors are
% thrown in an expandable manner by providing an undefined macro. In the
% following messages \lstinline|<key>| gets replaced with the problematic key's
% name, and \lstinline|<set>| with the corresponding set. If any errors during
% \kv\ handling are encountered, the entry in the comma separated list will be
% omitted after the error is thrown and the next \kv\ pair will be parsed.
%
% If you're using an undefined key you'll get:
% \begin{lstlisting}
% ! Undefined control sequence.
% <argument> \! expkv Error:
%                            unknown key (`<key>', set `<set>').
% \end{lstlisting}
%
% If you're using a key for which only a normal version and no |NoVal| version
% is defined, but don't provide a value, you'll get:
% \begin{lstlisting}
% ! Undefined control sequence.
% <argument> \! expkv Error:
%                            value required (`<key>', set `<set>').
% \end{lstlisting}
%
% If you're using a key for which only a |NoVal| version and no normal version
% is defined, but provide a value, you'll get:
% \begin{lstlisting}
% ! Undefined control sequence.
% <argument> \! expkv Error:
%                            value forbidden (`<key>', set `<set>').
% \end{lstlisting}
%
% If you're using a set for which you never executed one of the defining macros
% from \autoref{sec:define} you'll get a low level \TeX\ error, as that isn't
% actively tested by the parser (and hence will lead to undefined behaviour and
% not be gracefully ignored). The error will look like
% \begin{lstlisting}
% ! Missing \endcsname inserted.
% <to be read again>
%                    \! expkv Error: Set `<set>' undefined.
% \end{lstlisting}
%
% \subsection{License}
%
% Copyright \textcopyright\ 2020\unless\ifnum\year=2020--\the\year\fi\
% Jonathan P. Spratte
%
% \medskip\noindent
% This work may be distributed and/or modified under the conditions of the
% \LaTeX\ Project Public License (LPPL), either version 1.3c of this license or
% (at your option) any later version. The latest version of this license is in
% the file:
%
% \url{http://www.latex-project.org/lppl.txt}
%
% \noindent
% This work is ``maintained'' (as per LPPL maintenance status) by
%
% \mbox{Jonathan P. Spratte}.
%
% \end{documentation}^^A=<<
%
% \begin{implementation}^^A>>=
%
% \clearpage
%
% \section{Implementation}^^A>>=
%^^A the LaTeX package >>=
% \subsection{The \LaTeX\ Package}
% First we set up the \LaTeX\ package. That one doesn't really do much except
% |\input|ting the generic code and identifying itself as a package.
% \gobbledocstriptag
%<*pkg>
%    \begin{macrocode}
\def\ekv@tmp
  {%
    \ProvidesFile{expkv.tex}%
      [\ekvDate\space v\ekvVersion\space an expandable key=val implementation]%
  }
% \iffalse meta-comment
%
% File: expkv.dtx Copyright (C) 2020 Jonathan P. Spratte
%
% This work  may be  distributed and/or  modified under  the conditions  of the
% LaTeX Project Public License (LPPL),  either version 1.3c  of this license or
% (at your option) any later version.  The latest version of this license is in
% the file:
%
%   http://www.latex-project.org/lppl.txt
%
% ------------------------------------------------------------------------------
%
%<*driver>^^A>>=
\def\nameofplainTeX{plain}
\ifx\fmtname\nameofplainTeX\else
  \expandafter\begingroup
\fi
\input l3docstrip.tex
\askforoverwritefalse
\preamble

--------------------------------------------------------------
expkv -- an expandable key=val implementation
E-mail: jspratte@yahoo.de
Released under the LaTeX Project Public License v1.3c or later
See http://www.latex-project.org/lppl.txt
--------------------------------------------------------------

Copyright (C) 2020 Jonathan P. Spratte

This  work may be  distributed and/or  modified under  the conditions  of the
LaTeX Project Public License (LPPL),  either version 1.3c  of this license or
(at your option) any later version.  The latest version of this license is in
the file:

  http://www.latex-project.org/lppl.txt

This work is "maintained" (as per LPPL maintenance status) by
  Jonathan P. Spratte.

This work consists of the file  expkv.dtx
and the derived files           expkv.pdf
                                expkv.sty
                                expkv.tex

\endpreamble
% stop docstrip adding \endinput
\postamble
\endpostamble
\generate{\file{expkv.sty}{\from{expkv.dtx}{pkg}}}
\generate{\file{expkv.tex}{\from{expkv.dtx}{tex}}}
\ifx\fmtname\nameofplainTeX
  \expandafter\endbatchfile
\else
  \expandafter\endgroup
\fi
%
\IfFileExists{expkv.tex}{% \iffalse meta-comment
%
% File: expkv.dtx Copyright (C) 2020 Jonathan P. Spratte
%
% This work  may be  distributed and/or  modified under  the conditions  of the
% LaTeX Project Public License (LPPL),  either version 1.3c  of this license or
% (at your option) any later version.  The latest version of this license is in
% the file:
%
%   http://www.latex-project.org/lppl.txt
%
% ------------------------------------------------------------------------------
%
%<*driver>^^A>>=
\def\nameofplainTeX{plain}
\ifx\fmtname\nameofplainTeX\else
  \expandafter\begingroup
\fi
\input l3docstrip.tex
\askforoverwritefalse
\preamble

--------------------------------------------------------------
expkv -- an expandable key=val implementation
E-mail: jspratte@yahoo.de
Released under the LaTeX Project Public License v1.3c or later
See http://www.latex-project.org/lppl.txt
--------------------------------------------------------------

Copyright (C) 2020 Jonathan P. Spratte

This  work may be  distributed and/or  modified under  the conditions  of the
LaTeX Project Public License (LPPL),  either version 1.3c  of this license or
(at your option) any later version.  The latest version of this license is in
the file:

  http://www.latex-project.org/lppl.txt

This work is "maintained" (as per LPPL maintenance status) by
  Jonathan P. Spratte.

This work consists of the file  expkv.dtx
and the derived files           expkv.pdf
                                expkv.sty
                                expkv.tex

\endpreamble
% stop docstrip adding \endinput
\postamble
\endpostamble
\generate{\file{expkv.sty}{\from{expkv.dtx}{pkg}}}
\generate{\file{expkv.tex}{\from{expkv.dtx}{tex}}}
\ifx\fmtname\nameofplainTeX
  \expandafter\endbatchfile
\else
  \expandafter\endgroup
\fi
%
\IfFileExists{expkv.tex}{\input{expkv.tex}}{}
\ProvidesFile{expkv.dtx}
  [\csname ekvDate\endcsname\ an expandable key=val implementation]
\PassOptionsToPackage{full}{textcomp}
\documentclass{l3doc}
\RequirePackage[oldstylenums,nott]{kpfonts}
\input{glyphtounicode}
\pdfgentounicode=1
\RequirePackage{xfp} ^^A required for an example
\RequirePackage{listings}
\RequirePackage{booktabs}
\RequirePackage{array}
\RequirePackage{collcell}
\RequirePackage{siunitx}
\RequirePackage{xcolor}
\RequirePackage{caption}
\RequirePackage{microtype}
\RequirePackage{accsupp}
\lstset
  {
    ,flexiblecolumns=false
    ,basewidth=.53em
    ,gobble=2
    ,basicstyle=\fontfamily{jkp}\itshape
    ,morekeywords=^^A
      {^^A
        \ekvdef,\ekvdefNoVal,\ekvlet,\ekvletNoVal,\ekvletkv,\ekvletkvNoVal,^^A
        \ekvset,\ekvparse,\ekvVersion,\ekvDate,\ekvifdefined,^^A
        \ekvifdefinedNoVal,\ekvbreak,\ekvbreakPreSneak,\ekvbreakPostSneak,^^A
        \ekvsneak,\ekvsneakPre,\ekvchangeset
      }
    ,morecomment=[l]\%
    ,commentstyle=\color[gray]{0.4}
    ,literate={\{}{{\CodeSymbol\{}}{1}
              {\}}{{\CodeSymbol\}}}{1}
    ^^A,literate=*{<key>}{\key}{4}{<set>}{\set}{4}
  }
\newcommand*\CodeSymbol[1]{\textbf{#1}}
\RequirePackage{randtext}
\let\metaORIG\meta
\protected\def\meta #1{\texttt{\metaORIG{#1}}}
\renewcommand*\thefootnote{\fnsymbol{footnote}}
\definecolor{expkvred}{HTML}{9F393D}
\colorlet{expkvgrey}{black!75}
\makeatletter
\newcommand*\expkv
  {^^A
    \texorpdfstring
      {^^A
        \mbox
          {^^A
            \BeginAccSupp{ActualText=expkv}^^A
            \href{https://github.com/Skillmon/tex_expkv}
              {^^A
                \rmfamily
                \bfseries
                {\color{expkvgrey}e\kern-.05em x\kern-.05em}^^A
                \lower.493ex
                  \hbox{{\color{expkvgrey}P}\kern-.1em{\color{expkvred}k}}^^A
                \kern-.18em{\color{expkvred}v}^^A
              }^^A
            \EndAccSupp{}^^A
          }^^A
      }
      {expkv}^^A
  }
\newcommand*\expkvd
  {^^A
    \texorpdfstring
      {^^A
        \mbox
          {^^A
            \BeginAccSupp{ActualText=expkv-def}^^A
            \rmfamily
            \bfseries
            {\color{expkvgrey}e\kern-.05em x\kern-.05em}^^A
            \lower.493ex
              \hbox{{\color{expkvgrey}P}\kern-.1em{\color{expkvred}k}}^^A
            \kern-.18em{\color{expkvred}v}^^A
            {\color{expkvgrey}^^A
              \kern.05em\rule[-.1ex]{.08em}{1.2ex}\kern.05em\textsc{def}^^A
            }^^A
            \EndAccSupp{}^^A
          }^^A
      }
      {expkv-def}^^A
  }
\newcommand\kv{\meta{key}=\meta{value}}
\newcommand\key{\meta{key}}
\newcommand\val{\meta{value}}
\newcommand\set{\meta{set}}
\hypersetup{linkcolor=red!80!black,urlcolor=purple!80!black}
\DoNotIndex{\def,\edef,\,,\=,\begingroup,\catcode,\chardef,\csname,\endcsname}
\DoNotIndex{\endgroup,\endinput,\errmessage,\expandafter,\input,\let,\long}
\DoNotIndex{\noexpand,\protected,\ProvidesFile,\ProvidesPackage,\relax,\space}
\DoNotIndex{\@,\unexpanded,\string,\expanded}
\DoNotIndex{\ifcsname}
\DoNotIndex{\ifx}
\DoNotIndex{\else}
\DoNotIndex{\fi}
\@gobble\fi ^^A ignoring \ifx and \ifcsname, but only one \fi
\@ifdefinable\gobbledocstriptag{\def\gobbledocstriptag#1>{}}
\newcommand*\pmso[1] ^^A poor man's strike out
  {^^A
    \leavevmode
    \begingroup
      \sbox0{#1}^^A
      \rlap{\vrule height .6ex depth -.5ex width \wd0\relax}^^A
      \usebox0\relax
    \endgroup
  }
\makeatother
\begin{document}
  \title
    {^^A
      \texorpdfstring
        {^^A
          \huge\expkv
          \\[\medskipamount]
          \Large an expandable \kv\ implementation^^A
        }{expkv - an expandable <key>=<value> implementation}^^A
    }
  \date{\ekvDate\space v\ekvVersion}
  \author{Jonathan P. Spratte\thanks{\protect\randomize{jspratte@yahoo.de}}}
  \DocInput{expkv.dtx}
\end{document}
%</driver>^^A=<<
% \fi
%
% \maketitle
% \renewcommand*\thefootnote{\arabic{footnote}}
%
% \begin{abstract}
% \noindent\parfillskip=0pt
% \expkv\ provides a small interface for \kv\ parsing. The parsing macro is
% fully expandable, the \meta{code} of your keys might be not. \expkv\ is pretty
% fast, but not the fastest available \kv\ solution (\pkg{keyval} for instance
% is thrice as fast, but not expandable and it might strip braces it shouldn't
% have stripped).
% \end{abstract}
%
% \tableofcontents
%
% \begin{documentation}^^A>>=
%
% \section{Documentation}
%
% \expkv\ provides an expandable \kv\ parser. The \kv\ pairs should be given as
% a comma separated list and the separator between a \key\ and the associated
% \val\ should be an equal sign. Both, the commas and the equal signs, might be
% of category 12 (other) or 13 (active). To support this is necessary as for
% example \pkg{babel} turns characters active for some languages, for instance
% the equal sign is turned active for Turkish.
%
% \expkv\ is usable as generic code or as a \LaTeX\ package. To use it, just
% use one of:
% \begin{lstlisting}
% \usepackage{expkv} % LaTeX
% \input expkv       % plainTeX
% \end{lstlisting}
% The \LaTeX\ package doesn't do more than \file{expkv.tex}, except calling
% |\ProvidesPackage| and setting things up such that \file{expkv.tex} will use
% |\ProvidesFile|.
%
% \subsection{Setting up Keys}\label{sec:define}
% \expkv\ provides a rather simple approach to setting up keys, similar to
% \pkg{keyval}. However there is an auxiliary package named
% \href{https://github.com/Skillmon/tex_expkv-def}{\expkvd} (the files use
% \file{expkv-def} instead because \string| isn't allowed in package names by
% CTAN) which provides a more sophisticated interface, similar to well
% established packages like \pkg{pgfkeys} or \pkg{l3keys}.
%
% Keys in \expkv\ (as in almost all other \kv\ implementations) belong to a
% \emph{set} such that different sets can contain keys of the same name. Unlike
% many other implementations \expkv\ doesn't provide means to set a default
% value, instead we have keys that take values and keys that don't (the latter
% are called |NoVal| keys by \expkv), but both can have the same name (on the
% user level).
%
% The following macros are available to define new keys. Those macros containing
% ``|def|'' in their name can be prefixed by anything allowed to prefix |\def|,
% prefixes allowed for |\let| can prefix those with ``|let|'' in their name,
% accordingly. Neither \set\ nor \key\ are allowed to be empty for new keys and
% must not contain a |\par| or tokens that expand to it -- they must be legal
% inside of |\csname ...\endcsname|.
%
% \begin{function}{\ekvdef}
%   \begin{syntax}
%     \cs{ekvdef}\marg{set}\marg{key}\marg{code}
%   \end{syntax}
%   Defines a \key\ taking a value in a \set\ to expand to \meta{code}. In
%   \meta{code} you can use |#1| to refer to the given value.
% \end{function}
%
% \begin{function}{\ekvdefNoVal}
%   \begin{syntax}
%     \cs{ekvdefNoVal}\marg{set}\marg{key}\marg{code}
%   \end{syntax}
%   Defines a no value taking \key\ in a \set\ to expand to \meta{code}.
% \end{function}
%
% \begin{function}{\ekvlet}
%   \begin{syntax}
%     \cs{ekvlet}\marg{set}\marg{key}\meta{cs}
%   \end{syntax}
%   Let the value taking \key\ in \set\ to \meta{cs}, there are no checks on
%   \meta{cs} enforced.
% \end{function}
%
% \begin{function}{\ekvletNoVal}
%   \begin{syntax}
%     \cs{ekvletNoVal}\marg{set}\marg{key}\meta{cs}
%   \end{syntax}
%   Let the no value taking \key\ in \set\ to \meta{cs}, it is not checked
%   whether \meta{cs} exists or that it takes no parameter.
% \end{function}
%
% \begin{function}{\ekvletkv}
%   \begin{syntax}
%     \cs{ekvletkv}\marg{set}\marg{key}\marg{set2}\marg{key2}
%   \end{syntax}
%   Let the \key\ in \set\ to \meta{key2} in \meta{set2}, it is not checked
%   whether that second key exists.
% \end{function}
%
% \begin{function}{\ekvletkvNoVal}
%   \begin{syntax}
%     \cs{ekvletkvNoVal}\marg{set}\marg{key}\marg{set2}\marg{key2}
%   \end{syntax}
%   Let the \key\ in \set\ to \meta{key2} in \meta{set2}, it is not checked
%   whether that second key exists.
% \end{function}
%
% \subsection{Parsing Keys}
%
% \begin{function}{\ekvset}
%   \begin{syntax}
%     \cs{ekvset}\marg{set}\{\kv,\ldots\}
%   \end{syntax}
%   Splits \kv\ pairs on commas. From both \key\ and \val\ up to one space is
%   stripped from both ends, if then only a braced group remains the braces are
%   stripped as well. So |\ekvset{foo}{bar=baz}| and
%   |\ekvset{foo}{ {bar}= {baz} }| will both do
%   \texttt{\cs[no-index]{\meta{foobarcode}}\{baz\}}, so you can hide commas,
%   equal signs and spaces at the ends of either \key\ or \val\ by putting
%   braces around them. If you omit the equal sign the code of the key created
%   with the |NoVal| variants described in \autoref{sec:define} will be
%   executed. If \kv\ contains more than a single unhidden equal sign, it will
%   be split at the first one and the others are considered part of the value.
%   |\ekvset| should be nestable.
% \end{function}
%
% \begin{function}{\ekvparse}
%   \begin{syntax}
%     \cs{ekvparse}\meta{cs1}\meta{cs2}\{\kv,\ldots\}
%   \end{syntax}
%   This macro parses the \kv\ pairs and provides those list elements which are
%   only keys as the argument to \meta{cs1}, and those which are a \kv\ pair to
%   \meta{cs2} as two arguments. It is fully expandable as well and returns the
%   parsed list in |\unexpanded|, which has no effect outside of an |\expanded|
%   or |\edef| context\footnotemark.
%   If you need control over the necessary steps of expansion you
%   can use |\expanded| around it.
%
%   |\ekvbreak|, |\ekvsneak|, and |\ekvchangeset| and their relatives don't work
%   in |\ekvparse|.  It is analogue to \pkg{expl3}'s |\keyval_parse:NNn|, but
%   not with the same parsing rules -- |\keyval_parse:NNn| throws an error on
%   multiple equal signs per \kv\ pair and on empty \key\ names in a \kv\ pair,
%   both of which |\ekvparse| doesn't deal with.
%
%   As a small example:
% \begin{lstlisting}
% \ekvparse\handlekey\handlekeyval{foo = bar, key, baz={zzz}}
% \end{lstlisting}
%   would expand to
% \begin{lstlisting}
% \handlekeyval{foo}{bar}\handlekey{key}\handlekeyval{baz}{zzz}
% \end{lstlisting}
%   and afterwards |\handlekey| and |\handlekeyval| would have to further handle
%   the \key. There are no macros like these two contained in \expkv, you have
%   to set them up yourself if you want to use |\ekvparse| (of course the names
%   might differ). If you need the results of |\ekvparse| as the argument for
%   another macro, you should use |\expanded| as only then the input stream will
%   contain the output above:
% \begin{lstlisting}
% \expandafter\handle\expanded{\ekvparse\k\kv{foo = bar, key, baz={zzz}}}
% \end{lstlisting}
%   would expand to
% \begin{lstlisting}
% \handle\kv{foo}{bar}\k{key}\kv{baz}{zzz}
% \end{lstlisting}
% \end{function}
% \footnotetext 
%   {^^A
%     This is a change in behaviour, previously (v0.3 and before) \cs{ekvparse}
%     would expand in exactly two steps. This isn't always necessary, but makes
%     the parsing considerably slower. If this is necessary for your application
%     you can put an \cs[no-index]{expanded} around it and will still be faster
%     since you need only a single \cs[no-index]{expandafter} this way.^^A
%   }.
%
% \subsection{Miscellaneous}
%
% \subsubsection{Other Macros}
%
% \expkv\ provides some other macros which might be of interest.
%
% \begin{function}{\ekvVersion,\ekvDate}
%   These two macros store the version and date of the package.
% \end{function}
% \begin{function}{\ekvifdefined,\ekvifdefinedNoVal}
%   \begin{syntax}
%     \cs{ekvifdefined}\marg{set}\marg{key}\marg{true}\marg{false}
%     \cs{ekvifdefinedNoVal}\marg{set}\marg{key}\marg{true}\marg{false}
%   \end{syntax}
%   These two macros test whether there is a \key\ in \set. It is false if
%   either a hash table entry doesn't exist for that key or its meaning is
%   |\relax|.
% \end{function}
%
% \begin{function}{\ekvbreak,\ekvbreakPreSneak,\ekvbreakPostSneak}
%   \begin{syntax}
%     \cs{ekvbreak}\marg{after}
%   \end{syntax}
%   Gobbles the remainder of the current |\ekvset| macro and its argument list
%   and reinserts \meta{after}. So this can be used to break out of |\ekvset|.
%   The first variant will also gobble anything that has been sneaked out using
%   |\ekvsneak| or |\ekvsneakPre|, while |\ekvbreakPreSneak| will put
%   \meta{after} before anything that has been smuggled and |\ekvbreakPostSneak|
%   will put \meta{after} after the stuff that has been sneaked out.
% \end{function}
%
% \begin{function}{\ekvsneak,\ekvsneakPre}
%   \begin{syntax}
%     \cs{ekvsneak}\marg{after}
%   \end{syntax}
%   Puts \meta{after} after the effects of |\ekvset|. The first variant will put
%   \meta{after} after any other tokens which might have been sneaked before,
%   while |\ekvsneakPre| will put \meta{after} before other smuggled stuff.
%   This reads and reinserts the remainder of the current |\ekvset| macro and
%   its argument list to do its job. A small usage example is shown in
%   \autoref{sec:sneakex}.
% \end{function}
%
% \begin{function}{\ekvchangeset}
%   \begin{syntax}
%     \cs{ekvchangeset}\marg{new-set}
%   \end{syntax}
%   Replaces the current set with \meta{new-set}, so for the rest of the current
%   |\ekvset| call, that call behaves as if it was called with
%   \texttt{\cs[no-index]{ekvset}\marg{new-set}}. Just like |\ekvsneak| this
%   reads and reinserts the remainder of the current |\ekvset| macro to do its
%   job. It is comparable to using \texttt{\meta{key}/.cd} in \pkg{pgfkeys}.
% \end{function}
%
% \bigskip
%
% \begin{function}{\ekv@name,\ekv@name@set,\ekv@name@key}
%   \begin{syntax}
%     \cs{ekv@name}\marg{set}\marg{key}\\
%     \cs{ekv@name@set}\marg{set}\\
%     \cs{ekv@name@key}\marg{key}
%   \end{syntax}
%   The names of the macros that correspond to a key in a set are build with
%   these macros. The default definition of |\ekv@name@set| is
%   ``\texttt{\csname ekv@name@set\endcsname{\set}}'' and the default of
%   |\ekv@name@key| is ``\texttt{\csname ekv@name@key\endcsname{\key}}''.
%   The complete name is build using |\ekv@name| which is equivalent to
%   \texttt{\cs[no-index]{ekv@name@set}\marg{set}\cs{ekv@name@key}\marg{key}}.
%   For |NoVal| keys an additional |N| gets appended irrespective of these
%   macros' definition, so their name is
%   \cs[no-index]{\csname ekv@name\endcsname{\set}{\key}N}. You might redefine
%   |\ekv@name@set| and |\ekv@name@key| locally but \emph{don't redefine}
%   |\ekv@name|!
% \end{function}
%
% \subsubsection{Bugs}
%
% Just like \pkg{keyval}, \expkv\ is bug free. But if you find \pmso{bugs}hidden
% features\footnote{Thanks, David!} you can tell me about them either via mail
% (see the first page) or directly on GitHub if you have an account there:
% \url{https://github.com/Skillmon/tex_expkv}
%
% \subsubsection{Comparisons}
%
% Comparisons of speed are done with a very simple test key and the help of the
% \pkg{l3benchmark} package. The key and its usage should be equivalent to
% \begin{lstlisting}
% \protected\ekvdef{test}{height}{\def\myheight{#1}}
% \ekvset{test}{ height = 6 }
% \end{lstlisting}
% and only the usage of the key, not its definition, is benchmarked. For the
% impatient, the essence of these comparisons regarding speed and buggy
% behaviour is contained in \autoref{tab:comp}.
%
% As far as I know \expkv\ is the only fully expandable \kv\ parser. I tried to
% compare \expkv\ to every \kv\ package listed on
% \href{https://ctan.org/topic/keyval}{CTAN}, however, one might notice that
% some of those are missing from this list. That's because I didn't get the
% others to work due to bugs, or because they just provide wrappers around other
% packages in this list.
%
% In this subsubsection is no benchmark of |\ekvparse| and |\keyval_parse:NNn|
% contained, as most other packages don't provide equivalent features to my
% knowledge. |\ekvparse| is slightly faster than |\ekvset|, but keep in mind
% that it does less. The same is true for |\keyval_parse:NNn| compared to
% |\keys_set:nn| of \pkg{expl3} (where the difference is much bigger).
%
% \paragraph{\pkg{keyval}} is about two times faster and has a comparable
% feature set just a slightly different way how it handles keys without values.
% That might be considered a drawback, as it limits the versatility, but also as
% an advantage, as it might reduce doubled code. Keep in mind that as soon as
% someone loads \pkg{xkeyval} the performance of \pkg{keyval} gets replaced by
% \pkg{xkeyval}'s.
%
% Also \pkg{keyval} has a bug, which unfortunately can't really be resolved
% without breaking backwards compatibility for \emph{many} documents, namely it
% strips braces from the argument before stripping spaces if the argument isn't
% surrounded by spaces, also it might strip more than one set of braces. Hence
% all of the following are equivalent in their outcome, though the last two
% lines should result in something different than the first two:
% \begin{lstlisting}[belowskip=0pt]
% \setkeys{foo}{bar=baz}
% \setkeys{foo}{bar= {baz}}
% \setkeys{foo}{bar={ baz}}
% \setkeys{foo}{bar={{baz}}}
% \end{lstlisting}
%
% \paragraph{\pkg{xkeyval}} is roughly fourteen times slower, but it provides
% more functionality, e.g., it has choice keys, boolean keys, and so on. It
% contains the same bug as \pkg{keyval} as it has to be compatible with it by
% design (it replaces \pkg{keyval}'s frontend), but also adds even more cases in
% which braces are stripped that shouldn't be stripped, worsening the situation.
%
% \paragraph{\pkg{ltxkeys}} is over 300 times slower -- which is funny, because
% it aims to be ``[\ldots] faster than these earlier packages [referring to
% \pkg{keyval} and \pkg{xkeyval}].'' Since it aims to have a bigger feature set
% than \pkg{xkeyval}, it most definitely also has a bigger feature set than
% \expkv. Also, it can't parse |\long| input, so as soon as your values contain
% a |\par|, it'll throw errors. Furthermore, \pkg{ltxkeys} doesn't strip outer
% braces at all by design, which, imho, is a weird design choice. In addition
% \pkg{ltxkeys} loads \pkg{catoptions} which is known to introduce bugs (e.g.,
% see \url{https://tex.stackexchange.com/questions/461783}).
%
% \paragraph{\pkg{l3keys}} is almost five times slower, but has an, imho, great
% interface to define keys. It strips \emph{all} outer spaces, even if somehow
% multiple spaces ended up on either end. It offers more features, but is pretty
% much bound to \pkg{expl3} code. Whether that's a drawback is up to you.
%
% \paragraph{\pkg{pgfkeys}} is a bit more than two times slower for one key,
% but has an \emph{enormous} feature set. However, since adding additional keys
% doesn't add as much needed time for \pkg{pgfkeys} compared to \expkv, it gets
% faster than \expkv\ at around eight \kv\ pairs.
% It has the same or a very similar bug \pkg{keyval} has. The brace bug (and
% also the category fragility) can be fixed by \pkg{pgfkeyx}, but this package
% was last updated in 2012 and it slows down |\pgfkeys| by factor~$8$. Also I
% don't know whether this might introduce new bugs.
%
% \paragraph{\pkg{kvsetkeys} with \pkg{kvdefinekeys}} is about three times
% slower, but it works even if commas and equals have category codes different
% from 12 (just as some other packages in this list). Else the features of the
% keys are equal to those of \pkg{keyval}, the parser has more features, though.
%
% \paragraph{\pkg{options}} is a bit slower for only a single value, but gets a
% tad faster than \expkv\ at around 10 \kv\ pairs. It has a much bigger feature
% set. Unfortunately it also suffers from the premature unbracing bug
% \pkg{keyval} has.
%
% \paragraph{\pkg{simplekv}} is hard to compare because I don't speak French (so
% I don't understand the documentation) and from what I can see, there is no
% direct way to define the equivalent test key. Nevertheless, I tested the
% closest possible equivalent of my test key while siding for \pkg{simplekv}'s
% design not forcing something into it it doesn't seem to be designed for. It is
% more than five times slower and has hard to predict behaviour regarding brace
% and space stripping, similar to \pkg{keyval}. The tested definition was:
% \begin{lstlisting}[belowskip=0pt]
% \usepackage{simplekv}
% \setKVdefault[simplekv]{height={ abc}}                     % key setup
% \setKV[simplekv]{ height = 6 }                             % benchmarked
% \end{lstlisting}
%
% \paragraph{\pkg{yax}} is over eighteen times slower. It has a pretty strange
% syntax, imho, and again a direct equivalent is hard to define. It has the
% premature unbracing bug, too. Also somehow loading \pkg{yax} broke
% \pkg{options} for me. The tested definition was:
% \begin{lstlisting}[belowskip=0pt]
% \usepackage{yax}
% \defactiveparameter yax {\storevalue\myheight yax:height } % key setup
% \setparameterlist{yax}{ height = 6 }                       % benchmarked
% \end{lstlisting}
%
% \begin{table}
%   \def\fnsym{\textcolor{red!80!black}{*}}%
%   \sisetup{round-precision=1, round-mode=places}%
%   \begingroup
%   \centering
%   \newcommand*\yes{\textcolor{red!80!black}  {yes}}^^A
%   \newcommand*\no {\textcolor{green!80!black}{no}}^^A
%   \caption[Comparison of \kv\ packages]
%     {^^A
%       Comparison of \kv\ packages. The packages are ordered from
%       fastest to slowest for one \kv\ pair. Benchmarking was done using
%       \pkg{l3benchmark} and the scripts in the \file{Benchmarks} folder
%       of the \href{https://github.com/Skillmon/tex_expkv}{git repository}.
%       The columns $p_i$ are the polynomial coefficients of a linear fit to the
%       run-time, $p_0$ can be interpreted as the overhead for initialisation
%       and $p_1$ the cost per key. The $T_0$ column is the actual mean ops
%       needed for an empty list argument, as the linear fit doesn't match that
%       point well in general. The column ``BB'' lists whether the
%       parsing is affected by some sort of brace bug, ``CF'' stands for
%       category code fragile and lists whether the parsing breaks with active
%       commas or equal signs.^^A
%       \label{tab:comp}^^A
%     }
%   \begin{tabular}
%       {>{\collectcell\pkg}l<{\endcollectcell}*3{S[table-format=4.1]}ccc}
%     \toprule
%     \rmfamily Package & {$p_1$} & {$p_0$} & {$T_0$}& BB & CF & Date \\
%     \midrule
%     keyval    &   13.539 &    1.844 &    6.969 & \yes & \yes & 2014-10-28 \\
%     \expkv    &   26.411 &    2.391 &   12.610 & \no  & \no  & 2020-02-22 \\
%     options   &   25.532 &   10.518 &   21.393 & \yes & \yes & 2015-03-01 \\
%     pgfkeys   &   26.444 &   40.939 &   55.790 & \yes & \yes & 2020-01-08 \\
%     kvsetkeys & {\fnsym} & {\fnsym} &   41.620 & \no  & \no  & 2019-12-15 \\
%     l3keys    &  114.456 &   35.357 &   52.667 & \no  & \no  & 2020-02-14 \\
%     simplekv  &  160.523 &   10.809 &    8.698 & \yes & \yes & 2017-08-08 \\
%     xkeyval   &  260.872 &  180.435 &  164.167 & \yes & \yes & 2014-12-03 \\
%     yax       &  507.643 &   62.863 &  123.533 & \yes & \yes & 2010-01-22 \\
%     ltxkeys   & 3932.759 & 4737.818 & 5883.000 & \no  & \no  & 2012-11-17 \\
%     \bottomrule
%   \end{tabular}
%   \par
%   \endgroup
%   \medskip
%   \fnsym For \pkg{kvsetkeys} the linear model used for the other
%   packages is a poor fit, \pkg{kvsetkeys} seems to have approximately
%   quadratic run-time, the coefficients of the second degree polynomial fit are
%   $p_2=\num{9.911}$, $p_1=\num{40.523}$, and $p_0=\num{61.500}$. Of course the
%   other packages might not really have linear run-time, but at least from 1~to
%   20~keys the fits don't seem too bad (the maximum ratio $p_2/p_1$
%   for the other packages is \num{9.697e-3}). If one extrapolates the fits
%   for 100 \kv\ pairs one finds that most of them match pretty well, the
%   exception being \pkg{ltxkeys}, which behaves quadratic as well with
%   $p_2=\num{31.685}$, $p_1=\num{3267.368}$, and $p_0=\num{7177.582}$.
% \end{table}
%
% \subsection{Examples}
%
% \subsubsection{Standard Use-Case}
%
% Say we have a macro for which we want to create a \kv\ interface. The macro
% has a parameter, which is stored in the dimension |\ourdim| having a default
% value from its initialization. Now we want to be able to change that dimension
% with the |width| key to some specified value. For that we'd do
% \begin{lstlisting}
% \newdimen\ourdim
% \ourdim=150pt
% \protected\ekvdef{our}{width}{\ourdim=#1\relax}
% \end{lstlisting}
% as you can see, we use the set |our| here. We want the key to behave different
% if no value is specified. In that case the key should not use its initial
% value, but be smart and determine the available space from
% \cs[no-index]{hsize}, so we also define
% \begin{lstlisting}
% \protected\ekvdefNoVal{our}{width}{\ourdim=.9\hsize}
% \end{lstlisting}
% Now we set up our macro to use this \kv\ interface
% \begin{lstlisting}
% \protected\def\ourmacro#1{\begingroup\ekvset{our}{#1}\the\ourdim\endgroup}
% \end{lstlisting}
% Finally we can use our macro like in the following\\
% \begin{minipage}{6cm}
% \begin{lstlisting}
% \ourmacro{}\par
% \ourmacro{width}\par
% \ourmacro{width=5pt}\par
% \end{lstlisting}
% \end{minipage}^^A
% \begin{minipage}{\dimexpr\linewidth-6cm\relax}
%   \newdimen\ourdim\ourdim=150pt
%   \protected\ekvdef{our}{width}{\ourdim=#1\relax}^^A
%   \protected\ekvdefNoVal{our}{width}{\ourdim=.9\hsize}^^A
%   \protected\def\ourmacro#1^^A
%     {\begingroup\ekvset{our}{#1}\the\ourdim\endgroup}^^A
%   \ourmacro{}\par
%   \ourmacro{width}\par
%   \ourmacro{width=5pt}\par
% \end{minipage}
%
% \paragraph{The same key using \protect\expkvd}
% Using \expkvd\ we can set up the equivalent key using a \kv\ interface, after
% the following we could use |\ourmacro| in the same way as above. \expkvd\ will
% allocate and initialise |\ourdim| and define the |width| key |\protected| for
% us, so the result will be exactly the same -- with the exception that the
% default will use |\ourdim=.9\hsize\relax| instead.
% \begin{lstlisting}
% \input expkv-def                 % or \usepackage{expkv-def}
% \ekvdefinekeys{our}
%   {
%     dimen    width = \ourdim,
%     qdefault width = .9\hsize,
%     initial  width = 150pt
%   }
% \end{lstlisting}
%
% \subsubsection{An Expandable \kv\ Macro Using \cs[no-index]{ekvsneak}}
% \label{sec:sneakex}
%
% Let's set up an expandable macro, that uses a \kv\ interface. The problems
% we'll face for this are:
% \begin{enumerate}
%   \item ignoring duplicate keys
%   \item default values for keys which weren't used
%   \item providing the values as the correct argument to a macro (ordered)
% \end{enumerate}
%
% First we need to decide which \kv\ parsing macro we want to do this with,
% |\ekvset| or |\ekvparse|. For this example we also want to show the
% usage of |\ekvsneak|, hence we'll choose |\ekvset|. And we'll have to use
% |\ekvset| such that it builds a parsable list for our macro internals. To
% gain back control after |\ekvset| is done we have to put an internal of our
% macro at the start of that list, so we use an internal key that uses
% |\ekvsneakPre| after any user input.
%
% To ignore duplicates will be easy if the value of the key used last will be
% put first in the list, so the following will use |\ekvsneakPre| for the
% user-level keys. If we wanted some key for which the first usage should be the
% the binding one we would use |\ekvsneak| instead for that key.
%
% Providing default values can be done in different ways, we'll use a simple
% approach in which we'll just put the outcome of our keys if they were used
% with default values before the parsing list terminator.
%
% Ordering the keys can be done simply by searching for a specific token for
% each argument which acts like a flag, so our sneaked out values will include
% specific tokens acting as markers.
%
% Now that we have answers for our technical problems, we have to decide what
% our example macro should do. How about we define a macro that calculates the
% sine of a number and rounds that to a specified precision? As a small extra
% this macro should understand input in radian and degree and the used
% trigonometric function should be selectable as well. For the hard part of this
% task (expandably evaluating trigonometric functions) we'll use the \pkg{xfp}
% package.
%
% First we set up our keys according to our earlier considerations and set up
% the user facing macro |\sine|. The end marker of the parsing list will be a
% |\sine@stop| token, which we don't need to define and we put our defaults
% right before it.
%
% \begin{lstlisting}
% \RequirePackage{xfp}
% \makeatletter
% \ekvdef{expex}{f}{\ekvsneakPre{\f{#1}}}
% \ekvdef{expex}{round}{\ekvsneakPre{\rnd{#1}}}
% \ekvdefNoVal{expex}{degree}{\ekvsneakPre{\deg{d}}}
% \ekvdefNoVal{expex}{radian}{\ekvsneakPre{\deg{}}}
% \ekvdefNoVal{expex}{internal}{\ekvsneakPre{\sine@rnd}}
% \newcommand*\sine[2]
%   {\ekvset{expex}{#1,internal}\rnd{3}\deg{d}\f{sin}\sine@stop{#2}}
% \end{lstlisting}
% For the sake of simplicity we defined the macro |\sine| with two mandatory
% arguments, the first being the \kv\ list, the second the argument to the
% trigonometric function. We could've used \pkg{xparse}'s facilities here to
% define an expandable macro which takes an optional argument instead.
%
% Now we need to define some internal macros to extract the value of each key's
% last usage (remember that this will be the group after the first special
% flag-token). For that we use one delimited macro per key.
% \begin{lstlisting}
% \def\sine@rnd#1\rnd#2#3\sine@stop{\sine@deg#1#3\sine@stop{#2}}
% \def\sine@deg#1\deg#2#3\sine@stop{\sine@f#1#3\sine@stop{#2}}
% \def\sine@f#1\f#2#3\sine@stop{\sine@final{#2}}
% \end{lstlisting}
% After the macros |\sine@rnd|, |\sine@deg|, and |\sine@f| the macro
% |\sine@final| will see
% \texttt
%   {^^A
%     \cs[no-index]{sine@final}^^A
%     \marg{f}\marg{degree/radian}\marg{round}\marg{num}^^A
%   }.
% Now |\sine@final| has to expandably deal with those arguments such that the
% |\fpeval| macro of \pkg{xfp} gets the correct input. Luckily this is pretty
% straight forward in this example. In |\fpeval| the trigonometric functions
% have names such as |sin| or |cos| and the degree taking variants |sind| or
% |cosd|. And since the |degree| key puts a |d| in |#2| and the |radian| key
% leaves |#2| empty all we have to do to get the correct function name is stick
% the two together.
% \begin{lstlisting}
% \newcommand*\sine@final[4]{\fpeval{round(#1#2(#4),#3)}}
% \makeatother
% \end{lstlisting}
% Let's test our macro:\\
% \begin{minipage}{.7\linewidth}
%   \begin{lstlisting}[gobble=4]
%   \sine{}{60}\par
%   \sine{round=10}{60}\par
%   \sine{f=cos,radian}{pi}\par
%   \edef\myval{\sine{f=tan}{1}}\texttt{\meaning\myval}
%   \end{lstlisting}
% \end{minipage}\hfill
% \begin{minipage}{.2\linewidth}
%   \makeatletter
%   \ekvdef{expex}{f}{\ekvsneakPre{\f{#1}}}^^A
%   \ekvdef{expex}{round}{\ekvsneakPre{\rnd{#1}}}^^A
%   \ekvdefNoVal{expex}{degree}{\ekvsneakPre{\deg{d}}}^^A
%   \ekvdefNoVal{expex}{radian}{\ekvsneakPre{\deg{}}}^^A
%   \ekvdefNoVal{expex}{internal}{\ekvsneakPre{\sine@rnd}}^^A
%   \newcommand*\sine[2]
%     {\ekvset{expex}{#1,internal}\rnd{3}\deg{d}\f{sin}\sine@stop{#2}}^^A
%   \def\sine@rnd#1\rnd#2#3\sine@stop{\sine@deg#1#3\sine@stop{#2}}^^A
%   \def\sine@deg#1\deg#2#3\sine@stop{\sine@f#1#3\sine@stop{#2}}^^A
%   \def\sine@f#1\f#2#3\sine@stop{\sine@final{#2}}^^A
%   \newcommand*\sine@final[4]{\fpeval{round(#1#2(#4),#3)}}^^A
%   \makeatother
%   \sine{}{60}\par
%   \sine{round=10}{60}\par
%   \sine{f=cos,radian}{pi}\par
%   \edef\myval{\sine{f=tan}{1}}\texttt{\meaning\myval}
% \end{minipage}
%
%
% \subsection{Error Messages}
%
% \expkv\ should only send messages in case of errors, there are no warnings and
% no info messages. In this subsection those errors are listed.
%
% \subsubsection{Load Time}
%
% \file{expkv.tex} checks whether \eTeX\ is available. If it isn't, an error
% will be thrown using |\errmessage|:
% \begin{lstlisting}
% ! expkv Error: e-TeX required.
% \end{lstlisting}
%
% \subsubsection{Defining Keys}
%
% If you get any error from \expkv\ while you're trying to define a key, the
% definition will be aborted and gobbled.
%
% If you try to define a key with an empty set name you'll get:
% \begin{lstlisting}
% ! expkv Error: empty set name not allowed.
% \end{lstlisting}
%
% Similarly, if you try to define a key with an empty key name:
% \begin{lstlisting}
% ! expkv Error: empty key name not allowed.
% \end{lstlisting}
%
% Both of these messages are done in a way that doesn't throw additional errors
% due to |\global|, |\long|, etc., not being used correctly if you prefixed one
% of the defining macros.
%
% \subsubsection{Using Keys}
%
% This subsubsection contains the errors thrown during |\ekvset|. The errors are
% thrown in an expandable manner by providing an undefined macro. In the
% following messages \lstinline|<key>| gets replaced with the problematic key's
% name, and \lstinline|<set>| with the corresponding set. If any errors during
% \kv\ handling are encountered, the entry in the comma separated list will be
% omitted after the error is thrown and the next \kv\ pair will be parsed.
%
% If you're using an undefined key you'll get:
% \begin{lstlisting}
% ! Undefined control sequence.
% <argument> \! expkv Error:
%                            unknown key (`<key>', set `<set>').
% \end{lstlisting}
%
% If you're using a key for which only a normal version and no |NoVal| version
% is defined, but don't provide a value, you'll get:
% \begin{lstlisting}
% ! Undefined control sequence.
% <argument> \! expkv Error:
%                            value required (`<key>', set `<set>').
% \end{lstlisting}
%
% If you're using a key for which only a |NoVal| version and no normal version
% is defined, but provide a value, you'll get:
% \begin{lstlisting}
% ! Undefined control sequence.
% <argument> \! expkv Error:
%                            value forbidden (`<key>', set `<set>').
% \end{lstlisting}
%
% If you're using a set for which you never executed one of the defining macros
% from \autoref{sec:define} you'll get a low level \TeX\ error, as that isn't
% actively tested by the parser (and hence will lead to undefined behaviour and
% not be gracefully ignored). The error will look like
% \begin{lstlisting}
% ! Missing \endcsname inserted.
% <to be read again>
%                    \! expkv Error: Set `<set>' undefined.
% \end{lstlisting}
%
% \subsection{License}
%
% Copyright \textcopyright\ 2020\unless\ifnum\year=2020--\the\year\fi\
% Jonathan P. Spratte
%
% \medskip\noindent
% This work may be distributed and/or modified under the conditions of the
% \LaTeX\ Project Public License (LPPL), either version 1.3c of this license or
% (at your option) any later version. The latest version of this license is in
% the file:
%
% \url{http://www.latex-project.org/lppl.txt}
%
% \noindent
% This work is ``maintained'' (as per LPPL maintenance status) by
%
% \mbox{Jonathan P. Spratte}.
%
% \end{documentation}^^A=<<
%
% \begin{implementation}^^A>>=
%
% \clearpage
%
% \section{Implementation}^^A>>=
%^^A the LaTeX package >>=
% \subsection{The \LaTeX\ Package}
% First we set up the \LaTeX\ package. That one doesn't really do much except
% |\input|ting the generic code and identifying itself as a package.
% \gobbledocstriptag
%<*pkg>
%    \begin{macrocode}
\def\ekv@tmp
  {%
    \ProvidesFile{expkv.tex}%
      [\ekvDate\space v\ekvVersion\space an expandable key=val implementation]%
  }
\input{expkv.tex}
\ProvidesPackage{expkv}%
  [\ekvDate\space v\ekvVersion\space an expandable key=val implementation]
%    \end{macrocode}
% \gobbledocstriptag
%</pkg>
%^^A=<<
%^^A main file >>=
% \subsection{The Generic Code}
% The rest of this implementation will be the generic code.
% \gobbledocstriptag
%<*tex>
%
% Check whether \eTeX\ is available -- \expkv\ requires \eTeX.
%    \begin{macrocode}
\begingroup\expandafter\expandafter\expandafter\endgroup
\expandafter\ifx\csname numexpr\endcsname\relax
  \errmessage{expkv requires e-TeX}
  \expandafter\endinput
\fi
%    \end{macrocode}
%
% We make sure that it's only input once:
%    \begin{macrocode}
\expandafter\ifx\csname ekvVersion\endcsname\relax
\else
  \expandafter\endinput
\fi
%    \end{macrocode}
% \begin{macro}{\ekvVersion,\ekvDate}
% We're on our first input, so lets store the version and date in a macro.
%    \begin{macrocode}
\def\ekvVersion{0.5a}
\def\ekvDate{2020-02-27}
%    \end{macrocode}
% \end{macro}
%
% If the \LaTeX\ format is loaded we want to be a good file and report back who
% we are, for this the package will have defined |\ekv@tmp| to use
% |\ProvidesFile|, else this will expand to a |\relax| and do no harm.
%    \begin{macrocode}
\csname ekv@tmp\endcsname
%    \end{macrocode}
%
% Store the category code of |@| to later be able to reset it and change it to
% 11 for now.
%    \begin{macrocode}
\expandafter\chardef\csname ekv@tmp\endcsname=\catcode`\@
\catcode`\@=11
%    \end{macrocode}
% |\ekv@tmp| might later be reused to gobble any prefixes which might be
% provided to |\ekvdef| and similar in case the names are invalid, we just
% temporarily use it here as means to store the current category code of |@| to
% restore it at the end of the file, we never care for the actual definition of
% it.
%
% \begin{macro}[internal]
%   {
%     \@gobble,\@firstofone,\@firstoftwo,\@secondoftwo,
%     \ekv@gobbleto@stop,\ekv@fi@secondoftwo,\ekv@gobble@mark
%   }
% Since branching tests are often more versatile than |\if...\else...\fi|
% constructs, we define helpers that are branching pretty fast. Also here are
% some other utility functions that just grab some tokens. The ones that are
% also contained in \LaTeX\ don't use the |ekv| prefix.
%    \begin{macrocode}
\long\def\@gobble#1{}
\long\def\@firstofone#1{#1}
\long\def\@firstoftwo#1#2{#1}
\long\def\@secondoftwo#1#2{#2}
\long\def\ekv@fi@secondoftwo\fi\@firstoftwo#1#2{\fi#2}
\long\def\ekv@gobbleto@stop#1\ekv@stop{}
\def\ekv@gobble@mark\ekv@mark{}
%    \end{macrocode}
% \end{macro}
% As you can see |\ekv@gobbleto@stop| uses a special marker |\ekv@stop|. The
% package will use three such markers, the one you've seen already, |\ekv@mark|
% and |\ekv@nil|. Contrarily to how for instance \pkg{expl3} does things, we
% don't define them, as we don't need them to have an actual meaning. This has
% the advantage that if they somehow get expanded -- which should never happen
% if things work out -- they'll throw an error directly.
%
% \begin{macro}[internal]
%   {
%     \ekv@ifempty,\ekv@ifempty@,\ekv@ifempty@true,\ekv@ifempty@false,
%     \ekv@ifempty@true@F
%   }
% We can test for a lot of things building on an if-empty test, so lets define a
% really fast one. Since some tests might have reversed logic (true if something
% is not empty) we also set up macros for the reversed branches.
%    \begin{macrocode}
\long\def\ekv@ifempty#1%
  {%
    \ekv@ifempty@\ekv@ifempty@A#1\ekv@ifempty@B\ekv@ifempty@true
      \ekv@ifempty@A\ekv@ifempty@B\@secondoftwo
  }
\long\def\ekv@ifempty@#1\ekv@ifempty@A\ekv@ifempty@B{}
\long\def\ekv@ifempty@true\ekv@ifempty@A\ekv@ifempty@B\@secondoftwo#1#2{#1}
\long\def\ekv@ifempty@false\ekv@ifempty@A\ekv@ifempty@B\@firstoftwo#1#2{#2}
\long\def\ekv@ifempty@true@F\ekv@ifempty@A\ekv@ifempty@B\@firstofone#1{}
%    \end{macrocode}
% \end{macro}
%
% \begin{macro}[internal]{\ekv@ifblank@}
% The obvious test that can be based on an if-empty is if-blank, meaning a test
% checking whether the argument is empty or consists only of spaces. Our version
% here will be tweaked a bit, as we want to check this, but with one leading
% |\ekv@mark| token that is to be ignored.
%    \begin{macrocode}
\long\def\ekv@ifblank@\ekv@mark#1{\ekv@ifempty@\ekv@ifempty@A}
%    \end{macrocode}
% \end{macro}
%
% \begin{macro}[internal]{\ekv@ifdefined}
% We'll need to check whether something is defined quite frequently, so why not
% define a macro that does this. The following test is expandable, slower than
% the typical expandable test for undefined control sequences, but faster for
% defined ones. Since we want to be as fast as possible for correct input, this
% is to be preferred.
%    \begin{macrocode}
\def\ekv@ifdefined#1%
  {%
    \expandafter
    \ifx\csname\ifcsname #1\endcsname #1\else relax\fi\endcsname\relax
      \ekv@fi@secondoftwo
    \fi
    \@firstoftwo
  }
%    \end{macrocode}
% \end{macro}
  % \begin{macro}[internal]
%   {
%     \ekv@ifdefined@pair,\ekv@ifdefined@pair@,\ekv@ifdefined@key,
%     \ekv@ifdefined@key@
%   }
% Since we can save some time if we only have to create the control sequence
% once when we know beforehand how we want to use it, we build some other macros
% for those cases (which we'll have quite often, once per key usage).
%    \begin{macrocode}
\def\ekv@ifdefined@pair#1#2%
  {%
    \expandafter\ekv@ifdefined@pair@
      \csname
        \ifcsname #1{#2}\endcsname
          #1{#2}%
        \else
          relax%
        \fi
      \endcsname
  }
\def\ekv@ifdefined@pair@#1%
  {%
    \ifx#1\relax
      \ekv@fi@secondoftwo
    \fi
    \@firstoftwo
    {\ekv@set@pair@#1\ekv@mark}%
  }
\def\ekv@ifdefined@key#1#2%
  {%
    \expandafter\ekv@ifdefined@key@
      \csname
        \ifcsname #1{#2}N\endcsname
          #1{#2}N%
        \else
          relax%
        \fi
      \endcsname
  }
\def\ekv@ifdefined@key@#1%
  {%
    \ifx#1\relax
      \ekv@fi@secondoftwo
    \fi
    \@firstoftwo#1%
  }
%    \end{macrocode}
% \end{macro}
%
% \begin{macro}{\ekv@name,\ekv@name@set,\ekv@name@key}
% The keys will all follow the same naming scheme, so we define it here.
%    \begin{macrocode}
\def\ekv@name#1#2{\ekv@name@set{#1}\ekv@name@key{#2}}
\def\ekv@name@set#1{ekv#1(}
\def\ekv@name@key#1{#1)}
%    \end{macrocode}
% \end{macro}
% \begin{macro}[internal]{\ekv@undefined@set}
% We can misuse the macro name we use to expandably store the set-name in a
% single token -- since this increases performance drastically, especially for
% long set-names -- to throw a more meaningful error message in case a set isn't
% defined. The name of |\ekv@undefined@set| is a little bit misleading, as it is
% called in either case inside of |\csname|, but the result will be a control
% sequence with meaning |\relax| if the set is undefined, hence will break the
% |\csname| building the key-macro which will throw the error message.
%    \begin{macrocode}
\def\ekv@undefined@set#1{! expkv Error: Set `#1' undefined.}
%    \end{macrocode}
% \end{macro}
% \begin{macro}[internal]{\ekv@checkvalid}
% We place some restrictions on the allowed names, though, namely sets and
% keys are not allowed to be empty -- blanks are fine (meaning \mbox{set-
% or} key-names consisting of spaces).
%    \begin{macrocode}
\protected\def\ekv@checkvalid#1#2%
  {%
    \ekv@ifempty{#1}%
      {%
        \def\ekv@tmp{}%
        \errmessage{expkv Error: empty set name not allowed}%
      }%
      {%
        \ekv@ifempty{#2}%
          {%
            \def\ekv@tmp{}%
            \errmessage{expkv Error: empty key name not allowed}%
          }%
          \@secondoftwo
      }%
    \@gobble
  }
%    \end{macrocode}
% \end{macro}
% \begin{macro}{\ekvifdefined,\ekvifdefinedNoVal}
% And provide user-level macros to test whether a key is defined.
%    \begin{macrocode}
\def\ekvifdefined#1#2{\ekv@ifdefined{\ekv@name{#1}{#2}}}
\def\ekvifdefinedNoVal#1#2{\ekv@ifdefined{\ekv@name{#1}{#2}N}}
%    \end{macrocode}
% \end{macro}
%
% \begin{macro}
%   {\ekvdef,\ekvdefNoVal,\ekvlet,\ekvletNoVal,\ekvletkv,\ekvletkvNoVal}
% Set up the key defining macros |\ekvdef| etc.
%    \begin{macrocode}
\protected\long\def\ekvdef#1#2#3%
  {%
    \ekv@checkvalid{#1}{#2}%
      {%
        \expandafter\def\csname\ekv@name{#1}{#2}\endcsname##1{#3}%
        \ekv@defset{#1}%
      }%
  }
\protected\long\def\ekvdefNoVal#1#2#3%
  {%
    \ekv@checkvalid{#1}{#2}%
      {%
        \expandafter\def\csname\ekv@name{#1}{#2}N\endcsname{#3}%
        \ekv@defset{#1}%
      }%
  }
\protected\def\ekvlet#1#2#3%
  {%
    \ekv@checkvalid{#1}{#2}%
      {%
        \expandafter\let\csname\ekv@name{#1}{#2}\endcsname#3%
        \ekv@defset{#1}%
      }%
  }
\protected\def\ekvletNoVal#1#2#3%
  {%
    \ekv@checkvalid{#1}{#2}%
      {%
        \expandafter\let\csname\ekv@name{#1}{#2}N\endcsname#3%
        \ekv@defset{#1}%
      }%
  }
\protected\def\ekvletkv#1#2#3#4%
  {%
    \ekv@checkvalid{#1}{#2}%
      {%
        \expandafter\let\csname\ekv@name{#1}{#2}\expandafter\endcsname
        \csname\ekv@name{#3}{#4}\endcsname
        \ekv@defset{#1}%
      }%
  }
\protected\def\ekvletkvNoVal#1#2#3#4%
  {%
    \ekv@checkvalid{#1}{#2}%
      {%
        \expandafter\let\csname\ekv@name{#1}{#2}N\expandafter\endcsname
        \csname\ekv@name{#3}{#4}N\endcsname
        \ekv@defset{#1}%
      }%
  }
%    \end{macrocode}
% \end{macro}
% \begin{macro}[internal]{\ekv@defset}
%   In order to enhance the speed the set name given to |\ekvset| will be turned
%   into a control sequence pretty early, so we have to define that control
%   sequence.
%    \begin{macrocode}
\protected\def\ekv@defset#1%
  {%
    \expandafter\edef\csname\ekv@undefined@set{#1}\endcsname##1%
      {\ekv@name@set{#1}\ekv@name@key{##1}}%
  }
%    \end{macrocode}
% \end{macro}
%
% \begin{macro}{\ekvset}
% Set up |\ekvset|, which should not be affected by active commas and equal
% signs. The equal signs are a bit harder to cope with and we'll do that later,
% but replacing the active commas with commas of category other can be done
% beforehand. That's why we define |\ekvset| here with a temporary meaning just
% to set up the things with two different category codes. |#1| will be a
% \texttt{,\textsubscript{13}} and |#2| will be a \texttt{=\textsubscript{13}}.
%    \begin{macrocode}
\def\ekvset#1#2{%
\endgroup
\long\def\ekvset##1##2%
  {%
    \expandafter\ekv@set\csname\ekv@undefined@set{##1}\endcsname
      \ekv@mark##2#1\ekv@stop#1{}%
  }
%    \end{macrocode}
% \end{macro}
% \begin{macro}[internal]{\ekv@set}
% |\ekv@set| will split the \kv\ list at active commas. Then it has to check
% whether there were unprotected other commas and resplit there.
%    \begin{macrocode}
\long\def\ekv@set##1##2#1%
  {%
%    \end{macrocode}
% Test whether we're at the end, if so invoke |\ekv@endset|,
%    \begin{macrocode}
    \ekv@gobbleto@markstop##2\ekv@endset\ekv@mark\ekv@stop
%    \end{macrocode}
% else go on with other commas,
%    \begin{macrocode}
    \ekv@set@other##1##2,\ekv@stop,%
%    \end{macrocode}
% and get the next active comma delimited \kv\ pair.
%    \begin{macrocode}
    \ekv@set##1\ekv@mark
  }
%    \end{macrocode}
% \end{macro}
% \begin{macro}[internal]{\ekv@endset}
% |\ekv@endset| is a hungry little macro. It will eat everything that remains
% of |\ekv@set| and unbrace the sneaked stuff.
%    \begin{macrocode}
\long\def\ekv@endset
    \ekv@mark\ekv@stop\ekv@set@other##1,\ekv@stop,\ekv@set##2\ekv@mark
    ##3%
  {##3}
%    \end{macrocode}
% \end{macro}
% \begin{macro}[internal]{\ekv@set@other}
% The macro |\ekv@set@other| is guaranteed to get only single \kv\ pairs.
% First we test whether we're done, if not split at equal signs. It is faster to
% first split at category 12 equal signs and only after that on actives.
% If there is no equal sign, we need to test whether we got a blank argument and
% if not this is a |NoVal| key.
%    \begin{macrocode}
\long\def\ekv@set@other##1##2,%
  {%
    \ekv@gobbleto@markstop##2\ekv@endset@other\ekv@mark\ekv@stop
    \ekv@ifhas@eq@other##2=\ekv@ifempty@B\ekv@ifempty@false
      \ekv@ifempty@A\ekv@ifempty@B\@firstoftwo
      {\ekv@set@eq@other##1##2\ekv@stop}%
      {%
        \ekv@ifhas@eq@active##2#2\ekv@ifempty@B\ekv@ifempty@false
          \ekv@ifempty@A\ekv@ifempty@B\@firstoftwo
          {\ekv@set@eq@active##1##2\ekv@stop}%
          {%
            \ekv@ifblank@##2\ekv@nil\ekv@ifempty@B\ekv@ifempty@true@F
              \ekv@ifempty@A\ekv@ifempty@B\@firstofone
              {\ekv@strip{##2}\ekv@set@key##1}%
          }%
      }%
    \ekv@set@other##1\ekv@mark%
  }
%    \end{macrocode}
% \end{macro}
% \begin{macro}[internal]{\ekv@set@eq@other}
% |\ekv@set@eq@other| might not be the correct break point, there might be an
% active equal sign in the currently parsed key-name. If so, we have to resplit.
% If the split is correct strip the key-name of outer spaces and braces and feed
% it to |\ekv@set@pair|.
%    \begin{macrocode}
\long\def\ekv@set@eq@other##1##2=%
  {%
    \ekv@ifhas@eq@active##2#2\ekv@ifempty@B\ekv@ifempty@false
      \ekv@ifempty@A\ekv@ifempty@B\@firstoftwo
      {\ekv@set@eq@active##1##2=}%
      {\ekv@strip{##2}\ekv@set@pair##1}%
  }
%    \end{macrocode}
% \end{macro}
% \begin{macro}[internal]{\ekv@set@eq@active}
% We need to handle the active equal signs.
%    \begin{macrocode}
\long\def\ekv@set@eq@active##1##2#2%
  {%
    \ekv@strip{##2}\ekv@set@pair##1%
  }
%    \end{macrocode}
% \end{macro}
% \begin{macro}[internal]
%   {\ekv@ifhas@eq@other,\ekv@ifhas@eq@active,\ekv@endset@other}
% And we have to set up the testing macros for our equal signs and
% |\ekv@endset@other|.
%    \begin{macrocode}
\long\def\ekv@ifhas@eq@other\ekv@mark##1={\ekv@ifempty@\ekv@ifempty@A}
\long\def\ekv@ifhas@eq@active\ekv@mark##1#2{\ekv@ifempty@\ekv@ifempty@A}
\long\def\ekv@endset@other
    \ekv@mark\ekv@stop
    \ekv@ifhas@eq@other##1=\ekv@ifempty@B\ekv@ifempty@false
    \ekv@ifempty@A\ekv@ifempty@B\@firstoftwo
    ##2%
    \ekv@set@other##3\ekv@mark
  {}
%    \end{macrocode}
% \end{macro}
% \begin{macro}{\ekvbreak,\ekvbreakPreSneak,\ekvbreakPostSneak}
% Provide macros that can completely stop the parsing of |\ekvset|, who knows
% what it'll be useful for.
%    \begin{macrocode}
\long\def\ekvbreak##1##2\ekv@stop#1##3{##1}
\long\def\ekvbreakPreSneak ##1##2\ekv@stop#1##3{##1##3}
\long\def\ekvbreakPostSneak##1##2\ekv@stop#1##3{##3##1}
%    \end{macrocode}
% \end{macro}
% \begin{macro}{\ekvsneak,\ekvsneakPre}
% One last thing we want to do for |\ekvset| is to provide macros that just
% smuggle stuff after |\ekvset|'s effects.
%    \begin{macrocode}
\long\def\ekvsneak##1##2\ekv@stop#1##3%
  {%
    ##2\ekv@stop#1{##3##1}%
  }
\long\def\ekvsneakPre##1##2\ekv@stop#1##3%
  {%
    ##2\ekv@stop#1{##1##3}%
  }
%    \end{macrocode}
% \end{macro}
%
% \begin{macro}{\ekvparse}
% Additionally to the |\ekvset| macro we also want to provide an |\ekvparse|
% macro, that has the same scope as |\keyval_parse:NNn| from \pkg{expl3}.
% This is pretty analogue to the |\ekvset| implementation, we just put an
% |\unexpanded| here and there instead of other macros to stop the |\expanded|
% on our output.
%    \begin{macrocode}
\long\def\ekvparse##1##2##3%
  {%
    \ekv@parse##1##2\ekv@mark##3#1\ekv@stop#1%
  }
%    \end{macrocode}
% \end{macro}
% \begin{macro}[internal]{\ekv@parse}
%    \begin{macrocode}
\long\def\ekv@parse##1##2##3#1%
  {%
    \ekv@gobbleto@markstop##3\ekv@endparse\ekv@mark\ekv@stop
    \ekv@parse@other##1##2##3,\ekv@stop,%
    \ekv@parse##1##2\ekv@mark
  }
%    \end{macrocode}
% \end{macro}
% \begin{macro}[internal]{\ekv@endparse}
%    \begin{macrocode}
\long\def\ekv@endparse
    \ekv@mark\ekv@stop\ekv@parse@other##1,\ekv@stop,\ekv@parse##2\ekv@mark
  {}
%    \end{macrocode}
% \end{macro}
% \begin{macro}[internal]{\ekv@parse@other}
%    \begin{macrocode}
\long\def\ekv@parse@other##1##2##3,%
  {%
    \ekv@gobbleto@markstop##3\ekv@endparse@other\ekv@mark\ekv@stop
    \ekv@ifhas@eq@other##3=\ekv@ifempty@B\ekv@ifempty@false
      \ekv@ifempty@A\ekv@ifempty@B\@firstoftwo
      {\ekv@parse@eq@other##3\ekv@stop##2}%
      {%
        \ekv@ifhas@eq@active##3#2\ekv@ifempty@B\ekv@ifempty@false
          \ekv@ifempty@A\ekv@ifempty@B\@firstoftwo
          {\ekv@parse@eq@active##3\ekv@stop##2}%
          {%
            \ekv@ifblank@##3\ekv@nil\ekv@ifempty@B\ekv@ifempty@true@F
              \ekv@ifempty@A\ekv@ifempty@B\@firstofone
              {\ekv@strip{##3}\ekv@parse@key##1}%
          }%
      }%
    \ekv@parse@other##1##2\ekv@mark
  }
%    \end{macrocode}
% \end{macro}
% \begin{macro}[internal]{\ekv@parse@eq@other}
%    \begin{macrocode}
\long\def\ekv@parse@eq@other##1=%
  {%
    \ekv@ifhas@eq@active##1#2\ekv@ifempty@B\ekv@ifempty@false
      \ekv@ifempty@A\ekv@ifempty@B\@firstoftwo
      {\ekv@parse@eq@active##1=}%
      {\ekv@strip{##1}\ekv@parse@pair\ekv@mark}%
  }
%    \end{macrocode}
% \end{macro}
% \begin{macro}[internal]{\ekv@parse@eq@active}
%    \begin{macrocode}
\long\def\ekv@parse@eq@active##1#2%
  {%
    \ekv@strip{##1}\ekv@parse@pair\ekv@mark
  }
%    \end{macrocode}
% \end{macro}
% \begin{macro}[internal]{\ekv@endparse@other}
%    \begin{macrocode}
\long\def\ekv@endparse@other
    \ekv@mark\ekv@stop
    \ekv@ifhas@eq@other##1=\ekv@ifempty@B\ekv@ifempty@false
    \ekv@ifempty@A\ekv@ifempty@B\@firstoftwo
    ##2%
    \ekv@parse@other##3\ekv@mark
  {}
%    \end{macrocode}
% \end{macro}
% \begin{macro}[internal]{\ekv@parse@pair,\ekv@parse@pair@}
%    \begin{macrocode}
\long\def\ekv@parse@pair##1##2\ekv@stop
  {%
    \ekv@strip{##2}\ekv@parse@pair@{##1}%
  }
\long\def\ekv@parse@pair@##1##2##3%
  {%
    \unexpanded{##3{##2}{##1}}%
  }
%    \end{macrocode}
% \end{macro}
% \begin{macro}[internal]{\ekv@parse@key}
%    \begin{macrocode}
\long\def\ekv@parse@key##1##2%
  {%
    \unexpanded{##2{##1}}%
  }
%    \end{macrocode}
% \end{macro}
%
% Finally really setting things up with |\ekvset|'s temporary meaning:
%    \begin{macrocode}
}
\begingroup
\catcode`\,=13
\catcode`\==13
\ekvset,=
%    \end{macrocode}
%
% \begin{macro}{\ekvchangeset}
% Provide a macro that is able to switch out the current \set\ in |\ekvset|.
% This operation is slow (by comparison, it should be slightly faster than
% |\ekvsneak|), but allows for something similar to \pkg{pgfkeys}'s
% \texttt{\meta{key}/.cd} mechanism. However this operation is more expensive
% than |/.cd| as we can't just redefine some token to reflect this, but have to
% switch out the set expandably, so this works similar to the |\ekvsneak| macros
% reading and reinserting the remainder of the \kv\ list.
%    \begin{macrocode}
\def\ekvchangeset#1%
  {%
    \expandafter\ekv@changeset\csname\ekv@undefined@set{#1}\endcsname\ekv@mark
  }
%    \end{macrocode}
% \end{macro}
% \begin{macro}[internal]{\ekv@changeset}
% This macro does the real change-out of |\ekvchangeset|. We introduced an
% |\ekv@mark| to not accidentally remove some braces which we have to remove
% again.
%    \begin{macrocode}
\long\def\ekv@changeset#1#2\ekv@set@other#3#4\ekv@set#5%
  {%
    \ekv@gobble@mark#2\ekv@set@other#1#4\ekv@set#1%
  }
%    \end{macrocode}
% \end{macro}
%
%
% \begin{macro}[internal]{\ekv@gobbleto@markstop}
% The |\ekv@gobbleto@markstop| can be used to test for |\ekv@stop| similar to
% our if-empty test, but instead of using tokens which are used nowhere else
% (|\ekv@ifempty@A| and |\ekv@ifempty@B|) it uses |\ekv@mark| and |\ekv@stop|.
%    \begin{macrocode}
\long\def\ekv@gobbleto@markstop#1\ekv@mark\ekv@stop{}
%    \end{macrocode}
% \end{macro}
%
% \begin{macro}[internal]{\ekv@set@pair,\ekv@set@pair@}
% |\ekv@set@pair| gets invoked with the space and brace stripped key-name as its
% first argument, the set-macro as the second argument, and following that is
% the key-value right delimited by an |\ekv@stop|.
%    \begin{macrocode}
\long\def\ekv@set@pair#1#2%
  {%
    \ekv@ifdefined@pair#2{#1}%
      {%
%    \end{macrocode}
% This branch will be executed if the key is not defined as an argument grabbing
% one. If so test whether there is a |NoVal| key of the same name or whether the
% key is unknown. Throw a meaningful error message and gobble the value.
%    \begin{macrocode}
        \ekv@ifdefined{#2{#1}N}%
          \ekv@err@noarg
          \ekv@err@unknown
          #2{#1}%
        \ekv@gobbleto@stop
      }%
  }
%    \end{macrocode}
% |\ekv@ifdefined@pair| will call |\ekv@set@pair@| if the key is correctly
% defined. This will then grab the value, strip outer spaces and braces from it
% and feed it to the key-macro. Afterwards |\ekv@set@other| will take control
% again.
%    \begin{macrocode}
\long\def\ekv@set@pair@#1#2\ekv@stop
  {%
    \ekv@strip{#2}#1%
  }
%    \end{macrocode}
% \end{macro}
%
% \begin{macro}[internal]{\ekv@set@key}
% Analogous to |\ekv@set@pair|, |\ekv@set@key| lets |\ekv@ifdefined@key| test
% whether a |NoVal| key is defined, else it'll throw a meaningful error message.
% Since we don't have to grab any value |\ekv@ifdefined@key| will invoke the
% key-macro and we're done here, |\ekv@set@other| will take over again.
%    \begin{macrocode}
\long\def\ekv@set@key#1#2%
  {%
    \ekv@ifdefined@key#2{#1}%
      {%
        \ekv@ifdefined{#2{#1}}%
          \ekv@err@reqval
          \ekv@err@unknown
          #2{#1}%
      }%
  }
%    \end{macrocode}
% \end{macro}
%
% \begin{macro}[internal]{\ekv@err,\ekv@err@}
% Since |\ekvset| is fully expandable as long as the code of the keys is (which
% is unlikely) we want to somehow throw expandable errors, in our case via
% undefined control sequences.
%    \begin{macrocode}
\begingroup
\edef\ekv@err
  {%
    \endgroup
    \unexpanded{\long\def\ekv@err}##1%
      {%
        \unexpanded{\expandafter\ekv@err@\@firstofone}%
        {\expandafter\noexpand\csname ! expkv Error:\endcsname ##1.}%
        \unexpanded{\ekv@stop}%
      }%
  }
\ekv@err
\def\ekv@err@{\expandafter\ekv@gobbleto@stop}
%    \end{macrocode}
% \end{macro}
% \begin{macro}[internal]
%   {
%     \ekv@err@common,\ekv@err@common@,
%     \ekv@err@unknown,\ekv@err@noarg,\ekv@err@reqval
%   }
% Now we can use |\ekv@err| to set up some error messages so that we can later
% use those instead of the full strings.
%    \begin{macrocode}
\long\def\ekv@err@common #1#2{\expandafter\ekv@err@common@\string#2{#1}}
\long\def\ekv@err@common@#1`#2' #3.#4#5{\ekv@err{#4 (`#5', set `#2')}}
\long\def\ekv@err@unknown#1#2{\ekv@err@common{unknown key}#1{#2}}
\long\def\ekv@err@noarg  #1#2{\ekv@err@common{value forbidden}#1{#2}}
\long\def\ekv@err@reqval #1#2{\ekv@err@common{value required}#1{#2}}
%    \end{macrocode}
% \end{macro}
%
% \begin{macro}[internal]{\ekv@strip,\ekv@strip@a,\ekv@strip@b,\ekv@strip@c}
% Finally we borrow some ideas of \pkg{expl3}'s \pkg{l3tl} to strip spaces
% from keys and values. This |\ekv@strip| also strips one level of outer braces
% \emph{after} stripping spaces, so an input of | {abc} | becomes |abc| after
% stripping. It should be used with |#1| prefixed by |\ekv@mark|. Also this
% implementation at most strips \emph{one} space from both sides.
%    \begin{macrocode}
\def\ekv@strip#1%
  {%
    \long\def\ekv@strip##1%
      {%
        \ekv@strip@a
          ##1%
          \ekv@nil
          \ekv@mark#1%
          #1\ekv@nil{}%
        \ekv@stop
      }%
    \long\def\ekv@strip@a##1\ekv@mark#1##2\ekv@nil##3%
      {%
        \ekv@strip@b##3##1##2\ekv@nil
      }%
    \long\def\ekv@strip@b##1#1\ekv@nil
      {%
        \ekv@strip@c##1\ekv@nil
      }%
    \long\def\ekv@strip@c\ekv@mark##1\ekv@nil##2\ekv@stop##3%
      {%
        ##3{##1}%
      }%
  }
\ekv@strip{ }
%    \end{macrocode}
% \end{macro}
%
% Now everything that's left is to reset the category code of |@|.
%    \begin{macrocode}
\catcode`\@=\ekv@tmp
%    \end{macrocode}
%
% \gobbledocstriptag
%</tex>
%^^A=<<
%
%^^A=<<
%
% \end{implementation}^^A=<<
%
% \clearpage
% \PrintIndex
%
\endinput
%
^^A vim: ft=tex fdm=marker fmr=>>=,=<<
}{}
\ProvidesFile{expkv.dtx}
  [\csname ekvDate\endcsname\ an expandable key=val implementation]
\PassOptionsToPackage{full}{textcomp}
\documentclass{l3doc}
\RequirePackage[oldstylenums,nott]{kpfonts}
\input{glyphtounicode}
\pdfgentounicode=1
\RequirePackage{xfp} ^^A required for an example
\RequirePackage{listings}
\RequirePackage{booktabs}
\RequirePackage{array}
\RequirePackage{collcell}
\RequirePackage{siunitx}
\RequirePackage{xcolor}
\RequirePackage{caption}
\RequirePackage{microtype}
\RequirePackage{accsupp}
\lstset
  {
    ,flexiblecolumns=false
    ,basewidth=.53em
    ,gobble=2
    ,basicstyle=\fontfamily{jkp}\itshape
    ,morekeywords=^^A
      {^^A
        \ekvdef,\ekvdefNoVal,\ekvlet,\ekvletNoVal,\ekvletkv,\ekvletkvNoVal,^^A
        \ekvset,\ekvparse,\ekvVersion,\ekvDate,\ekvifdefined,^^A
        \ekvifdefinedNoVal,\ekvbreak,\ekvbreakPreSneak,\ekvbreakPostSneak,^^A
        \ekvsneak,\ekvsneakPre,\ekvchangeset
      }
    ,morecomment=[l]\%
    ,commentstyle=\color[gray]{0.4}
    ,literate={\{}{{\CodeSymbol\{}}{1}
              {\}}{{\CodeSymbol\}}}{1}
    ^^A,literate=*{<key>}{\key}{4}{<set>}{\set}{4}
  }
\newcommand*\CodeSymbol[1]{\textbf{#1}}
\RequirePackage{randtext}
\let\metaORIG\meta
\protected\def\meta #1{\texttt{\metaORIG{#1}}}
\renewcommand*\thefootnote{\fnsymbol{footnote}}
\definecolor{expkvred}{HTML}{9F393D}
\colorlet{expkvgrey}{black!75}
\makeatletter
\newcommand*\expkv
  {^^A
    \texorpdfstring
      {^^A
        \mbox
          {^^A
            \BeginAccSupp{ActualText=expkv}^^A
            \href{https://github.com/Skillmon/tex_expkv}
              {^^A
                \rmfamily
                \bfseries
                {\color{expkvgrey}e\kern-.05em x\kern-.05em}^^A
                \lower.493ex
                  \hbox{{\color{expkvgrey}P}\kern-.1em{\color{expkvred}k}}^^A
                \kern-.18em{\color{expkvred}v}^^A
              }^^A
            \EndAccSupp{}^^A
          }^^A
      }
      {expkv}^^A
  }
\newcommand*\expkvd
  {^^A
    \texorpdfstring
      {^^A
        \mbox
          {^^A
            \BeginAccSupp{ActualText=expkv-def}^^A
            \rmfamily
            \bfseries
            {\color{expkvgrey}e\kern-.05em x\kern-.05em}^^A
            \lower.493ex
              \hbox{{\color{expkvgrey}P}\kern-.1em{\color{expkvred}k}}^^A
            \kern-.18em{\color{expkvred}v}^^A
            {\color{expkvgrey}^^A
              \kern.05em\rule[-.1ex]{.08em}{1.2ex}\kern.05em\textsc{def}^^A
            }^^A
            \EndAccSupp{}^^A
          }^^A
      }
      {expkv-def}^^A
  }
\newcommand\kv{\meta{key}=\meta{value}}
\newcommand\key{\meta{key}}
\newcommand\val{\meta{value}}
\newcommand\set{\meta{set}}
\hypersetup{linkcolor=red!80!black,urlcolor=purple!80!black}
\DoNotIndex{\def,\edef,\,,\=,\begingroup,\catcode,\chardef,\csname,\endcsname}
\DoNotIndex{\endgroup,\endinput,\errmessage,\expandafter,\input,\let,\long}
\DoNotIndex{\noexpand,\protected,\ProvidesFile,\ProvidesPackage,\relax,\space}
\DoNotIndex{\@,\unexpanded,\string,\expanded}
\DoNotIndex{\ifcsname}
\DoNotIndex{\ifx}
\DoNotIndex{\else}
\DoNotIndex{\fi}
\@gobble\fi ^^A ignoring \ifx and \ifcsname, but only one \fi
\@ifdefinable\gobbledocstriptag{\def\gobbledocstriptag#1>{}}
\newcommand*\pmso[1] ^^A poor man's strike out
  {^^A
    \leavevmode
    \begingroup
      \sbox0{#1}^^A
      \rlap{\vrule height .6ex depth -.5ex width \wd0\relax}^^A
      \usebox0\relax
    \endgroup
  }
\makeatother
\begin{document}
  \title
    {^^A
      \texorpdfstring
        {^^A
          \huge\expkv
          \\[\medskipamount]
          \Large an expandable \kv\ implementation^^A
        }{expkv - an expandable <key>=<value> implementation}^^A
    }
  \date{\ekvDate\space v\ekvVersion}
  \author{Jonathan P. Spratte\thanks{\protect\randomize{jspratte@yahoo.de}}}
  \DocInput{expkv.dtx}
\end{document}
%</driver>^^A=<<
% \fi
%
% \maketitle
% \renewcommand*\thefootnote{\arabic{footnote}}
%
% \begin{abstract}
% \noindent\parfillskip=0pt
% \expkv\ provides a small interface for \kv\ parsing. The parsing macro is
% fully expandable, the \meta{code} of your keys might be not. \expkv\ is pretty
% fast, but not the fastest available \kv\ solution (\pkg{keyval} for instance
% is thrice as fast, but not expandable and it might strip braces it shouldn't
% have stripped).
% \end{abstract}
%
% \tableofcontents
%
% \begin{documentation}^^A>>=
%
% \section{Documentation}
%
% \expkv\ provides an expandable \kv\ parser. The \kv\ pairs should be given as
% a comma separated list and the separator between a \key\ and the associated
% \val\ should be an equal sign. Both, the commas and the equal signs, might be
% of category 12 (other) or 13 (active). To support this is necessary as for
% example \pkg{babel} turns characters active for some languages, for instance
% the equal sign is turned active for Turkish.
%
% \expkv\ is usable as generic code or as a \LaTeX\ package. To use it, just
% use one of:
% \begin{lstlisting}
% \usepackage{expkv} % LaTeX
% \input expkv       % plainTeX
% \end{lstlisting}
% The \LaTeX\ package doesn't do more than \file{expkv.tex}, except calling
% |\ProvidesPackage| and setting things up such that \file{expkv.tex} will use
% |\ProvidesFile|.
%
% \subsection{Setting up Keys}\label{sec:define}
% \expkv\ provides a rather simple approach to setting up keys, similar to
% \pkg{keyval}. However there is an auxiliary package named
% \href{https://github.com/Skillmon/tex_expkv-def}{\expkvd} (the files use
% \file{expkv-def} instead because \string| isn't allowed in package names by
% CTAN) which provides a more sophisticated interface, similar to well
% established packages like \pkg{pgfkeys} or \pkg{l3keys}.
%
% Keys in \expkv\ (as in almost all other \kv\ implementations) belong to a
% \emph{set} such that different sets can contain keys of the same name. Unlike
% many other implementations \expkv\ doesn't provide means to set a default
% value, instead we have keys that take values and keys that don't (the latter
% are called |NoVal| keys by \expkv), but both can have the same name (on the
% user level).
%
% The following macros are available to define new keys. Those macros containing
% ``|def|'' in their name can be prefixed by anything allowed to prefix |\def|,
% prefixes allowed for |\let| can prefix those with ``|let|'' in their name,
% accordingly. Neither \set\ nor \key\ are allowed to be empty for new keys and
% must not contain a |\par| or tokens that expand to it -- they must be legal
% inside of |\csname ...\endcsname|.
%
% \begin{function}{\ekvdef}
%   \begin{syntax}
%     \cs{ekvdef}\marg{set}\marg{key}\marg{code}
%   \end{syntax}
%   Defines a \key\ taking a value in a \set\ to expand to \meta{code}. In
%   \meta{code} you can use |#1| to refer to the given value.
% \end{function}
%
% \begin{function}{\ekvdefNoVal}
%   \begin{syntax}
%     \cs{ekvdefNoVal}\marg{set}\marg{key}\marg{code}
%   \end{syntax}
%   Defines a no value taking \key\ in a \set\ to expand to \meta{code}.
% \end{function}
%
% \begin{function}{\ekvlet}
%   \begin{syntax}
%     \cs{ekvlet}\marg{set}\marg{key}\meta{cs}
%   \end{syntax}
%   Let the value taking \key\ in \set\ to \meta{cs}, there are no checks on
%   \meta{cs} enforced.
% \end{function}
%
% \begin{function}{\ekvletNoVal}
%   \begin{syntax}
%     \cs{ekvletNoVal}\marg{set}\marg{key}\meta{cs}
%   \end{syntax}
%   Let the no value taking \key\ in \set\ to \meta{cs}, it is not checked
%   whether \meta{cs} exists or that it takes no parameter.
% \end{function}
%
% \begin{function}{\ekvletkv}
%   \begin{syntax}
%     \cs{ekvletkv}\marg{set}\marg{key}\marg{set2}\marg{key2}
%   \end{syntax}
%   Let the \key\ in \set\ to \meta{key2} in \meta{set2}, it is not checked
%   whether that second key exists.
% \end{function}
%
% \begin{function}{\ekvletkvNoVal}
%   \begin{syntax}
%     \cs{ekvletkvNoVal}\marg{set}\marg{key}\marg{set2}\marg{key2}
%   \end{syntax}
%   Let the \key\ in \set\ to \meta{key2} in \meta{set2}, it is not checked
%   whether that second key exists.
% \end{function}
%
% \subsection{Parsing Keys}
%
% \begin{function}{\ekvset}
%   \begin{syntax}
%     \cs{ekvset}\marg{set}\{\kv,\ldots\}
%   \end{syntax}
%   Splits \kv\ pairs on commas. From both \key\ and \val\ up to one space is
%   stripped from both ends, if then only a braced group remains the braces are
%   stripped as well. So |\ekvset{foo}{bar=baz}| and
%   |\ekvset{foo}{ {bar}= {baz} }| will both do
%   \texttt{\cs[no-index]{\meta{foobarcode}}\{baz\}}, so you can hide commas,
%   equal signs and spaces at the ends of either \key\ or \val\ by putting
%   braces around them. If you omit the equal sign the code of the key created
%   with the |NoVal| variants described in \autoref{sec:define} will be
%   executed. If \kv\ contains more than a single unhidden equal sign, it will
%   be split at the first one and the others are considered part of the value.
%   |\ekvset| should be nestable.
% \end{function}
%
% \begin{function}{\ekvparse}
%   \begin{syntax}
%     \cs{ekvparse}\meta{cs1}\meta{cs2}\{\kv,\ldots\}
%   \end{syntax}
%   This macro parses the \kv\ pairs and provides those list elements which are
%   only keys as the argument to \meta{cs1}, and those which are a \kv\ pair to
%   \meta{cs2} as two arguments. It is fully expandable as well and returns the
%   parsed list in |\unexpanded|, which has no effect outside of an |\expanded|
%   or |\edef| context\footnotemark.
%   If you need control over the necessary steps of expansion you
%   can use |\expanded| around it.
%
%   |\ekvbreak|, |\ekvsneak|, and |\ekvchangeset| and their relatives don't work
%   in |\ekvparse|.  It is analogue to \pkg{expl3}'s |\keyval_parse:NNn|, but
%   not with the same parsing rules -- |\keyval_parse:NNn| throws an error on
%   multiple equal signs per \kv\ pair and on empty \key\ names in a \kv\ pair,
%   both of which |\ekvparse| doesn't deal with.
%
%   As a small example:
% \begin{lstlisting}
% \ekvparse\handlekey\handlekeyval{foo = bar, key, baz={zzz}}
% \end{lstlisting}
%   would expand to
% \begin{lstlisting}
% \handlekeyval{foo}{bar}\handlekey{key}\handlekeyval{baz}{zzz}
% \end{lstlisting}
%   and afterwards |\handlekey| and |\handlekeyval| would have to further handle
%   the \key. There are no macros like these two contained in \expkv, you have
%   to set them up yourself if you want to use |\ekvparse| (of course the names
%   might differ). If you need the results of |\ekvparse| as the argument for
%   another macro, you should use |\expanded| as only then the input stream will
%   contain the output above:
% \begin{lstlisting}
% \expandafter\handle\expanded{\ekvparse\k\kv{foo = bar, key, baz={zzz}}}
% \end{lstlisting}
%   would expand to
% \begin{lstlisting}
% \handle\kv{foo}{bar}\k{key}\kv{baz}{zzz}
% \end{lstlisting}
% \end{function}
% \footnotetext 
%   {^^A
%     This is a change in behaviour, previously (v0.3 and before) \cs{ekvparse}
%     would expand in exactly two steps. This isn't always necessary, but makes
%     the parsing considerably slower. If this is necessary for your application
%     you can put an \cs[no-index]{expanded} around it and will still be faster
%     since you need only a single \cs[no-index]{expandafter} this way.^^A
%   }.
%
% \subsection{Miscellaneous}
%
% \subsubsection{Other Macros}
%
% \expkv\ provides some other macros which might be of interest.
%
% \begin{function}{\ekvVersion,\ekvDate}
%   These two macros store the version and date of the package.
% \end{function}
% \begin{function}{\ekvifdefined,\ekvifdefinedNoVal}
%   \begin{syntax}
%     \cs{ekvifdefined}\marg{set}\marg{key}\marg{true}\marg{false}
%     \cs{ekvifdefinedNoVal}\marg{set}\marg{key}\marg{true}\marg{false}
%   \end{syntax}
%   These two macros test whether there is a \key\ in \set. It is false if
%   either a hash table entry doesn't exist for that key or its meaning is
%   |\relax|.
% \end{function}
%
% \begin{function}{\ekvbreak,\ekvbreakPreSneak,\ekvbreakPostSneak}
%   \begin{syntax}
%     \cs{ekvbreak}\marg{after}
%   \end{syntax}
%   Gobbles the remainder of the current |\ekvset| macro and its argument list
%   and reinserts \meta{after}. So this can be used to break out of |\ekvset|.
%   The first variant will also gobble anything that has been sneaked out using
%   |\ekvsneak| or |\ekvsneakPre|, while |\ekvbreakPreSneak| will put
%   \meta{after} before anything that has been smuggled and |\ekvbreakPostSneak|
%   will put \meta{after} after the stuff that has been sneaked out.
% \end{function}
%
% \begin{function}{\ekvsneak,\ekvsneakPre}
%   \begin{syntax}
%     \cs{ekvsneak}\marg{after}
%   \end{syntax}
%   Puts \meta{after} after the effects of |\ekvset|. The first variant will put
%   \meta{after} after any other tokens which might have been sneaked before,
%   while |\ekvsneakPre| will put \meta{after} before other smuggled stuff.
%   This reads and reinserts the remainder of the current |\ekvset| macro and
%   its argument list to do its job. A small usage example is shown in
%   \autoref{sec:sneakex}.
% \end{function}
%
% \begin{function}{\ekvchangeset}
%   \begin{syntax}
%     \cs{ekvchangeset}\marg{new-set}
%   \end{syntax}
%   Replaces the current set with \meta{new-set}, so for the rest of the current
%   |\ekvset| call, that call behaves as if it was called with
%   \texttt{\cs[no-index]{ekvset}\marg{new-set}}. Just like |\ekvsneak| this
%   reads and reinserts the remainder of the current |\ekvset| macro to do its
%   job. It is comparable to using \texttt{\meta{key}/.cd} in \pkg{pgfkeys}.
% \end{function}
%
% \bigskip
%
% \begin{function}{\ekv@name,\ekv@name@set,\ekv@name@key}
%   \begin{syntax}
%     \cs{ekv@name}\marg{set}\marg{key}\\
%     \cs{ekv@name@set}\marg{set}\\
%     \cs{ekv@name@key}\marg{key}
%   \end{syntax}
%   The names of the macros that correspond to a key in a set are build with
%   these macros. The default definition of |\ekv@name@set| is
%   ``\texttt{\csname ekv@name@set\endcsname{\set}}'' and the default of
%   |\ekv@name@key| is ``\texttt{\csname ekv@name@key\endcsname{\key}}''.
%   The complete name is build using |\ekv@name| which is equivalent to
%   \texttt{\cs[no-index]{ekv@name@set}\marg{set}\cs{ekv@name@key}\marg{key}}.
%   For |NoVal| keys an additional |N| gets appended irrespective of these
%   macros' definition, so their name is
%   \cs[no-index]{\csname ekv@name\endcsname{\set}{\key}N}. You might redefine
%   |\ekv@name@set| and |\ekv@name@key| locally but \emph{don't redefine}
%   |\ekv@name|!
% \end{function}
%
% \subsubsection{Bugs}
%
% Just like \pkg{keyval}, \expkv\ is bug free. But if you find \pmso{bugs}hidden
% features\footnote{Thanks, David!} you can tell me about them either via mail
% (see the first page) or directly on GitHub if you have an account there:
% \url{https://github.com/Skillmon/tex_expkv}
%
% \subsubsection{Comparisons}
%
% Comparisons of speed are done with a very simple test key and the help of the
% \pkg{l3benchmark} package. The key and its usage should be equivalent to
% \begin{lstlisting}
% \protected\ekvdef{test}{height}{\def\myheight{#1}}
% \ekvset{test}{ height = 6 }
% \end{lstlisting}
% and only the usage of the key, not its definition, is benchmarked. For the
% impatient, the essence of these comparisons regarding speed and buggy
% behaviour is contained in \autoref{tab:comp}.
%
% As far as I know \expkv\ is the only fully expandable \kv\ parser. I tried to
% compare \expkv\ to every \kv\ package listed on
% \href{https://ctan.org/topic/keyval}{CTAN}, however, one might notice that
% some of those are missing from this list. That's because I didn't get the
% others to work due to bugs, or because they just provide wrappers around other
% packages in this list.
%
% In this subsubsection is no benchmark of |\ekvparse| and |\keyval_parse:NNn|
% contained, as most other packages don't provide equivalent features to my
% knowledge. |\ekvparse| is slightly faster than |\ekvset|, but keep in mind
% that it does less. The same is true for |\keyval_parse:NNn| compared to
% |\keys_set:nn| of \pkg{expl3} (where the difference is much bigger).
%
% \paragraph{\pkg{keyval}} is about two times faster and has a comparable
% feature set just a slightly different way how it handles keys without values.
% That might be considered a drawback, as it limits the versatility, but also as
% an advantage, as it might reduce doubled code. Keep in mind that as soon as
% someone loads \pkg{xkeyval} the performance of \pkg{keyval} gets replaced by
% \pkg{xkeyval}'s.
%
% Also \pkg{keyval} has a bug, which unfortunately can't really be resolved
% without breaking backwards compatibility for \emph{many} documents, namely it
% strips braces from the argument before stripping spaces if the argument isn't
% surrounded by spaces, also it might strip more than one set of braces. Hence
% all of the following are equivalent in their outcome, though the last two
% lines should result in something different than the first two:
% \begin{lstlisting}[belowskip=0pt]
% \setkeys{foo}{bar=baz}
% \setkeys{foo}{bar= {baz}}
% \setkeys{foo}{bar={ baz}}
% \setkeys{foo}{bar={{baz}}}
% \end{lstlisting}
%
% \paragraph{\pkg{xkeyval}} is roughly fourteen times slower, but it provides
% more functionality, e.g., it has choice keys, boolean keys, and so on. It
% contains the same bug as \pkg{keyval} as it has to be compatible with it by
% design (it replaces \pkg{keyval}'s frontend), but also adds even more cases in
% which braces are stripped that shouldn't be stripped, worsening the situation.
%
% \paragraph{\pkg{ltxkeys}} is over 300 times slower -- which is funny, because
% it aims to be ``[\ldots] faster than these earlier packages [referring to
% \pkg{keyval} and \pkg{xkeyval}].'' Since it aims to have a bigger feature set
% than \pkg{xkeyval}, it most definitely also has a bigger feature set than
% \expkv. Also, it can't parse |\long| input, so as soon as your values contain
% a |\par|, it'll throw errors. Furthermore, \pkg{ltxkeys} doesn't strip outer
% braces at all by design, which, imho, is a weird design choice. In addition
% \pkg{ltxkeys} loads \pkg{catoptions} which is known to introduce bugs (e.g.,
% see \url{https://tex.stackexchange.com/questions/461783}).
%
% \paragraph{\pkg{l3keys}} is almost five times slower, but has an, imho, great
% interface to define keys. It strips \emph{all} outer spaces, even if somehow
% multiple spaces ended up on either end. It offers more features, but is pretty
% much bound to \pkg{expl3} code. Whether that's a drawback is up to you.
%
% \paragraph{\pkg{pgfkeys}} is a bit more than two times slower for one key,
% but has an \emph{enormous} feature set. However, since adding additional keys
% doesn't add as much needed time for \pkg{pgfkeys} compared to \expkv, it gets
% faster than \expkv\ at around eight \kv\ pairs.
% It has the same or a very similar bug \pkg{keyval} has. The brace bug (and
% also the category fragility) can be fixed by \pkg{pgfkeyx}, but this package
% was last updated in 2012 and it slows down |\pgfkeys| by factor~$8$. Also I
% don't know whether this might introduce new bugs.
%
% \paragraph{\pkg{kvsetkeys} with \pkg{kvdefinekeys}} is about three times
% slower, but it works even if commas and equals have category codes different
% from 12 (just as some other packages in this list). Else the features of the
% keys are equal to those of \pkg{keyval}, the parser has more features, though.
%
% \paragraph{\pkg{options}} is a bit slower for only a single value, but gets a
% tad faster than \expkv\ at around 10 \kv\ pairs. It has a much bigger feature
% set. Unfortunately it also suffers from the premature unbracing bug
% \pkg{keyval} has.
%
% \paragraph{\pkg{simplekv}} is hard to compare because I don't speak French (so
% I don't understand the documentation) and from what I can see, there is no
% direct way to define the equivalent test key. Nevertheless, I tested the
% closest possible equivalent of my test key while siding for \pkg{simplekv}'s
% design not forcing something into it it doesn't seem to be designed for. It is
% more than five times slower and has hard to predict behaviour regarding brace
% and space stripping, similar to \pkg{keyval}. The tested definition was:
% \begin{lstlisting}[belowskip=0pt]
% \usepackage{simplekv}
% \setKVdefault[simplekv]{height={ abc}}                     % key setup
% \setKV[simplekv]{ height = 6 }                             % benchmarked
% \end{lstlisting}
%
% \paragraph{\pkg{yax}} is over eighteen times slower. It has a pretty strange
% syntax, imho, and again a direct equivalent is hard to define. It has the
% premature unbracing bug, too. Also somehow loading \pkg{yax} broke
% \pkg{options} for me. The tested definition was:
% \begin{lstlisting}[belowskip=0pt]
% \usepackage{yax}
% \defactiveparameter yax {\storevalue\myheight yax:height } % key setup
% \setparameterlist{yax}{ height = 6 }                       % benchmarked
% \end{lstlisting}
%
% \begin{table}
%   \def\fnsym{\textcolor{red!80!black}{*}}%
%   \sisetup{round-precision=1, round-mode=places}%
%   \begingroup
%   \centering
%   \newcommand*\yes{\textcolor{red!80!black}  {yes}}^^A
%   \newcommand*\no {\textcolor{green!80!black}{no}}^^A
%   \caption[Comparison of \kv\ packages]
%     {^^A
%       Comparison of \kv\ packages. The packages are ordered from
%       fastest to slowest for one \kv\ pair. Benchmarking was done using
%       \pkg{l3benchmark} and the scripts in the \file{Benchmarks} folder
%       of the \href{https://github.com/Skillmon/tex_expkv}{git repository}.
%       The columns $p_i$ are the polynomial coefficients of a linear fit to the
%       run-time, $p_0$ can be interpreted as the overhead for initialisation
%       and $p_1$ the cost per key. The $T_0$ column is the actual mean ops
%       needed for an empty list argument, as the linear fit doesn't match that
%       point well in general. The column ``BB'' lists whether the
%       parsing is affected by some sort of brace bug, ``CF'' stands for
%       category code fragile and lists whether the parsing breaks with active
%       commas or equal signs.^^A
%       \label{tab:comp}^^A
%     }
%   \begin{tabular}
%       {>{\collectcell\pkg}l<{\endcollectcell}*3{S[table-format=4.1]}ccc}
%     \toprule
%     \rmfamily Package & {$p_1$} & {$p_0$} & {$T_0$}& BB & CF & Date \\
%     \midrule
%     keyval    &   13.539 &    1.844 &    6.969 & \yes & \yes & 2014-10-28 \\
%     \expkv    &   26.411 &    2.391 &   12.610 & \no  & \no  & 2020-02-22 \\
%     options   &   25.532 &   10.518 &   21.393 & \yes & \yes & 2015-03-01 \\
%     pgfkeys   &   26.444 &   40.939 &   55.790 & \yes & \yes & 2020-01-08 \\
%     kvsetkeys & {\fnsym} & {\fnsym} &   41.620 & \no  & \no  & 2019-12-15 \\
%     l3keys    &  114.456 &   35.357 &   52.667 & \no  & \no  & 2020-02-14 \\
%     simplekv  &  160.523 &   10.809 &    8.698 & \yes & \yes & 2017-08-08 \\
%     xkeyval   &  260.872 &  180.435 &  164.167 & \yes & \yes & 2014-12-03 \\
%     yax       &  507.643 &   62.863 &  123.533 & \yes & \yes & 2010-01-22 \\
%     ltxkeys   & 3932.759 & 4737.818 & 5883.000 & \no  & \no  & 2012-11-17 \\
%     \bottomrule
%   \end{tabular}
%   \par
%   \endgroup
%   \medskip
%   \fnsym For \pkg{kvsetkeys} the linear model used for the other
%   packages is a poor fit, \pkg{kvsetkeys} seems to have approximately
%   quadratic run-time, the coefficients of the second degree polynomial fit are
%   $p_2=\num{9.911}$, $p_1=\num{40.523}$, and $p_0=\num{61.500}$. Of course the
%   other packages might not really have linear run-time, but at least from 1~to
%   20~keys the fits don't seem too bad (the maximum ratio $p_2/p_1$
%   for the other packages is \num{9.697e-3}). If one extrapolates the fits
%   for 100 \kv\ pairs one finds that most of them match pretty well, the
%   exception being \pkg{ltxkeys}, which behaves quadratic as well with
%   $p_2=\num{31.685}$, $p_1=\num{3267.368}$, and $p_0=\num{7177.582}$.
% \end{table}
%
% \subsection{Examples}
%
% \subsubsection{Standard Use-Case}
%
% Say we have a macro for which we want to create a \kv\ interface. The macro
% has a parameter, which is stored in the dimension |\ourdim| having a default
% value from its initialization. Now we want to be able to change that dimension
% with the |width| key to some specified value. For that we'd do
% \begin{lstlisting}
% \newdimen\ourdim
% \ourdim=150pt
% \protected\ekvdef{our}{width}{\ourdim=#1\relax}
% \end{lstlisting}
% as you can see, we use the set |our| here. We want the key to behave different
% if no value is specified. In that case the key should not use its initial
% value, but be smart and determine the available space from
% \cs[no-index]{hsize}, so we also define
% \begin{lstlisting}
% \protected\ekvdefNoVal{our}{width}{\ourdim=.9\hsize}
% \end{lstlisting}
% Now we set up our macro to use this \kv\ interface
% \begin{lstlisting}
% \protected\def\ourmacro#1{\begingroup\ekvset{our}{#1}\the\ourdim\endgroup}
% \end{lstlisting}
% Finally we can use our macro like in the following\\
% \begin{minipage}{6cm}
% \begin{lstlisting}
% \ourmacro{}\par
% \ourmacro{width}\par
% \ourmacro{width=5pt}\par
% \end{lstlisting}
% \end{minipage}^^A
% \begin{minipage}{\dimexpr\linewidth-6cm\relax}
%   \newdimen\ourdim\ourdim=150pt
%   \protected\ekvdef{our}{width}{\ourdim=#1\relax}^^A
%   \protected\ekvdefNoVal{our}{width}{\ourdim=.9\hsize}^^A
%   \protected\def\ourmacro#1^^A
%     {\begingroup\ekvset{our}{#1}\the\ourdim\endgroup}^^A
%   \ourmacro{}\par
%   \ourmacro{width}\par
%   \ourmacro{width=5pt}\par
% \end{minipage}
%
% \paragraph{The same key using \protect\expkvd}
% Using \expkvd\ we can set up the equivalent key using a \kv\ interface, after
% the following we could use |\ourmacro| in the same way as above. \expkvd\ will
% allocate and initialise |\ourdim| and define the |width| key |\protected| for
% us, so the result will be exactly the same -- with the exception that the
% default will use |\ourdim=.9\hsize\relax| instead.
% \begin{lstlisting}
% \input expkv-def                 % or \usepackage{expkv-def}
% \ekvdefinekeys{our}
%   {
%     dimen    width = \ourdim,
%     qdefault width = .9\hsize,
%     initial  width = 150pt
%   }
% \end{lstlisting}
%
% \subsubsection{An Expandable \kv\ Macro Using \cs[no-index]{ekvsneak}}
% \label{sec:sneakex}
%
% Let's set up an expandable macro, that uses a \kv\ interface. The problems
% we'll face for this are:
% \begin{enumerate}
%   \item ignoring duplicate keys
%   \item default values for keys which weren't used
%   \item providing the values as the correct argument to a macro (ordered)
% \end{enumerate}
%
% First we need to decide which \kv\ parsing macro we want to do this with,
% |\ekvset| or |\ekvparse|. For this example we also want to show the
% usage of |\ekvsneak|, hence we'll choose |\ekvset|. And we'll have to use
% |\ekvset| such that it builds a parsable list for our macro internals. To
% gain back control after |\ekvset| is done we have to put an internal of our
% macro at the start of that list, so we use an internal key that uses
% |\ekvsneakPre| after any user input.
%
% To ignore duplicates will be easy if the value of the key used last will be
% put first in the list, so the following will use |\ekvsneakPre| for the
% user-level keys. If we wanted some key for which the first usage should be the
% the binding one we would use |\ekvsneak| instead for that key.
%
% Providing default values can be done in different ways, we'll use a simple
% approach in which we'll just put the outcome of our keys if they were used
% with default values before the parsing list terminator.
%
% Ordering the keys can be done simply by searching for a specific token for
% each argument which acts like a flag, so our sneaked out values will include
% specific tokens acting as markers.
%
% Now that we have answers for our technical problems, we have to decide what
% our example macro should do. How about we define a macro that calculates the
% sine of a number and rounds that to a specified precision? As a small extra
% this macro should understand input in radian and degree and the used
% trigonometric function should be selectable as well. For the hard part of this
% task (expandably evaluating trigonometric functions) we'll use the \pkg{xfp}
% package.
%
% First we set up our keys according to our earlier considerations and set up
% the user facing macro |\sine|. The end marker of the parsing list will be a
% |\sine@stop| token, which we don't need to define and we put our defaults
% right before it.
%
% \begin{lstlisting}
% \RequirePackage{xfp}
% \makeatletter
% \ekvdef{expex}{f}{\ekvsneakPre{\f{#1}}}
% \ekvdef{expex}{round}{\ekvsneakPre{\rnd{#1}}}
% \ekvdefNoVal{expex}{degree}{\ekvsneakPre{\deg{d}}}
% \ekvdefNoVal{expex}{radian}{\ekvsneakPre{\deg{}}}
% \ekvdefNoVal{expex}{internal}{\ekvsneakPre{\sine@rnd}}
% \newcommand*\sine[2]
%   {\ekvset{expex}{#1,internal}\rnd{3}\deg{d}\f{sin}\sine@stop{#2}}
% \end{lstlisting}
% For the sake of simplicity we defined the macro |\sine| with two mandatory
% arguments, the first being the \kv\ list, the second the argument to the
% trigonometric function. We could've used \pkg{xparse}'s facilities here to
% define an expandable macro which takes an optional argument instead.
%
% Now we need to define some internal macros to extract the value of each key's
% last usage (remember that this will be the group after the first special
% flag-token). For that we use one delimited macro per key.
% \begin{lstlisting}
% \def\sine@rnd#1\rnd#2#3\sine@stop{\sine@deg#1#3\sine@stop{#2}}
% \def\sine@deg#1\deg#2#3\sine@stop{\sine@f#1#3\sine@stop{#2}}
% \def\sine@f#1\f#2#3\sine@stop{\sine@final{#2}}
% \end{lstlisting}
% After the macros |\sine@rnd|, |\sine@deg|, and |\sine@f| the macro
% |\sine@final| will see
% \texttt
%   {^^A
%     \cs[no-index]{sine@final}^^A
%     \marg{f}\marg{degree/radian}\marg{round}\marg{num}^^A
%   }.
% Now |\sine@final| has to expandably deal with those arguments such that the
% |\fpeval| macro of \pkg{xfp} gets the correct input. Luckily this is pretty
% straight forward in this example. In |\fpeval| the trigonometric functions
% have names such as |sin| or |cos| and the degree taking variants |sind| or
% |cosd|. And since the |degree| key puts a |d| in |#2| and the |radian| key
% leaves |#2| empty all we have to do to get the correct function name is stick
% the two together.
% \begin{lstlisting}
% \newcommand*\sine@final[4]{\fpeval{round(#1#2(#4),#3)}}
% \makeatother
% \end{lstlisting}
% Let's test our macro:\\
% \begin{minipage}{.7\linewidth}
%   \begin{lstlisting}[gobble=4]
%   \sine{}{60}\par
%   \sine{round=10}{60}\par
%   \sine{f=cos,radian}{pi}\par
%   \edef\myval{\sine{f=tan}{1}}\texttt{\meaning\myval}
%   \end{lstlisting}
% \end{minipage}\hfill
% \begin{minipage}{.2\linewidth}
%   \makeatletter
%   \ekvdef{expex}{f}{\ekvsneakPre{\f{#1}}}^^A
%   \ekvdef{expex}{round}{\ekvsneakPre{\rnd{#1}}}^^A
%   \ekvdefNoVal{expex}{degree}{\ekvsneakPre{\deg{d}}}^^A
%   \ekvdefNoVal{expex}{radian}{\ekvsneakPre{\deg{}}}^^A
%   \ekvdefNoVal{expex}{internal}{\ekvsneakPre{\sine@rnd}}^^A
%   \newcommand*\sine[2]
%     {\ekvset{expex}{#1,internal}\rnd{3}\deg{d}\f{sin}\sine@stop{#2}}^^A
%   \def\sine@rnd#1\rnd#2#3\sine@stop{\sine@deg#1#3\sine@stop{#2}}^^A
%   \def\sine@deg#1\deg#2#3\sine@stop{\sine@f#1#3\sine@stop{#2}}^^A
%   \def\sine@f#1\f#2#3\sine@stop{\sine@final{#2}}^^A
%   \newcommand*\sine@final[4]{\fpeval{round(#1#2(#4),#3)}}^^A
%   \makeatother
%   \sine{}{60}\par
%   \sine{round=10}{60}\par
%   \sine{f=cos,radian}{pi}\par
%   \edef\myval{\sine{f=tan}{1}}\texttt{\meaning\myval}
% \end{minipage}
%
%
% \subsection{Error Messages}
%
% \expkv\ should only send messages in case of errors, there are no warnings and
% no info messages. In this subsection those errors are listed.
%
% \subsubsection{Load Time}
%
% \file{expkv.tex} checks whether \eTeX\ is available. If it isn't, an error
% will be thrown using |\errmessage|:
% \begin{lstlisting}
% ! expkv Error: e-TeX required.
% \end{lstlisting}
%
% \subsubsection{Defining Keys}
%
% If you get any error from \expkv\ while you're trying to define a key, the
% definition will be aborted and gobbled.
%
% If you try to define a key with an empty set name you'll get:
% \begin{lstlisting}
% ! expkv Error: empty set name not allowed.
% \end{lstlisting}
%
% Similarly, if you try to define a key with an empty key name:
% \begin{lstlisting}
% ! expkv Error: empty key name not allowed.
% \end{lstlisting}
%
% Both of these messages are done in a way that doesn't throw additional errors
% due to |\global|, |\long|, etc., not being used correctly if you prefixed one
% of the defining macros.
%
% \subsubsection{Using Keys}
%
% This subsubsection contains the errors thrown during |\ekvset|. The errors are
% thrown in an expandable manner by providing an undefined macro. In the
% following messages \lstinline|<key>| gets replaced with the problematic key's
% name, and \lstinline|<set>| with the corresponding set. If any errors during
% \kv\ handling are encountered, the entry in the comma separated list will be
% omitted after the error is thrown and the next \kv\ pair will be parsed.
%
% If you're using an undefined key you'll get:
% \begin{lstlisting}
% ! Undefined control sequence.
% <argument> \! expkv Error:
%                            unknown key (`<key>', set `<set>').
% \end{lstlisting}
%
% If you're using a key for which only a normal version and no |NoVal| version
% is defined, but don't provide a value, you'll get:
% \begin{lstlisting}
% ! Undefined control sequence.
% <argument> \! expkv Error:
%                            value required (`<key>', set `<set>').
% \end{lstlisting}
%
% If you're using a key for which only a |NoVal| version and no normal version
% is defined, but provide a value, you'll get:
% \begin{lstlisting}
% ! Undefined control sequence.
% <argument> \! expkv Error:
%                            value forbidden (`<key>', set `<set>').
% \end{lstlisting}
%
% If you're using a set for which you never executed one of the defining macros
% from \autoref{sec:define} you'll get a low level \TeX\ error, as that isn't
% actively tested by the parser (and hence will lead to undefined behaviour and
% not be gracefully ignored). The error will look like
% \begin{lstlisting}
% ! Missing \endcsname inserted.
% <to be read again>
%                    \! expkv Error: Set `<set>' undefined.
% \end{lstlisting}
%
% \subsection{License}
%
% Copyright \textcopyright\ 2020\unless\ifnum\year=2020--\the\year\fi\
% Jonathan P. Spratte
%
% \medskip\noindent
% This work may be distributed and/or modified under the conditions of the
% \LaTeX\ Project Public License (LPPL), either version 1.3c of this license or
% (at your option) any later version. The latest version of this license is in
% the file:
%
% \url{http://www.latex-project.org/lppl.txt}
%
% \noindent
% This work is ``maintained'' (as per LPPL maintenance status) by
%
% \mbox{Jonathan P. Spratte}.
%
% \end{documentation}^^A=<<
%
% \begin{implementation}^^A>>=
%
% \clearpage
%
% \section{Implementation}^^A>>=
%^^A the LaTeX package >>=
% \subsection{The \LaTeX\ Package}
% First we set up the \LaTeX\ package. That one doesn't really do much except
% |\input|ting the generic code and identifying itself as a package.
% \gobbledocstriptag
%<*pkg>
%    \begin{macrocode}
\def\ekv@tmp
  {%
    \ProvidesFile{expkv.tex}%
      [\ekvDate\space v\ekvVersion\space an expandable key=val implementation]%
  }
% \iffalse meta-comment
%
% File: expkv.dtx Copyright (C) 2020 Jonathan P. Spratte
%
% This work  may be  distributed and/or  modified under  the conditions  of the
% LaTeX Project Public License (LPPL),  either version 1.3c  of this license or
% (at your option) any later version.  The latest version of this license is in
% the file:
%
%   http://www.latex-project.org/lppl.txt
%
% ------------------------------------------------------------------------------
%
%<*driver>^^A>>=
\def\nameofplainTeX{plain}
\ifx\fmtname\nameofplainTeX\else
  \expandafter\begingroup
\fi
\input l3docstrip.tex
\askforoverwritefalse
\preamble

--------------------------------------------------------------
expkv -- an expandable key=val implementation
E-mail: jspratte@yahoo.de
Released under the LaTeX Project Public License v1.3c or later
See http://www.latex-project.org/lppl.txt
--------------------------------------------------------------

Copyright (C) 2020 Jonathan P. Spratte

This  work may be  distributed and/or  modified under  the conditions  of the
LaTeX Project Public License (LPPL),  either version 1.3c  of this license or
(at your option) any later version.  The latest version of this license is in
the file:

  http://www.latex-project.org/lppl.txt

This work is "maintained" (as per LPPL maintenance status) by
  Jonathan P. Spratte.

This work consists of the file  expkv.dtx
and the derived files           expkv.pdf
                                expkv.sty
                                expkv.tex

\endpreamble
% stop docstrip adding \endinput
\postamble
\endpostamble
\generate{\file{expkv.sty}{\from{expkv.dtx}{pkg}}}
\generate{\file{expkv.tex}{\from{expkv.dtx}{tex}}}
\ifx\fmtname\nameofplainTeX
  \expandafter\endbatchfile
\else
  \expandafter\endgroup
\fi
%
\IfFileExists{expkv.tex}{\input{expkv.tex}}{}
\ProvidesFile{expkv.dtx}
  [\csname ekvDate\endcsname\ an expandable key=val implementation]
\PassOptionsToPackage{full}{textcomp}
\documentclass{l3doc}
\RequirePackage[oldstylenums,nott]{kpfonts}
\input{glyphtounicode}
\pdfgentounicode=1
\RequirePackage{xfp} ^^A required for an example
\RequirePackage{listings}
\RequirePackage{booktabs}
\RequirePackage{array}
\RequirePackage{collcell}
\RequirePackage{siunitx}
\RequirePackage{xcolor}
\RequirePackage{caption}
\RequirePackage{microtype}
\RequirePackage{accsupp}
\lstset
  {
    ,flexiblecolumns=false
    ,basewidth=.53em
    ,gobble=2
    ,basicstyle=\fontfamily{jkp}\itshape
    ,morekeywords=^^A
      {^^A
        \ekvdef,\ekvdefNoVal,\ekvlet,\ekvletNoVal,\ekvletkv,\ekvletkvNoVal,^^A
        \ekvset,\ekvparse,\ekvVersion,\ekvDate,\ekvifdefined,^^A
        \ekvifdefinedNoVal,\ekvbreak,\ekvbreakPreSneak,\ekvbreakPostSneak,^^A
        \ekvsneak,\ekvsneakPre,\ekvchangeset
      }
    ,morecomment=[l]\%
    ,commentstyle=\color[gray]{0.4}
    ,literate={\{}{{\CodeSymbol\{}}{1}
              {\}}{{\CodeSymbol\}}}{1}
    ^^A,literate=*{<key>}{\key}{4}{<set>}{\set}{4}
  }
\newcommand*\CodeSymbol[1]{\textbf{#1}}
\RequirePackage{randtext}
\let\metaORIG\meta
\protected\def\meta #1{\texttt{\metaORIG{#1}}}
\renewcommand*\thefootnote{\fnsymbol{footnote}}
\definecolor{expkvred}{HTML}{9F393D}
\colorlet{expkvgrey}{black!75}
\makeatletter
\newcommand*\expkv
  {^^A
    \texorpdfstring
      {^^A
        \mbox
          {^^A
            \BeginAccSupp{ActualText=expkv}^^A
            \href{https://github.com/Skillmon/tex_expkv}
              {^^A
                \rmfamily
                \bfseries
                {\color{expkvgrey}e\kern-.05em x\kern-.05em}^^A
                \lower.493ex
                  \hbox{{\color{expkvgrey}P}\kern-.1em{\color{expkvred}k}}^^A
                \kern-.18em{\color{expkvred}v}^^A
              }^^A
            \EndAccSupp{}^^A
          }^^A
      }
      {expkv}^^A
  }
\newcommand*\expkvd
  {^^A
    \texorpdfstring
      {^^A
        \mbox
          {^^A
            \BeginAccSupp{ActualText=expkv-def}^^A
            \rmfamily
            \bfseries
            {\color{expkvgrey}e\kern-.05em x\kern-.05em}^^A
            \lower.493ex
              \hbox{{\color{expkvgrey}P}\kern-.1em{\color{expkvred}k}}^^A
            \kern-.18em{\color{expkvred}v}^^A
            {\color{expkvgrey}^^A
              \kern.05em\rule[-.1ex]{.08em}{1.2ex}\kern.05em\textsc{def}^^A
            }^^A
            \EndAccSupp{}^^A
          }^^A
      }
      {expkv-def}^^A
  }
\newcommand\kv{\meta{key}=\meta{value}}
\newcommand\key{\meta{key}}
\newcommand\val{\meta{value}}
\newcommand\set{\meta{set}}
\hypersetup{linkcolor=red!80!black,urlcolor=purple!80!black}
\DoNotIndex{\def,\edef,\,,\=,\begingroup,\catcode,\chardef,\csname,\endcsname}
\DoNotIndex{\endgroup,\endinput,\errmessage,\expandafter,\input,\let,\long}
\DoNotIndex{\noexpand,\protected,\ProvidesFile,\ProvidesPackage,\relax,\space}
\DoNotIndex{\@,\unexpanded,\string,\expanded}
\DoNotIndex{\ifcsname}
\DoNotIndex{\ifx}
\DoNotIndex{\else}
\DoNotIndex{\fi}
\@gobble\fi ^^A ignoring \ifx and \ifcsname, but only one \fi
\@ifdefinable\gobbledocstriptag{\def\gobbledocstriptag#1>{}}
\newcommand*\pmso[1] ^^A poor man's strike out
  {^^A
    \leavevmode
    \begingroup
      \sbox0{#1}^^A
      \rlap{\vrule height .6ex depth -.5ex width \wd0\relax}^^A
      \usebox0\relax
    \endgroup
  }
\makeatother
\begin{document}
  \title
    {^^A
      \texorpdfstring
        {^^A
          \huge\expkv
          \\[\medskipamount]
          \Large an expandable \kv\ implementation^^A
        }{expkv - an expandable <key>=<value> implementation}^^A
    }
  \date{\ekvDate\space v\ekvVersion}
  \author{Jonathan P. Spratte\thanks{\protect\randomize{jspratte@yahoo.de}}}
  \DocInput{expkv.dtx}
\end{document}
%</driver>^^A=<<
% \fi
%
% \maketitle
% \renewcommand*\thefootnote{\arabic{footnote}}
%
% \begin{abstract}
% \noindent\parfillskip=0pt
% \expkv\ provides a small interface for \kv\ parsing. The parsing macro is
% fully expandable, the \meta{code} of your keys might be not. \expkv\ is pretty
% fast, but not the fastest available \kv\ solution (\pkg{keyval} for instance
% is thrice as fast, but not expandable and it might strip braces it shouldn't
% have stripped).
% \end{abstract}
%
% \tableofcontents
%
% \begin{documentation}^^A>>=
%
% \section{Documentation}
%
% \expkv\ provides an expandable \kv\ parser. The \kv\ pairs should be given as
% a comma separated list and the separator between a \key\ and the associated
% \val\ should be an equal sign. Both, the commas and the equal signs, might be
% of category 12 (other) or 13 (active). To support this is necessary as for
% example \pkg{babel} turns characters active for some languages, for instance
% the equal sign is turned active for Turkish.
%
% \expkv\ is usable as generic code or as a \LaTeX\ package. To use it, just
% use one of:
% \begin{lstlisting}
% \usepackage{expkv} % LaTeX
% \input expkv       % plainTeX
% \end{lstlisting}
% The \LaTeX\ package doesn't do more than \file{expkv.tex}, except calling
% |\ProvidesPackage| and setting things up such that \file{expkv.tex} will use
% |\ProvidesFile|.
%
% \subsection{Setting up Keys}\label{sec:define}
% \expkv\ provides a rather simple approach to setting up keys, similar to
% \pkg{keyval}. However there is an auxiliary package named
% \href{https://github.com/Skillmon/tex_expkv-def}{\expkvd} (the files use
% \file{expkv-def} instead because \string| isn't allowed in package names by
% CTAN) which provides a more sophisticated interface, similar to well
% established packages like \pkg{pgfkeys} or \pkg{l3keys}.
%
% Keys in \expkv\ (as in almost all other \kv\ implementations) belong to a
% \emph{set} such that different sets can contain keys of the same name. Unlike
% many other implementations \expkv\ doesn't provide means to set a default
% value, instead we have keys that take values and keys that don't (the latter
% are called |NoVal| keys by \expkv), but both can have the same name (on the
% user level).
%
% The following macros are available to define new keys. Those macros containing
% ``|def|'' in their name can be prefixed by anything allowed to prefix |\def|,
% prefixes allowed for |\let| can prefix those with ``|let|'' in their name,
% accordingly. Neither \set\ nor \key\ are allowed to be empty for new keys and
% must not contain a |\par| or tokens that expand to it -- they must be legal
% inside of |\csname ...\endcsname|.
%
% \begin{function}{\ekvdef}
%   \begin{syntax}
%     \cs{ekvdef}\marg{set}\marg{key}\marg{code}
%   \end{syntax}
%   Defines a \key\ taking a value in a \set\ to expand to \meta{code}. In
%   \meta{code} you can use |#1| to refer to the given value.
% \end{function}
%
% \begin{function}{\ekvdefNoVal}
%   \begin{syntax}
%     \cs{ekvdefNoVal}\marg{set}\marg{key}\marg{code}
%   \end{syntax}
%   Defines a no value taking \key\ in a \set\ to expand to \meta{code}.
% \end{function}
%
% \begin{function}{\ekvlet}
%   \begin{syntax}
%     \cs{ekvlet}\marg{set}\marg{key}\meta{cs}
%   \end{syntax}
%   Let the value taking \key\ in \set\ to \meta{cs}, there are no checks on
%   \meta{cs} enforced.
% \end{function}
%
% \begin{function}{\ekvletNoVal}
%   \begin{syntax}
%     \cs{ekvletNoVal}\marg{set}\marg{key}\meta{cs}
%   \end{syntax}
%   Let the no value taking \key\ in \set\ to \meta{cs}, it is not checked
%   whether \meta{cs} exists or that it takes no parameter.
% \end{function}
%
% \begin{function}{\ekvletkv}
%   \begin{syntax}
%     \cs{ekvletkv}\marg{set}\marg{key}\marg{set2}\marg{key2}
%   \end{syntax}
%   Let the \key\ in \set\ to \meta{key2} in \meta{set2}, it is not checked
%   whether that second key exists.
% \end{function}
%
% \begin{function}{\ekvletkvNoVal}
%   \begin{syntax}
%     \cs{ekvletkvNoVal}\marg{set}\marg{key}\marg{set2}\marg{key2}
%   \end{syntax}
%   Let the \key\ in \set\ to \meta{key2} in \meta{set2}, it is not checked
%   whether that second key exists.
% \end{function}
%
% \subsection{Parsing Keys}
%
% \begin{function}{\ekvset}
%   \begin{syntax}
%     \cs{ekvset}\marg{set}\{\kv,\ldots\}
%   \end{syntax}
%   Splits \kv\ pairs on commas. From both \key\ and \val\ up to one space is
%   stripped from both ends, if then only a braced group remains the braces are
%   stripped as well. So |\ekvset{foo}{bar=baz}| and
%   |\ekvset{foo}{ {bar}= {baz} }| will both do
%   \texttt{\cs[no-index]{\meta{foobarcode}}\{baz\}}, so you can hide commas,
%   equal signs and spaces at the ends of either \key\ or \val\ by putting
%   braces around them. If you omit the equal sign the code of the key created
%   with the |NoVal| variants described in \autoref{sec:define} will be
%   executed. If \kv\ contains more than a single unhidden equal sign, it will
%   be split at the first one and the others are considered part of the value.
%   |\ekvset| should be nestable.
% \end{function}
%
% \begin{function}{\ekvparse}
%   \begin{syntax}
%     \cs{ekvparse}\meta{cs1}\meta{cs2}\{\kv,\ldots\}
%   \end{syntax}
%   This macro parses the \kv\ pairs and provides those list elements which are
%   only keys as the argument to \meta{cs1}, and those which are a \kv\ pair to
%   \meta{cs2} as two arguments. It is fully expandable as well and returns the
%   parsed list in |\unexpanded|, which has no effect outside of an |\expanded|
%   or |\edef| context\footnotemark.
%   If you need control over the necessary steps of expansion you
%   can use |\expanded| around it.
%
%   |\ekvbreak|, |\ekvsneak|, and |\ekvchangeset| and their relatives don't work
%   in |\ekvparse|.  It is analogue to \pkg{expl3}'s |\keyval_parse:NNn|, but
%   not with the same parsing rules -- |\keyval_parse:NNn| throws an error on
%   multiple equal signs per \kv\ pair and on empty \key\ names in a \kv\ pair,
%   both of which |\ekvparse| doesn't deal with.
%
%   As a small example:
% \begin{lstlisting}
% \ekvparse\handlekey\handlekeyval{foo = bar, key, baz={zzz}}
% \end{lstlisting}
%   would expand to
% \begin{lstlisting}
% \handlekeyval{foo}{bar}\handlekey{key}\handlekeyval{baz}{zzz}
% \end{lstlisting}
%   and afterwards |\handlekey| and |\handlekeyval| would have to further handle
%   the \key. There are no macros like these two contained in \expkv, you have
%   to set them up yourself if you want to use |\ekvparse| (of course the names
%   might differ). If you need the results of |\ekvparse| as the argument for
%   another macro, you should use |\expanded| as only then the input stream will
%   contain the output above:
% \begin{lstlisting}
% \expandafter\handle\expanded{\ekvparse\k\kv{foo = bar, key, baz={zzz}}}
% \end{lstlisting}
%   would expand to
% \begin{lstlisting}
% \handle\kv{foo}{bar}\k{key}\kv{baz}{zzz}
% \end{lstlisting}
% \end{function}
% \footnotetext 
%   {^^A
%     This is a change in behaviour, previously (v0.3 and before) \cs{ekvparse}
%     would expand in exactly two steps. This isn't always necessary, but makes
%     the parsing considerably slower. If this is necessary for your application
%     you can put an \cs[no-index]{expanded} around it and will still be faster
%     since you need only a single \cs[no-index]{expandafter} this way.^^A
%   }.
%
% \subsection{Miscellaneous}
%
% \subsubsection{Other Macros}
%
% \expkv\ provides some other macros which might be of interest.
%
% \begin{function}{\ekvVersion,\ekvDate}
%   These two macros store the version and date of the package.
% \end{function}
% \begin{function}{\ekvifdefined,\ekvifdefinedNoVal}
%   \begin{syntax}
%     \cs{ekvifdefined}\marg{set}\marg{key}\marg{true}\marg{false}
%     \cs{ekvifdefinedNoVal}\marg{set}\marg{key}\marg{true}\marg{false}
%   \end{syntax}
%   These two macros test whether there is a \key\ in \set. It is false if
%   either a hash table entry doesn't exist for that key or its meaning is
%   |\relax|.
% \end{function}
%
% \begin{function}{\ekvbreak,\ekvbreakPreSneak,\ekvbreakPostSneak}
%   \begin{syntax}
%     \cs{ekvbreak}\marg{after}
%   \end{syntax}
%   Gobbles the remainder of the current |\ekvset| macro and its argument list
%   and reinserts \meta{after}. So this can be used to break out of |\ekvset|.
%   The first variant will also gobble anything that has been sneaked out using
%   |\ekvsneak| or |\ekvsneakPre|, while |\ekvbreakPreSneak| will put
%   \meta{after} before anything that has been smuggled and |\ekvbreakPostSneak|
%   will put \meta{after} after the stuff that has been sneaked out.
% \end{function}
%
% \begin{function}{\ekvsneak,\ekvsneakPre}
%   \begin{syntax}
%     \cs{ekvsneak}\marg{after}
%   \end{syntax}
%   Puts \meta{after} after the effects of |\ekvset|. The first variant will put
%   \meta{after} after any other tokens which might have been sneaked before,
%   while |\ekvsneakPre| will put \meta{after} before other smuggled stuff.
%   This reads and reinserts the remainder of the current |\ekvset| macro and
%   its argument list to do its job. A small usage example is shown in
%   \autoref{sec:sneakex}.
% \end{function}
%
% \begin{function}{\ekvchangeset}
%   \begin{syntax}
%     \cs{ekvchangeset}\marg{new-set}
%   \end{syntax}
%   Replaces the current set with \meta{new-set}, so for the rest of the current
%   |\ekvset| call, that call behaves as if it was called with
%   \texttt{\cs[no-index]{ekvset}\marg{new-set}}. Just like |\ekvsneak| this
%   reads and reinserts the remainder of the current |\ekvset| macro to do its
%   job. It is comparable to using \texttt{\meta{key}/.cd} in \pkg{pgfkeys}.
% \end{function}
%
% \bigskip
%
% \begin{function}{\ekv@name,\ekv@name@set,\ekv@name@key}
%   \begin{syntax}
%     \cs{ekv@name}\marg{set}\marg{key}\\
%     \cs{ekv@name@set}\marg{set}\\
%     \cs{ekv@name@key}\marg{key}
%   \end{syntax}
%   The names of the macros that correspond to a key in a set are build with
%   these macros. The default definition of |\ekv@name@set| is
%   ``\texttt{\csname ekv@name@set\endcsname{\set}}'' and the default of
%   |\ekv@name@key| is ``\texttt{\csname ekv@name@key\endcsname{\key}}''.
%   The complete name is build using |\ekv@name| which is equivalent to
%   \texttt{\cs[no-index]{ekv@name@set}\marg{set}\cs{ekv@name@key}\marg{key}}.
%   For |NoVal| keys an additional |N| gets appended irrespective of these
%   macros' definition, so their name is
%   \cs[no-index]{\csname ekv@name\endcsname{\set}{\key}N}. You might redefine
%   |\ekv@name@set| and |\ekv@name@key| locally but \emph{don't redefine}
%   |\ekv@name|!
% \end{function}
%
% \subsubsection{Bugs}
%
% Just like \pkg{keyval}, \expkv\ is bug free. But if you find \pmso{bugs}hidden
% features\footnote{Thanks, David!} you can tell me about them either via mail
% (see the first page) or directly on GitHub if you have an account there:
% \url{https://github.com/Skillmon/tex_expkv}
%
% \subsubsection{Comparisons}
%
% Comparisons of speed are done with a very simple test key and the help of the
% \pkg{l3benchmark} package. The key and its usage should be equivalent to
% \begin{lstlisting}
% \protected\ekvdef{test}{height}{\def\myheight{#1}}
% \ekvset{test}{ height = 6 }
% \end{lstlisting}
% and only the usage of the key, not its definition, is benchmarked. For the
% impatient, the essence of these comparisons regarding speed and buggy
% behaviour is contained in \autoref{tab:comp}.
%
% As far as I know \expkv\ is the only fully expandable \kv\ parser. I tried to
% compare \expkv\ to every \kv\ package listed on
% \href{https://ctan.org/topic/keyval}{CTAN}, however, one might notice that
% some of those are missing from this list. That's because I didn't get the
% others to work due to bugs, or because they just provide wrappers around other
% packages in this list.
%
% In this subsubsection is no benchmark of |\ekvparse| and |\keyval_parse:NNn|
% contained, as most other packages don't provide equivalent features to my
% knowledge. |\ekvparse| is slightly faster than |\ekvset|, but keep in mind
% that it does less. The same is true for |\keyval_parse:NNn| compared to
% |\keys_set:nn| of \pkg{expl3} (where the difference is much bigger).
%
% \paragraph{\pkg{keyval}} is about two times faster and has a comparable
% feature set just a slightly different way how it handles keys without values.
% That might be considered a drawback, as it limits the versatility, but also as
% an advantage, as it might reduce doubled code. Keep in mind that as soon as
% someone loads \pkg{xkeyval} the performance of \pkg{keyval} gets replaced by
% \pkg{xkeyval}'s.
%
% Also \pkg{keyval} has a bug, which unfortunately can't really be resolved
% without breaking backwards compatibility for \emph{many} documents, namely it
% strips braces from the argument before stripping spaces if the argument isn't
% surrounded by spaces, also it might strip more than one set of braces. Hence
% all of the following are equivalent in their outcome, though the last two
% lines should result in something different than the first two:
% \begin{lstlisting}[belowskip=0pt]
% \setkeys{foo}{bar=baz}
% \setkeys{foo}{bar= {baz}}
% \setkeys{foo}{bar={ baz}}
% \setkeys{foo}{bar={{baz}}}
% \end{lstlisting}
%
% \paragraph{\pkg{xkeyval}} is roughly fourteen times slower, but it provides
% more functionality, e.g., it has choice keys, boolean keys, and so on. It
% contains the same bug as \pkg{keyval} as it has to be compatible with it by
% design (it replaces \pkg{keyval}'s frontend), but also adds even more cases in
% which braces are stripped that shouldn't be stripped, worsening the situation.
%
% \paragraph{\pkg{ltxkeys}} is over 300 times slower -- which is funny, because
% it aims to be ``[\ldots] faster than these earlier packages [referring to
% \pkg{keyval} and \pkg{xkeyval}].'' Since it aims to have a bigger feature set
% than \pkg{xkeyval}, it most definitely also has a bigger feature set than
% \expkv. Also, it can't parse |\long| input, so as soon as your values contain
% a |\par|, it'll throw errors. Furthermore, \pkg{ltxkeys} doesn't strip outer
% braces at all by design, which, imho, is a weird design choice. In addition
% \pkg{ltxkeys} loads \pkg{catoptions} which is known to introduce bugs (e.g.,
% see \url{https://tex.stackexchange.com/questions/461783}).
%
% \paragraph{\pkg{l3keys}} is almost five times slower, but has an, imho, great
% interface to define keys. It strips \emph{all} outer spaces, even if somehow
% multiple spaces ended up on either end. It offers more features, but is pretty
% much bound to \pkg{expl3} code. Whether that's a drawback is up to you.
%
% \paragraph{\pkg{pgfkeys}} is a bit more than two times slower for one key,
% but has an \emph{enormous} feature set. However, since adding additional keys
% doesn't add as much needed time for \pkg{pgfkeys} compared to \expkv, it gets
% faster than \expkv\ at around eight \kv\ pairs.
% It has the same or a very similar bug \pkg{keyval} has. The brace bug (and
% also the category fragility) can be fixed by \pkg{pgfkeyx}, but this package
% was last updated in 2012 and it slows down |\pgfkeys| by factor~$8$. Also I
% don't know whether this might introduce new bugs.
%
% \paragraph{\pkg{kvsetkeys} with \pkg{kvdefinekeys}} is about three times
% slower, but it works even if commas and equals have category codes different
% from 12 (just as some other packages in this list). Else the features of the
% keys are equal to those of \pkg{keyval}, the parser has more features, though.
%
% \paragraph{\pkg{options}} is a bit slower for only a single value, but gets a
% tad faster than \expkv\ at around 10 \kv\ pairs. It has a much bigger feature
% set. Unfortunately it also suffers from the premature unbracing bug
% \pkg{keyval} has.
%
% \paragraph{\pkg{simplekv}} is hard to compare because I don't speak French (so
% I don't understand the documentation) and from what I can see, there is no
% direct way to define the equivalent test key. Nevertheless, I tested the
% closest possible equivalent of my test key while siding for \pkg{simplekv}'s
% design not forcing something into it it doesn't seem to be designed for. It is
% more than five times slower and has hard to predict behaviour regarding brace
% and space stripping, similar to \pkg{keyval}. The tested definition was:
% \begin{lstlisting}[belowskip=0pt]
% \usepackage{simplekv}
% \setKVdefault[simplekv]{height={ abc}}                     % key setup
% \setKV[simplekv]{ height = 6 }                             % benchmarked
% \end{lstlisting}
%
% \paragraph{\pkg{yax}} is over eighteen times slower. It has a pretty strange
% syntax, imho, and again a direct equivalent is hard to define. It has the
% premature unbracing bug, too. Also somehow loading \pkg{yax} broke
% \pkg{options} for me. The tested definition was:
% \begin{lstlisting}[belowskip=0pt]
% \usepackage{yax}
% \defactiveparameter yax {\storevalue\myheight yax:height } % key setup
% \setparameterlist{yax}{ height = 6 }                       % benchmarked
% \end{lstlisting}
%
% \begin{table}
%   \def\fnsym{\textcolor{red!80!black}{*}}%
%   \sisetup{round-precision=1, round-mode=places}%
%   \begingroup
%   \centering
%   \newcommand*\yes{\textcolor{red!80!black}  {yes}}^^A
%   \newcommand*\no {\textcolor{green!80!black}{no}}^^A
%   \caption[Comparison of \kv\ packages]
%     {^^A
%       Comparison of \kv\ packages. The packages are ordered from
%       fastest to slowest for one \kv\ pair. Benchmarking was done using
%       \pkg{l3benchmark} and the scripts in the \file{Benchmarks} folder
%       of the \href{https://github.com/Skillmon/tex_expkv}{git repository}.
%       The columns $p_i$ are the polynomial coefficients of a linear fit to the
%       run-time, $p_0$ can be interpreted as the overhead for initialisation
%       and $p_1$ the cost per key. The $T_0$ column is the actual mean ops
%       needed for an empty list argument, as the linear fit doesn't match that
%       point well in general. The column ``BB'' lists whether the
%       parsing is affected by some sort of brace bug, ``CF'' stands for
%       category code fragile and lists whether the parsing breaks with active
%       commas or equal signs.^^A
%       \label{tab:comp}^^A
%     }
%   \begin{tabular}
%       {>{\collectcell\pkg}l<{\endcollectcell}*3{S[table-format=4.1]}ccc}
%     \toprule
%     \rmfamily Package & {$p_1$} & {$p_0$} & {$T_0$}& BB & CF & Date \\
%     \midrule
%     keyval    &   13.539 &    1.844 &    6.969 & \yes & \yes & 2014-10-28 \\
%     \expkv    &   26.411 &    2.391 &   12.610 & \no  & \no  & 2020-02-22 \\
%     options   &   25.532 &   10.518 &   21.393 & \yes & \yes & 2015-03-01 \\
%     pgfkeys   &   26.444 &   40.939 &   55.790 & \yes & \yes & 2020-01-08 \\
%     kvsetkeys & {\fnsym} & {\fnsym} &   41.620 & \no  & \no  & 2019-12-15 \\
%     l3keys    &  114.456 &   35.357 &   52.667 & \no  & \no  & 2020-02-14 \\
%     simplekv  &  160.523 &   10.809 &    8.698 & \yes & \yes & 2017-08-08 \\
%     xkeyval   &  260.872 &  180.435 &  164.167 & \yes & \yes & 2014-12-03 \\
%     yax       &  507.643 &   62.863 &  123.533 & \yes & \yes & 2010-01-22 \\
%     ltxkeys   & 3932.759 & 4737.818 & 5883.000 & \no  & \no  & 2012-11-17 \\
%     \bottomrule
%   \end{tabular}
%   \par
%   \endgroup
%   \medskip
%   \fnsym For \pkg{kvsetkeys} the linear model used for the other
%   packages is a poor fit, \pkg{kvsetkeys} seems to have approximately
%   quadratic run-time, the coefficients of the second degree polynomial fit are
%   $p_2=\num{9.911}$, $p_1=\num{40.523}$, and $p_0=\num{61.500}$. Of course the
%   other packages might not really have linear run-time, but at least from 1~to
%   20~keys the fits don't seem too bad (the maximum ratio $p_2/p_1$
%   for the other packages is \num{9.697e-3}). If one extrapolates the fits
%   for 100 \kv\ pairs one finds that most of them match pretty well, the
%   exception being \pkg{ltxkeys}, which behaves quadratic as well with
%   $p_2=\num{31.685}$, $p_1=\num{3267.368}$, and $p_0=\num{7177.582}$.
% \end{table}
%
% \subsection{Examples}
%
% \subsubsection{Standard Use-Case}
%
% Say we have a macro for which we want to create a \kv\ interface. The macro
% has a parameter, which is stored in the dimension |\ourdim| having a default
% value from its initialization. Now we want to be able to change that dimension
% with the |width| key to some specified value. For that we'd do
% \begin{lstlisting}
% \newdimen\ourdim
% \ourdim=150pt
% \protected\ekvdef{our}{width}{\ourdim=#1\relax}
% \end{lstlisting}
% as you can see, we use the set |our| here. We want the key to behave different
% if no value is specified. In that case the key should not use its initial
% value, but be smart and determine the available space from
% \cs[no-index]{hsize}, so we also define
% \begin{lstlisting}
% \protected\ekvdefNoVal{our}{width}{\ourdim=.9\hsize}
% \end{lstlisting}
% Now we set up our macro to use this \kv\ interface
% \begin{lstlisting}
% \protected\def\ourmacro#1{\begingroup\ekvset{our}{#1}\the\ourdim\endgroup}
% \end{lstlisting}
% Finally we can use our macro like in the following\\
% \begin{minipage}{6cm}
% \begin{lstlisting}
% \ourmacro{}\par
% \ourmacro{width}\par
% \ourmacro{width=5pt}\par
% \end{lstlisting}
% \end{minipage}^^A
% \begin{minipage}{\dimexpr\linewidth-6cm\relax}
%   \newdimen\ourdim\ourdim=150pt
%   \protected\ekvdef{our}{width}{\ourdim=#1\relax}^^A
%   \protected\ekvdefNoVal{our}{width}{\ourdim=.9\hsize}^^A
%   \protected\def\ourmacro#1^^A
%     {\begingroup\ekvset{our}{#1}\the\ourdim\endgroup}^^A
%   \ourmacro{}\par
%   \ourmacro{width}\par
%   \ourmacro{width=5pt}\par
% \end{minipage}
%
% \paragraph{The same key using \protect\expkvd}
% Using \expkvd\ we can set up the equivalent key using a \kv\ interface, after
% the following we could use |\ourmacro| in the same way as above. \expkvd\ will
% allocate and initialise |\ourdim| and define the |width| key |\protected| for
% us, so the result will be exactly the same -- with the exception that the
% default will use |\ourdim=.9\hsize\relax| instead.
% \begin{lstlisting}
% \input expkv-def                 % or \usepackage{expkv-def}
% \ekvdefinekeys{our}
%   {
%     dimen    width = \ourdim,
%     qdefault width = .9\hsize,
%     initial  width = 150pt
%   }
% \end{lstlisting}
%
% \subsubsection{An Expandable \kv\ Macro Using \cs[no-index]{ekvsneak}}
% \label{sec:sneakex}
%
% Let's set up an expandable macro, that uses a \kv\ interface. The problems
% we'll face for this are:
% \begin{enumerate}
%   \item ignoring duplicate keys
%   \item default values for keys which weren't used
%   \item providing the values as the correct argument to a macro (ordered)
% \end{enumerate}
%
% First we need to decide which \kv\ parsing macro we want to do this with,
% |\ekvset| or |\ekvparse|. For this example we also want to show the
% usage of |\ekvsneak|, hence we'll choose |\ekvset|. And we'll have to use
% |\ekvset| such that it builds a parsable list for our macro internals. To
% gain back control after |\ekvset| is done we have to put an internal of our
% macro at the start of that list, so we use an internal key that uses
% |\ekvsneakPre| after any user input.
%
% To ignore duplicates will be easy if the value of the key used last will be
% put first in the list, so the following will use |\ekvsneakPre| for the
% user-level keys. If we wanted some key for which the first usage should be the
% the binding one we would use |\ekvsneak| instead for that key.
%
% Providing default values can be done in different ways, we'll use a simple
% approach in which we'll just put the outcome of our keys if they were used
% with default values before the parsing list terminator.
%
% Ordering the keys can be done simply by searching for a specific token for
% each argument which acts like a flag, so our sneaked out values will include
% specific tokens acting as markers.
%
% Now that we have answers for our technical problems, we have to decide what
% our example macro should do. How about we define a macro that calculates the
% sine of a number and rounds that to a specified precision? As a small extra
% this macro should understand input in radian and degree and the used
% trigonometric function should be selectable as well. For the hard part of this
% task (expandably evaluating trigonometric functions) we'll use the \pkg{xfp}
% package.
%
% First we set up our keys according to our earlier considerations and set up
% the user facing macro |\sine|. The end marker of the parsing list will be a
% |\sine@stop| token, which we don't need to define and we put our defaults
% right before it.
%
% \begin{lstlisting}
% \RequirePackage{xfp}
% \makeatletter
% \ekvdef{expex}{f}{\ekvsneakPre{\f{#1}}}
% \ekvdef{expex}{round}{\ekvsneakPre{\rnd{#1}}}
% \ekvdefNoVal{expex}{degree}{\ekvsneakPre{\deg{d}}}
% \ekvdefNoVal{expex}{radian}{\ekvsneakPre{\deg{}}}
% \ekvdefNoVal{expex}{internal}{\ekvsneakPre{\sine@rnd}}
% \newcommand*\sine[2]
%   {\ekvset{expex}{#1,internal}\rnd{3}\deg{d}\f{sin}\sine@stop{#2}}
% \end{lstlisting}
% For the sake of simplicity we defined the macro |\sine| with two mandatory
% arguments, the first being the \kv\ list, the second the argument to the
% trigonometric function. We could've used \pkg{xparse}'s facilities here to
% define an expandable macro which takes an optional argument instead.
%
% Now we need to define some internal macros to extract the value of each key's
% last usage (remember that this will be the group after the first special
% flag-token). For that we use one delimited macro per key.
% \begin{lstlisting}
% \def\sine@rnd#1\rnd#2#3\sine@stop{\sine@deg#1#3\sine@stop{#2}}
% \def\sine@deg#1\deg#2#3\sine@stop{\sine@f#1#3\sine@stop{#2}}
% \def\sine@f#1\f#2#3\sine@stop{\sine@final{#2}}
% \end{lstlisting}
% After the macros |\sine@rnd|, |\sine@deg|, and |\sine@f| the macro
% |\sine@final| will see
% \texttt
%   {^^A
%     \cs[no-index]{sine@final}^^A
%     \marg{f}\marg{degree/radian}\marg{round}\marg{num}^^A
%   }.
% Now |\sine@final| has to expandably deal with those arguments such that the
% |\fpeval| macro of \pkg{xfp} gets the correct input. Luckily this is pretty
% straight forward in this example. In |\fpeval| the trigonometric functions
% have names such as |sin| or |cos| and the degree taking variants |sind| or
% |cosd|. And since the |degree| key puts a |d| in |#2| and the |radian| key
% leaves |#2| empty all we have to do to get the correct function name is stick
% the two together.
% \begin{lstlisting}
% \newcommand*\sine@final[4]{\fpeval{round(#1#2(#4),#3)}}
% \makeatother
% \end{lstlisting}
% Let's test our macro:\\
% \begin{minipage}{.7\linewidth}
%   \begin{lstlisting}[gobble=4]
%   \sine{}{60}\par
%   \sine{round=10}{60}\par
%   \sine{f=cos,radian}{pi}\par
%   \edef\myval{\sine{f=tan}{1}}\texttt{\meaning\myval}
%   \end{lstlisting}
% \end{minipage}\hfill
% \begin{minipage}{.2\linewidth}
%   \makeatletter
%   \ekvdef{expex}{f}{\ekvsneakPre{\f{#1}}}^^A
%   \ekvdef{expex}{round}{\ekvsneakPre{\rnd{#1}}}^^A
%   \ekvdefNoVal{expex}{degree}{\ekvsneakPre{\deg{d}}}^^A
%   \ekvdefNoVal{expex}{radian}{\ekvsneakPre{\deg{}}}^^A
%   \ekvdefNoVal{expex}{internal}{\ekvsneakPre{\sine@rnd}}^^A
%   \newcommand*\sine[2]
%     {\ekvset{expex}{#1,internal}\rnd{3}\deg{d}\f{sin}\sine@stop{#2}}^^A
%   \def\sine@rnd#1\rnd#2#3\sine@stop{\sine@deg#1#3\sine@stop{#2}}^^A
%   \def\sine@deg#1\deg#2#3\sine@stop{\sine@f#1#3\sine@stop{#2}}^^A
%   \def\sine@f#1\f#2#3\sine@stop{\sine@final{#2}}^^A
%   \newcommand*\sine@final[4]{\fpeval{round(#1#2(#4),#3)}}^^A
%   \makeatother
%   \sine{}{60}\par
%   \sine{round=10}{60}\par
%   \sine{f=cos,radian}{pi}\par
%   \edef\myval{\sine{f=tan}{1}}\texttt{\meaning\myval}
% \end{minipage}
%
%
% \subsection{Error Messages}
%
% \expkv\ should only send messages in case of errors, there are no warnings and
% no info messages. In this subsection those errors are listed.
%
% \subsubsection{Load Time}
%
% \file{expkv.tex} checks whether \eTeX\ is available. If it isn't, an error
% will be thrown using |\errmessage|:
% \begin{lstlisting}
% ! expkv Error: e-TeX required.
% \end{lstlisting}
%
% \subsubsection{Defining Keys}
%
% If you get any error from \expkv\ while you're trying to define a key, the
% definition will be aborted and gobbled.
%
% If you try to define a key with an empty set name you'll get:
% \begin{lstlisting}
% ! expkv Error: empty set name not allowed.
% \end{lstlisting}
%
% Similarly, if you try to define a key with an empty key name:
% \begin{lstlisting}
% ! expkv Error: empty key name not allowed.
% \end{lstlisting}
%
% Both of these messages are done in a way that doesn't throw additional errors
% due to |\global|, |\long|, etc., not being used correctly if you prefixed one
% of the defining macros.
%
% \subsubsection{Using Keys}
%
% This subsubsection contains the errors thrown during |\ekvset|. The errors are
% thrown in an expandable manner by providing an undefined macro. In the
% following messages \lstinline|<key>| gets replaced with the problematic key's
% name, and \lstinline|<set>| with the corresponding set. If any errors during
% \kv\ handling are encountered, the entry in the comma separated list will be
% omitted after the error is thrown and the next \kv\ pair will be parsed.
%
% If you're using an undefined key you'll get:
% \begin{lstlisting}
% ! Undefined control sequence.
% <argument> \! expkv Error:
%                            unknown key (`<key>', set `<set>').
% \end{lstlisting}
%
% If you're using a key for which only a normal version and no |NoVal| version
% is defined, but don't provide a value, you'll get:
% \begin{lstlisting}
% ! Undefined control sequence.
% <argument> \! expkv Error:
%                            value required (`<key>', set `<set>').
% \end{lstlisting}
%
% If you're using a key for which only a |NoVal| version and no normal version
% is defined, but provide a value, you'll get:
% \begin{lstlisting}
% ! Undefined control sequence.
% <argument> \! expkv Error:
%                            value forbidden (`<key>', set `<set>').
% \end{lstlisting}
%
% If you're using a set for which you never executed one of the defining macros
% from \autoref{sec:define} you'll get a low level \TeX\ error, as that isn't
% actively tested by the parser (and hence will lead to undefined behaviour and
% not be gracefully ignored). The error will look like
% \begin{lstlisting}
% ! Missing \endcsname inserted.
% <to be read again>
%                    \! expkv Error: Set `<set>' undefined.
% \end{lstlisting}
%
% \subsection{License}
%
% Copyright \textcopyright\ 2020\unless\ifnum\year=2020--\the\year\fi\
% Jonathan P. Spratte
%
% \medskip\noindent
% This work may be distributed and/or modified under the conditions of the
% \LaTeX\ Project Public License (LPPL), either version 1.3c of this license or
% (at your option) any later version. The latest version of this license is in
% the file:
%
% \url{http://www.latex-project.org/lppl.txt}
%
% \noindent
% This work is ``maintained'' (as per LPPL maintenance status) by
%
% \mbox{Jonathan P. Spratte}.
%
% \end{documentation}^^A=<<
%
% \begin{implementation}^^A>>=
%
% \clearpage
%
% \section{Implementation}^^A>>=
%^^A the LaTeX package >>=
% \subsection{The \LaTeX\ Package}
% First we set up the \LaTeX\ package. That one doesn't really do much except
% |\input|ting the generic code and identifying itself as a package.
% \gobbledocstriptag
%<*pkg>
%    \begin{macrocode}
\def\ekv@tmp
  {%
    \ProvidesFile{expkv.tex}%
      [\ekvDate\space v\ekvVersion\space an expandable key=val implementation]%
  }
\input{expkv.tex}
\ProvidesPackage{expkv}%
  [\ekvDate\space v\ekvVersion\space an expandable key=val implementation]
%    \end{macrocode}
% \gobbledocstriptag
%</pkg>
%^^A=<<
%^^A main file >>=
% \subsection{The Generic Code}
% The rest of this implementation will be the generic code.
% \gobbledocstriptag
%<*tex>
%
% Check whether \eTeX\ is available -- \expkv\ requires \eTeX.
%    \begin{macrocode}
\begingroup\expandafter\expandafter\expandafter\endgroup
\expandafter\ifx\csname numexpr\endcsname\relax
  \errmessage{expkv requires e-TeX}
  \expandafter\endinput
\fi
%    \end{macrocode}
%
% We make sure that it's only input once:
%    \begin{macrocode}
\expandafter\ifx\csname ekvVersion\endcsname\relax
\else
  \expandafter\endinput
\fi
%    \end{macrocode}
% \begin{macro}{\ekvVersion,\ekvDate}
% We're on our first input, so lets store the version and date in a macro.
%    \begin{macrocode}
\def\ekvVersion{0.5a}
\def\ekvDate{2020-02-27}
%    \end{macrocode}
% \end{macro}
%
% If the \LaTeX\ format is loaded we want to be a good file and report back who
% we are, for this the package will have defined |\ekv@tmp| to use
% |\ProvidesFile|, else this will expand to a |\relax| and do no harm.
%    \begin{macrocode}
\csname ekv@tmp\endcsname
%    \end{macrocode}
%
% Store the category code of |@| to later be able to reset it and change it to
% 11 for now.
%    \begin{macrocode}
\expandafter\chardef\csname ekv@tmp\endcsname=\catcode`\@
\catcode`\@=11
%    \end{macrocode}
% |\ekv@tmp| might later be reused to gobble any prefixes which might be
% provided to |\ekvdef| and similar in case the names are invalid, we just
% temporarily use it here as means to store the current category code of |@| to
% restore it at the end of the file, we never care for the actual definition of
% it.
%
% \begin{macro}[internal]
%   {
%     \@gobble,\@firstofone,\@firstoftwo,\@secondoftwo,
%     \ekv@gobbleto@stop,\ekv@fi@secondoftwo,\ekv@gobble@mark
%   }
% Since branching tests are often more versatile than |\if...\else...\fi|
% constructs, we define helpers that are branching pretty fast. Also here are
% some other utility functions that just grab some tokens. The ones that are
% also contained in \LaTeX\ don't use the |ekv| prefix.
%    \begin{macrocode}
\long\def\@gobble#1{}
\long\def\@firstofone#1{#1}
\long\def\@firstoftwo#1#2{#1}
\long\def\@secondoftwo#1#2{#2}
\long\def\ekv@fi@secondoftwo\fi\@firstoftwo#1#2{\fi#2}
\long\def\ekv@gobbleto@stop#1\ekv@stop{}
\def\ekv@gobble@mark\ekv@mark{}
%    \end{macrocode}
% \end{macro}
% As you can see |\ekv@gobbleto@stop| uses a special marker |\ekv@stop|. The
% package will use three such markers, the one you've seen already, |\ekv@mark|
% and |\ekv@nil|. Contrarily to how for instance \pkg{expl3} does things, we
% don't define them, as we don't need them to have an actual meaning. This has
% the advantage that if they somehow get expanded -- which should never happen
% if things work out -- they'll throw an error directly.
%
% \begin{macro}[internal]
%   {
%     \ekv@ifempty,\ekv@ifempty@,\ekv@ifempty@true,\ekv@ifempty@false,
%     \ekv@ifempty@true@F
%   }
% We can test for a lot of things building on an if-empty test, so lets define a
% really fast one. Since some tests might have reversed logic (true if something
% is not empty) we also set up macros for the reversed branches.
%    \begin{macrocode}
\long\def\ekv@ifempty#1%
  {%
    \ekv@ifempty@\ekv@ifempty@A#1\ekv@ifempty@B\ekv@ifempty@true
      \ekv@ifempty@A\ekv@ifempty@B\@secondoftwo
  }
\long\def\ekv@ifempty@#1\ekv@ifempty@A\ekv@ifempty@B{}
\long\def\ekv@ifempty@true\ekv@ifempty@A\ekv@ifempty@B\@secondoftwo#1#2{#1}
\long\def\ekv@ifempty@false\ekv@ifempty@A\ekv@ifempty@B\@firstoftwo#1#2{#2}
\long\def\ekv@ifempty@true@F\ekv@ifempty@A\ekv@ifempty@B\@firstofone#1{}
%    \end{macrocode}
% \end{macro}
%
% \begin{macro}[internal]{\ekv@ifblank@}
% The obvious test that can be based on an if-empty is if-blank, meaning a test
% checking whether the argument is empty or consists only of spaces. Our version
% here will be tweaked a bit, as we want to check this, but with one leading
% |\ekv@mark| token that is to be ignored.
%    \begin{macrocode}
\long\def\ekv@ifblank@\ekv@mark#1{\ekv@ifempty@\ekv@ifempty@A}
%    \end{macrocode}
% \end{macro}
%
% \begin{macro}[internal]{\ekv@ifdefined}
% We'll need to check whether something is defined quite frequently, so why not
% define a macro that does this. The following test is expandable, slower than
% the typical expandable test for undefined control sequences, but faster for
% defined ones. Since we want to be as fast as possible for correct input, this
% is to be preferred.
%    \begin{macrocode}
\def\ekv@ifdefined#1%
  {%
    \expandafter
    \ifx\csname\ifcsname #1\endcsname #1\else relax\fi\endcsname\relax
      \ekv@fi@secondoftwo
    \fi
    \@firstoftwo
  }
%    \end{macrocode}
% \end{macro}
  % \begin{macro}[internal]
%   {
%     \ekv@ifdefined@pair,\ekv@ifdefined@pair@,\ekv@ifdefined@key,
%     \ekv@ifdefined@key@
%   }
% Since we can save some time if we only have to create the control sequence
% once when we know beforehand how we want to use it, we build some other macros
% for those cases (which we'll have quite often, once per key usage).
%    \begin{macrocode}
\def\ekv@ifdefined@pair#1#2%
  {%
    \expandafter\ekv@ifdefined@pair@
      \csname
        \ifcsname #1{#2}\endcsname
          #1{#2}%
        \else
          relax%
        \fi
      \endcsname
  }
\def\ekv@ifdefined@pair@#1%
  {%
    \ifx#1\relax
      \ekv@fi@secondoftwo
    \fi
    \@firstoftwo
    {\ekv@set@pair@#1\ekv@mark}%
  }
\def\ekv@ifdefined@key#1#2%
  {%
    \expandafter\ekv@ifdefined@key@
      \csname
        \ifcsname #1{#2}N\endcsname
          #1{#2}N%
        \else
          relax%
        \fi
      \endcsname
  }
\def\ekv@ifdefined@key@#1%
  {%
    \ifx#1\relax
      \ekv@fi@secondoftwo
    \fi
    \@firstoftwo#1%
  }
%    \end{macrocode}
% \end{macro}
%
% \begin{macro}{\ekv@name,\ekv@name@set,\ekv@name@key}
% The keys will all follow the same naming scheme, so we define it here.
%    \begin{macrocode}
\def\ekv@name#1#2{\ekv@name@set{#1}\ekv@name@key{#2}}
\def\ekv@name@set#1{ekv#1(}
\def\ekv@name@key#1{#1)}
%    \end{macrocode}
% \end{macro}
% \begin{macro}[internal]{\ekv@undefined@set}
% We can misuse the macro name we use to expandably store the set-name in a
% single token -- since this increases performance drastically, especially for
% long set-names -- to throw a more meaningful error message in case a set isn't
% defined. The name of |\ekv@undefined@set| is a little bit misleading, as it is
% called in either case inside of |\csname|, but the result will be a control
% sequence with meaning |\relax| if the set is undefined, hence will break the
% |\csname| building the key-macro which will throw the error message.
%    \begin{macrocode}
\def\ekv@undefined@set#1{! expkv Error: Set `#1' undefined.}
%    \end{macrocode}
% \end{macro}
% \begin{macro}[internal]{\ekv@checkvalid}
% We place some restrictions on the allowed names, though, namely sets and
% keys are not allowed to be empty -- blanks are fine (meaning \mbox{set-
% or} key-names consisting of spaces).
%    \begin{macrocode}
\protected\def\ekv@checkvalid#1#2%
  {%
    \ekv@ifempty{#1}%
      {%
        \def\ekv@tmp{}%
        \errmessage{expkv Error: empty set name not allowed}%
      }%
      {%
        \ekv@ifempty{#2}%
          {%
            \def\ekv@tmp{}%
            \errmessage{expkv Error: empty key name not allowed}%
          }%
          \@secondoftwo
      }%
    \@gobble
  }
%    \end{macrocode}
% \end{macro}
% \begin{macro}{\ekvifdefined,\ekvifdefinedNoVal}
% And provide user-level macros to test whether a key is defined.
%    \begin{macrocode}
\def\ekvifdefined#1#2{\ekv@ifdefined{\ekv@name{#1}{#2}}}
\def\ekvifdefinedNoVal#1#2{\ekv@ifdefined{\ekv@name{#1}{#2}N}}
%    \end{macrocode}
% \end{macro}
%
% \begin{macro}
%   {\ekvdef,\ekvdefNoVal,\ekvlet,\ekvletNoVal,\ekvletkv,\ekvletkvNoVal}
% Set up the key defining macros |\ekvdef| etc.
%    \begin{macrocode}
\protected\long\def\ekvdef#1#2#3%
  {%
    \ekv@checkvalid{#1}{#2}%
      {%
        \expandafter\def\csname\ekv@name{#1}{#2}\endcsname##1{#3}%
        \ekv@defset{#1}%
      }%
  }
\protected\long\def\ekvdefNoVal#1#2#3%
  {%
    \ekv@checkvalid{#1}{#2}%
      {%
        \expandafter\def\csname\ekv@name{#1}{#2}N\endcsname{#3}%
        \ekv@defset{#1}%
      }%
  }
\protected\def\ekvlet#1#2#3%
  {%
    \ekv@checkvalid{#1}{#2}%
      {%
        \expandafter\let\csname\ekv@name{#1}{#2}\endcsname#3%
        \ekv@defset{#1}%
      }%
  }
\protected\def\ekvletNoVal#1#2#3%
  {%
    \ekv@checkvalid{#1}{#2}%
      {%
        \expandafter\let\csname\ekv@name{#1}{#2}N\endcsname#3%
        \ekv@defset{#1}%
      }%
  }
\protected\def\ekvletkv#1#2#3#4%
  {%
    \ekv@checkvalid{#1}{#2}%
      {%
        \expandafter\let\csname\ekv@name{#1}{#2}\expandafter\endcsname
        \csname\ekv@name{#3}{#4}\endcsname
        \ekv@defset{#1}%
      }%
  }
\protected\def\ekvletkvNoVal#1#2#3#4%
  {%
    \ekv@checkvalid{#1}{#2}%
      {%
        \expandafter\let\csname\ekv@name{#1}{#2}N\expandafter\endcsname
        \csname\ekv@name{#3}{#4}N\endcsname
        \ekv@defset{#1}%
      }%
  }
%    \end{macrocode}
% \end{macro}
% \begin{macro}[internal]{\ekv@defset}
%   In order to enhance the speed the set name given to |\ekvset| will be turned
%   into a control sequence pretty early, so we have to define that control
%   sequence.
%    \begin{macrocode}
\protected\def\ekv@defset#1%
  {%
    \expandafter\edef\csname\ekv@undefined@set{#1}\endcsname##1%
      {\ekv@name@set{#1}\ekv@name@key{##1}}%
  }
%    \end{macrocode}
% \end{macro}
%
% \begin{macro}{\ekvset}
% Set up |\ekvset|, which should not be affected by active commas and equal
% signs. The equal signs are a bit harder to cope with and we'll do that later,
% but replacing the active commas with commas of category other can be done
% beforehand. That's why we define |\ekvset| here with a temporary meaning just
% to set up the things with two different category codes. |#1| will be a
% \texttt{,\textsubscript{13}} and |#2| will be a \texttt{=\textsubscript{13}}.
%    \begin{macrocode}
\def\ekvset#1#2{%
\endgroup
\long\def\ekvset##1##2%
  {%
    \expandafter\ekv@set\csname\ekv@undefined@set{##1}\endcsname
      \ekv@mark##2#1\ekv@stop#1{}%
  }
%    \end{macrocode}
% \end{macro}
% \begin{macro}[internal]{\ekv@set}
% |\ekv@set| will split the \kv\ list at active commas. Then it has to check
% whether there were unprotected other commas and resplit there.
%    \begin{macrocode}
\long\def\ekv@set##1##2#1%
  {%
%    \end{macrocode}
% Test whether we're at the end, if so invoke |\ekv@endset|,
%    \begin{macrocode}
    \ekv@gobbleto@markstop##2\ekv@endset\ekv@mark\ekv@stop
%    \end{macrocode}
% else go on with other commas,
%    \begin{macrocode}
    \ekv@set@other##1##2,\ekv@stop,%
%    \end{macrocode}
% and get the next active comma delimited \kv\ pair.
%    \begin{macrocode}
    \ekv@set##1\ekv@mark
  }
%    \end{macrocode}
% \end{macro}
% \begin{macro}[internal]{\ekv@endset}
% |\ekv@endset| is a hungry little macro. It will eat everything that remains
% of |\ekv@set| and unbrace the sneaked stuff.
%    \begin{macrocode}
\long\def\ekv@endset
    \ekv@mark\ekv@stop\ekv@set@other##1,\ekv@stop,\ekv@set##2\ekv@mark
    ##3%
  {##3}
%    \end{macrocode}
% \end{macro}
% \begin{macro}[internal]{\ekv@set@other}
% The macro |\ekv@set@other| is guaranteed to get only single \kv\ pairs.
% First we test whether we're done, if not split at equal signs. It is faster to
% first split at category 12 equal signs and only after that on actives.
% If there is no equal sign, we need to test whether we got a blank argument and
% if not this is a |NoVal| key.
%    \begin{macrocode}
\long\def\ekv@set@other##1##2,%
  {%
    \ekv@gobbleto@markstop##2\ekv@endset@other\ekv@mark\ekv@stop
    \ekv@ifhas@eq@other##2=\ekv@ifempty@B\ekv@ifempty@false
      \ekv@ifempty@A\ekv@ifempty@B\@firstoftwo
      {\ekv@set@eq@other##1##2\ekv@stop}%
      {%
        \ekv@ifhas@eq@active##2#2\ekv@ifempty@B\ekv@ifempty@false
          \ekv@ifempty@A\ekv@ifempty@B\@firstoftwo
          {\ekv@set@eq@active##1##2\ekv@stop}%
          {%
            \ekv@ifblank@##2\ekv@nil\ekv@ifempty@B\ekv@ifempty@true@F
              \ekv@ifempty@A\ekv@ifempty@B\@firstofone
              {\ekv@strip{##2}\ekv@set@key##1}%
          }%
      }%
    \ekv@set@other##1\ekv@mark%
  }
%    \end{macrocode}
% \end{macro}
% \begin{macro}[internal]{\ekv@set@eq@other}
% |\ekv@set@eq@other| might not be the correct break point, there might be an
% active equal sign in the currently parsed key-name. If so, we have to resplit.
% If the split is correct strip the key-name of outer spaces and braces and feed
% it to |\ekv@set@pair|.
%    \begin{macrocode}
\long\def\ekv@set@eq@other##1##2=%
  {%
    \ekv@ifhas@eq@active##2#2\ekv@ifempty@B\ekv@ifempty@false
      \ekv@ifempty@A\ekv@ifempty@B\@firstoftwo
      {\ekv@set@eq@active##1##2=}%
      {\ekv@strip{##2}\ekv@set@pair##1}%
  }
%    \end{macrocode}
% \end{macro}
% \begin{macro}[internal]{\ekv@set@eq@active}
% We need to handle the active equal signs.
%    \begin{macrocode}
\long\def\ekv@set@eq@active##1##2#2%
  {%
    \ekv@strip{##2}\ekv@set@pair##1%
  }
%    \end{macrocode}
% \end{macro}
% \begin{macro}[internal]
%   {\ekv@ifhas@eq@other,\ekv@ifhas@eq@active,\ekv@endset@other}
% And we have to set up the testing macros for our equal signs and
% |\ekv@endset@other|.
%    \begin{macrocode}
\long\def\ekv@ifhas@eq@other\ekv@mark##1={\ekv@ifempty@\ekv@ifempty@A}
\long\def\ekv@ifhas@eq@active\ekv@mark##1#2{\ekv@ifempty@\ekv@ifempty@A}
\long\def\ekv@endset@other
    \ekv@mark\ekv@stop
    \ekv@ifhas@eq@other##1=\ekv@ifempty@B\ekv@ifempty@false
    \ekv@ifempty@A\ekv@ifempty@B\@firstoftwo
    ##2%
    \ekv@set@other##3\ekv@mark
  {}
%    \end{macrocode}
% \end{macro}
% \begin{macro}{\ekvbreak,\ekvbreakPreSneak,\ekvbreakPostSneak}
% Provide macros that can completely stop the parsing of |\ekvset|, who knows
% what it'll be useful for.
%    \begin{macrocode}
\long\def\ekvbreak##1##2\ekv@stop#1##3{##1}
\long\def\ekvbreakPreSneak ##1##2\ekv@stop#1##3{##1##3}
\long\def\ekvbreakPostSneak##1##2\ekv@stop#1##3{##3##1}
%    \end{macrocode}
% \end{macro}
% \begin{macro}{\ekvsneak,\ekvsneakPre}
% One last thing we want to do for |\ekvset| is to provide macros that just
% smuggle stuff after |\ekvset|'s effects.
%    \begin{macrocode}
\long\def\ekvsneak##1##2\ekv@stop#1##3%
  {%
    ##2\ekv@stop#1{##3##1}%
  }
\long\def\ekvsneakPre##1##2\ekv@stop#1##3%
  {%
    ##2\ekv@stop#1{##1##3}%
  }
%    \end{macrocode}
% \end{macro}
%
% \begin{macro}{\ekvparse}
% Additionally to the |\ekvset| macro we also want to provide an |\ekvparse|
% macro, that has the same scope as |\keyval_parse:NNn| from \pkg{expl3}.
% This is pretty analogue to the |\ekvset| implementation, we just put an
% |\unexpanded| here and there instead of other macros to stop the |\expanded|
% on our output.
%    \begin{macrocode}
\long\def\ekvparse##1##2##3%
  {%
    \ekv@parse##1##2\ekv@mark##3#1\ekv@stop#1%
  }
%    \end{macrocode}
% \end{macro}
% \begin{macro}[internal]{\ekv@parse}
%    \begin{macrocode}
\long\def\ekv@parse##1##2##3#1%
  {%
    \ekv@gobbleto@markstop##3\ekv@endparse\ekv@mark\ekv@stop
    \ekv@parse@other##1##2##3,\ekv@stop,%
    \ekv@parse##1##2\ekv@mark
  }
%    \end{macrocode}
% \end{macro}
% \begin{macro}[internal]{\ekv@endparse}
%    \begin{macrocode}
\long\def\ekv@endparse
    \ekv@mark\ekv@stop\ekv@parse@other##1,\ekv@stop,\ekv@parse##2\ekv@mark
  {}
%    \end{macrocode}
% \end{macro}
% \begin{macro}[internal]{\ekv@parse@other}
%    \begin{macrocode}
\long\def\ekv@parse@other##1##2##3,%
  {%
    \ekv@gobbleto@markstop##3\ekv@endparse@other\ekv@mark\ekv@stop
    \ekv@ifhas@eq@other##3=\ekv@ifempty@B\ekv@ifempty@false
      \ekv@ifempty@A\ekv@ifempty@B\@firstoftwo
      {\ekv@parse@eq@other##3\ekv@stop##2}%
      {%
        \ekv@ifhas@eq@active##3#2\ekv@ifempty@B\ekv@ifempty@false
          \ekv@ifempty@A\ekv@ifempty@B\@firstoftwo
          {\ekv@parse@eq@active##3\ekv@stop##2}%
          {%
            \ekv@ifblank@##3\ekv@nil\ekv@ifempty@B\ekv@ifempty@true@F
              \ekv@ifempty@A\ekv@ifempty@B\@firstofone
              {\ekv@strip{##3}\ekv@parse@key##1}%
          }%
      }%
    \ekv@parse@other##1##2\ekv@mark
  }
%    \end{macrocode}
% \end{macro}
% \begin{macro}[internal]{\ekv@parse@eq@other}
%    \begin{macrocode}
\long\def\ekv@parse@eq@other##1=%
  {%
    \ekv@ifhas@eq@active##1#2\ekv@ifempty@B\ekv@ifempty@false
      \ekv@ifempty@A\ekv@ifempty@B\@firstoftwo
      {\ekv@parse@eq@active##1=}%
      {\ekv@strip{##1}\ekv@parse@pair\ekv@mark}%
  }
%    \end{macrocode}
% \end{macro}
% \begin{macro}[internal]{\ekv@parse@eq@active}
%    \begin{macrocode}
\long\def\ekv@parse@eq@active##1#2%
  {%
    \ekv@strip{##1}\ekv@parse@pair\ekv@mark
  }
%    \end{macrocode}
% \end{macro}
% \begin{macro}[internal]{\ekv@endparse@other}
%    \begin{macrocode}
\long\def\ekv@endparse@other
    \ekv@mark\ekv@stop
    \ekv@ifhas@eq@other##1=\ekv@ifempty@B\ekv@ifempty@false
    \ekv@ifempty@A\ekv@ifempty@B\@firstoftwo
    ##2%
    \ekv@parse@other##3\ekv@mark
  {}
%    \end{macrocode}
% \end{macro}
% \begin{macro}[internal]{\ekv@parse@pair,\ekv@parse@pair@}
%    \begin{macrocode}
\long\def\ekv@parse@pair##1##2\ekv@stop
  {%
    \ekv@strip{##2}\ekv@parse@pair@{##1}%
  }
\long\def\ekv@parse@pair@##1##2##3%
  {%
    \unexpanded{##3{##2}{##1}}%
  }
%    \end{macrocode}
% \end{macro}
% \begin{macro}[internal]{\ekv@parse@key}
%    \begin{macrocode}
\long\def\ekv@parse@key##1##2%
  {%
    \unexpanded{##2{##1}}%
  }
%    \end{macrocode}
% \end{macro}
%
% Finally really setting things up with |\ekvset|'s temporary meaning:
%    \begin{macrocode}
}
\begingroup
\catcode`\,=13
\catcode`\==13
\ekvset,=
%    \end{macrocode}
%
% \begin{macro}{\ekvchangeset}
% Provide a macro that is able to switch out the current \set\ in |\ekvset|.
% This operation is slow (by comparison, it should be slightly faster than
% |\ekvsneak|), but allows for something similar to \pkg{pgfkeys}'s
% \texttt{\meta{key}/.cd} mechanism. However this operation is more expensive
% than |/.cd| as we can't just redefine some token to reflect this, but have to
% switch out the set expandably, so this works similar to the |\ekvsneak| macros
% reading and reinserting the remainder of the \kv\ list.
%    \begin{macrocode}
\def\ekvchangeset#1%
  {%
    \expandafter\ekv@changeset\csname\ekv@undefined@set{#1}\endcsname\ekv@mark
  }
%    \end{macrocode}
% \end{macro}
% \begin{macro}[internal]{\ekv@changeset}
% This macro does the real change-out of |\ekvchangeset|. We introduced an
% |\ekv@mark| to not accidentally remove some braces which we have to remove
% again.
%    \begin{macrocode}
\long\def\ekv@changeset#1#2\ekv@set@other#3#4\ekv@set#5%
  {%
    \ekv@gobble@mark#2\ekv@set@other#1#4\ekv@set#1%
  }
%    \end{macrocode}
% \end{macro}
%
%
% \begin{macro}[internal]{\ekv@gobbleto@markstop}
% The |\ekv@gobbleto@markstop| can be used to test for |\ekv@stop| similar to
% our if-empty test, but instead of using tokens which are used nowhere else
% (|\ekv@ifempty@A| and |\ekv@ifempty@B|) it uses |\ekv@mark| and |\ekv@stop|.
%    \begin{macrocode}
\long\def\ekv@gobbleto@markstop#1\ekv@mark\ekv@stop{}
%    \end{macrocode}
% \end{macro}
%
% \begin{macro}[internal]{\ekv@set@pair,\ekv@set@pair@}
% |\ekv@set@pair| gets invoked with the space and brace stripped key-name as its
% first argument, the set-macro as the second argument, and following that is
% the key-value right delimited by an |\ekv@stop|.
%    \begin{macrocode}
\long\def\ekv@set@pair#1#2%
  {%
    \ekv@ifdefined@pair#2{#1}%
      {%
%    \end{macrocode}
% This branch will be executed if the key is not defined as an argument grabbing
% one. If so test whether there is a |NoVal| key of the same name or whether the
% key is unknown. Throw a meaningful error message and gobble the value.
%    \begin{macrocode}
        \ekv@ifdefined{#2{#1}N}%
          \ekv@err@noarg
          \ekv@err@unknown
          #2{#1}%
        \ekv@gobbleto@stop
      }%
  }
%    \end{macrocode}
% |\ekv@ifdefined@pair| will call |\ekv@set@pair@| if the key is correctly
% defined. This will then grab the value, strip outer spaces and braces from it
% and feed it to the key-macro. Afterwards |\ekv@set@other| will take control
% again.
%    \begin{macrocode}
\long\def\ekv@set@pair@#1#2\ekv@stop
  {%
    \ekv@strip{#2}#1%
  }
%    \end{macrocode}
% \end{macro}
%
% \begin{macro}[internal]{\ekv@set@key}
% Analogous to |\ekv@set@pair|, |\ekv@set@key| lets |\ekv@ifdefined@key| test
% whether a |NoVal| key is defined, else it'll throw a meaningful error message.
% Since we don't have to grab any value |\ekv@ifdefined@key| will invoke the
% key-macro and we're done here, |\ekv@set@other| will take over again.
%    \begin{macrocode}
\long\def\ekv@set@key#1#2%
  {%
    \ekv@ifdefined@key#2{#1}%
      {%
        \ekv@ifdefined{#2{#1}}%
          \ekv@err@reqval
          \ekv@err@unknown
          #2{#1}%
      }%
  }
%    \end{macrocode}
% \end{macro}
%
% \begin{macro}[internal]{\ekv@err,\ekv@err@}
% Since |\ekvset| is fully expandable as long as the code of the keys is (which
% is unlikely) we want to somehow throw expandable errors, in our case via
% undefined control sequences.
%    \begin{macrocode}
\begingroup
\edef\ekv@err
  {%
    \endgroup
    \unexpanded{\long\def\ekv@err}##1%
      {%
        \unexpanded{\expandafter\ekv@err@\@firstofone}%
        {\expandafter\noexpand\csname ! expkv Error:\endcsname ##1.}%
        \unexpanded{\ekv@stop}%
      }%
  }
\ekv@err
\def\ekv@err@{\expandafter\ekv@gobbleto@stop}
%    \end{macrocode}
% \end{macro}
% \begin{macro}[internal]
%   {
%     \ekv@err@common,\ekv@err@common@,
%     \ekv@err@unknown,\ekv@err@noarg,\ekv@err@reqval
%   }
% Now we can use |\ekv@err| to set up some error messages so that we can later
% use those instead of the full strings.
%    \begin{macrocode}
\long\def\ekv@err@common #1#2{\expandafter\ekv@err@common@\string#2{#1}}
\long\def\ekv@err@common@#1`#2' #3.#4#5{\ekv@err{#4 (`#5', set `#2')}}
\long\def\ekv@err@unknown#1#2{\ekv@err@common{unknown key}#1{#2}}
\long\def\ekv@err@noarg  #1#2{\ekv@err@common{value forbidden}#1{#2}}
\long\def\ekv@err@reqval #1#2{\ekv@err@common{value required}#1{#2}}
%    \end{macrocode}
% \end{macro}
%
% \begin{macro}[internal]{\ekv@strip,\ekv@strip@a,\ekv@strip@b,\ekv@strip@c}
% Finally we borrow some ideas of \pkg{expl3}'s \pkg{l3tl} to strip spaces
% from keys and values. This |\ekv@strip| also strips one level of outer braces
% \emph{after} stripping spaces, so an input of | {abc} | becomes |abc| after
% stripping. It should be used with |#1| prefixed by |\ekv@mark|. Also this
% implementation at most strips \emph{one} space from both sides.
%    \begin{macrocode}
\def\ekv@strip#1%
  {%
    \long\def\ekv@strip##1%
      {%
        \ekv@strip@a
          ##1%
          \ekv@nil
          \ekv@mark#1%
          #1\ekv@nil{}%
        \ekv@stop
      }%
    \long\def\ekv@strip@a##1\ekv@mark#1##2\ekv@nil##3%
      {%
        \ekv@strip@b##3##1##2\ekv@nil
      }%
    \long\def\ekv@strip@b##1#1\ekv@nil
      {%
        \ekv@strip@c##1\ekv@nil
      }%
    \long\def\ekv@strip@c\ekv@mark##1\ekv@nil##2\ekv@stop##3%
      {%
        ##3{##1}%
      }%
  }
\ekv@strip{ }
%    \end{macrocode}
% \end{macro}
%
% Now everything that's left is to reset the category code of |@|.
%    \begin{macrocode}
\catcode`\@=\ekv@tmp
%    \end{macrocode}
%
% \gobbledocstriptag
%</tex>
%^^A=<<
%
%^^A=<<
%
% \end{implementation}^^A=<<
%
% \clearpage
% \PrintIndex
%
\endinput
%
^^A vim: ft=tex fdm=marker fmr=>>=,=<<

\ProvidesPackage{expkv}%
  [\ekvDate\space v\ekvVersion\space an expandable key=val implementation]
%    \end{macrocode}
% \gobbledocstriptag
%</pkg>
%^^A=<<
%^^A main file >>=
% \subsection{The Generic Code}
% The rest of this implementation will be the generic code.
% \gobbledocstriptag
%<*tex>
%
% Check whether \eTeX\ is available -- \expkv\ requires \eTeX.
%    \begin{macrocode}
\begingroup\expandafter\expandafter\expandafter\endgroup
\expandafter\ifx\csname numexpr\endcsname\relax
  \errmessage{expkv requires e-TeX}
  \expandafter\endinput
\fi
%    \end{macrocode}
%
% We make sure that it's only input once:
%    \begin{macrocode}
\expandafter\ifx\csname ekvVersion\endcsname\relax
\else
  \expandafter\endinput
\fi
%    \end{macrocode}
% \begin{macro}{\ekvVersion,\ekvDate}
% We're on our first input, so lets store the version and date in a macro.
%    \begin{macrocode}
\def\ekvVersion{0.5a}
\def\ekvDate{2020-02-27}
%    \end{macrocode}
% \end{macro}
%
% If the \LaTeX\ format is loaded we want to be a good file and report back who
% we are, for this the package will have defined |\ekv@tmp| to use
% |\ProvidesFile|, else this will expand to a |\relax| and do no harm.
%    \begin{macrocode}
\csname ekv@tmp\endcsname
%    \end{macrocode}
%
% Store the category code of |@| to later be able to reset it and change it to
% 11 for now.
%    \begin{macrocode}
\expandafter\chardef\csname ekv@tmp\endcsname=\catcode`\@
\catcode`\@=11
%    \end{macrocode}
% |\ekv@tmp| might later be reused to gobble any prefixes which might be
% provided to |\ekvdef| and similar in case the names are invalid, we just
% temporarily use it here as means to store the current category code of |@| to
% restore it at the end of the file, we never care for the actual definition of
% it.
%
% \begin{macro}[internal]
%   {
%     \@gobble,\@firstofone,\@firstoftwo,\@secondoftwo,
%     \ekv@gobbleto@stop,\ekv@fi@secondoftwo,\ekv@gobble@mark
%   }
% Since branching tests are often more versatile than |\if...\else...\fi|
% constructs, we define helpers that are branching pretty fast. Also here are
% some other utility functions that just grab some tokens. The ones that are
% also contained in \LaTeX\ don't use the |ekv| prefix.
%    \begin{macrocode}
\long\def\@gobble#1{}
\long\def\@firstofone#1{#1}
\long\def\@firstoftwo#1#2{#1}
\long\def\@secondoftwo#1#2{#2}
\long\def\ekv@fi@secondoftwo\fi\@firstoftwo#1#2{\fi#2}
\long\def\ekv@gobbleto@stop#1\ekv@stop{}
\def\ekv@gobble@mark\ekv@mark{}
%    \end{macrocode}
% \end{macro}
% As you can see |\ekv@gobbleto@stop| uses a special marker |\ekv@stop|. The
% package will use three such markers, the one you've seen already, |\ekv@mark|
% and |\ekv@nil|. Contrarily to how for instance \pkg{expl3} does things, we
% don't define them, as we don't need them to have an actual meaning. This has
% the advantage that if they somehow get expanded -- which should never happen
% if things work out -- they'll throw an error directly.
%
% \begin{macro}[internal]
%   {
%     \ekv@ifempty,\ekv@ifempty@,\ekv@ifempty@true,\ekv@ifempty@false,
%     \ekv@ifempty@true@F
%   }
% We can test for a lot of things building on an if-empty test, so lets define a
% really fast one. Since some tests might have reversed logic (true if something
% is not empty) we also set up macros for the reversed branches.
%    \begin{macrocode}
\long\def\ekv@ifempty#1%
  {%
    \ekv@ifempty@\ekv@ifempty@A#1\ekv@ifempty@B\ekv@ifempty@true
      \ekv@ifempty@A\ekv@ifempty@B\@secondoftwo
  }
\long\def\ekv@ifempty@#1\ekv@ifempty@A\ekv@ifempty@B{}
\long\def\ekv@ifempty@true\ekv@ifempty@A\ekv@ifempty@B\@secondoftwo#1#2{#1}
\long\def\ekv@ifempty@false\ekv@ifempty@A\ekv@ifempty@B\@firstoftwo#1#2{#2}
\long\def\ekv@ifempty@true@F\ekv@ifempty@A\ekv@ifempty@B\@firstofone#1{}
%    \end{macrocode}
% \end{macro}
%
% \begin{macro}[internal]{\ekv@ifblank@}
% The obvious test that can be based on an if-empty is if-blank, meaning a test
% checking whether the argument is empty or consists only of spaces. Our version
% here will be tweaked a bit, as we want to check this, but with one leading
% |\ekv@mark| token that is to be ignored.
%    \begin{macrocode}
\long\def\ekv@ifblank@\ekv@mark#1{\ekv@ifempty@\ekv@ifempty@A}
%    \end{macrocode}
% \end{macro}
%
% \begin{macro}[internal]{\ekv@ifdefined}
% We'll need to check whether something is defined quite frequently, so why not
% define a macro that does this. The following test is expandable, slower than
% the typical expandable test for undefined control sequences, but faster for
% defined ones. Since we want to be as fast as possible for correct input, this
% is to be preferred.
%    \begin{macrocode}
\def\ekv@ifdefined#1%
  {%
    \expandafter
    \ifx\csname\ifcsname #1\endcsname #1\else relax\fi\endcsname\relax
      \ekv@fi@secondoftwo
    \fi
    \@firstoftwo
  }
%    \end{macrocode}
% \end{macro}
  % \begin{macro}[internal]
%   {
%     \ekv@ifdefined@pair,\ekv@ifdefined@pair@,\ekv@ifdefined@key,
%     \ekv@ifdefined@key@
%   }
% Since we can save some time if we only have to create the control sequence
% once when we know beforehand how we want to use it, we build some other macros
% for those cases (which we'll have quite often, once per key usage).
%    \begin{macrocode}
\def\ekv@ifdefined@pair#1#2%
  {%
    \expandafter\ekv@ifdefined@pair@
      \csname
        \ifcsname #1{#2}\endcsname
          #1{#2}%
        \else
          relax%
        \fi
      \endcsname
  }
\def\ekv@ifdefined@pair@#1%
  {%
    \ifx#1\relax
      \ekv@fi@secondoftwo
    \fi
    \@firstoftwo
    {\ekv@set@pair@#1\ekv@mark}%
  }
\def\ekv@ifdefined@key#1#2%
  {%
    \expandafter\ekv@ifdefined@key@
      \csname
        \ifcsname #1{#2}N\endcsname
          #1{#2}N%
        \else
          relax%
        \fi
      \endcsname
  }
\def\ekv@ifdefined@key@#1%
  {%
    \ifx#1\relax
      \ekv@fi@secondoftwo
    \fi
    \@firstoftwo#1%
  }
%    \end{macrocode}
% \end{macro}
%
% \begin{macro}{\ekv@name,\ekv@name@set,\ekv@name@key}
% The keys will all follow the same naming scheme, so we define it here.
%    \begin{macrocode}
\def\ekv@name#1#2{\ekv@name@set{#1}\ekv@name@key{#2}}
\def\ekv@name@set#1{ekv#1(}
\def\ekv@name@key#1{#1)}
%    \end{macrocode}
% \end{macro}
% \begin{macro}[internal]{\ekv@undefined@set}
% We can misuse the macro name we use to expandably store the set-name in a
% single token -- since this increases performance drastically, especially for
% long set-names -- to throw a more meaningful error message in case a set isn't
% defined. The name of |\ekv@undefined@set| is a little bit misleading, as it is
% called in either case inside of |\csname|, but the result will be a control
% sequence with meaning |\relax| if the set is undefined, hence will break the
% |\csname| building the key-macro which will throw the error message.
%    \begin{macrocode}
\def\ekv@undefined@set#1{! expkv Error: Set `#1' undefined.}
%    \end{macrocode}
% \end{macro}
% \begin{macro}[internal]{\ekv@checkvalid}
% We place some restrictions on the allowed names, though, namely sets and
% keys are not allowed to be empty -- blanks are fine (meaning \mbox{set-
% or} key-names consisting of spaces).
%    \begin{macrocode}
\protected\def\ekv@checkvalid#1#2%
  {%
    \ekv@ifempty{#1}%
      {%
        \def\ekv@tmp{}%
        \errmessage{expkv Error: empty set name not allowed}%
      }%
      {%
        \ekv@ifempty{#2}%
          {%
            \def\ekv@tmp{}%
            \errmessage{expkv Error: empty key name not allowed}%
          }%
          \@secondoftwo
      }%
    \@gobble
  }
%    \end{macrocode}
% \end{macro}
% \begin{macro}{\ekvifdefined,\ekvifdefinedNoVal}
% And provide user-level macros to test whether a key is defined.
%    \begin{macrocode}
\def\ekvifdefined#1#2{\ekv@ifdefined{\ekv@name{#1}{#2}}}
\def\ekvifdefinedNoVal#1#2{\ekv@ifdefined{\ekv@name{#1}{#2}N}}
%    \end{macrocode}
% \end{macro}
%
% \begin{macro}
%   {\ekvdef,\ekvdefNoVal,\ekvlet,\ekvletNoVal,\ekvletkv,\ekvletkvNoVal}
% Set up the key defining macros |\ekvdef| etc.
%    \begin{macrocode}
\protected\long\def\ekvdef#1#2#3%
  {%
    \ekv@checkvalid{#1}{#2}%
      {%
        \expandafter\def\csname\ekv@name{#1}{#2}\endcsname##1{#3}%
        \ekv@defset{#1}%
      }%
  }
\protected\long\def\ekvdefNoVal#1#2#3%
  {%
    \ekv@checkvalid{#1}{#2}%
      {%
        \expandafter\def\csname\ekv@name{#1}{#2}N\endcsname{#3}%
        \ekv@defset{#1}%
      }%
  }
\protected\def\ekvlet#1#2#3%
  {%
    \ekv@checkvalid{#1}{#2}%
      {%
        \expandafter\let\csname\ekv@name{#1}{#2}\endcsname#3%
        \ekv@defset{#1}%
      }%
  }
\protected\def\ekvletNoVal#1#2#3%
  {%
    \ekv@checkvalid{#1}{#2}%
      {%
        \expandafter\let\csname\ekv@name{#1}{#2}N\endcsname#3%
        \ekv@defset{#1}%
      }%
  }
\protected\def\ekvletkv#1#2#3#4%
  {%
    \ekv@checkvalid{#1}{#2}%
      {%
        \expandafter\let\csname\ekv@name{#1}{#2}\expandafter\endcsname
        \csname\ekv@name{#3}{#4}\endcsname
        \ekv@defset{#1}%
      }%
  }
\protected\def\ekvletkvNoVal#1#2#3#4%
  {%
    \ekv@checkvalid{#1}{#2}%
      {%
        \expandafter\let\csname\ekv@name{#1}{#2}N\expandafter\endcsname
        \csname\ekv@name{#3}{#4}N\endcsname
        \ekv@defset{#1}%
      }%
  }
%    \end{macrocode}
% \end{macro}
% \begin{macro}[internal]{\ekv@defset}
%   In order to enhance the speed the set name given to |\ekvset| will be turned
%   into a control sequence pretty early, so we have to define that control
%   sequence.
%    \begin{macrocode}
\protected\def\ekv@defset#1%
  {%
    \expandafter\edef\csname\ekv@undefined@set{#1}\endcsname##1%
      {\ekv@name@set{#1}\ekv@name@key{##1}}%
  }
%    \end{macrocode}
% \end{macro}
%
% \begin{macro}{\ekvset}
% Set up |\ekvset|, which should not be affected by active commas and equal
% signs. The equal signs are a bit harder to cope with and we'll do that later,
% but replacing the active commas with commas of category other can be done
% beforehand. That's why we define |\ekvset| here with a temporary meaning just
% to set up the things with two different category codes. |#1| will be a
% \texttt{,\textsubscript{13}} and |#2| will be a \texttt{=\textsubscript{13}}.
%    \begin{macrocode}
\def\ekvset#1#2{%
\endgroup
\long\def\ekvset##1##2%
  {%
    \expandafter\ekv@set\csname\ekv@undefined@set{##1}\endcsname
      \ekv@mark##2#1\ekv@stop#1{}%
  }
%    \end{macrocode}
% \end{macro}
% \begin{macro}[internal]{\ekv@set}
% |\ekv@set| will split the \kv\ list at active commas. Then it has to check
% whether there were unprotected other commas and resplit there.
%    \begin{macrocode}
\long\def\ekv@set##1##2#1%
  {%
%    \end{macrocode}
% Test whether we're at the end, if so invoke |\ekv@endset|,
%    \begin{macrocode}
    \ekv@gobbleto@markstop##2\ekv@endset\ekv@mark\ekv@stop
%    \end{macrocode}
% else go on with other commas,
%    \begin{macrocode}
    \ekv@set@other##1##2,\ekv@stop,%
%    \end{macrocode}
% and get the next active comma delimited \kv\ pair.
%    \begin{macrocode}
    \ekv@set##1\ekv@mark
  }
%    \end{macrocode}
% \end{macro}
% \begin{macro}[internal]{\ekv@endset}
% |\ekv@endset| is a hungry little macro. It will eat everything that remains
% of |\ekv@set| and unbrace the sneaked stuff.
%    \begin{macrocode}
\long\def\ekv@endset
    \ekv@mark\ekv@stop\ekv@set@other##1,\ekv@stop,\ekv@set##2\ekv@mark
    ##3%
  {##3}
%    \end{macrocode}
% \end{macro}
% \begin{macro}[internal]{\ekv@set@other}
% The macro |\ekv@set@other| is guaranteed to get only single \kv\ pairs.
% First we test whether we're done, if not split at equal signs. It is faster to
% first split at category 12 equal signs and only after that on actives.
% If there is no equal sign, we need to test whether we got a blank argument and
% if not this is a |NoVal| key.
%    \begin{macrocode}
\long\def\ekv@set@other##1##2,%
  {%
    \ekv@gobbleto@markstop##2\ekv@endset@other\ekv@mark\ekv@stop
    \ekv@ifhas@eq@other##2=\ekv@ifempty@B\ekv@ifempty@false
      \ekv@ifempty@A\ekv@ifempty@B\@firstoftwo
      {\ekv@set@eq@other##1##2\ekv@stop}%
      {%
        \ekv@ifhas@eq@active##2#2\ekv@ifempty@B\ekv@ifempty@false
          \ekv@ifempty@A\ekv@ifempty@B\@firstoftwo
          {\ekv@set@eq@active##1##2\ekv@stop}%
          {%
            \ekv@ifblank@##2\ekv@nil\ekv@ifempty@B\ekv@ifempty@true@F
              \ekv@ifempty@A\ekv@ifempty@B\@firstofone
              {\ekv@strip{##2}\ekv@set@key##1}%
          }%
      }%
    \ekv@set@other##1\ekv@mark%
  }
%    \end{macrocode}
% \end{macro}
% \begin{macro}[internal]{\ekv@set@eq@other}
% |\ekv@set@eq@other| might not be the correct break point, there might be an
% active equal sign in the currently parsed key-name. If so, we have to resplit.
% If the split is correct strip the key-name of outer spaces and braces and feed
% it to |\ekv@set@pair|.
%    \begin{macrocode}
\long\def\ekv@set@eq@other##1##2=%
  {%
    \ekv@ifhas@eq@active##2#2\ekv@ifempty@B\ekv@ifempty@false
      \ekv@ifempty@A\ekv@ifempty@B\@firstoftwo
      {\ekv@set@eq@active##1##2=}%
      {\ekv@strip{##2}\ekv@set@pair##1}%
  }
%    \end{macrocode}
% \end{macro}
% \begin{macro}[internal]{\ekv@set@eq@active}
% We need to handle the active equal signs.
%    \begin{macrocode}
\long\def\ekv@set@eq@active##1##2#2%
  {%
    \ekv@strip{##2}\ekv@set@pair##1%
  }
%    \end{macrocode}
% \end{macro}
% \begin{macro}[internal]
%   {\ekv@ifhas@eq@other,\ekv@ifhas@eq@active,\ekv@endset@other}
% And we have to set up the testing macros for our equal signs and
% |\ekv@endset@other|.
%    \begin{macrocode}
\long\def\ekv@ifhas@eq@other\ekv@mark##1={\ekv@ifempty@\ekv@ifempty@A}
\long\def\ekv@ifhas@eq@active\ekv@mark##1#2{\ekv@ifempty@\ekv@ifempty@A}
\long\def\ekv@endset@other
    \ekv@mark\ekv@stop
    \ekv@ifhas@eq@other##1=\ekv@ifempty@B\ekv@ifempty@false
    \ekv@ifempty@A\ekv@ifempty@B\@firstoftwo
    ##2%
    \ekv@set@other##3\ekv@mark
  {}
%    \end{macrocode}
% \end{macro}
% \begin{macro}{\ekvbreak,\ekvbreakPreSneak,\ekvbreakPostSneak}
% Provide macros that can completely stop the parsing of |\ekvset|, who knows
% what it'll be useful for.
%    \begin{macrocode}
\long\def\ekvbreak##1##2\ekv@stop#1##3{##1}
\long\def\ekvbreakPreSneak ##1##2\ekv@stop#1##3{##1##3}
\long\def\ekvbreakPostSneak##1##2\ekv@stop#1##3{##3##1}
%    \end{macrocode}
% \end{macro}
% \begin{macro}{\ekvsneak,\ekvsneakPre}
% One last thing we want to do for |\ekvset| is to provide macros that just
% smuggle stuff after |\ekvset|'s effects.
%    \begin{macrocode}
\long\def\ekvsneak##1##2\ekv@stop#1##3%
  {%
    ##2\ekv@stop#1{##3##1}%
  }
\long\def\ekvsneakPre##1##2\ekv@stop#1##3%
  {%
    ##2\ekv@stop#1{##1##3}%
  }
%    \end{macrocode}
% \end{macro}
%
% \begin{macro}{\ekvparse}
% Additionally to the |\ekvset| macro we also want to provide an |\ekvparse|
% macro, that has the same scope as |\keyval_parse:NNn| from \pkg{expl3}.
% This is pretty analogue to the |\ekvset| implementation, we just put an
% |\unexpanded| here and there instead of other macros to stop the |\expanded|
% on our output.
%    \begin{macrocode}
\long\def\ekvparse##1##2##3%
  {%
    \ekv@parse##1##2\ekv@mark##3#1\ekv@stop#1%
  }
%    \end{macrocode}
% \end{macro}
% \begin{macro}[internal]{\ekv@parse}
%    \begin{macrocode}
\long\def\ekv@parse##1##2##3#1%
  {%
    \ekv@gobbleto@markstop##3\ekv@endparse\ekv@mark\ekv@stop
    \ekv@parse@other##1##2##3,\ekv@stop,%
    \ekv@parse##1##2\ekv@mark
  }
%    \end{macrocode}
% \end{macro}
% \begin{macro}[internal]{\ekv@endparse}
%    \begin{macrocode}
\long\def\ekv@endparse
    \ekv@mark\ekv@stop\ekv@parse@other##1,\ekv@stop,\ekv@parse##2\ekv@mark
  {}
%    \end{macrocode}
% \end{macro}
% \begin{macro}[internal]{\ekv@parse@other}
%    \begin{macrocode}
\long\def\ekv@parse@other##1##2##3,%
  {%
    \ekv@gobbleto@markstop##3\ekv@endparse@other\ekv@mark\ekv@stop
    \ekv@ifhas@eq@other##3=\ekv@ifempty@B\ekv@ifempty@false
      \ekv@ifempty@A\ekv@ifempty@B\@firstoftwo
      {\ekv@parse@eq@other##3\ekv@stop##2}%
      {%
        \ekv@ifhas@eq@active##3#2\ekv@ifempty@B\ekv@ifempty@false
          \ekv@ifempty@A\ekv@ifempty@B\@firstoftwo
          {\ekv@parse@eq@active##3\ekv@stop##2}%
          {%
            \ekv@ifblank@##3\ekv@nil\ekv@ifempty@B\ekv@ifempty@true@F
              \ekv@ifempty@A\ekv@ifempty@B\@firstofone
              {\ekv@strip{##3}\ekv@parse@key##1}%
          }%
      }%
    \ekv@parse@other##1##2\ekv@mark
  }
%    \end{macrocode}
% \end{macro}
% \begin{macro}[internal]{\ekv@parse@eq@other}
%    \begin{macrocode}
\long\def\ekv@parse@eq@other##1=%
  {%
    \ekv@ifhas@eq@active##1#2\ekv@ifempty@B\ekv@ifempty@false
      \ekv@ifempty@A\ekv@ifempty@B\@firstoftwo
      {\ekv@parse@eq@active##1=}%
      {\ekv@strip{##1}\ekv@parse@pair\ekv@mark}%
  }
%    \end{macrocode}
% \end{macro}
% \begin{macro}[internal]{\ekv@parse@eq@active}
%    \begin{macrocode}
\long\def\ekv@parse@eq@active##1#2%
  {%
    \ekv@strip{##1}\ekv@parse@pair\ekv@mark
  }
%    \end{macrocode}
% \end{macro}
% \begin{macro}[internal]{\ekv@endparse@other}
%    \begin{macrocode}
\long\def\ekv@endparse@other
    \ekv@mark\ekv@stop
    \ekv@ifhas@eq@other##1=\ekv@ifempty@B\ekv@ifempty@false
    \ekv@ifempty@A\ekv@ifempty@B\@firstoftwo
    ##2%
    \ekv@parse@other##3\ekv@mark
  {}
%    \end{macrocode}
% \end{macro}
% \begin{macro}[internal]{\ekv@parse@pair,\ekv@parse@pair@}
%    \begin{macrocode}
\long\def\ekv@parse@pair##1##2\ekv@stop
  {%
    \ekv@strip{##2}\ekv@parse@pair@{##1}%
  }
\long\def\ekv@parse@pair@##1##2##3%
  {%
    \unexpanded{##3{##2}{##1}}%
  }
%    \end{macrocode}
% \end{macro}
% \begin{macro}[internal]{\ekv@parse@key}
%    \begin{macrocode}
\long\def\ekv@parse@key##1##2%
  {%
    \unexpanded{##2{##1}}%
  }
%    \end{macrocode}
% \end{macro}
%
% Finally really setting things up with |\ekvset|'s temporary meaning:
%    \begin{macrocode}
}
\begingroup
\catcode`\,=13
\catcode`\==13
\ekvset,=
%    \end{macrocode}
%
% \begin{macro}{\ekvchangeset}
% Provide a macro that is able to switch out the current \set\ in |\ekvset|.
% This operation is slow (by comparison, it should be slightly faster than
% |\ekvsneak|), but allows for something similar to \pkg{pgfkeys}'s
% \texttt{\meta{key}/.cd} mechanism. However this operation is more expensive
% than |/.cd| as we can't just redefine some token to reflect this, but have to
% switch out the set expandably, so this works similar to the |\ekvsneak| macros
% reading and reinserting the remainder of the \kv\ list.
%    \begin{macrocode}
\def\ekvchangeset#1%
  {%
    \expandafter\ekv@changeset\csname\ekv@undefined@set{#1}\endcsname\ekv@mark
  }
%    \end{macrocode}
% \end{macro}
% \begin{macro}[internal]{\ekv@changeset}
% This macro does the real change-out of |\ekvchangeset|. We introduced an
% |\ekv@mark| to not accidentally remove some braces which we have to remove
% again.
%    \begin{macrocode}
\long\def\ekv@changeset#1#2\ekv@set@other#3#4\ekv@set#5%
  {%
    \ekv@gobble@mark#2\ekv@set@other#1#4\ekv@set#1%
  }
%    \end{macrocode}
% \end{macro}
%
%
% \begin{macro}[internal]{\ekv@gobbleto@markstop}
% The |\ekv@gobbleto@markstop| can be used to test for |\ekv@stop| similar to
% our if-empty test, but instead of using tokens which are used nowhere else
% (|\ekv@ifempty@A| and |\ekv@ifempty@B|) it uses |\ekv@mark| and |\ekv@stop|.
%    \begin{macrocode}
\long\def\ekv@gobbleto@markstop#1\ekv@mark\ekv@stop{}
%    \end{macrocode}
% \end{macro}
%
% \begin{macro}[internal]{\ekv@set@pair,\ekv@set@pair@}
% |\ekv@set@pair| gets invoked with the space and brace stripped key-name as its
% first argument, the set-macro as the second argument, and following that is
% the key-value right delimited by an |\ekv@stop|.
%    \begin{macrocode}
\long\def\ekv@set@pair#1#2%
  {%
    \ekv@ifdefined@pair#2{#1}%
      {%
%    \end{macrocode}
% This branch will be executed if the key is not defined as an argument grabbing
% one. If so test whether there is a |NoVal| key of the same name or whether the
% key is unknown. Throw a meaningful error message and gobble the value.
%    \begin{macrocode}
        \ekv@ifdefined{#2{#1}N}%
          \ekv@err@noarg
          \ekv@err@unknown
          #2{#1}%
        \ekv@gobbleto@stop
      }%
  }
%    \end{macrocode}
% |\ekv@ifdefined@pair| will call |\ekv@set@pair@| if the key is correctly
% defined. This will then grab the value, strip outer spaces and braces from it
% and feed it to the key-macro. Afterwards |\ekv@set@other| will take control
% again.
%    \begin{macrocode}
\long\def\ekv@set@pair@#1#2\ekv@stop
  {%
    \ekv@strip{#2}#1%
  }
%    \end{macrocode}
% \end{macro}
%
% \begin{macro}[internal]{\ekv@set@key}
% Analogous to |\ekv@set@pair|, |\ekv@set@key| lets |\ekv@ifdefined@key| test
% whether a |NoVal| key is defined, else it'll throw a meaningful error message.
% Since we don't have to grab any value |\ekv@ifdefined@key| will invoke the
% key-macro and we're done here, |\ekv@set@other| will take over again.
%    \begin{macrocode}
\long\def\ekv@set@key#1#2%
  {%
    \ekv@ifdefined@key#2{#1}%
      {%
        \ekv@ifdefined{#2{#1}}%
          \ekv@err@reqval
          \ekv@err@unknown
          #2{#1}%
      }%
  }
%    \end{macrocode}
% \end{macro}
%
% \begin{macro}[internal]{\ekv@err,\ekv@err@}
% Since |\ekvset| is fully expandable as long as the code of the keys is (which
% is unlikely) we want to somehow throw expandable errors, in our case via
% undefined control sequences.
%    \begin{macrocode}
\begingroup
\edef\ekv@err
  {%
    \endgroup
    \unexpanded{\long\def\ekv@err}##1%
      {%
        \unexpanded{\expandafter\ekv@err@\@firstofone}%
        {\expandafter\noexpand\csname ! expkv Error:\endcsname ##1.}%
        \unexpanded{\ekv@stop}%
      }%
  }
\ekv@err
\def\ekv@err@{\expandafter\ekv@gobbleto@stop}
%    \end{macrocode}
% \end{macro}
% \begin{macro}[internal]
%   {
%     \ekv@err@common,\ekv@err@common@,
%     \ekv@err@unknown,\ekv@err@noarg,\ekv@err@reqval
%   }
% Now we can use |\ekv@err| to set up some error messages so that we can later
% use those instead of the full strings.
%    \begin{macrocode}
\long\def\ekv@err@common #1#2{\expandafter\ekv@err@common@\string#2{#1}}
\long\def\ekv@err@common@#1`#2' #3.#4#5{\ekv@err{#4 (`#5', set `#2')}}
\long\def\ekv@err@unknown#1#2{\ekv@err@common{unknown key}#1{#2}}
\long\def\ekv@err@noarg  #1#2{\ekv@err@common{value forbidden}#1{#2}}
\long\def\ekv@err@reqval #1#2{\ekv@err@common{value required}#1{#2}}
%    \end{macrocode}
% \end{macro}
%
% \begin{macro}[internal]{\ekv@strip,\ekv@strip@a,\ekv@strip@b,\ekv@strip@c}
% Finally we borrow some ideas of \pkg{expl3}'s \pkg{l3tl} to strip spaces
% from keys and values. This |\ekv@strip| also strips one level of outer braces
% \emph{after} stripping spaces, so an input of | {abc} | becomes |abc| after
% stripping. It should be used with |#1| prefixed by |\ekv@mark|. Also this
% implementation at most strips \emph{one} space from both sides.
%    \begin{macrocode}
\def\ekv@strip#1%
  {%
    \long\def\ekv@strip##1%
      {%
        \ekv@strip@a
          ##1%
          \ekv@nil
          \ekv@mark#1%
          #1\ekv@nil{}%
        \ekv@stop
      }%
    \long\def\ekv@strip@a##1\ekv@mark#1##2\ekv@nil##3%
      {%
        \ekv@strip@b##3##1##2\ekv@nil
      }%
    \long\def\ekv@strip@b##1#1\ekv@nil
      {%
        \ekv@strip@c##1\ekv@nil
      }%
    \long\def\ekv@strip@c\ekv@mark##1\ekv@nil##2\ekv@stop##3%
      {%
        ##3{##1}%
      }%
  }
\ekv@strip{ }
%    \end{macrocode}
% \end{macro}
%
% Now everything that's left is to reset the category code of |@|.
%    \begin{macrocode}
\catcode`\@=\ekv@tmp
%    \end{macrocode}
%
% \gobbledocstriptag
%</tex>
%^^A=<<
%
%^^A=<<
%
% \end{implementation}^^A=<<
%
% \clearpage
% \PrintIndex
%
\endinput
%
^^A vim: ft=tex fdm=marker fmr=>>=,=<<

\ProvidesPackage{expkv}%
  [\ekvDate\space v\ekvVersion\space an expandable key=val implementation]
%    \end{macrocode}
% \gobbledocstriptag
%</pkg>
%^^A=<<
%^^A main file >>=
% \subsection{The Generic Code}
% The rest of this implementation will be the generic code.
% \gobbledocstriptag
%<*tex>
%
% Check whether \eTeX\ is available -- \expkv\ requires \eTeX.
%    \begin{macrocode}
\begingroup\expandafter\expandafter\expandafter\endgroup
\expandafter\ifx\csname numexpr\endcsname\relax
  \errmessage{expkv requires e-TeX}
  \expandafter\endinput
\fi
%    \end{macrocode}
%
% We make sure that it's only input once:
%    \begin{macrocode}
\expandafter\ifx\csname ekvVersion\endcsname\relax
\else
  \expandafter\endinput
\fi
%    \end{macrocode}
% \begin{macro}{\ekvVersion,\ekvDate}
% We're on our first input, so lets store the version and date in a macro.
%    \begin{macrocode}
\def\ekvVersion{0.5a}
\def\ekvDate{2020-02-27}
%    \end{macrocode}
% \end{macro}
%
% If the \LaTeX\ format is loaded we want to be a good file and report back who
% we are, for this the package will have defined |\ekv@tmp| to use
% |\ProvidesFile|, else this will expand to a |\relax| and do no harm.
%    \begin{macrocode}
\csname ekv@tmp\endcsname
%    \end{macrocode}
%
% Store the category code of |@| to later be able to reset it and change it to
% 11 for now.
%    \begin{macrocode}
\expandafter\chardef\csname ekv@tmp\endcsname=\catcode`\@
\catcode`\@=11
%    \end{macrocode}
% |\ekv@tmp| might later be reused to gobble any prefixes which might be
% provided to |\ekvdef| and similar in case the names are invalid, we just
% temporarily use it here as means to store the current category code of |@| to
% restore it at the end of the file, we never care for the actual definition of
% it.
%
% \begin{macro}[internal]
%   {
%     \@gobble,\@firstofone,\@firstoftwo,\@secondoftwo,
%     \ekv@gobbleto@stop,\ekv@fi@secondoftwo,\ekv@gobble@mark
%   }
% Since branching tests are often more versatile than |\if...\else...\fi|
% constructs, we define helpers that are branching pretty fast. Also here are
% some other utility functions that just grab some tokens. The ones that are
% also contained in \LaTeX\ don't use the |ekv| prefix.
%    \begin{macrocode}
\long\def\@gobble#1{}
\long\def\@firstofone#1{#1}
\long\def\@firstoftwo#1#2{#1}
\long\def\@secondoftwo#1#2{#2}
\long\def\ekv@fi@secondoftwo\fi\@firstoftwo#1#2{\fi#2}
\long\def\ekv@gobbleto@stop#1\ekv@stop{}
\def\ekv@gobble@mark\ekv@mark{}
%    \end{macrocode}
% \end{macro}
% As you can see |\ekv@gobbleto@stop| uses a special marker |\ekv@stop|. The
% package will use three such markers, the one you've seen already, |\ekv@mark|
% and |\ekv@nil|. Contrarily to how for instance \pkg{expl3} does things, we
% don't define them, as we don't need them to have an actual meaning. This has
% the advantage that if they somehow get expanded -- which should never happen
% if things work out -- they'll throw an error directly.
%
% \begin{macro}[internal]
%   {
%     \ekv@ifempty,\ekv@ifempty@,\ekv@ifempty@true,\ekv@ifempty@false,
%     \ekv@ifempty@true@F
%   }
% We can test for a lot of things building on an if-empty test, so lets define a
% really fast one. Since some tests might have reversed logic (true if something
% is not empty) we also set up macros for the reversed branches.
%    \begin{macrocode}
\long\def\ekv@ifempty#1%
  {%
    \ekv@ifempty@\ekv@ifempty@A#1\ekv@ifempty@B\ekv@ifempty@true
      \ekv@ifempty@A\ekv@ifempty@B\@secondoftwo
  }
\long\def\ekv@ifempty@#1\ekv@ifempty@A\ekv@ifempty@B{}
\long\def\ekv@ifempty@true\ekv@ifempty@A\ekv@ifempty@B\@secondoftwo#1#2{#1}
\long\def\ekv@ifempty@false\ekv@ifempty@A\ekv@ifempty@B\@firstoftwo#1#2{#2}
\long\def\ekv@ifempty@true@F\ekv@ifempty@A\ekv@ifempty@B\@firstofone#1{}
%    \end{macrocode}
% \end{macro}
%
% \begin{macro}[internal]{\ekv@ifblank@}
% The obvious test that can be based on an if-empty is if-blank, meaning a test
% checking whether the argument is empty or consists only of spaces. Our version
% here will be tweaked a bit, as we want to check this, but with one leading
% |\ekv@mark| token that is to be ignored.
%    \begin{macrocode}
\long\def\ekv@ifblank@\ekv@mark#1{\ekv@ifempty@\ekv@ifempty@A}
%    \end{macrocode}
% \end{macro}
%
% \begin{macro}[internal]{\ekv@ifdefined}
% We'll need to check whether something is defined quite frequently, so why not
% define a macro that does this. The following test is expandable, slower than
% the typical expandable test for undefined control sequences, but faster for
% defined ones. Since we want to be as fast as possible for correct input, this
% is to be preferred.
%    \begin{macrocode}
\def\ekv@ifdefined#1%
  {%
    \expandafter
    \ifx\csname\ifcsname #1\endcsname #1\else relax\fi\endcsname\relax
      \ekv@fi@secondoftwo
    \fi
    \@firstoftwo
  }
%    \end{macrocode}
% \end{macro}
  % \begin{macro}[internal]
%   {
%     \ekv@ifdefined@pair,\ekv@ifdefined@pair@,\ekv@ifdefined@key,
%     \ekv@ifdefined@key@
%   }
% Since we can save some time if we only have to create the control sequence
% once when we know beforehand how we want to use it, we build some other macros
% for those cases (which we'll have quite often, once per key usage).
%    \begin{macrocode}
\def\ekv@ifdefined@pair#1#2%
  {%
    \expandafter\ekv@ifdefined@pair@
      \csname
        \ifcsname #1{#2}\endcsname
          #1{#2}%
        \else
          relax%
        \fi
      \endcsname
  }
\def\ekv@ifdefined@pair@#1%
  {%
    \ifx#1\relax
      \ekv@fi@secondoftwo
    \fi
    \@firstoftwo
    {\ekv@set@pair@#1\ekv@mark}%
  }
\def\ekv@ifdefined@key#1#2%
  {%
    \expandafter\ekv@ifdefined@key@
      \csname
        \ifcsname #1{#2}N\endcsname
          #1{#2}N%
        \else
          relax%
        \fi
      \endcsname
  }
\def\ekv@ifdefined@key@#1%
  {%
    \ifx#1\relax
      \ekv@fi@secondoftwo
    \fi
    \@firstoftwo#1%
  }
%    \end{macrocode}
% \end{macro}
%
% \begin{macro}{\ekv@name,\ekv@name@set,\ekv@name@key}
% The keys will all follow the same naming scheme, so we define it here.
%    \begin{macrocode}
\def\ekv@name#1#2{\ekv@name@set{#1}\ekv@name@key{#2}}
\def\ekv@name@set#1{ekv#1(}
\def\ekv@name@key#1{#1)}
%    \end{macrocode}
% \end{macro}
% \begin{macro}[internal]{\ekv@undefined@set}
% We can misuse the macro name we use to expandably store the set-name in a
% single token -- since this increases performance drastically, especially for
% long set-names -- to throw a more meaningful error message in case a set isn't
% defined. The name of |\ekv@undefined@set| is a little bit misleading, as it is
% called in either case inside of |\csname|, but the result will be a control
% sequence with meaning |\relax| if the set is undefined, hence will break the
% |\csname| building the key-macro which will throw the error message.
%    \begin{macrocode}
\def\ekv@undefined@set#1{! expkv Error: Set `#1' undefined.}
%    \end{macrocode}
% \end{macro}
% \begin{macro}[internal]{\ekv@checkvalid}
% We place some restrictions on the allowed names, though, namely sets and
% keys are not allowed to be empty -- blanks are fine (meaning \mbox{set-
% or} key-names consisting of spaces).
%    \begin{macrocode}
\protected\def\ekv@checkvalid#1#2%
  {%
    \ekv@ifempty{#1}%
      {%
        \def\ekv@tmp{}%
        \errmessage{expkv Error: empty set name not allowed}%
      }%
      {%
        \ekv@ifempty{#2}%
          {%
            \def\ekv@tmp{}%
            \errmessage{expkv Error: empty key name not allowed}%
          }%
          \@secondoftwo
      }%
    \@gobble
  }
%    \end{macrocode}
% \end{macro}
% \begin{macro}{\ekvifdefined,\ekvifdefinedNoVal}
% And provide user-level macros to test whether a key is defined.
%    \begin{macrocode}
\def\ekvifdefined#1#2{\ekv@ifdefined{\ekv@name{#1}{#2}}}
\def\ekvifdefinedNoVal#1#2{\ekv@ifdefined{\ekv@name{#1}{#2}N}}
%    \end{macrocode}
% \end{macro}
%
% \begin{macro}
%   {\ekvdef,\ekvdefNoVal,\ekvlet,\ekvletNoVal,\ekvletkv,\ekvletkvNoVal}
% Set up the key defining macros |\ekvdef| etc.
%    \begin{macrocode}
\protected\long\def\ekvdef#1#2#3%
  {%
    \ekv@checkvalid{#1}{#2}%
      {%
        \expandafter\def\csname\ekv@name{#1}{#2}\endcsname##1{#3}%
        \ekv@defset{#1}%
      }%
  }
\protected\long\def\ekvdefNoVal#1#2#3%
  {%
    \ekv@checkvalid{#1}{#2}%
      {%
        \expandafter\def\csname\ekv@name{#1}{#2}N\endcsname{#3}%
        \ekv@defset{#1}%
      }%
  }
\protected\def\ekvlet#1#2#3%
  {%
    \ekv@checkvalid{#1}{#2}%
      {%
        \expandafter\let\csname\ekv@name{#1}{#2}\endcsname#3%
        \ekv@defset{#1}%
      }%
  }
\protected\def\ekvletNoVal#1#2#3%
  {%
    \ekv@checkvalid{#1}{#2}%
      {%
        \expandafter\let\csname\ekv@name{#1}{#2}N\endcsname#3%
        \ekv@defset{#1}%
      }%
  }
\protected\def\ekvletkv#1#2#3#4%
  {%
    \ekv@checkvalid{#1}{#2}%
      {%
        \expandafter\let\csname\ekv@name{#1}{#2}\expandafter\endcsname
        \csname\ekv@name{#3}{#4}\endcsname
        \ekv@defset{#1}%
      }%
  }
\protected\def\ekvletkvNoVal#1#2#3#4%
  {%
    \ekv@checkvalid{#1}{#2}%
      {%
        \expandafter\let\csname\ekv@name{#1}{#2}N\expandafter\endcsname
        \csname\ekv@name{#3}{#4}N\endcsname
        \ekv@defset{#1}%
      }%
  }
%    \end{macrocode}
% \end{macro}
% \begin{macro}[internal]{\ekv@defset}
%   In order to enhance the speed the set name given to |\ekvset| will be turned
%   into a control sequence pretty early, so we have to define that control
%   sequence.
%    \begin{macrocode}
\protected\def\ekv@defset#1%
  {%
    \expandafter\edef\csname\ekv@undefined@set{#1}\endcsname##1%
      {\ekv@name@set{#1}\ekv@name@key{##1}}%
  }
%    \end{macrocode}
% \end{macro}
%
% \begin{macro}{\ekvset}
% Set up |\ekvset|, which should not be affected by active commas and equal
% signs. The equal signs are a bit harder to cope with and we'll do that later,
% but replacing the active commas with commas of category other can be done
% beforehand. That's why we define |\ekvset| here with a temporary meaning just
% to set up the things with two different category codes. |#1| will be a
% \texttt{,\textsubscript{13}} and |#2| will be a \texttt{=\textsubscript{13}}.
%    \begin{macrocode}
\def\ekvset#1#2{%
\endgroup
\long\def\ekvset##1##2%
  {%
    \expandafter\ekv@set\csname\ekv@undefined@set{##1}\endcsname
      \ekv@mark##2#1\ekv@stop#1{}%
  }
%    \end{macrocode}
% \end{macro}
% \begin{macro}[internal]{\ekv@set}
% |\ekv@set| will split the \kv\ list at active commas. Then it has to check
% whether there were unprotected other commas and resplit there.
%    \begin{macrocode}
\long\def\ekv@set##1##2#1%
  {%
%    \end{macrocode}
% Test whether we're at the end, if so invoke |\ekv@endset|,
%    \begin{macrocode}
    \ekv@gobbleto@markstop##2\ekv@endset\ekv@mark\ekv@stop
%    \end{macrocode}
% else go on with other commas,
%    \begin{macrocode}
    \ekv@set@other##1##2,\ekv@stop,%
%    \end{macrocode}
% and get the next active comma delimited \kv\ pair.
%    \begin{macrocode}
    \ekv@set##1\ekv@mark
  }
%    \end{macrocode}
% \end{macro}
% \begin{macro}[internal]{\ekv@endset}
% |\ekv@endset| is a hungry little macro. It will eat everything that remains
% of |\ekv@set| and unbrace the sneaked stuff.
%    \begin{macrocode}
\long\def\ekv@endset
    \ekv@mark\ekv@stop\ekv@set@other##1,\ekv@stop,\ekv@set##2\ekv@mark
    ##3%
  {##3}
%    \end{macrocode}
% \end{macro}
% \begin{macro}[internal]{\ekv@set@other}
% The macro |\ekv@set@other| is guaranteed to get only single \kv\ pairs.
% First we test whether we're done, if not split at equal signs. It is faster to
% first split at category 12 equal signs and only after that on actives.
% If there is no equal sign, we need to test whether we got a blank argument and
% if not this is a |NoVal| key.
%    \begin{macrocode}
\long\def\ekv@set@other##1##2,%
  {%
    \ekv@gobbleto@markstop##2\ekv@endset@other\ekv@mark\ekv@stop
    \ekv@ifhas@eq@other##2=\ekv@ifempty@B\ekv@ifempty@false
      \ekv@ifempty@A\ekv@ifempty@B\@firstoftwo
      {\ekv@set@eq@other##1##2\ekv@stop}%
      {%
        \ekv@ifhas@eq@active##2#2\ekv@ifempty@B\ekv@ifempty@false
          \ekv@ifempty@A\ekv@ifempty@B\@firstoftwo
          {\ekv@set@eq@active##1##2\ekv@stop}%
          {%
            \ekv@ifblank@##2\ekv@nil\ekv@ifempty@B\ekv@ifempty@true@F
              \ekv@ifempty@A\ekv@ifempty@B\@firstofone
              {\ekv@strip{##2}\ekv@set@key##1}%
          }%
      }%
    \ekv@set@other##1\ekv@mark%
  }
%    \end{macrocode}
% \end{macro}
% \begin{macro}[internal]{\ekv@set@eq@other}
% |\ekv@set@eq@other| might not be the correct break point, there might be an
% active equal sign in the currently parsed key-name. If so, we have to resplit.
% If the split is correct strip the key-name of outer spaces and braces and feed
% it to |\ekv@set@pair|.
%    \begin{macrocode}
\long\def\ekv@set@eq@other##1##2=%
  {%
    \ekv@ifhas@eq@active##2#2\ekv@ifempty@B\ekv@ifempty@false
      \ekv@ifempty@A\ekv@ifempty@B\@firstoftwo
      {\ekv@set@eq@active##1##2=}%
      {\ekv@strip{##2}\ekv@set@pair##1}%
  }
%    \end{macrocode}
% \end{macro}
% \begin{macro}[internal]{\ekv@set@eq@active}
% We need to handle the active equal signs.
%    \begin{macrocode}
\long\def\ekv@set@eq@active##1##2#2%
  {%
    \ekv@strip{##2}\ekv@set@pair##1%
  }
%    \end{macrocode}
% \end{macro}
% \begin{macro}[internal]
%   {\ekv@ifhas@eq@other,\ekv@ifhas@eq@active,\ekv@endset@other}
% And we have to set up the testing macros for our equal signs and
% |\ekv@endset@other|.
%    \begin{macrocode}
\long\def\ekv@ifhas@eq@other\ekv@mark##1={\ekv@ifempty@\ekv@ifempty@A}
\long\def\ekv@ifhas@eq@active\ekv@mark##1#2{\ekv@ifempty@\ekv@ifempty@A}
\long\def\ekv@endset@other
    \ekv@mark\ekv@stop
    \ekv@ifhas@eq@other##1=\ekv@ifempty@B\ekv@ifempty@false
    \ekv@ifempty@A\ekv@ifempty@B\@firstoftwo
    ##2%
    \ekv@set@other##3\ekv@mark
  {}
%    \end{macrocode}
% \end{macro}
% \begin{macro}{\ekvbreak,\ekvbreakPreSneak,\ekvbreakPostSneak}
% Provide macros that can completely stop the parsing of |\ekvset|, who knows
% what it'll be useful for.
%    \begin{macrocode}
\long\def\ekvbreak##1##2\ekv@stop#1##3{##1}
\long\def\ekvbreakPreSneak ##1##2\ekv@stop#1##3{##1##3}
\long\def\ekvbreakPostSneak##1##2\ekv@stop#1##3{##3##1}
%    \end{macrocode}
% \end{macro}
% \begin{macro}{\ekvsneak,\ekvsneakPre}
% One last thing we want to do for |\ekvset| is to provide macros that just
% smuggle stuff after |\ekvset|'s effects.
%    \begin{macrocode}
\long\def\ekvsneak##1##2\ekv@stop#1##3%
  {%
    ##2\ekv@stop#1{##3##1}%
  }
\long\def\ekvsneakPre##1##2\ekv@stop#1##3%
  {%
    ##2\ekv@stop#1{##1##3}%
  }
%    \end{macrocode}
% \end{macro}
%
% \begin{macro}{\ekvparse}
% Additionally to the |\ekvset| macro we also want to provide an |\ekvparse|
% macro, that has the same scope as |\keyval_parse:NNn| from \pkg{expl3}.
% This is pretty analogue to the |\ekvset| implementation, we just put an
% |\unexpanded| here and there instead of other macros to stop the |\expanded|
% on our output.
%    \begin{macrocode}
\long\def\ekvparse##1##2##3%
  {%
    \ekv@parse##1##2\ekv@mark##3#1\ekv@stop#1%
  }
%    \end{macrocode}
% \end{macro}
% \begin{macro}[internal]{\ekv@parse}
%    \begin{macrocode}
\long\def\ekv@parse##1##2##3#1%
  {%
    \ekv@gobbleto@markstop##3\ekv@endparse\ekv@mark\ekv@stop
    \ekv@parse@other##1##2##3,\ekv@stop,%
    \ekv@parse##1##2\ekv@mark
  }
%    \end{macrocode}
% \end{macro}
% \begin{macro}[internal]{\ekv@endparse}
%    \begin{macrocode}
\long\def\ekv@endparse
    \ekv@mark\ekv@stop\ekv@parse@other##1,\ekv@stop,\ekv@parse##2\ekv@mark
  {}
%    \end{macrocode}
% \end{macro}
% \begin{macro}[internal]{\ekv@parse@other}
%    \begin{macrocode}
\long\def\ekv@parse@other##1##2##3,%
  {%
    \ekv@gobbleto@markstop##3\ekv@endparse@other\ekv@mark\ekv@stop
    \ekv@ifhas@eq@other##3=\ekv@ifempty@B\ekv@ifempty@false
      \ekv@ifempty@A\ekv@ifempty@B\@firstoftwo
      {\ekv@parse@eq@other##3\ekv@stop##2}%
      {%
        \ekv@ifhas@eq@active##3#2\ekv@ifempty@B\ekv@ifempty@false
          \ekv@ifempty@A\ekv@ifempty@B\@firstoftwo
          {\ekv@parse@eq@active##3\ekv@stop##2}%
          {%
            \ekv@ifblank@##3\ekv@nil\ekv@ifempty@B\ekv@ifempty@true@F
              \ekv@ifempty@A\ekv@ifempty@B\@firstofone
              {\ekv@strip{##3}\ekv@parse@key##1}%
          }%
      }%
    \ekv@parse@other##1##2\ekv@mark
  }
%    \end{macrocode}
% \end{macro}
% \begin{macro}[internal]{\ekv@parse@eq@other}
%    \begin{macrocode}
\long\def\ekv@parse@eq@other##1=%
  {%
    \ekv@ifhas@eq@active##1#2\ekv@ifempty@B\ekv@ifempty@false
      \ekv@ifempty@A\ekv@ifempty@B\@firstoftwo
      {\ekv@parse@eq@active##1=}%
      {\ekv@strip{##1}\ekv@parse@pair\ekv@mark}%
  }
%    \end{macrocode}
% \end{macro}
% \begin{macro}[internal]{\ekv@parse@eq@active}
%    \begin{macrocode}
\long\def\ekv@parse@eq@active##1#2%
  {%
    \ekv@strip{##1}\ekv@parse@pair\ekv@mark
  }
%    \end{macrocode}
% \end{macro}
% \begin{macro}[internal]{\ekv@endparse@other}
%    \begin{macrocode}
\long\def\ekv@endparse@other
    \ekv@mark\ekv@stop
    \ekv@ifhas@eq@other##1=\ekv@ifempty@B\ekv@ifempty@false
    \ekv@ifempty@A\ekv@ifempty@B\@firstoftwo
    ##2%
    \ekv@parse@other##3\ekv@mark
  {}
%    \end{macrocode}
% \end{macro}
% \begin{macro}[internal]{\ekv@parse@pair,\ekv@parse@pair@}
%    \begin{macrocode}
\long\def\ekv@parse@pair##1##2\ekv@stop
  {%
    \ekv@strip{##2}\ekv@parse@pair@{##1}%
  }
\long\def\ekv@parse@pair@##1##2##3%
  {%
    \unexpanded{##3{##2}{##1}}%
  }
%    \end{macrocode}
% \end{macro}
% \begin{macro}[internal]{\ekv@parse@key}
%    \begin{macrocode}
\long\def\ekv@parse@key##1##2%
  {%
    \unexpanded{##2{##1}}%
  }
%    \end{macrocode}
% \end{macro}
%
% Finally really setting things up with |\ekvset|'s temporary meaning:
%    \begin{macrocode}
}
\begingroup
\catcode`\,=13
\catcode`\==13
\ekvset,=
%    \end{macrocode}
%
% \begin{macro}{\ekvchangeset}
% Provide a macro that is able to switch out the current \set\ in |\ekvset|.
% This operation is slow (by comparison, it should be slightly faster than
% |\ekvsneak|), but allows for something similar to \pkg{pgfkeys}'s
% \texttt{\meta{key}/.cd} mechanism. However this operation is more expensive
% than |/.cd| as we can't just redefine some token to reflect this, but have to
% switch out the set expandably, so this works similar to the |\ekvsneak| macros
% reading and reinserting the remainder of the \kv\ list.
%    \begin{macrocode}
\def\ekvchangeset#1%
  {%
    \expandafter\ekv@changeset\csname\ekv@undefined@set{#1}\endcsname\ekv@mark
  }
%    \end{macrocode}
% \end{macro}
% \begin{macro}[internal]{\ekv@changeset}
% This macro does the real change-out of |\ekvchangeset|. We introduced an
% |\ekv@mark| to not accidentally remove some braces which we have to remove
% again.
%    \begin{macrocode}
\long\def\ekv@changeset#1#2\ekv@set@other#3#4\ekv@set#5%
  {%
    \ekv@gobble@mark#2\ekv@set@other#1#4\ekv@set#1%
  }
%    \end{macrocode}
% \end{macro}
%
%
% \begin{macro}[internal]{\ekv@gobbleto@markstop}
% The |\ekv@gobbleto@markstop| can be used to test for |\ekv@stop| similar to
% our if-empty test, but instead of using tokens which are used nowhere else
% (|\ekv@ifempty@A| and |\ekv@ifempty@B|) it uses |\ekv@mark| and |\ekv@stop|.
%    \begin{macrocode}
\long\def\ekv@gobbleto@markstop#1\ekv@mark\ekv@stop{}
%    \end{macrocode}
% \end{macro}
%
% \begin{macro}[internal]{\ekv@set@pair,\ekv@set@pair@}
% |\ekv@set@pair| gets invoked with the space and brace stripped key-name as its
% first argument, the set-macro as the second argument, and following that is
% the key-value right delimited by an |\ekv@stop|.
%    \begin{macrocode}
\long\def\ekv@set@pair#1#2%
  {%
    \ekv@ifdefined@pair#2{#1}%
      {%
%    \end{macrocode}
% This branch will be executed if the key is not defined as an argument grabbing
% one. If so test whether there is a |NoVal| key of the same name or whether the
% key is unknown. Throw a meaningful error message and gobble the value.
%    \begin{macrocode}
        \ekv@ifdefined{#2{#1}N}%
          \ekv@err@noarg
          \ekv@err@unknown
          #2{#1}%
        \ekv@gobbleto@stop
      }%
  }
%    \end{macrocode}
% |\ekv@ifdefined@pair| will call |\ekv@set@pair@| if the key is correctly
% defined. This will then grab the value, strip outer spaces and braces from it
% and feed it to the key-macro. Afterwards |\ekv@set@other| will take control
% again.
%    \begin{macrocode}
\long\def\ekv@set@pair@#1#2\ekv@stop
  {%
    \ekv@strip{#2}#1%
  }
%    \end{macrocode}
% \end{macro}
%
% \begin{macro}[internal]{\ekv@set@key}
% Analogous to |\ekv@set@pair|, |\ekv@set@key| lets |\ekv@ifdefined@key| test
% whether a |NoVal| key is defined, else it'll throw a meaningful error message.
% Since we don't have to grab any value |\ekv@ifdefined@key| will invoke the
% key-macro and we're done here, |\ekv@set@other| will take over again.
%    \begin{macrocode}
\long\def\ekv@set@key#1#2%
  {%
    \ekv@ifdefined@key#2{#1}%
      {%
        \ekv@ifdefined{#2{#1}}%
          \ekv@err@reqval
          \ekv@err@unknown
          #2{#1}%
      }%
  }
%    \end{macrocode}
% \end{macro}
%
% \begin{macro}[internal]{\ekv@err,\ekv@err@}
% Since |\ekvset| is fully expandable as long as the code of the keys is (which
% is unlikely) we want to somehow throw expandable errors, in our case via
% undefined control sequences.
%    \begin{macrocode}
\begingroup
\edef\ekv@err
  {%
    \endgroup
    \unexpanded{\long\def\ekv@err}##1%
      {%
        \unexpanded{\expandafter\ekv@err@\@firstofone}%
        {\expandafter\noexpand\csname ! expkv Error:\endcsname ##1.}%
        \unexpanded{\ekv@stop}%
      }%
  }
\ekv@err
\def\ekv@err@{\expandafter\ekv@gobbleto@stop}
%    \end{macrocode}
% \end{macro}
% \begin{macro}[internal]
%   {
%     \ekv@err@common,\ekv@err@common@,
%     \ekv@err@unknown,\ekv@err@noarg,\ekv@err@reqval
%   }
% Now we can use |\ekv@err| to set up some error messages so that we can later
% use those instead of the full strings.
%    \begin{macrocode}
\long\def\ekv@err@common #1#2{\expandafter\ekv@err@common@\string#2{#1}}
\long\def\ekv@err@common@#1`#2' #3.#4#5{\ekv@err{#4 (`#5', set `#2')}}
\long\def\ekv@err@unknown#1#2{\ekv@err@common{unknown key}#1{#2}}
\long\def\ekv@err@noarg  #1#2{\ekv@err@common{value forbidden}#1{#2}}
\long\def\ekv@err@reqval #1#2{\ekv@err@common{value required}#1{#2}}
%    \end{macrocode}
% \end{macro}
%
% \begin{macro}[internal]{\ekv@strip,\ekv@strip@a,\ekv@strip@b,\ekv@strip@c}
% Finally we borrow some ideas of \pkg{expl3}'s \pkg{l3tl} to strip spaces
% from keys and values. This |\ekv@strip| also strips one level of outer braces
% \emph{after} stripping spaces, so an input of | {abc} | becomes |abc| after
% stripping. It should be used with |#1| prefixed by |\ekv@mark|. Also this
% implementation at most strips \emph{one} space from both sides.
%    \begin{macrocode}
\def\ekv@strip#1%
  {%
    \long\def\ekv@strip##1%
      {%
        \ekv@strip@a
          ##1%
          \ekv@nil
          \ekv@mark#1%
          #1\ekv@nil{}%
        \ekv@stop
      }%
    \long\def\ekv@strip@a##1\ekv@mark#1##2\ekv@nil##3%
      {%
        \ekv@strip@b##3##1##2\ekv@nil
      }%
    \long\def\ekv@strip@b##1#1\ekv@nil
      {%
        \ekv@strip@c##1\ekv@nil
      }%
    \long\def\ekv@strip@c\ekv@mark##1\ekv@nil##2\ekv@stop##3%
      {%
        ##3{##1}%
      }%
  }
\ekv@strip{ }
%    \end{macrocode}
% \end{macro}
%
% Now everything that's left is to reset the category code of |@|.
%    \begin{macrocode}
\catcode`\@=\ekv@tmp
%    \end{macrocode}
%
% \gobbledocstriptag
%</tex>
%^^A=<<
%
%^^A=<<
%
% \end{implementation}^^A=<<
%
% \clearpage
% \PrintIndex
%
\endinput
%
^^A vim: ft=tex fdm=marker fmr=>>=,=<<

\ProvidesPackage{expkv}%
  [\ekv@date\space v\ekv@version\space an expandable key=val implementation]
%</pkg>
%    \end{macrocode}
%^^A=<<
%^^A main file >>=
% The rest of this implementation will be the generic code.
%    \begin{macrocode}
%<*tex>
%    \end{macrocode}
%
% Check whether \eTeX\ is available -- \expkv\ requires \eTeX.
%    \begin{macrocode}
\begingroup\expandafter\expandafter\expandafter\endgroup
\expandafter\ifx\csname numexpr\endcsname\relax
  \errmessage{expkv requires e-TeX}
  \expandafter\endinput
\fi
%    \end{macrocode}
%
% We make sure that it's only input once:
%    \begin{macrocode}
\expandafter\ifx\csname ekv@version\endcsname\relax
\else
  \expandafter\endinput
\fi
%    \end{macrocode}
%
% Change the category code of |@| and define the version and date:
%    \begin{macrocode}
\expandafter\chardef\csname ekv@waste\endcsname\catcode`\@
\catcode`\@=11
\def\ekv@version{0.1}
\def\ekv@date{2020-01-11}
%    \end{macrocode}
% |\ekv@waste| might later be reused to gobble any prefixes which might be
% provided to |\ekvdef| and similar in case the names are invalid, we just
% temporarily use it here as means to store the current category code of |@| to
% restore it at the end of the file, we never care for the actual definition of
% it.
%
% If the \LaTeX\ format is loaded we want to be a good file and report back who
% we are, for this the package will have defined |\ekv@stop| to use
% |\ProvidesFile|, else this will expand to a |\relax| and do no harm. Later
% |\ekv@stop| will be (re-)defined.
%    \begin{macrocode}
\csname ekv@stop\endcsname
%    \end{macrocode}
%
% We'll need a few special tokens as delimiters here and there, but only
% |\ekv@stop| needs to have a meaning, as it is used in an |\ifx| comparison.
%    \begin{macrocode}
\def\ekv@stop{\ekv@err{expanded \ekv@stop}}
%    \end{macrocode}
%
% Since branching tests are often more versatile than |\if...\else...\fi|
% constructs, we define helpers that are branching pretty fast. Also here are
% some other utility functions that just grab some tokens. The ones that are
% also contained in \LaTeX\ don't use the |ekv| prefix.
%    \begin{macrocode}
\long\def\ekv@gobbletostop#1\ekv@stop{}
\long\def\@firstofone#1{#1}
\long\def\@firstoftwo#1#2{#1}
\long\def\@secondoftwo#1#2{#2}
\long\def\ekv@fi@gobble\fi\@firstofone#1{\fi}
\long\def\ekv@fi@secondoftwo\fi\@firstoftwo#1#2{\fi#2}
%    \end{macrocode}
%
% We can test for a lot of things building on an if-empty test, so lets define a
% really fast one. Since some tests might have reversed logic (true if something
% is not empty) we also set up macros for the reversed branches.
%    \begin{macrocode}
\long\def\ekv@ifempty#1%
  {%
    \ekv@ifempty@\ekv@ifempty@A#1\ekv@ifempty@B\ekv@ifempty@true
      \ekv@ifempty@A\ekv@ifempty@B\@secondoftwo
  }
\long\def\ekv@ifempty@#1\ekv@ifempty@A\ekv@ifempty@B{}
\long\def\ekv@ifempty@true\ekv@ifempty@A\ekv@ifempty@B\@secondoftwo#1#2{#1}
\long\def\ekv@ifempty@false\ekv@ifempty@A\ekv@ifempty@B\@firstoftwo#1#2{#2}
\long\def\ekv@ifempty@true@F\ekv@ifempty@A\ekv@ifempty@B\@firstofone#1{}
%    \end{macrocode}
%
% The obvious test that can be based on an if-empty is if-blank, meaning a test
% checking whether the argument is empty or consists only of spaces. Our version
% here will be tweaked a bit, as we want to check this, but with one leading
% |\ekv@mark| token that is to be ignored.
%    \begin{macrocode}
\long\def\ekv@ifblankF#1%
  {%
    \ekv@ifblank@#1\ekv@nil\ekv@ifempty@B\ekv@ifempty@true@F
      \ekv@ifempty@A\ekv@ifempty@B\@firstofone
  }
\long\def\ekv@ifblank@\ekv@mark#1{\ekv@ifempty@\ekv@ifempty@A}
%    \end{macrocode}
%
% We'll need to check whether something is defined quite frequently, so why not
% define a macro that does this. The following test is expandable, slower than
% the typical expandable test for undefined control sequences, but faster for
% defined ones. Since we want to be as fast as possible for correct input, this
% is to be preferred.
%    \begin{macrocode}
\def\ekv@ifdefined#1%
  {%
    \expandafter
    \ifx\csname\ifcsname #1\endcsname #1\else relax\fi\endcsname\relax
      \ekv@fi@secondoftwo
    \fi
    \@firstoftwo
  }
%    \end{macrocode}
%
% The keys will all follow the same naming scheme, so we define it here.
% \begin{macro}{\ekv@name}
%    \begin{macrocode}
\def\ekv@name#1#2{ekv(set #1)(key #2)}
%    \end{macrocode}
% \end{macro}
% We place some restrictions on the allowed names, though, namely sets and
% keys are not allowed to be empty -- blanks are fine (meaning \mbox{set-
% or} key-names consisting of spaces).
%    \begin{macrocode}
\protected\long\def\ekv@checkvalid#1#2#3%
  {%
    \ekv@ifempty{#1}
      {%
        \def\ekv@waste{}%
        \errmessage{expkv Error: empty set name not allowed}%
      }
      {%
        \ekv@ifempty{#2}
          {%
            \def\ekv@waste{}%
            \errmessage{expkv Error: empty key name not allowed}%
          }
          {#3}%
      }%
  }
%    \end{macrocode}
% And provide user-level macros to test whether a key is defined.
% \begin{macro}{\ekvifdefined,\ekvifNoValdefined}
%    \begin{macrocode}
\def\ekvifdefined#1#2{\ekv@ifdefined{\ekv@name{#1}{#2}}}
\def\ekvifNoValdefined#1#2{\ekv@ifdefined{\ekv@name{#1}{#2}:n}}
%    \end{macrocode}
% \end{macro}
%
% Set up the key defining macros |\ekvdef| etc.
% \begin{macro}{\ekvdef}
%    \begin{macrocode}
\protected\long\def\ekvdef#1#2#3%
  {%
    \ekv@checkvalid{#1}{#2}
      {\expandafter\def\csname\ekv@name{#1}{#2}\endcsname##1{#3}}%
  }
%    \end{macrocode}
% \end{macro}
% \begin{macro}{\ekvdefNoVal}
%    \begin{macrocode}
\protected\long\def\ekvdefNoVal#1#2#3%
  {%
    \ekv@checkvalid{#1}{#2}
      {\expandafter\def\csname\ekv@name{#1}{#2}:n\endcsname{#3}}%
  }
%    \end{macrocode}
% \end{macro}
% \begin{macro}{\ekvlet}
%    \begin{macrocode}
\protected\def\ekvlet#1#2%
  {%
    \ekv@checkvalid{#1}{#2}
      {\expandafter\let\csname\ekv@name{#1}{#2}\endcsname}%
  }
%    \end{macrocode}
% \end{macro}
% \begin{macro}{\ekvletNoVal}
%    \begin{macrocode}
\protected\def\ekvletNoVal#1#2%
  {%
    \ekv@checkvalid{#1}{#2}
      {\expandafter\let\csname\ekv@name{#1}{#2}:n\endcsname}%
  }
%    \end{macrocode}
% \end{macro}
% \begin{macro}{\ekvletkv}
%    \begin{macrocode}
\protected\def\ekvletkv#1#2#3#4%
  {%
    \ekv@checkvalid{#1}{#2}
      {%
        \expandafter\let\csname\ekv@name{#1}{#2}\expandafter\endcsname
        \csname\ekv@name{#1}{#2}\endcsname
      }%
  }
%    \end{macrocode}
% \end{macro}
% \begin{macro}{\ekvletkvNoVal}
%    \begin{macrocode}
\protected\def\ekvletkvNoVal#1#2#3#4%
  {%
    \ekv@checkvalid{#1}{#2}
      {%
        \expandafter\let\csname\ekv@name{#1}{#2}:n\expandafter\endcsname
        \csname\ekv@name{#1}{#2}:n\endcsname
      }%
  }
%    \end{macrocode}
% \end{macro}
%
% And set up |\ekvset|, which should not be affected by active commas and equal
% signs. The equal signs are a bit harder to cope with and we'll do that later,
% but replacing the active commas with commas of category other can be done
% beforehand. That's why we define |\ekvset| here with a temporary meaning just
% to set up the things with two different category codes. |#1| will be a
% \texttt{,\textsubscript{13}} and |#2| will be a \texttt{=\textsubscript{13}}.
% \begin{macro}{\ekvset}
%    \begin{macrocode}
\def\ekvset#1#2{%
\endgroup
\long\def\ekvset##1##2%
  {%
    \expanded
      {%
        \unexpanded{\ekv@parse{##1}}%
        \ekv@splitcomma@active\ekv@mark##2#1\ekv@stop#1%
      }%
      \ekv@mark\ekv@stop,{}%
  }
%    \end{macrocode}
% \end{macro}
% The macro |\ekv@splitcomma@active| will replace all the active commas with
% others, place an |\ekv@mark| after each comma, and remove empty list elements.
%    \begin{macrocode}
\long\def\ekv@splitcomma@active##1#1%
  {%
    \ekv@ifstop##1\ekv@endsplit\ekv@mark\ekv@stop\@firstofone
      {%
        \ekv@splitcomma@other##1,\ekv@stop,%
        \ekv@splitcomma@active\ekv@mark
      }%
  }
\long\def\ekv@splitcomma@other##1,%
  {%
    \ekv@ifstop##1\ekv@endsplit\ekv@mark\ekv@stop\@firstofone
      {\ekv@ifblankF{##1}{\unexpanded{##1},}\ekv@splitcomma@other\ekv@mark}%
  }
%    \end{macrocode}
% Additionally we'll need a macro which can determine if the key-name includes
% an active equal sign, meaning it was broken at the wrong equal sign, and
% resplits the key-name pair.
%    \begin{macrocode}
\long\def\ekv@ifhaseq@active##1%
  {%
    \ekv@ifhaseq@active@##1#2\ekv@ifempty@B\ekv@ifempty@false
    \ekv@ifempty@A\ekv@ifempty@B\@firstoftwo
  }
\long\def\ekv@ifhaseq@active@##1#2{\ekv@ifempty@\ekv@ifempty@A}
\long\def\ekv@spliteq@active##1##2#2%
  {%
    \ekv@strip{##2}\ekv@parse@pair@a{##1}%
  }
%    \end{macrocode}
% Finally really setting things up with |\ekvset|'s temporary meaning:
%    \begin{macrocode}
}
\begingroup
\catcode`\,13
\catcode`\=13
\ekvset,=
%    \end{macrocode}
%
% The first thing |\ekvset| has to do is normalize active commas and equals, so
% that we're category code safe (e.g., in \LaTeX\ babel might make one of these
% active depending on the language choice). For that we define |\ekv@normalize|.
% In the following |#1| will be an active comma and |#2| will be an active equal
% sign.
%    \begin{macrocode}
\long\def\ekv@endsplit\ekv@mark\ekv@stop\@firstofone#1{}
%    \end{macrocode}
%
% |\ekvset@parse| needs to split the second argument to |\ekvset| at the commas,
% but since we don't want to accidentally strip braces around anything we put a
% leading |\ekv@mark| at the begin of |#2|.
%    \begin{macrocode}
\long\def\ekv@parse#1#2,%
  {%
%    \end{macrocode}
% First check whether we got an equal sign or an active equal sign.
%    \begin{macrocode}
    \ekv@ifhaseq@other{#2}
      {\ekv@parse@pair{#1}#2\ekv@stop}
      {%
        \ekv@ifhaseq@active{#2}%
          {\ekv@spliteq@active{#1}#2\ekv@stop}%
          {%
%    \end{macrocode}
% If we don't have both, check whether we're done.
%    \begin{macrocode}
            \ekv@ifstop#2\ekv@endparse\ekv@mark\ekv@stop\@firstofone
              {%
%    \end{macrocode}
% If we don't have an equal sign and aren't done, then we have a |NoVal|
% situation.
%    \begin{macrocode}
                \ekv@strip{#2}\ekv@parse@key{#1}%
%    \end{macrocode}
% Finally grab the next key.
%    \begin{macrocode}
              }%
          }%
      }%
    \ekv@parse{#1}%
  }
%    \end{macrocode}
%
% The |\ekv@ifstop| test works similar to our if-empty test, but instead of
% using tokens which are used nowhere else (|\ekv@ifempty@A| and
% |\ekv@ifempty@B|) we use |\ekv@mark| and |\ekv@stop|.
%    \begin{macrocode}
\long\def\ekv@ifstop#1\ekv@mark\ekv@stop{}
%    \end{macrocode}
%
% The |\ekv@endparse| macro will have to gobble some tokens to end |\ekvset|'s
% effects.
%    \begin{macrocode}
\long\def\ekv@endparse\ekv@mark\ekv@stop\@firstofone#1\ekv@parse#2#3{#3}
%    \end{macrocode}
%
%
% The |\ekv@ifhaseq@other| test is one of the tests which can be reduced to an
% if-empty test with reversed logic.
%    \begin{macrocode}
\long\def\ekv@ifhaseq@other#1%
  {%
    \ekv@ifhaseq@other@#1=\ekv@ifempty@B\ekv@ifempty@false
      \ekv@ifempty@A\ekv@ifempty@B\@firstoftwo
  }
\long\def\ekv@ifhaseq@other@#1={\ekv@ifempty@\ekv@ifempty@A}
%    \end{macrocode}
%
% |\ekv@parse@pair| needs to split the argument at the |=| sign and check
% whether the key is defined.
%    \begin{macrocode}
\long\def\ekv@parse@pair#1#2=%
  {%
    \ekv@ifhaseq@active{#2}{\ekv@spliteq@active{#1}#2=}
      {\ekv@strip{#2}\ekv@parse@pair@a{#1}}%
  }
\long\def\ekv@parse@pair@a#1#2%
  {%
    \ekv@ifdefined{\ekv@name{#2}{#1}}
      {\expandafter\ekv@parse@pair@b\csname\ekv@name{#2}{#1}\endcsname\ekv@mark}
      {%
        \ekv@ifdefined{\ekv@name{#2}{#1}:n}
          \ekv@err@noarg
          \ekv@err@unknown
          {#2}{#1}%
        \ekv@gobbletostop
      }%
  }
\long\def\ekv@parse@pair@b#1#2\ekv@stop
  {%
    \ekv@strip{#2}#1%
  }
%    \end{macrocode}
%
%    \begin{macrocode}
\long\def\ekv@parse@key#1#2%
  {%
    \ekv@ifdefined{\ekv@name{#2}{#1}:n}
      {\csname\ekv@name{#2}{#1}:n\endcsname}
      {%
        \ekv@ifdefined{\ekv@name{#2}{#1}}
          \ekv@err@reqval
          \ekv@err@unknown
          {#2}{#1}%
      }%
  }
%    \end{macrocode}
%
% Provide macros that can completely stop the parsing of |\ekvset|, who knows
% what it'll be useful for.
% \begin{macro}{\ekvbreak,\ekvbreakPreSneak,\ekvbreakPostSneak}
%    \begin{macrocode}
\long\def\ekvbreak#1#2\ekv@stop,#3{#1}
\long\def\ekvbreakPreSneak #1#2\ekv@stop,#3{#1#3}
\long\def\ekvbreakPostSneak#1#2\ekv@stop,#3{#3#1}
%    \end{macrocode}
% \end{macro}
% And provide macros that just smuggle stuff after |\ekvset|'s effects.
% \begin{macro}{\ekvsneak,\ekvsneakPre}
%    \begin{macrocode}
\long\def\ekvsneak#1#2\ekv@stop,#3%
  {%
    #2\ekv@stop,{#3#1}%
  }
\long\def\ekvsneakPre#1#2\ekv@stop,#3%
  {%
    #2\ekv@stop,{#1#3}%
  }
%    \end{macrocode}
% \end{macro}
%
% Since |\ekvset| is fully expandable as long as the code of the keys is (which
% is unlikely) we want to somehow throw expandable errors, in our case via
% undefined control sequences.
%    \begin{macrocode}
\begingroup
\edef\ekv@err
  {%
    \endgroup
    \unexpanded{\long\def\ekv@err}##1%
      {%
        \unexpanded{\expandafter\ekv@err@\@firstofone}%
        {\expandafter\noexpand\csname ! expkv Error:\endcsname ##1.}%
        \unexpanded{\ekv@stop}%
      }%
  }
\ekv@err
\def\ekv@err@{\expandafter\ekv@gobbletostop}
%    \end{macrocode}
% Now we can use |\ekv@err| to set up some error messages so that we can later
% use those instead of the full strings.
%    \begin{macrocode}
\long\def\ekv@err@unknown#1#2{\ekv@err{unknown key (`#2', set `#1')}}
\long\def\ekv@err@noarg  #1#2{\ekv@err{value forbidden (`#2', set `#1')}}
\long\def\ekv@err@reqval #1#2{\ekv@err{value required (`#2', set `#1')}}
%    \end{macrocode}
%
% Finally we borrow some ideas of \pkg{expl3}'s \pkg{l3tl} to strip spaces
% from keys and values. This |\ekv@strip| also strips one level of outer braces
% \emph{after} stripping spaces, so an input of | {abc} | becomes |abc| after
% stripping. It should be used with |#1| prefixed by |\ekv@mark|.
%    \begin{macrocode}
\def\ekv@strip#1%
  {%
    \long\def\ekv@strip##1%
      {%
        \ekv@strip@a
          ##1%
          \ekv@nil
          \ekv@mark#1%
          #1\ekv@nil{}%
        \ekv@stop
      }%
    \long\def\ekv@strip@a##1\ekv@mark#1##2\ekv@nil##3%
      {%
        \ekv@strip@b##3##1##2\ekv@nil
      }%
    \long\def\ekv@strip@b##1#1\ekv@nil
      {%
        \ekv@strip@c##1\ekv@nil
      }%
    \long\def\ekv@strip@c\ekv@mark##1\ekv@nil##2\ekv@stop##3%
      {%
        ##3{##1}%
      }%
  }
\ekv@strip{ }
%    \end{macrocode}
%
% Now everything that's left is to reset the category code of |@|.
%    \begin{macrocode}
\catcode`\@\ekv@waste
%    \end{macrocode}
%
%
%    \begin{macrocode}
%</tex>
%    \end{macrocode}
%^^A=<<
%
%^^A=<<
%
% \end{implementation}^^A=<<
%
\endinput
%
^^A vim: ft=tex fdm=marker fmr=>>=,=<<
